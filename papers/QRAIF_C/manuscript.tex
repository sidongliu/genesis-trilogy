% ============================================================
% Q-RAIF Paper C: The Realizability Bridge (FINAL)
% ============================================================

\documentclass[12pt]{article}
\usepackage[margin=1in]{geometry}
\usepackage{amsmath,amssymb,amsthm}
\usepackage[hidelinks]{hyperref}
\usepackage{booktabs}
\usepackage{physics}

% Theorem environments — matching Paper A
\newtheorem{theorem}{Theorem}
\newtheorem{lemma}[theorem]{Lemma}
\newtheorem{definition}[theorem]{Definition}
\newtheorem{corollary}[theorem]{Corollary}
\newtheorem{remark}[theorem]{Remark}

\title{The Realizability Bridge: Algebraic Closure\\
in the Q-RAIF Framework}

\author{Sidong Liu, PhD \\
\small iBioStratix Ltd \\
\small \texttt{sidongliu@hotmail.com}}

\date{8th February 2026}

\begin{document}
\maketitle

% ============================================================
\begin{abstract}
This addendum provides a minimal mathematical bridge between the two foundational papers of the \textbf{Q-RAIF (Quantum Reference Algebra for Information Flow)} framework.
Paper A~\cite{Liu2026QRAIF_A} establishes that the observable algebra of a holographically consistent universe must contain $Cl(1,3)$ as its minimal Clifford-compatible structure.
Paper B~\cite{Liu2026QRAIF_B} establishes that the control algebra of a persistent subsystem must be Cliffordian $Cl(V,q)$ to ensure Lyapunov stability under entropic constraints.

Here we prove the \textbf{Closure Theorem}: any \emph{physically realizable} control algebra must embed into the environmental algebra as a subalgebra.
We formalize the required feedback synchrony via a \emph{Same-Clock} co-indexing lemma, ensuring the feedback loop is thermodynamically potent.

This note does not modify Papers A or B; it supplies only the realizability bridge needed for algebraic closure.

\medskip
\noindent\textbf{Keywords}: Q-RAIF, realizability, representation, operator algebra, Clifford algebra, open quantum systems, Lyapunov stability, algebraic closure
\end{abstract}

% ============================================================
\section{Introduction}
\label{sec:intro}

\subsection{Context: The Q-RAIF Program}

The Quantum Reference Algebra for Information Flow (Q-RAIF) framework investigates what algebraic structures are \emph{necessary}---as opposed to merely convenient---for the self-consistent description of physical reality and persistence within it.
The program builds on the Holographic Alaya-Field Framework (HAFF)~\cite{Liu2026PaperA,Liu2026PaperB}, which establishes that geometry emerges from coarse-graining of observable algebras.

\begin{center}
\begin{tabular}{@{}llll@{}}
\toprule
\textbf{Paper} & \textbf{Question} & \textbf{Analogy} & \textbf{Result} \\
\midrule
HAFF~\cite{Liu2026PaperA} & How does geometry emerge? & Ocean & Algebra $\to$ Geometry \\
Q-RAIF A~\cite{Liu2026QRAIF_A} & What algebra does geometry need? & Water & $Cl(1,3)$ \\
Q-RAIF B~\cite{Liu2026QRAIF_B} & What algebra does survival need? & Fish & $Cl(V,q)$ \\
This work & Must the fish fit the water? & Bridge & $Cl(V,q) \hookrightarrow Cl(1,3)$ \\
\bottomrule
\end{tabular}
\end{center}

\subsection{The Logical Gap}

Papers A and B independently arrive at Clifford algebra from opposite directions.
Both papers explicitly note that this convergence is \emph{heuristic rather than deductive}~\cite{Liu2026QRAIF_A,Liu2026QRAIF_B}.
The present note closes the gap by proving a realizability constraint: the internal control algebra of any persistent subsystem must be representable within the external observable algebra.

\subsection{Scope}

This addendum introduces no new physical assumptions.
It uses only the objects and results already established in Papers A and B, and derives their mutual constraint.
Papers A and B remain unmodified.

% ============================================================
\section{Setup and Prerequisites}
\label{sec:setup}

Let $\mathcal{U}$ be a universe described by the Q-RAIF framework.
\begin{itemize}
    \item \textbf{Environment (``water'').} Let $\mathcal{A}_{\mathrm{ext}}$ denote the algebra of observables accessible at the holographic boundary.
    Paper A~\cite{Liu2026QRAIF_A} argues that $\mathcal{A}_{\mathrm{ext}}$ must contain $Cl(1,3)$ as its minimal Clifford-compatible subalgebra (Theorem~1 of Paper A, ``Clifford Compatibility'').
    \item \textbf{Subsystem (``fish'').} Let $\mathcal{O}_{\mathrm{int}}$ denote the internal control algebra of a persistent subsystem $R\subset\mathcal{U}$.
    Paper B~\cite{Liu2026QRAIF_B} argues that thermodynamic persistence requires $\mathcal{O}_{\mathrm{int}} \cong Cl(V,q)$ for some $(V,q)$ (Theorem~1 of Paper B, ``Persistence Compatibility'').
\end{itemize}

The remaining logical gap is the relationship between $\mathcal{O}_{\mathrm{int}}$ and $\mathcal{A}_{\mathrm{ext}}$: can a stable Clifford control algebra exist while being structurally disjoint from the available environmental observables?

% ============================================================
\section{Realizability and Same-Clock Co-Indexing}
\label{sec:realizability}

\begin{definition}[Algebraic Realizability]
\label{def:realizability}
A control algebra $\mathcal{O}_{\mathrm{int}}$ is \textbf{physically realizable} within an environment $\mathcal{A}_{\mathrm{ext}}$ if there exists a homomorphism
\begin{equation}
    \phi: \mathcal{O}_{\mathrm{int}} \to \mathcal{A}_{\mathrm{ext}}
\end{equation}
such that $\mathrm{Im}(\phi)$ has non-zero action on the interaction Hamiltonian $H_{\mathrm{int}}$, i.e., $[\mathrm{Im}(\phi), H_{\mathrm{int}}] \neq 0$.
This ensures that the controller can physically influence the system-environment boundary.
\end{definition}

Let $I$ be an operational/causal index set (e.g., proper-time frames or discretized event slices).
For a subset $J\subseteq I$, write $\mathcal{A}|_J$ for the restriction of an algebra $\mathcal{A}$ to the index set $J$.

\begin{lemma}[Same-Clock / Co-Indexing]
\label{lem:clock}
For a feedback loop to be causally closed and thermodynamically potent (capable of entropy export~\cite{Seifert2012}), there must exist non-null index overlap between control and feedback windows: there exist $J_{\mathrm{ctrl}},J_{\mathrm{env}}\subseteq I$ such that
\begin{enumerate}
    \item \textbf{Non-null intersection:} $J_{\mathrm{ctrl}}\cap J_{\mathrm{env}}\neq\emptyset$.
    \item \textbf{Window integrity:} on any critical lookback window $W\subseteq J_{\mathrm{ctrl}}\cap J_{\mathrm{env}}$ used to define the controller, $\mathcal{A}_{\mathrm{ext}}|_W$ is well-defined (no holes on $W$).
\end{enumerate}
\end{lemma}

\begin{proof}
If $J_{\mathrm{ctrl}}\cap J_{\mathrm{env}}=\emptyset$, the control action is operationally decoupled from environmental feedback, so no entropy export channel exists; persistence (NESS~\cite{Seifert2012}) fails.
If window integrity fails on a critical lookback window $W$, the feedback map---and thus the Lyapunov descent condition (Eq.~(4) of Paper B~\cite{Liu2026QRAIF_B})---is not definable on the operational window.
Therefore both conditions are necessary.
\end{proof}

% ============================================================
\subsection{Semisimplicity of Clifford Algebras}

\begin{lemma}[Injectivity from Semisimplicity]
\label{lem:semisimple}
Let $(V,q)$ be a finite-dimensional real vector space with
non-degenerate quadratic form.
Then $Cl(V,q)$ is semisimple.
If\/ $\dim V$ is even, $Cl(V,q)$ is simple, and every
non-zero algebra homomorphism
$\phi: Cl(V,q) \to \mathcal{A}$ is injective.
\end{lemma}

\begin{proof}
By the periodicity theorem for real Clifford
algebras~\cite{ABS1964,LawsonMichelsohn1989}, $Cl(V,q)$
with non-degenerate~$q$ is isomorphic to a matrix algebra
$M_n(K)$ or a direct sum $M_n(K) \oplus M_n(K)$,
where $K \in \{\mathbb{R}, \mathbb{C}, \mathbb{H}\}$
depends on the signature and dimension modulo~$8$.
In either case the algebra is semisimple.

When $\dim V$ is even, $Cl(V,q)$ is simple (a single matrix
block).
The kernel of any algebra homomorphism is a two-sided ideal;
a simple algebra admits no proper non-trivial ideals,
so $\ker\phi = \{0\}$ whenever $\phi \neq 0$.
\end{proof}

% ============================================================
\section{The Closure Theorem}
\label{sec:closure}

\begin{theorem}[Q-RAIF Algebraic Closure]
\label{thm:closure}
Assume $\mathcal{A}_{\mathrm{ext}} \supseteq Cl(1,3)$ (Paper A~\cite{Liu2026QRAIF_A}).
Let $R$ be a persistent subsystem whose control algebra satisfies $\mathcal{O}_{\mathrm{int}}\cong Cl(V,q)$ (Paper B~\cite{Liu2026QRAIF_B}).
If $\mathcal{O}_{\mathrm{int}}$ is realizable in $\mathcal{A}_{\mathrm{ext}}$ (Definition~\ref{def:realizability}) and the Same-Clock conditions of Lemma~\ref{lem:clock} hold, then the effective control algebra
\begin{equation}
    \mathcal{O}_{\mathrm{eff}} := \mathrm{Im}(\phi) \subseteq \mathcal{A}_{\mathrm{ext}}
\end{equation}
is a Clifford subalgebra of the external geometry.
\end{theorem}

\begin{proof}
By realizability, there exists a homomorphism $\phi:\mathcal{O}_{\mathrm{int}} \to \mathcal{A}_{\mathrm{ext}}$ with non-trivial image.
The operational content of the controller is its image $\mathcal{O}_{\mathrm{eff}} = \mathrm{Im}(\phi)$.
Since $\mathcal{O}_{\mathrm{int}} \cong Cl(V,q)$ by the persistence requirement (Theorem~1 of Paper B), and $\phi$ is structure-preserving, $\mathcal{O}_{\mathrm{eff}}$ inherits the Clifford relations $v^2 = q(v)\mathbf{1}$~\cite{Hestenes1966}.
Moreover, by Lemma~\ref{lem:semisimple}, if\/ $\dim V$ is even then $Cl(V,q)$ is simple and $\phi$ is necessarily injective; the image is therefore isomorphic to $Cl(V,q)$ itself, giving a genuine embedding $Cl(V,q) \hookrightarrow \mathcal{A}_{\mathrm{ext}}$.
In the physically relevant case ($\dim V = 4$, signature $(1,3)$ or compatible sub-signature), the even-dimensionality condition is satisfied.
Since $\mathcal{O}_{\mathrm{eff}} \subseteq \mathcal{A}_{\mathrm{ext}}$, the internal geometry $(V,q)$ is induced by a restriction of the ambient algebraic structure.
\end{proof}

\begin{corollary}[No Ghost Algebra]
A control algebra that is mathematically stable (Cliffordian) but not representable in $\mathcal{A}_{\mathrm{ext}}$ is not physically realizable.
In particular, a control structure with signature incompatible with $(1,3)$ cannot underwrite persistent feedback in a universe whose observable algebra contains $Cl(1,3)$.
\end{corollary}

% ============================================================
\section{Discussion}
\label{sec:discussion}

\subsection{What This Result Does and Does Not Show}

\textbf{Does show:}
Realizability forces the internal control algebra of a persistent subsystem to embed into the external observable algebra.
Combined with Papers A and B, this converts the previously heuristic convergence ($Cl(V,q)$ from stability, $Cl(1,3)$ from geometry) into a constrained embedding: $Cl(V,q) \hookrightarrow Cl(1,3)$.

\textbf{Does not show:}
That $\phi$ must be injective in general---however,
by Lemma~\ref{lem:semisimple}, injectivity \emph{is}
guaranteed when $\dim V$ is even
(which includes the physically relevant case $\dim V = 4$).
For odd $\dim V$, the image $\mathrm{Im}(\phi)$ is
isomorphic to a simple factor of $Cl(V,q)$ and still
carries the Clifford structure.
That the specific signature $(V,q)$ is uniquely determined---only that it must be compatible with $(1,3)$.
That this constitutes a derivation of physics from first principles---it is a consistency constraint within the Q-RAIF framework.

\subsection{The Bridge Statement}

\begin{remark}[Closing the Loop]
Paper A fixes the realizable operator content of the world ($\mathcal{A}_{\mathrm{ext}}$).
Paper B fixes the algebraic form required for persistence ($\mathcal{O}_{\mathrm{int}}$).
Theorem~\ref{thm:closure} locks them together: realizable persistence forces the agent's control algebra to be built from the same algebraic atoms as its environment.
The fish's gills must be made of water's molecules.
\end{remark}

\subsection{Connection to HAFF}

Within the HAFF program~\cite{Liu2026PaperA,Liu2026PaperB}, geometry emerges from coarse-graining of observable algebras.
The Closure Theorem adds a further structural consequence: not only does the world's geometry emerge from its algebra, but any persistent subsystem's internal geometry is \emph{constrained to be a restriction} of that emergent geometry.
This is algebraic natural selection operating at the level of geometric structure.

% ============================================================
\begin{thebibliography}{99}

\bibitem{Liu2026QRAIF_A}
S.~Liu, \emph{Algebraic Constraints on the Emergence of Lorentzian Metrics in Entropic Gravity Frameworks}, Zenodo (2026), DOI: \href{https://doi.org/10.5281/zenodo.18525876}{10.5281/zenodo.18525876}.

\bibitem{Liu2026QRAIF_B}
S.~Liu, \emph{Thermodynamic Stability Constraints on the Operator Algebra of Persistent Open Quantum Subsystems}, Zenodo (2026), DOI: \href{https://doi.org/10.5281/zenodo.18525890}{10.5281/zenodo.18525890}.

\bibitem{Liu2026PaperA}
S.~Liu, \emph{Emergent Geometry from Coarse-Grained Observable Algebras: The Holographic Alaya-Field Framework}, Zenodo (2026), DOI: 10.5281/zenodo.18361706.

\bibitem{Liu2026PaperB}
S.~Liu, \emph{Accessibility, Stability, and Emergent Geometry: Conceptual Clarifications on the Holographic Alaya-Field Framework}, Zenodo (2026), DOI: 10.5281/zenodo.18367060.

\bibitem{ABS1964}
M.~F.~Atiyah, R.~Bott and A.~Shapiro,
\emph{Clifford modules},
Topology \textbf{3}, Suppl.~1, 3--38 (1964).

\bibitem{Hestenes1966}
D.~Hestenes, \emph{Space-Time Algebra}, Gordon and Breach (1966).

\bibitem{LawsonMichelsohn1989}
H.~B.~Lawson and M.-L.~Michelsohn,
\emph{Spin Geometry},
Princeton University Press (1989).

\bibitem{Seifert2012}
U.~Seifert, \emph{Stochastic thermodynamics, fluctuation theorems and molecular machines}, Rep.\ Prog.\ Phys.\ \textbf{75}, 126001 (2012).

\end{thebibliography}

\end{document}
