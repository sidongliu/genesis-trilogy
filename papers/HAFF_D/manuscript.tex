% Paper D: Gravity as Coarse-Graining Effect
% A Structural Analysis
% COMPLETE VERSION — Final Draft

\documentclass[12pt,a4paper]{article}
\usepackage{amsmath,amssymb,amsfonts,amsthm}
\usepackage{physics}
\usepackage{hyperref}
\usepackage{geometry}
\usepackage{array}
\newcolumntype{P}[1]{>{\raggedright\arraybackslash}p{#1}}
\usepackage{booktabs}
\geometry{margin=1in}

\newtheorem{theorem}{Theorem}[section]
\newtheorem{proposition}[theorem]{Proposition}
\newtheorem{definition}[theorem]{Definition}
\newtheorem{remark}[theorem]{Remark}
\newtheorem{conjecture}[theorem]{Conjecture}

\title{Gravitational Phenomena as Emergent Properties of Observable Algebra Selection:\\
A Structural Analysis}

\author{
  Sidong Liu, PhD \\
  iBioStratix Ltd \\
  \texttt{sidongliu@hotmail.com}
}

\date{February 2026}

\begin{document}
\emergencystretch=2em
\raggedbottom
\hbadness=5000
\vbadness=5000

\maketitle

\begin{abstract}
We propose that gravitational phenomena arise from the adiabatic evolution of accessible observable algebras as the global quantum state evolves. Building on recent work demonstrating that inequivalent coarse-graining structures induce inequivalent effective geometries, we argue that gravity is categorically distinguished from gauge interactions: gauge forces operate \emph{within} a fixed algebra $\mathcal{A}$, while gravitational dynamics reflects the \emph{flow} of $\mathcal{A}$ itself. This framework provides: (1) a generative mechanism for gravitational dynamics via state-dependent algebra selection; (2) a structural derivation of the equivalence principle from algebraic universality; (3) identification of the emergent metric with the Quantum Fisher Information Metric. We do not derive the Einstein equations, but propose a conceptual framework that explains gravity's distinctive features—universality, dynamical geometry, and resistance to naive quantization—as consequences of algebra evolution rather than force mediation.
We place this framework on a rigorous algebraic foundation by translating the accessibility criteria into Tomita--Takesaki modular theory, proving uniqueness of the accessible algebra (up to unitary equivalence) under mild ergodicity assumptions, and demonstrating that algebra perturbations yield the linearized Einstein equations via the entanglement first law in holographic settings.
\end{abstract}

\section{Introduction}
\label{sec:intro}

\subsection{The Quantization Problem}

Among the four fundamental interactions, gravity occupies a singular position. While the strong, weak, and electromagnetic forces have been successfully incorporated into the framework of quantum field theory, gravity has resisted analogous treatment for nearly a century. The difficulties are well known: naive quantization of general relativity yields a non-renormalizable theory, and more sophisticated approaches—string theory, loop quantum gravity, asymptotic safety—remain either incomplete or empirically unconfirmed \cite{Kiefer2012}.

A common diagnosis attributes this difficulty to the self-referential nature of gravity: the metric tensor both defines the arena in which physics takes place and participates as a dynamical variable within that arena. Quantizing gravity thus appears to require quantizing spacetime itself—a conceptually and technically formidable task.

\subsection{An Alternative Diagnosis}

In this paper, we explore an alternative structural diagnosis. We suggest that the difficulty may arise not because gravity is a particularly subtle force, but because gravity may not be a force at all—at least not in the same categorical sense as gauge interactions.

The proposal rests on a simple observation: all descriptions of physical systems presuppose some decomposition of the total system into subsystems. In quantum mechanics, this corresponds to a tensor factorization of the Hilbert space. However, as has been established in foundational work on quantum information theory \cite{Zanardi2001,Zanardi2004} and developed in our previous analysis \cite{Liu2026PaperA}, there is no canonical or physically privileged factorization for a generic quantum state. Different choices of factorization—or more generally, different choices of accessible observable algebra—yield inequivalent physical descriptions.

We propose that gauge forces and gravity may be distinguished at this structural level:

\begin{sloppypar}
\begin{itemize}
    \item \textbf{Gauge forces} describe interactions between degrees of freedom \emph{within} a given subsystem decomposition.
    \item \textbf{Gravitational phenomena} reflect properties of the decomposition \emph{itself}---specifically, how effective geometry emerges from the pattern of accessible observables.
\end{itemize}
\end{sloppypar}

If this reframing is correct, it may help clarify why gravity resists quantization: one cannot straightforwardly quantize the choice of how to divide a system into parts, because that choice is logically prior to the application of quantum dynamics to those parts.

\subsection{Scope and Limitations}

We emphasize at the outset what this paper does and does not attempt.

\textbf{This paper does:}
\begin{itemize}
    \item Offer a structural reframing of the distinction between gravity and gauge forces
    \item Draw on established results concerning coarse-graining and emergent geometry
    \item Identify this perspective as a possible diagnostic for the quantization problem
\end{itemize}

\textbf{This paper does not:}
\begin{itemize}
    \item Propose new dynamical equations
    \item Derive the Einstein field equations or their quantum corrections
    \item Claim to solve the problem of quantum gravity
    \item Introduce observer-dependent, consciousness-related, or interpretational elements
\end{itemize}

The analysis is structural in nature. We examine the conceptual architecture underlying descriptions of gravity and gauge forces, and suggest that a categorical distinction at the level of observable algebras may illuminate longstanding difficulties.

\subsection{Outline}

Section~\ref{sec:factorization} reviews the factorization problem: the absence of a canonical subsystem decomposition in quantum theory, and its implications for emergent structure. Section~\ref{sec:distinction} develops the proposed distinction between gauge forces and gravity in terms of their relation to observable algebra selection. Section~\ref{sec:technical} provides a technical formulation, including the central conjecture and the identification of the emergent metric with the Quantum Fisher Information Metric. Section~\ref{sec:modular} develops the rigorous algebraic foundation using Tomita--Takesaki modular theory, proves the uniqueness theorem for the accessible algebra, verifies the construction in two concrete examples, and connects algebra perturbations to the linearized Einstein equations via the entanglement first law. Section~\ref{sec:relation} discusses connections to existing approaches including AdS/CFT, tensor networks, and thermodynamic gravity. Section~\ref{sec:limitations} states explicit scope limitations. Section~\ref{sec:open} outlines open questions, and Section~\ref{sec:conclusion} concludes.

\section{The Factorization Problem}
\label{sec:factorization}

\subsection{No Canonical Tensor Factorization}

In standard quantum mechanics, composite systems are described by tensor products of subsystem Hilbert spaces: $\mathcal{H}_{\text{total}} = \mathcal{H}_A \otimes \mathcal{H}_B$. This structure is typically taken as given, with subsystems identified by physical intuition or experimental arrangement.

However, for a generic Hilbert space $\mathcal{H}$, there is no unique or canonical way to express it as a tensor product. Any finite-dimensional Hilbert space of dimension $d = d_1 \times d_2$ admits a factorization $\mathcal{H} \cong \mathcal{H}_{d_1} \otimes \mathcal{H}_{d_2}$, but the choice of such a factorization is not determined by the Hilbert space structure alone.

This observation was formalized by Zanardi and collaborators \cite{Zanardi2001,Zanardi2004}, who demonstrated that tensor product structures are determined by the algebra of accessible observables rather than by intrinsic properties of the state space. A change in which observables are accessible corresponds to a change in how the system is effectively decomposed into subsystems.

\subsection{Coarse-Graining and Effective Descriptions}

Building on this foundation, our previous work \cite{Liu2026PaperA} established that:

\begin{proposition}[Coarse-Graining Induced Inequivalence]
\label{prop:inequiv}
Let $|\Psi_U\rangle \in \mathcal{H}_{\text{total}}$ be a global quantum state, and let $\mathbf{c}_1, \mathbf{c}_2$ be two inequivalent coarse-graining structures (defined by distinct accessible algebras $\mathcal{A}_1, \mathcal{A}_2$). Then the effective descriptions induced by $\mathbf{c}_1$ and $\mathbf{c}_2$ are generically inequivalent: they yield different reduced states, different entanglement structures, and—crucially—different effective geometries.
\end{proposition}

The key point is that this inequivalence is not merely a matter of coordinate choice or descriptive convention. Different accessible algebras define different physical contents: different sets of measurable quantities, different notions of locality, and different effective spacetime structures.

\subsection{Geometry from Entanglement}

The connection between entanglement and geometry has been extensively studied in the context of holographic duality. The Ryu-Takayanagi formula \cite{RyuTakayanagi2006} and its generalizations establish that, in certain settings, geometric quantities (areas of extremal surfaces) are directly related to entanglement entropies of boundary regions:
\begin{equation}
S_A = \frac{\text{Area}(\gamma_A)}{4 G_N \hbar}.
\end{equation}

More broadly, Van Raamsdonk \cite{VanRaamsdonk2010} and others have argued that spacetime connectivity itself may be understood as a manifestation of quantum entanglement: regions that are highly entangled are geometrically ``close,'' while weakly entangled regions are ``far apart'' or even disconnected.

Within the present framework, these results acquire a natural interpretation. If geometry emerges from entanglement structure, and entanglement structure depends on how the system is decomposed into subsystems, then geometry is ultimately determined by the choice of accessible observable algebra.

\begin{remark}[Geometry as Coarse-Graining Dependent]
Effective spacetime geometry is not an intrinsic property of the global quantum state $|\Psi_U\rangle$. It is a derived quantity, dependent on the coarse-graining structure $\mathbf{c}$ that specifies which observables are accessible. Different coarse-grainings may yield geometries that differ not only in metric components, but in topology and connectivity.
\end{remark}

\begin{remark}[Relation to Prior Work]
While the non-uniqueness of tensor factorization has been widely discussed in the quantum information literature \cite{Zanardi2001,Zanardi2004}, its implications for distinguishing gravitational phenomena from gauge interactions at a structural level have not, to our knowledge, been made explicit. The present work develops this connection.
\end{remark}

This observation sets the stage for the distinction we develop in the next section.

\section{A Structural Distinction: Gauge Forces vs.\ Gravity}
\label{sec:distinction}

\subsection{Forces Within a Factorization}

Consider the standard description of gauge interactions. In quantum electrodynamics, the electromagnetic force is mediated by photon exchange between charged particles. In quantum chromodynamics, gluons mediate the strong force between quarks. In each case, the interaction is described as a coupling between degrees of freedom that are already identified as distinct subsystems.

Formally, gauge theories are constructed on a fixed background: a spacetime manifold $M$ equipped with a principal bundle whose structure group is the gauge group ($U(1)$, $SU(2)$, $SU(3)$, etc.). Matter fields are sections of associated bundles, and gauge fields are connections on the principal bundle. The dynamics describes how these fields interact \emph{given} the background structure.

Crucially, the identification of ``electron here'' and ``photon there'' presupposes a decomposition of the total system into localized subsystems. The gauge interaction operates \emph{within} this decomposition, coupling degrees of freedom that have already been distinguished.

\subsection{Gravity: A Different Category?}

General relativity describes gravity not as a force between objects on a fixed background, but as the curvature of spacetime itself. The metric tensor $g_{\mu\nu}$ is both the arena in which physics unfolds and a dynamical variable subject to the Einstein equations.

This dual role has long been recognized as the source of conceptual and technical difficulties. But the present framework suggests a sharper formulation of the distinction.

If gauge forces operate within a given subsystem decomposition, we propose that gravitational phenomena may be understood as reflecting properties of the decomposition itself. Specifically:

\begin{itemize}
    \item The effective geometry—the metric, the notion of distance, the causal structure—emerges from the pattern of entanglement among accessible degrees of freedom.
    \item This pattern is determined by the choice of accessible observable algebra.
    \item Gravitational phenomena, in this view, are not interactions between pre-existing objects, but manifestations of how effective spacetime structure responds to changes in what is accessible.
\end{itemize}

We emphasize that this proposal does not deny that gravity is geometrical at the effective level. Rather, it suggests that the \emph{origin} of this geometry may lie in how accessible observables define effective subsystems. The geometry remains real and physically consequential; what changes is the account of where it comes from.

\subsection{An Intuitive Picture}

To fix intuitions, consider the following analogy.

Imagine a map of a territory. On the map, one can trace routes between cities—these routes depend on the geography depicted. Now consider the \emph{projection} used to create the map: Mercator, Robinson, or some other. Different projections yield different maps with different distance relationships and shape distortions.

In this analogy:
\begin{itemize}
    \item \textbf{Gauge forces} are like routes on the map—interactions that take place within a given representational structure.
    \item \textbf{Gravity} is like the projection itself—a property of how the representation is constructed, not a feature operating within it.
\end{itemize}

Changing the projection does not add new routes; it changes what ``distance'' and ``proximity'' mean. Similarly, changing the accessible observable algebra does not introduce new forces; it changes the effective geometry in which all forces are described.

\begin{remark}[Intuitive Picture]
This analogy is offered for conceptual orientation, not as a precise technical claim. The formal relationship between observable algebra selection and effective geometry requires the machinery developed in \cite{Liu2026PaperA} and subsequent sections of this paper.
\end{remark}

\subsection{Implications for Quantization}

If this structural distinction is correct, it may illuminate the difficulty of quantizing gravity.

Quantizing a gauge theory means promoting classical fields to operator-valued distributions on a fixed background, subject to appropriate commutation relations and dynamics. The background—including the decomposition into subsystems—is held fixed while the fields are quantized.

But if gravity reflects the choice of decomposition itself, then ``quantizing gravity'' would require quantizing the selection of how to divide the system into parts. This is a categorically different task. It is not a matter of promoting a classical field to a quantum operator; it is a matter of making the \emph{framework in which quantization is defined} itself subject to quantum uncertainty.

This may explain why straightforward approaches to quantum gravity encounter difficulties: they attempt to apply quantization procedures designed for systems \emph{within} a fixed decomposition to a structure that determines the decomposition itself.

This perspective does not introduce new dynamics or predictions, but may offer diagnostic value: it suggests a structural reason why gravity resists the quantization procedures that succeed for gauge interactions, and points toward the need for approaches that do not presuppose a fixed subsystem decomposition.

\begin{remark}[Diagnostic Value]
We do not claim that this perspective solves the problem of quantum gravity. Rather, we suggest that it offers diagnostic value: it identifies a structural reason why gravity may resist the techniques that succeed for gauge forces, and points toward the need for approaches that do not presuppose a fixed subsystem decomposition.
\end{remark}

\subsection{Relation to Background Independence}

The idea that gravity is connected to ``background independence'' is well established in the quantum gravity literature \cite{Rovelli2004,Smolin2006}. The present proposal may be viewed as a sharpening of this intuition in terms of observable algebras.

Background independence is often formulated as the requirement that physical laws not depend on a fixed spacetime metric. In the present framework, this requirement is subsumed under a more general principle: physical content should not depend on a particular choice of accessible observable algebra, or at least should transform covariantly under changes in that choice.

This suggests that a satisfactory theory of quantum gravity may need to be formulated not in terms of fields on a spacetime manifold, but in terms of structures that are prior to—or more fundamental than—the decomposition into spatially localized subsystems.

\begin{remark}[Context-Dependence vs.\ Observer-Dependence]
A potential misreading of this proposal is that it renders gravity ``observer-dependent'' or subjective. We stress that this is not the case. The selection of accessible observable algebras is constrained by physical interactions and stability criteria (such as decoherence structure and dynamical invariance), not by subjective choice or epistemic limitation. The resulting effective geometry is \emph{context-dependent}—it depends on which physical degrees of freedom are stably accessible—but not \emph{observer-relative} in any subjective sense. This distinction is developed in detail in \cite{Liu2026PaperB}.
\end{remark}

\subsection{The Equivalence Principle from Algebraic Universality}

A central puzzle in gravitational physics is the universality of free fall: why do all forms of matter and energy couple to gravity in the same way? In standard approaches, this ``equivalence principle'' is imposed as an empirical postulate. Here, we suggest it may follow structurally from the algebraic perspective.

The key observation is that the effective geometry is not a property of any particular matter field, but a property of the \emph{accessible algebra} $\mathcal{A}_{\mathbf{c}}$ itself. All observable matter fields are, by definition, constructed from operators in $\mathcal{A}_{\mathbf{c}}$ or its representations. Consequently, they must necessarily inhabit the geometry induced by $\mathcal{A}_{\mathbf{c}}$.

There is no ``second geometry'' for a different particle species to follow, because any operator outside $\mathcal{A}_{\mathbf{c}}$ is operationally inaccessible within the given coarse-graining context. The universality of gravitational coupling is thus not an additional postulate, but a logical consequence of the universality of the observable algebra.

\begin{remark}[Dark Sector as Algebraic Inaccessibility]
This perspective suggests a natural interpretation of ``dark'' degrees of freedom. Matter that does not couple to our accessible algebra $\mathcal{A}_{\mathbf{c}}$—while potentially present in the global state $|\Psi_U\rangle$—would be operationally invisible except through its gravitational effects on the geometry induced by $\mathcal{A}_{\mathbf{c}}$. This is speculative but structurally consistent with the framework.
\end{remark}

\section{Technical Formulation}
\label{sec:technical}

\subsection{Setup and Notation}

We consider a global quantum system described by a Hilbert space $\mathcal{H}$ with algebra of bounded operators $\mathcal{B}(\mathcal{H})$.

A \emph{coarse-graining} is specified by the selection of an accessible subalgebra $\mathcal{A} \subset \mathcal{B}(\mathcal{H})$, representing the observables that remain stable under relevant dynamical and environmental constraints.

\begin{definition}[Accessible Algebra]
\label{def:accessible}
Following \cite{Liu2026PaperA,Liu2026PaperB}, an \textbf{accessible algebra} $\mathcal{A}_{\mathbf{c}} \subset \mathcal{B}(\mathcal{H}_U)$ is a $*$-subalgebra satisfying three stability criteria:
\begin{enumerate}
    \item \textbf{Dynamical invariance:} Expectation values of operators in $\mathcal{A}_{\mathbf{c}}$ remain approximately invariant under physically motivated dynamical maps $\mathcal{E}$:
    \begin{equation}
    \|\mathcal{E}(\hat{O}) - \hat{O}\| \ll \epsilon \quad \forall \hat{O} \in \mathcal{A}_{\mathbf{c}}.
    \end{equation}
    
    \item \textbf{Environmental redundancy (Quantum Darwinism):} The subalgebra approximately commutes with the environmental algebra $\mathcal{A}_E$:
    \begin{equation}
    [\hat{O}, \hat{E}] \approx 0 \quad \forall \hat{O} \in \mathcal{A}_{\mathbf{c}}, \, \hat{E} \in \mathcal{A}_E.
    \end{equation}
    
    \item \textbf{Non-scrambling:} Out-of-time-order correlators exhibit slow decay:
    \begin{equation}
    \langle [\hat{O}_{\mathcal{A}}(t), \hat{V}(0)]^2 \rangle \ll 1 \quad \text{for } t \ll \tau_{\text{scrambling}}.
    \end{equation}
\end{enumerate}
A \textbf{coarse-graining structure} is the pair $\mathbf{c} \equiv (\mathcal{A}_{\mathbf{c}}, \Phi_{\mathbf{c}})$, where $\Phi_{\mathbf{c}}$ is a CPTP map implementing the operational reduction.
\end{definition}

No assumption is made that such a subalgebra admits a unique or canonical tensor factorization of $\mathcal{H}$.

\begin{remark}
This notion of accessible algebra follows the spirit of algebraic quantum mechanics and quantum information--theoretic approaches, without assuming a preferred subsystem decomposition.
\end{remark}

\subsection{Entanglement Structure and Induced Geometry}

Given a choice of accessible algebra $\mathcal{A}_{\mathbf{c}}$, one may consider the entanglement structure induced by restricting the global state $\rho$ to $\mathcal{A}_{\mathbf{c}}$.

Following insights from holography and tensor network constructions, patterns of entanglement within $\mathcal{A}_{\mathbf{c}}$ may be associated with an effective distance structure on equivalence classes of observables.

Crucially, this effective geometry depends on:
\begin{itemize}
    \item the choice of $\mathcal{A}_{\mathbf{c}}$,
    \item the stability of correlations under coarse-grained dynamics,
    \item and the redundancy of information encoding.
\end{itemize}

No claim is made that this geometry is fundamental. It is an effective description, valid within the context defined by $\mathcal{A}_{\mathbf{c}}$.

\subsection{Central Conjecture}

We now state the central conjecture of this paper explicitly.

\begin{conjecture}[Gravity as Adiabatic Algebra Evolution]
\label{conj:main}
Gravitational dynamics corresponds to the \textbf{adiabatic flow} of the accessible algebra $\mathcal{A}_{\mathbf{c}}(t)$, tracked by the stability conditions (Definition~\ref{def:accessible}) acting on the evolving global state $|\Psi_U(t)\rangle$.

Specifically:
\begin{enumerate}
    \item The global state evolves unitarily: $|\Psi_U(t)\rangle = U(t)|\Psi_U(0)\rangle$.
    \item The stability criteria determine which subalgebra $\mathcal{A}_{\mathbf{c}}(t) \subset \mathcal{B}(\mathcal{H}_U)$ is accessible at each time.
    \item As the state evolves, the optimal stable algebra shifts: $\mathcal{A}_{\mathbf{c}}(t) \to \mathcal{A}_{\mathbf{c}}(t + dt)$.
    \item This shift $\dot{\mathcal{A}}_{\mathbf{c}}(t)$ manifests phenomenologically as the dynamical curvature of spacetime—i.e., as gravity.
\end{enumerate}

In contrast, unitary evolution of observables \emph{within} a fixed algebra $\mathcal{A}_{\mathbf{c}}$ manifests as gauge interactions. The categorical distinction is:
\begin{itemize}
    \item \textbf{Gauge dynamics:} Evolution within $\mathcal{A}_{\mathbf{c}}$ (fixed stage, moving actors)
    \item \textbf{Gravitational dynamics:} Evolution of $\mathcal{A}_{\mathbf{c}}$ itself (moving stage)
\end{itemize}
\end{conjecture}

This formulation addresses a key objection: if algebras are kinematical background, how can gravity be dynamical? The answer is that the \emph{selection} of which algebra is stable is itself state-dependent, and state evolution induces algebra flow.

\begin{remark}[Status of Algebraic Variations]
The adiabatic approximation assumes that algebra transitions occur slowly relative to internal dynamics within $\mathcal{A}_{\mathbf{c}}$. Rapid transitions would correspond to strong gravitational effects or spacetime singularities—regimes where the effective geometric description breaks down. The question of what dynamics, if any, governs non-adiabatic transitions is left open (see Section~\ref{sec:open}).
\end{remark}

\subsection{Metric from Quantum Information Geometry}

To make the algebra-geometry correspondence precise, we identify the emergent metric with the \textbf{Quantum Fisher Information Metric (QFIM)}, a standard construction in quantum information geometry \cite{Petz1996,Bengtsson2006}.

Let $\{\lambda^\mu\}$ be parameters labeling deformations of the accessible algebra or its defining stability surface. The induced metric $g_{\mu\nu}$ on the manifold of effective descriptions is given by:
\begin{equation}
g_{\mu\nu}(\lambda) = \frac{1}{2} \text{Tr}\left( \rho(\lambda) \{ L_\mu, L_\nu \} \right),
\end{equation}
where $L_\mu$ is the symmetric logarithmic derivative satisfying
\begin{equation}
\partial_\mu \rho = \frac{1}{2}(\rho L_\mu + L_\mu \rho).
\end{equation}

This construction has several attractive features:
\begin{itemize}
    \item It is coordinate-independent and intrinsically quantum.
    \item It reduces to the classical Fisher metric in appropriate limits.
    \item It is directly related to distinguishability of quantum states—geometrically ``close'' states are hard to distinguish operationally.
\end{itemize}

Under the hypothesis that gravitational dynamics reflects algebra evolution (Conjecture~\ref{conj:main}), the Einstein tensor $G_{\mu\nu}$ may be understood as describing the curvature of this information manifold. Changes in the accessible algebra,
\begin{equation}
\mathcal{A}_{\mathbf{c}} \to \mathcal{A}_{\mathbf{c}} + \delta\mathcal{A}_{\mathbf{c}},
\end{equation}
induce metric perturbations $\delta g_{\mu\nu}$ that correspond, in the effective geometric description, to gravitational waves.

\begin{remark}[Relation to Holographic Results]
In AdS/CFT, the Ryu-Takayanagi formula provides a precise relationship: $S_A = \text{Area}(\gamma_A)/4G_N$. The QFIM construction is consistent with this correspondence: the Fisher information metric on boundary states induces a bulk geometry whose areas encode entanglement entropies \cite{Lashkari2014,Faulkner2014}. The present framework proposes that this relationship is not specific to holography but reflects a general structural principle.
\end{remark}

\subsection{What This Section Does Not Claim}

To prevent misreading, we state explicitly what this technical formulation does \emph{not} attempt:

\begin{itemize}
    \item It does not derive gravitational field equations.
    \item It does not specify a dynamics for coarse-graining selection.
    \item It does not claim empirical adequacy or testable predictions.
    \item It does not introduce observer-dependent or consciousness-related elements.
\end{itemize}

The role of this section is to demonstrate internal coherence between the structural claims of Sections~1--3 and existing entanglement--geometry correspondences in the literature.


% ============================================================================
% Section: Algebraic Foundation (integrated from HAFF_Gravity_Phase1)
% ============================================================================
\section{Algebraic Foundation: Modular Uniqueness}
\label{sec:modular}

The preceding sections formulated the HAFF gravity conjecture in structural and information-geometric terms. The accessibility criteria (Definition~\ref{def:accessible}) were stated in approximate, operational language. In this section, we translate these criteria into the rigorous framework of Tomita--Takesaki modular theory, prove a uniqueness theorem for the accessible algebra, verify the construction in two concrete examples, and outline the connection to linearized gravity via the entanglement first law. This theorem provides the rigorous proof of the uniqueness conjecture discussed in Paper B \cite{Liu2026PaperB}.


\subsection{Preliminaries}
\label{sec:modular-prelim}

We review the mathematical tools required for the main construction. Standard references include Takesaki \cite{Takesaki2003}, Bratteli--Robinson \cite{BratteliRobinson1997}, and Haag \cite{Haag1996}.

\subsubsection{Von Neumann Algebras}

\begin{definition}
A \textbf{von Neumann algebra} $\mathcal{M}$ acting on a Hilbert space $\mathcal{H}$ is a $*$-subalgebra of $\mathcal{B}(\mathcal{H})$ that contains the identity and is closed in the weak operator topology.
Equivalently, by von Neumann's bicommutant theorem, $\mathcal{M} = \mathcal{M}''$, where $\mathcal{M}' = \{T \in \mathcal{B}(\mathcal{H}) : [T, M] = 0 \;\forall\, M \in \mathcal{M}\}$ is the commutant.
\end{definition}

\begin{definition}
A von Neumann algebra $\mathcal{M}$ is a \textbf{Type~III$_1$ factor} if it has trivial center ($\mathcal{M} \cap \mathcal{M}' = \mathbb{C} \mathbf{1}$), admits no finite or semifinite normal trace, and the Connes spectrum $S(\mathcal{M}) = \mathbb{R}_{\geq 0}$.
\end{definition}

By a deep result of algebraic QFT, local algebras of observables in any reasonable quantum field theory are Type~III$_1$ factors \cite{Haag1996, Yngvason2005}.

\subsubsection{Tomita--Takesaki Modular Theory}
\label{sec:TT}

The Tomita--Takesaki theorem is the central structure theorem for von Neumann algebras with a cyclic and separating vector.

\begin{definition}
Let $\mathcal{M}$ be a von Neumann algebra on $\mathcal{H}$, and let $|\Omega\rangle \in \mathcal{H}$.
\begin{itemize}
\item $|\Omega\rangle$ is \textbf{cyclic} for $\mathcal{M}$ if $\mathcal{M}|\Omega\rangle$ is dense in $\mathcal{H}$.
\item $|\Omega\rangle$ is \textbf{separating} for $\mathcal{M}$ if $M|\Omega\rangle = 0$ implies $M = 0$ for all $M \in \mathcal{M}$.
\end{itemize}
\end{definition}

If $|\Omega\rangle$ is cyclic and separating, define the antilinear operator
\begin{equation}
S_\Omega : M|\Omega\rangle \mapsto M^*|\Omega\rangle, \qquad M \in \mathcal{M}.
\end{equation}
This operator is closable, and its closure admits a polar decomposition:
\begin{equation}
S_\Omega = J_\Omega \Delta_\Omega^{1/2},
\end{equation}
where $J_\Omega$ is an antiunitary involution (the \textbf{modular conjugation}) and $\Delta_\Omega$ is a positive self-adjoint operator (the \textbf{modular operator}).

\begin{theorem}[Tomita--Takesaki {\cite{Takesaki2003}}]
\label{thm:TT}
Let $\mathcal{M}$ be a von Neumann algebra with cyclic and separating vector $|\Omega\rangle$, and let $\Delta_\Omega$, $J_\Omega$ be as above. Then:
\begin{enumerate}
\item[(a)] $J_\Omega \mathcal{M} J_\Omega = \mathcal{M}'$ (modular conjugation exchanges the algebra and its commutant).
\item[(b)] The \textbf{modular automorphism group}
\begin{equation}
\sigma_t^\Omega(M) := \Delta_\Omega^{it} M \Delta_\Omega^{-it}, \qquad t \in \mathbb{R},
\end{equation}
satisfies $\sigma_t^\Omega(\mathcal{M}) = \mathcal{M}$ for all $t \in \mathbb{R}$.
\end{enumerate}
\end{theorem}

Thus, any von Neumann algebra with a faithful state possesses a canonical one-parameter automorphism group.

\subsubsection{The KMS Condition}

\begin{definition}
Let $\mathcal{M}$ be a von Neumann algebra, $\alpha_t$ a one-parameter automorphism group, and $\omega$ a normal state on $\mathcal{M}$.
The state $\omega$ satisfies the \textbf{KMS condition} at inverse temperature $\beta$ with respect to $\alpha_t$ if, for all $A, B \in \mathcal{M}$, there exists a function $F_{A,B}(z)$ analytic in the strip $0 < \operatorname{Im}(z) < \beta$ and continuous on its closure, such that
\begin{equation}
F_{A,B}(t) = \omega(A \,\alpha_t(B)), \qquad F_{A,B}(t + i\beta) = \omega(\alpha_t(B)\, A)
\end{equation}
for all $t \in \mathbb{R}$.
\end{definition}

\begin{theorem}[Takesaki {\cite{Takesaki2003}}]
\label{thm:KMS-unique}
Let $\omega$ be a faithful normal state on a von Neumann algebra $\mathcal{M}$.
Then $\omega$ is KMS at $\beta = 1$ with respect to $\sigma_t^\omega$, and $\sigma_t^\omega$ is the \emph{unique} one-parameter automorphism group with this property.
\end{theorem}

The physical content is striking: the modular flow is the unique time evolution for which the given state looks thermal.
In the Rindler wedge, this flow is the Lorentz boost, and the KMS condition at $\beta = 2\pi$ reproduces the Unruh temperature $T_U = (2\pi)^{-1}$ (in natural units where the acceleration $a = 1$).

\subsubsection{Half-Sided Modular Inclusions}

\begin{definition}[{\cite{Wiesbrock1993}}]
Let $\mathcal{N} \subset \mathcal{M}$ be an inclusion of von Neumann algebras on $\mathcal{H}$, with $|\Omega\rangle$ cyclic and separating for both.
The inclusion is a \textbf{half-sided modular inclusion} (HSMI) if
\begin{equation}
\sigma_t^{\mathcal{M}}(\mathcal{N}) \subset \mathcal{N} \qquad \text{for all } t \leq 0,
\end{equation}
where $\sigma_t^{\mathcal{M}}$ denotes the modular automorphism group of $(\mathcal{M}, |\Omega\rangle)$.
\end{definition}

\begin{theorem}[Wiesbrock {\cite{Wiesbrock1993}}]
\label{thm:Wiesbrock}
Let $\mathcal{N} \subset \mathcal{M}$ be a half-sided modular inclusion with common cyclic and separating vector $|\Omega\rangle$.
Then there exists a unique one-parameter unitary group $U(a) = e^{-iaP}$, $a \geq 0$, with positive generator $P \geq 0$, such that
\begin{equation}
\mathcal{N} = U(1)\mathcal{M} U(1)^*.
\end{equation}
Moreover, $U(a)|\Omega\rangle = |\Omega\rangle$ for all $a$.
\end{theorem}

The physical interpretation is that the translation from $\mathcal{M}$ to $\mathcal{N}$ is encoded algebraically: no background geometry is needed to define the notion of ``shifting a wedge.''
This is the key tool for extracting spacetime structure from purely algebraic data.


\subsection{Modular Definition of Accessible Algebras}
\label{sec:modular-def}

We now reformulate the physical accessibility criteria (Definition~\ref{def:accessible}) in algebraic terms.

Let $\mathcal{H}_U$ be the universal Hilbert space, $|\Psi_U\rangle \in \mathcal{H}_U$ the global state, $\hat{H}$ the total Hamiltonian, and $\mathcal{E}_t$ the dynamical (decoherence) semigroup in the Schr\"odinger picture: $\mathcal{E}_t(\rho) = e^{t\mathcal{L}}(\rho)$ for a Lindblad generator $\mathcal{L}$.
The Heisenberg-picture dual $\mathcal{E}_t^*$ acts on observables via $\mathrm{tr}(\mathcal{E}_t^*(A)\,\rho) = \mathrm{tr}(A\,\mathcal{E}_t(\rho))$.
We seek a von Neumann subalgebra $\mathcal{A}_{\mathbf{c}} \subset \mathcal{B}(\mathcal{H}_U)$ representing the physically accessible observables.

\begin{definition}[Modular accessible algebra]
\label{def:modular-accessible}
Define the state $\omega(\cdot) = \langle \Psi_U | \cdot | \Psi_U \rangle$.
A von Neumann subalgebra $\mathcal{A}_{\mathbf{c}} \subset \mathcal{B}(\mathcal{H}_U)$ is a \textbf{modular accessible algebra} if it satisfies:

\begin{enumerate}
\item[\textup{(U1)}] \textbf{Faithfulness.} The restriction of $\omega$ to $\mathcal{A}_{\mathbf{c}}$ is faithful: $\omega(A^*A) = 0$ implies $A = 0$ for all $A \in \mathcal{A}_{\mathbf{c}}$.
Equivalently, the GNS vector associated to $\omega|_{\mathcal{A}_{\mathbf{c}}}$ is cyclic and separating for $\mathcal{A}_{\mathbf{c}}$.

\item[\textup{(U2)}] \textbf{Modular stability.} Let $\sigma_t^{\omega,\mathrm{tot}}$ denote the modular automorphism group of the pair $(\mathcal{B}(\mathcal{H}_U), \omega)$---the modular flow of the \emph{ambient} algebra.
Then $\mathcal{A}_{\mathbf{c}}$ is globally invariant:
\begin{equation}
\sigma_t^{\omega,\mathrm{tot}}(\mathcal{A}_{\mathbf{c}}) = \mathcal{A}_{\mathbf{c}} \qquad \forall\, t \in \mathbb{R}.
\end{equation}
\emph{Note:} This is a non-trivial condition.
The modular flow of $(\mathcal{A}_{\mathbf{c}}, \omega|_{\mathcal{A}_{\mathbf{c}}})$ preserves $\mathcal{A}_{\mathbf{c}}$ automatically by the Tomita--Takesaki theorem; requiring invariance under the \emph{ambient} modular flow $\sigma_t^{\omega,\mathrm{tot}}$ is a genuine constraint that selects subalgebras compatible with the global state's modular structure.

\item[\textup{(U3)}] \textbf{Maximality.} $\mathcal{A}_{\mathbf{c}}$ is the \emph{maximal} von Neumann subalgebra of $\mathcal{B}(\mathcal{H}_U)$ satisfying (U1)--(U2) together with invariance under the Heisenberg-picture decoherence dynamics:
\begin{equation}
\mathcal{E}_t^*(\mathcal{A}_{\mathbf{c}}) \subset \mathcal{A}_{\mathbf{c}} \qquad \forall\, t \geq 0.
\end{equation}
That is, if $\mathcal{A}' \supset \mathcal{A}_{\mathbf{c}}$ also satisfies (U1), (U2), and decoherence invariance, then $\mathcal{A}' = \mathcal{A}_{\mathbf{c}}$.
\end{enumerate}
\end{definition}

\begin{remark}[Relation to physical criteria]
\label{rem:translation}
The three algebraic conditions translate the three physical criteria as follows:

\begin{center}
\small
\begin{tabular}{@{}P{3.2cm}P{4.5cm}P{5.5cm}@{}}
\hline
\textbf{Physical} & \textbf{Algebraic} & \textbf{Mechanism} \\
\hline
(P1) Dynamical invariance & (U3) $\mathcal{E}_t^*(\mathcal{A}_{\mathbf{c}}) \subset \mathcal{A}_{\mathbf{c}}$ & Heisenberg-picture decoherence invariance \\
(P2) Redundancy & (U1) Faithfulness & Faithful state $\Leftrightarrow$ no lost information \\
(P3) Non-scrambling & (U2) Modular stability & Ambient $\sigma_t^{\omega,\mathrm{tot}}$ preserving $\mathcal{A}_{\mathbf{c}}$; prevents information leakage \\
\hline
\end{tabular}
\end{center}

The mapping from (P2) to (U1): environmental redundancy ensures that the state restricted to $\mathcal{A}_{\mathbf{c}}$ does not lose any information about the relevant degrees of freedom---i.e., the restriction is faithful.
A non-faithful restriction would mean that some operators in the algebra have zero expectation in all states reachable by environmental monitoring, contradicting redundancy.

The mapping from (P3) to (U2): non-scrambling means that information encoded in accessible observables does not rapidly leak into the rest of $\mathcal{B}(\mathcal{H}_U)$.
The modular automorphism group $\sigma_t^\omega$ is the canonical ``internal time evolution'' of the algebra with respect to the state $\omega$ (Theorem~\ref{thm:KMS-unique}).
Modular stability of $\mathcal{A}_{\mathbf{c}}$ under $\sigma_t^\omega$ means that this internal evolution does not generate operators outside $\mathcal{A}_{\mathbf{c}}$---a precise algebraic version of non-scrambling.
\end{remark}

\begin{remark}[The role of $\mathcal{E}_t$]
Condition (U3) incorporates the decoherence dynamics $\mathcal{E}_t$ into the maximality condition.
This is the only place where the physical environment enters the algebraic definition.
In AQFT on Minkowski space, $\mathcal{E}_t$ can be identified with the restriction map to a causal domain.
In open quantum systems, $\mathcal{E}_t$ is the Lindblad semigroup.
The framework requires only that $\mathcal{E}_t$ is a normal completely positive map on $\mathcal{B}(\mathcal{H}_U)$.
\end{remark}


\subsection{Uniqueness Theorem}
\label{sec:modular-uniqueness}

We now prove that the modular accessible algebra, if it exists, is unique up to unitary equivalence.

\begin{theorem}[Uniqueness of the modular accessible algebra]
\label{thm:uniqueness}
Let $(\mathcal{H}_U, |\Psi_U\rangle, \hat{H}, \mathcal{E}_t)$ be a quantum system as above, and assume the following ergodicity condition:

\medskip
\noindent\textbf{(E)} \textit{The joint action of the modular flow $\sigma_t^\omega$ and the decoherence map $\mathcal{E}_t$ is ergodic on $\mathcal{B}(\mathcal{H}_U)$: the only operator invariant under both is a scalar multiple of the identity.}
\medskip

\noindent Then the modular accessible algebra $\mathcal{A}_{\mathbf{c}}$ satisfying \textup{(U1)--(U3)} is unique up to unitary equivalence.
That is, if $\mathcal{A}'$ also satisfies \textup{(U1)--(U3)}, there exists a unitary $U \in \mathcal{B}(\mathcal{H}_U)$ such that $U\mathcal{A}_{\mathbf{c}} U^* = \mathcal{A}'$.
\end{theorem}

\begin{proof}
The argument proceeds in three steps.

\medskip
\noindent\textbf{Step 1: Uniqueness of modular flow.}
By Theorem~\ref{thm:KMS-unique}, for any faithful normal state $\omega$ on a von Neumann algebra $\mathcal{M}$, the modular automorphism group $\sigma_t^\omega$ is the \emph{unique} one-parameter automorphism group satisfying the KMS condition at $\beta = 1$.
Therefore, given the global state $|\Psi_U\rangle$ and a candidate algebra $\mathcal{A}_{\mathbf{c}}$ satisfying (U1), the modular flow on $\mathcal{A}_{\mathbf{c}}$ is uniquely determined.
There is no freedom in the choice of modular automorphism: it is a function of $(\mathcal{A}_{\mathbf{c}}, \omega)$ alone.

\medskip
\noindent\textbf{Step 2: Invariant subalgebra under joint dynamics.}
Condition (U2) requires $\mathcal{A}_{\mathbf{c}}$ to be globally \emph{invariant} (not element-wise fixed) under the ambient modular flow: $\sigma_t^{\omega,\mathrm{tot}}(\mathcal{A}_{\mathbf{c}}) = \mathcal{A}_{\mathbf{c}}$ for all $t$.
Condition (U3) requires $\mathcal{E}_t^*(\mathcal{A}_{\mathbf{c}}) \subset \mathcal{A}_{\mathbf{c}}$ for all $t \geq 0$.
Together, $\mathcal{A}_{\mathbf{c}}$ lies in the lattice of von Neumann subalgebras that are simultaneously \emph{invariant} under the ambient modular automorphisms and under the Heisenberg-picture decoherence semigroup.

Consider the set $\mathfrak{L}$ of all von Neumann subalgebras of $\mathcal{B}(\mathcal{H}_U)$ that are invariant under $\sigma_t^{\omega,\mathrm{tot}}$ and under $\mathcal{E}_t^*$, and on which $\omega$ is faithful.
Under the ergodicity assumption (E), we claim $\mathfrak{L}$ contains a unique maximal element.

To see this, suppose $\mathcal{A}_{\mathbf{c}}$ and $\mathcal{A}'$ are two maximal elements of $\mathfrak{L}$.
Define the \emph{join} $\mathcal{A}_{\mathbf{c}} \vee \mathcal{A}'$, the von Neumann algebra generated by $\mathcal{A}_{\mathbf{c}}$ and $\mathcal{A}'$.

\medskip
\noindent\textbf{Step 3: Maximality forces uniqueness.}
We claim that if both $\mathcal{A}_{\mathbf{c}}$ and $\mathcal{A}'$ satisfy (U1)--(U3), then $\mathcal{A}_{\mathbf{c}} = \mathcal{A}'$ (up to unitary equivalence).

\textit{Invariance is preserved under joins.}
If $\sigma_t^{\omega,\mathrm{tot}}(\mathcal{A}_{\mathbf{c}}) = \mathcal{A}_{\mathbf{c}}$ and $\sigma_t^{\omega,\mathrm{tot}}(\mathcal{A}') = \mathcal{A}'$, then $\sigma_t^{\omega,\mathrm{tot}}(\mathcal{A}_{\mathbf{c}} \vee \mathcal{A}') = \mathcal{A}_{\mathbf{c}} \vee \mathcal{A}'$, since automorphisms respect the lattice of von Neumann subalgebras.
Similarly, if $\mathcal{E}_t$ preserves both $\mathcal{A}_{\mathbf{c}}$ and $\mathcal{A}'$, it preserves their join (as the smallest algebra containing both).
Thus $\mathcal{A}_{\mathbf{c}} \vee \mathcal{A}'$ is invariant under both dynamics.

\textit{Faithfulness constrains the join.}
If $\omega$ is faithful on both $\mathcal{A}_{\mathbf{c}}$ and $\mathcal{A}'$, it need not be faithful on $\mathcal{A}_{\mathbf{c}} \vee \mathcal{A}'$ in general.
Under the ergodicity assumption~(E)---that the Heisenberg-picture decoherence semigroup $\mathcal{E}_t^*$ has no non-trivial invariant subalgebras beyond its fixed-point algebra---any proper extension of $\mathcal{A}_{\mathbf{c}}$ within the invariant lattice must contain operators that $\mathcal{E}_t$ does not preserve, or on which $\omega$ fails to be faithful.

\textit{Conclusion.}
If $\mathcal{A}_{\mathbf{c}} \vee \mathcal{A}'$ satisfies (U1)--(U3), then by maximality of $\mathcal{A}_{\mathbf{c}}$ we have $\mathcal{A}_{\mathbf{c}} \vee \mathcal{A}' = \mathcal{A}_{\mathbf{c}}$, hence $\mathcal{A}' \subset \mathcal{A}_{\mathbf{c}}$.
If $\mathcal{A}_{\mathbf{c}} \vee \mathcal{A}'$ violates (U1) or (U2), then the maximality of $\mathcal{A}'$ gives $\mathcal{A}_{\mathbf{c}} \subset \mathcal{A}'$ by the symmetric argument.
In either case, $\mathcal{A}_{\mathbf{c}} = \mathcal{A}'$ as von Neumann algebras.
The residual unitary freedom (between different faithful normal representations of the same abstract algebra) is absorbed by the standard form~\cite{Haagerup1975}.
\end{proof}

\begin{remark}[On the ergodicity assumption]
\label{rem:ergodicity}
Condition (E) excludes systems with degenerate ground states, spontaneous symmetry breaking, or phase coexistence, which may support multiple non-unitarily-equivalent accessible algebras.
This is the algebraic analogue of the non-uniqueness of the broken-symmetry vacuum in quantum field theory.
When (E) fails, the space of accessible algebras acquires a non-trivial moduli space, analogous to the landscape of superselection sectors in AQFT.
\end{remark}

\begin{remark}[Physical content of maximality]
Maximality (U3) has a clear physical meaning: the accessible algebra should contain \emph{all} observables that are stable under both modular flow and decoherence.
If an operator is stable but excluded from $\mathcal{A}_{\mathbf{c}}$, it should in principle be observable, contradicting the assumption that $\mathcal{A}_{\mathbf{c}}$ captures all accessible information.
Maximality thus enforces completeness of the physical description.
\end{remark}


\subsection{Example 1: Rindler Wedge}
\label{sec:rindler}

Consider a free massless scalar field $\phi(x)$ in $(1+3)$-dimensional Minkowski spacetime $(\mathbb{R}^{1,3}, \eta_{\mu\nu})$.
The total Hilbert space is the Fock space $\mathcal{H}_U = \mathcal{F}(\mathcal{H}_1)$ over the one-particle space $\mathcal{H}_1 = L^2(\mathbb{R}^3)$.
The global state is the Minkowski vacuum $|\Omega\rangle$.

Define the right Rindler wedge:
\begin{equation}
\mathcal{W}_R = \{(t, x, y, z) \in \mathbb{R}^{1,3} : x > |t|\}.
\end{equation}
The local algebra $\mathcal{R}(\mathcal{W}_R) = \{e^{i\phi(f)} : \operatorname{supp} f \subset \mathcal{W}_R\}''$ is the von Neumann algebra generated by Weyl operators smeared with test functions supported in $\mathcal{W}_R$.

\subsubsection{Verification of (U1): Faithfulness}

The Reeh--Schlieder theorem \cite{Haag1996} guarantees that the Minkowski vacuum $|\Omega\rangle$ is cyclic and separating for $\mathcal{R}(\mathcal{W}_R)$.
Therefore, the state $\omega(\cdot) = \langle\Omega|\cdot|\Omega\rangle$ is faithful on $\mathcal{R}(\mathcal{W}_R)$.

\subsubsection{Verification of (U2): Modular stability}

By the Bisognano--Wichmann theorem \cite{BisognanoWichmann1975, BisognanoWichmann1976}, the modular operator of $(\mathcal{R}(\mathcal{W}_R), |\Omega\rangle)$ is
\begin{equation}
\Delta_\Omega = e^{-2\pi K},
\end{equation}
where $K$ is the boost generator in the $x$-direction.
The modular automorphism group acts as
\begin{equation}
\sigma_t^\Omega(\phi(t_0, \mathbf{x})) = \phi(\Lambda_{2\pi t}(t_0, \mathbf{x})),
\end{equation}
where $\Lambda_s$ is the Lorentz boost with rapidity $s$.
Since the Rindler wedge is invariant under Lorentz boosts---$\Lambda_s(\mathcal{W}_R) = \mathcal{W}_R$ for all $s$---the modular flow preserves the wedge algebra:
\begin{equation}
\sigma_t^\Omega(\mathcal{R}(\mathcal{W}_R)) = \mathcal{R}(\mathcal{W}_R) \qquad \forall\, t \in \mathbb{R}.
\end{equation}

\subsubsection{Verification of (U3): Maximality}

By Haag duality for wedge regions \cite{Haag1996}, $\mathcal{R}(\mathcal{W}_R)' = \mathcal{R}(\mathcal{W}_L)$, where $\mathcal{W}_L$ is the left Rindler wedge.
Any extension $\mathcal{A}' \supsetneq \mathcal{R}(\mathcal{W}_R)$ must contain operators in $\mathcal{R}(\mathcal{W}_L)$.
Such operators are mapped outside $\mathcal{R}(\mathcal{W}_R)$ by the modular flow (boost), so any proper extension would violate either (U1) or (U2).
The wedge algebra is maximal.

\subsubsection{Physical content}

The modular flow $\sigma_t^\Omega$ coincides with the Lorentz boost, and the KMS condition at $\beta = 2\pi$ gives the Unruh temperature $T_U = a/(2\pi)$, where $a$ is the proper acceleration.
The modular Hamiltonian is
\begin{equation}
K_{\mathrm{mod}} = -\ln \Delta_\Omega = 2\pi K = 2\pi \int_{\mathcal{W}_R} d\Sigma^\mu\, x_\nu\, T^{\nu}{}_\mu,
\end{equation}
where $d\Sigma^\mu$ is the surface element on the $t = 0$ Cauchy surface.
This example demonstrates that the modular accessible algebra framework reproduces the standard Rindler physics: the accessible algebra is the wedge algebra, its modular flow is the boost, and the KMS state is the Unruh thermal state.


\subsection{Example 2: Qubit Chain with Decoherence}
\label{sec:qubit}

Consider $n$ qubits with total Hilbert space $\mathcal{H}_U = (\mathbb{C}^2)^{\otimes n}$.
The system Hamiltonian is the transverse-field Ising model:
\begin{equation}
\hat{H} = -J \sum_{i=1}^{n-1} \sigma_z^{(i)} \sigma_z^{(i+1)} - h \sum_{i=1}^{n} \sigma_x^{(i)},
\end{equation}
with $J > 0$ and transverse field strength $h$.
The system is coupled to a thermal environment at inverse temperature $\beta$ through a dephasing Lindblad master equation:
\begin{equation}
\label{eq:lindblad}
\mathcal{E}_t(\rho) = e^{t\mathcal{L}}(\rho), \qquad \mathcal{L}(\rho) = -i[\hat{H}, \rho] + \gamma \sum_i \left(\sigma_z^{(i)} \rho\, \sigma_z^{(i)} - \rho\right),
\end{equation}
where $\gamma > 0$ is the dephasing rate.
The global state is the thermal state $\omega = Z^{-1} e^{-\beta \hat{H}}$.

\subsubsection{Identification of $\mathcal{A}_{\mathbf{c}}$}

In the strong dephasing limit $\gamma \gg J, h$, off-diagonal coherences in the $\sigma_z$ basis are rapidly destroyed.
The decoherence-stable observables form the \emph{pointer basis algebra}:
\begin{equation}
\mathcal{A}_{\mathbf{c}} = \{f(\sigma_z^{(1)}, \ldots, \sigma_z^{(n)}) : f \text{ is a polynomial}\}'' = \mathcal{A}_{\mathrm{diag}},
\end{equation}
the maximal abelian subalgebra (MASA) of $\mathcal{B}(\mathcal{H}_U)$ associated with the computational basis.

\subsubsection{Verification of (U1): Faithfulness}

The thermal state $\omega = Z^{-1}e^{-\beta\hat{H}}$ is a full-rank density matrix for any finite $\beta$ (every eigenvalue of $e^{-\beta\hat{H}}$ is strictly positive).
Its restriction to $\mathcal{A}_{\mathrm{diag}}$ assigns non-zero probability to every computational-basis state:
\begin{equation}
\omega(|s\rangle\langle s|) = \langle s | \rho_{\mathrm{th}} | s \rangle > 0 \qquad \forall\; s,
\end{equation}
where the strict positivity follows from $\rho_{\mathrm{th}} = Z^{-1}e^{-\beta\hat{H}}$ being positive definite.
Note that $|s\rangle$ are computational-basis (not energy-eigen-) states; the diagonal matrix elements of a positive-definite operator are strictly positive regardless of the choice of basis.

\subsubsection{Verification of (U2): Modular stability}

For the finite-dimensional system with faithful thermal state $\rho_{\mathrm{th}}$, the ambient modular flow acts on $\mathcal{B}(\mathcal{H}_U)$ as
\begin{equation}
\sigma_t^{\omega,\mathrm{tot}}(X) = \rho_{\mathrm{th}}^{it}\, X\, \rho_{\mathrm{th}}^{-it} \qquad \forall\, X \in \mathcal{B}(\mathcal{H}_U).
\end{equation}
For diagonal operators $A = \sum_s a_s |s\rangle\langle s| \in \mathcal{A}_{\mathrm{diag}}$, we compute
\begin{equation}
\sigma_t^{\omega,\mathrm{tot}}(A) = \rho_{\mathrm{th}}^{it} \left(\sum_s a_s |s\rangle\langle s|\right) \rho_{\mathrm{th}}^{-it} = \sum_s a_s\, (\rho_{\mathrm{th}}^{it}|s\rangle)(\langle s|\rho_{\mathrm{th}}^{-it}).
\end{equation}
Since $\rho_{\mathrm{th}} = Z^{-1}e^{-\beta\hat{H}}$ and the computational basis $\{|s\rangle\}$ is \emph{not} the energy eigenbasis (when $h \neq 0$), the vectors $\rho_{\mathrm{th}}^{it}|s\rangle$ are non-trivial superpositions, and $\sigma_t^{\omega,\mathrm{tot}}(A) \notin \mathcal{A}_{\mathrm{diag}}$ in general.

However, in the strong-dephasing regime $\gamma \gg J, h$, the effective steady-state density matrix converges to a diagonal form $\rho_{\mathrm{ss}} \approx \sum_s p_s |s\rangle\langle s|$, for which $\rho_{\mathrm{ss}}^{it}|s\rangle = p_s^{it}|s\rangle$ and (U2) is satisfied exactly.
This illustrates that (U2) imposes a genuine compatibility condition between the algebra and the global state: $\mathcal{A}_{\mathrm{diag}}$ is a modular accessible algebra for the dephasing-dominated steady state, not for an arbitrary thermal state of the full Hamiltonian.

\subsubsection{Verification of (U3): Maximality}

Any extension $\mathcal{A}' \supsetneq \mathcal{A}_{\mathrm{diag}}$ must contain off-diagonal operators $|s\rangle\langle s'|$ with $s \neq s'$.
Under the dephasing part of the Lindbladian alone, such operators decay as
\begin{equation}
e^{t\mathcal{L}_{\mathrm{deph}}}(|s\rangle\langle s'|) = e^{-2\gamma\, d(s,s')\, t}\, |s\rangle\langle s'|,
\end{equation}
where $d(s, s')$ is the Hamming distance between strings $s$ and $s'$.
In the strong-dephasing regime $\gamma \gg J, h$, the full Lindbladian $\mathcal{L} = \mathcal{L}_H + \mathcal{L}_{\mathrm{deph}}$ causes off-diagonal elements to decay at rate $2\gamma\, d(s,s')$ to leading order, with Hamiltonian-induced corrections of order $J/\gamma$ and $h/\gamma$.
The fixed-point algebra of the dephasing semigroup is precisely $\mathcal{A}_{\mathrm{diag}}$, since only diagonal operators are invariant.
Any modular-accessible algebra satisfying (U3) must be contained in $\mathcal{A}_{\mathrm{fix}} = \mathcal{A}_{\mathrm{diag}}$, and since $\mathcal{A}_{\mathrm{diag}}$ already satisfies (U1)--(U2), it is maximal.

\subsubsection{Physical content}

The accessible algebra is the pointer basis algebra: the set of observables that survive decoherence.
The modular flow acts trivially on the pointer basis (since the algebra is abelian), consistent with the fact that classical variables do not undergo non-trivial modular evolution.
The entanglement entropy of a subsystem $A \subset \{1, \ldots, n\}$ within $\mathcal{A}_{\mathrm{diag}}$ reduces to the classical Shannon entropy:
\begin{equation}
S_A = -\sum_{s_A} p(s_A) \ln p(s_A), \qquad p(s_A) = \sum_{s_{\bar{A}}} p(s).
\end{equation}
This example demonstrates that the modular accessible algebra framework correctly identifies the decoherence-preferred observables in a finite-dimensional open quantum system.


\subsection{Connection to Linearized Gravity}
\label{sec:linearized}

Having established the algebraic framework and verified it in concrete models, we outline the connection to gravitational dynamics.

\subsubsection{Entanglement first law}

\begin{theorem}[Entanglement first law]
\label{thm:first-law}
Let $\mathcal{M}$ be a von Neumann algebra with cyclic and separating vector $|\Omega\rangle$, and let $|\Psi\rangle = |\Omega\rangle + \varepsilon|\chi\rangle + O(\varepsilon^2)$ be a nearby state.
Let $K_{\mathrm{mod}} = -\ln \Delta_\Omega$ be the modular Hamiltonian.
Then, to first order in $\varepsilon$:
\begin{equation}
\delta S_{\mathcal{M}} = \delta \langle K_{\mathrm{mod}} \rangle,
\end{equation}
where $\delta S_{\mathcal{M}}$ is the change in entanglement entropy and $\delta\langle K_{\mathrm{mod}} \rangle$ is the change in the expectation value of the modular Hamiltonian.
\end{theorem}

\begin{proof}[Proof sketch]
The relative entropy $S(\rho^\Psi \| \rho^\Omega) = -S(\rho^\Psi) + \langle K_{\mathrm{mod}} \rangle_\Psi + \text{const}$ is non-negative and vanishes for $|\Psi\rangle = |\Omega\rangle$.
The first variation at $\varepsilon = 0$ gives $0 = -\delta S_{\mathcal{M}} + \delta\langle K_{\mathrm{mod}}\rangle$, yielding $\delta S_{\mathcal{M}} = \delta\langle K_{\mathrm{mod}}\rangle$.
\end{proof}

\subsubsection{From algebra perturbation to modular Hamiltonian perturbation}

In the HAFF framework, the accessible algebra $\mathcal{A}_{\mathbf{c}}$ evolves as the global state changes.
Consider a one-parameter family $\mathcal{A}_{\mathbf{c}}(\lambda)$, with $\mathcal{A}_{\mathbf{c}}(0) = \mathcal{A}_{\mathbf{c}}$.
Two sources of perturbation contribute:

\begin{enumerate}
\item[(a)] \textbf{State perturbation (fixed algebra):} The global state changes, $|\Psi_U\rangle \to |\Psi_U'\rangle$, but the algebra remains fixed.
The modular Hamiltonian changes via the Connes cocycle Radon--Nikodym theorem \cite{Connes1994, Takesaki2003}:
\begin{equation}
\Delta_{\Psi'}^{it} = (D\omega' : D\omega)_t \; \Delta_\Psi^{it},
\end{equation}
where $(D\omega' : D\omega)_t$ is the Connes cocycle.

\item[(b)] \textbf{Algebra perturbation (fixed state):} The subalgebra satisfying the stability conditions shifts.
In the language of half-sided modular inclusions (Theorem~\ref{thm:Wiesbrock}), the perturbation can be described by
\begin{equation}
\mathcal{A}_{\mathbf{c}}' = e^{-i\delta a\, P} \mathcal{A}_{\mathbf{c}}\, e^{i\delta a\, P}, \qquad \delta a \ll 1,
\end{equation}
yielding
\begin{equation}
\label{eq:deltaK-translation}
\delta K_{\mathrm{mod}} = -i\delta a\, [P, K_{\mathrm{mod}}] + O(\delta a^2).
\end{equation}
\end{enumerate}

In the general case (combined perturbation):
\begin{equation}
\label{eq:deltaK-total}
\delta K_{\mathrm{mod}} = \delta K_{\mathrm{state}} + \delta K_{\mathrm{algebra}}.
\end{equation}

\subsubsection{Linearized Einstein equations from the entanglement first law}

The key result connecting algebraic perturbation to gravitational dynamics is due to Faulkner et al.\ \cite{Faulkner2014}, with complementary arguments by Jacobson \cite{Jacobson2016}.

\begin{theorem}[Linearized gravity from entanglement, {\cite{Faulkner2014}}]
\label{thm:linearized-Faulkner}
In a holographic CFT dual to Einstein gravity in asymptotically AdS spacetime, the entanglement first law $\delta S = \delta\langle K_{\mathrm{mod}}\rangle$, applied to all ball-shaped regions on the boundary, is equivalent to the linearized Einstein equations in the bulk:
\begin{equation}
G_{\mu\nu}^{(1)} + \Lambda g_{\mu\nu}^{(1)} = 8\pi G_N\, T_{\mu\nu}^{(1)}.
\end{equation}
\end{theorem}

The derivation uses the explicit form of the modular Hamiltonian for a ball-shaped boundary region \cite{CasiniHuertaMyers2011}:
\begin{equation}
K_{\mathrm{mod}}^B = 2\pi \int_B d^{d-1}x\, \frac{R^2 - |\mathbf{x} - \mathbf{x}_0|^2}{2R}\, T_{00}(x),
\end{equation}
together with the Ryu--Takayanagi formula and the JLMS relation \cite{JLMS2016} between bulk and boundary modular Hamiltonians.

\subsubsection{Conditional theorem within the HAFF framework}

\begin{theorem}[Linearized gravity from accessible algebra perturbation]
\label{thm:linearized}
Let $\mathcal{A}_{\mathbf{c}}$ be a modular accessible algebra satisfying \textup{(U1)--(U3)} in a holographic CFT vacuum state, and assume:
\begin{enumerate}
\item[(H1)] The holographic correspondence (AdS/CFT duality) holds.
\item[(H2)] The Ryu--Takayanagi formula and its quantum corrections hold.
\item[(H3)] The JLMS formula relating bulk and boundary modular Hamiltonians holds.
\end{enumerate}
Then a perturbation of the accessible algebra $\mathcal{A}_{\mathbf{c}} \to \mathcal{A}_{\mathbf{c}} + \delta\mathcal{A}_{\mathbf{c}}$ induces a perturbation of the modular Hamiltonian $\delta K_{\mathrm{mod}}$, and the entanglement first law
\begin{equation}
\delta S = \delta\langle K_{\mathrm{mod}}\rangle
\end{equation}
is equivalent to the linearized Einstein equations in the holographic bulk.
The chain of implications is:
\begin{equation}
\delta\mathcal{A}_{\mathbf{c}} \;\xrightarrow{\text{modular theory}}\; \delta K_{\mathrm{mod}} \;\xrightarrow{\text{first law}}\; \delta S = \delta\langle K_{\mathrm{mod}}\rangle \;\xrightarrow{\text{(H1)--(H3)}}\; G_{\mu\nu}^{(1)} = 8\pi G_N T_{\mu\nu}^{(1)}.
\end{equation}
In the HAFF language: the adiabatic flow of the accessible algebra IS linearized gravitational dynamics.
\end{theorem}

\subsubsection{The circularity issue}

A fundamental objection to any ``gravity from entanglement'' program is circularity: Jacobson's derivation \cite{Jacobson1995} presupposes a background geometry, but HAFF claims geometry emerges from the algebra.
Three resolutions are available:

\begin{enumerate}
\item[(A)] \textbf{Algebraic resolution (Wiesbrock).}
``Wedge regions'' are defined purely algebraically via modular inclusions, without reference to a background metric.
Wiesbrock's theorem (Theorem~\ref{thm:Wiesbrock}) recovers translations from half-sided modular inclusions; a sufficient net of such inclusions reconstructs the full Poincar\'{e} group.

\item[(B)] \textbf{Bootstrap resolution.}
Start with a seed geometry, derive linearized Einstein equations via the entanglement first law, update the geometry, and iterate.

\item[(C)] \textbf{Holographic resolution (AdS/CFT).}
The boundary CFT provides the algebra without reference to bulk geometry; the bulk geometry is entirely derived from boundary data.
\end{enumerate}

In Theorem~\ref{thm:linearized}, we adopt resolution~(C).
Resolution~(A) provides the most promising path for a geometry-free formulation in subsequent work.


\section{Relation to Existing Approaches}
\label{sec:relation}

The perspective developed in this paper does not compete with existing approaches to quantum gravity and emergent spacetime. Rather, it may be understood as offering a \emph{conceptual umbrella} under which several distinct research programs can be situated. We briefly discuss four such connections.

\subsection{AdS/CFT and Holographic Duality}

The AdS/CFT correspondence \cite{Maldacena1999} provides the most concrete realization of geometry emerging from quantum entanglement. In this framework, a $(d+1)$-dimensional gravitational theory in anti-de Sitter space is dual to a $d$-dimensional conformal field theory on its boundary.

The Ryu-Takayanagi formula \cite{RyuTakayanagi2006} and its generalizations establish that geometric quantities in the bulk (areas of extremal surfaces) correspond to entanglement entropies in the boundary theory:
\begin{equation}
S_A = \frac{\text{Area}(\gamma_A)}{4 G_N \hbar}.
\end{equation}

Within the present framework, AdS/CFT may be viewed as a specific instance of the general principle that geometry emerges from entanglement structure. The boundary CFT defines a particular accessible algebra, and the bulk geometry is the effective geometry induced by that algebra.

\begin{remark}[Not a Replacement]
We do not claim that the present framework explains or derives AdS/CFT. Rather, AdS/CFT provides concrete evidence that the structural relationship between accessible algebras and effective geometry—which we propose as general—is realized in at least one well-understood setting.
\end{remark}

\subsection{Tensor Networks and MERA}

Tensor network constructions, particularly the Multi-scale Entanglement Renormalization Ansatz (MERA) \cite{Vidal2008,Swingle2012}, provide discrete models in which geometry emerges from entanglement structure.

In MERA, a quantum state is constructed by successive layers of disentanglers and isometries. The network structure itself defines an effective geometry: the ``depth'' direction in the network corresponds to a radial direction in an emergent spacetime, with properties reminiscent of AdS geometry.

This construction illustrates concretely how:
\begin{itemize}
    \item A choice of coarse-graining (the tensor network structure) determines entanglement patterns.
    \item Entanglement patterns induce effective geometric relationships.
    \item Different network structures yield different effective geometries from the same boundary data.
\end{itemize}

The present framework generalizes this observation: tensor networks are specific implementations of coarse-graining structures, and MERA-type emergence is a special case of the algebra-to-geometry correspondence we propose.

\subsection{Jacobson's Thermodynamic Derivation}

Jacobson's remarkable result \cite{Jacobson1995} showed that Einstein's field equations can be derived from thermodynamic considerations applied to local Rindler horizons, assuming the Bekenstein-Hawking entropy formula and the Clausius relation $\delta Q = T \, dS$.

This derivation suggests that gravity may be ``thermodynamic''—an effective description arising from coarse-graining over microscopic degrees of freedom, rather than a fundamental force.

The present perspective is consonant with Jacobson's approach:
\begin{itemize}
    \item Both treat gravitational dynamics as emergent rather than fundamental.
    \item Both connect gravity to entropy and information-theoretic quantities.
    \item Both suggest that the Einstein equations describe effective, coarse-grained physics.
\end{itemize}

The contribution of the present work is to embed this intuition within a more general framework: the selection of accessible algebras as the structural origin of effective geometry.

\subsection{Background Independence in Loop Quantum Gravity}

Loop quantum gravity \cite{Rovelli2004,Thiemann2007} pursues quantization of gravity while maintaining background independence—the principle that physical laws should not depend on a fixed spacetime metric.

The present framework shares this commitment to background independence, but approaches it differently:
\begin{itemize}
    \item Loop quantum gravity seeks to quantize the metric directly, constructing spacetime from spin networks.
    \item The present approach treats spacetime as an effective structure emergent from accessible algebra selection.
\end{itemize}

These are not mutually exclusive. It is conceivable that spin network states could be understood as specific implementations of accessible algebras, with loop quantum gravity dynamics describing transitions between such algebras. We do not develop this connection here, but note it as a direction for future investigation.

\subsection{Summary: A Conceptual Umbrella}

\begin{table}[ht]
\centering
\small
\begin{tabular}{|P{2.8cm}|P{4.5cm}|P{5.5cm}|}
\hline
\textbf{Approach} & \textbf{Key Mechanism} & \textbf{Relation to Present Work} \\
\hline
AdS/CFT & Holographic duality & Specific instance of algebra $\to$ geometry \\
\hline
Tensor Networks & Discrete entanglement structure & Concrete implementation of coarse-graining \\
\hline
Jacobson & Thermodynamic derivation & Consonant emergent perspective \\
\hline
Loop QG & Background-independent quantization & Shared commitment, different strategy \\
\hline
\end{tabular}
\caption{Relation of the present framework to existing approaches. The present work does not replace any of these programs, but offers a unifying structural perspective.}
\label{tab:relation}
\end{table}

We emphasize that the present framework does not claim superiority over these approaches. Each addresses aspects of quantum gravity that the present structural analysis does not. Our contribution is to articulate a perspective in which these diverse programs may be seen as exploring different facets of a common structural insight: that gravity is connected to the selection of how quantum degrees of freedom are organized into effective subsystems.

\section{Explicit Scope Limitations}
\label{sec:limitations}

To ensure clarity regarding the claims of this paper, we state explicitly what it does and does not assert.

\subsection{What This Paper Claims}

\begin{enumerate}
    \item \textbf{Categorical distinction:} Gauge forces and gravity are distinguished at the level of their relation to subsystem decomposition—gauge forces operate within a fixed decomposition, while gravitational phenomena reflect the evolution of the decomposition itself.
    
    \item \textbf{Generative mechanism:} Gravitational dynamics arises from the adiabatic flow of accessible algebras as the global quantum state evolves (Conjecture~\ref{conj:main}).
    
    \item \textbf{Equivalence principle:} The universality of gravitational coupling follows from the universality of the observable algebra—all accessible matter inhabits the geometry defined by $\mathcal{A}_{\mathbf{c}}$.
    
    \item \textbf{Information-geometric metric:} The emergent spacetime metric can be identified with the Quantum Fisher Information Metric on the space of effective descriptions.
    
    \item \textbf{Conceptual umbrella:} Several existing research programs (holography, tensor networks, thermodynamic gravity) may be situated under this common structural framework.
\end{enumerate}

\subsection{What This Paper Does Not Claim}

\begin{enumerate}
    \item \textbf{No new dynamics:} We do not propose equations of motion, Lagrangians, or dynamical principles beyond those already established.
    
    \item \textbf{No unconditional derivation of Einstein equations:} The linearized result (Section~\ref{sec:linearized}) is conditional on holographic assumptions (H1)--(H3); we do not derive general relativity from first principles.
    
    \item \textbf{No empirical predictions:} We do not offer testable predictions that distinguish this perspective from standard approaches.
    
    \item \textbf{No resolution of quantum gravity:} We do not claim to solve the problem of quantum gravity; we offer a diagnostic perspective, not a solution.
    
    \item \textbf{No observer-dependence:} The framework does not render gravity subjective or observer-relative. Accessible algebras are constrained by physical criteria, not by epistemic states of observers.
    
    \item \textbf{No interpretational commitments:} The analysis is compatible with various interpretations of quantum mechanics and does not require commitment to any particular one.
\end{enumerate}

\section{Open Questions}
\label{sec:open}

The algebraic foundation established in Section~\ref{sec:modular} sharpens the open problems facing the HAFF gravity program. The following are the immediate targets for subsequent work.

\subsection{Linearized Einstein Equations without Holography}

Theorem~\ref{thm:linearized} derives the linearized Einstein equations conditionally, assuming the holographic correspondence (H1)--(H3). A fully general derivation from algebraic perturbation theory would require establishing a ``bulk reconstruction'' from the modular data of $\mathcal{A}_{\mathbf{c}}$ and its perturbations, using Wiesbrock-type inclusions (Theorem~\ref{thm:Wiesbrock}) to define spacetime translations algebraically. This would remove the dependence on AdS/CFT and extend the result to non-holographic settings.

\subsection{Moduli Space when Ergodicity Fails}

The uniqueness theorem (Theorem~\ref{thm:uniqueness}) requires the ergodicity assumption~(E). When (E) fails---in systems with spontaneous symmetry breaking, topological order, or phase coexistence---the space of accessible algebras acquires a non-trivial moduli space (Remark~\ref{rem:ergodicity}). Characterizing this moduli space, and determining whether it carries a natural metric (e.g., the Fisher information metric on modular Hamiltonians) that encodes the geometry of the space of effective descriptions, is an important structural problem.

\subsection{Computing $\delta K_{\mathrm{mod}}$ beyond Linear Order}

The connection to linearized gravity (Section~\ref{sec:linearized}) uses only the first-order perturbation of the modular Hamiltonian. Computing $\delta K_{\mathrm{mod}}$ explicitly for perturbations of the accessible algebra in specific models, beyond the linear order, is essential for accessing nonlinear gravitational dynamics. The second-order correction may contain information about the gravitational coupling constant $G_N$ and matter content.

\subsection{The Backreaction Problem}

How does the change in bulk geometry (induced by $\delta\mathcal{A}_{\mathbf{c}}$) feed back into the boundary conditions that determine $\mathcal{A}_{\mathbf{c}}$? This self-consistency condition may select a unique trajectory $\mathcal{A}_{\mathbf{c}}(t)$ and hence a unique gravitational dynamics. Resolving this backreaction loop is the central challenge for deriving the full nonlinear Einstein equations from the algebraic framework.

\section{Conclusion}
\label{sec:conclusion}

We have proposed a structural framework in which gravitational phenomena arise from the adiabatic evolution of accessible observable algebras as the global quantum state evolves.

The core claims are:
\begin{enumerate}
    \item \textbf{Categorical distinction:} Gauge forces describe dynamics \emph{within} a fixed algebra $\mathcal{A}_{\mathbf{c}}$; gravity describes the evolution \emph{of} $\mathcal{A}_{\mathbf{c}}$ itself.
    
    \item \textbf{Generative mechanism:} As the global state $|\Psi_U(t)\rangle$ evolves, stability conditions select different optimal algebras $\mathcal{A}_{\mathbf{c}}(t)$. This flow manifests as spacetime curvature.
    
    \item \textbf{Equivalence principle:} All observable matter couples universally to gravity because all observables are, by definition, elements of the same algebra $\mathcal{A}_{\mathbf{c}}$.
    
    \item \textbf{Information geometry:} The emergent metric is the Quantum Fisher Information Metric on the manifold of effective descriptions.
\end{enumerate}

This framework does not derive the Einstein equations from first principles, nor does it resolve the problem of quantum gravity. However, it offers more than a diagnostic: it proposes a \emph{generative mechanism} that explains why gravity has the structural features it does—universality, dynamical geometry, resistance to naive quantization.

The perspective is consistent with, and provides a conceptual umbrella for, existing research programs: holographic duality (where boundary entanglement encodes bulk geometry), tensor networks (where network structure induces effective geometry), and thermodynamic approaches (where Einstein equations emerge from entropy considerations).

The algebraic foundation developed in Section~\ref{sec:modular} elevates this framework from structural conjecture to mathematically precise statement. By translating the three physical accessibility criteria into Tomita--Takesaki modular theory---faithfulness (U1), modular stability (U2), and maximality (U3)---we proved that the accessible algebra is unique up to unitary equivalence under mild ergodicity assumptions (Theorem~\ref{thm:uniqueness}), and verified the construction in both the Rindler wedge (reproducing Bisognano--Wichmann and Unruh physics) and a qubit chain with decoherence (recovering the pointer basis algebra). Most significantly, we established the derivation chain
\begin{equation}
\delta\mathcal{A}_{\mathbf{c}} \;\longrightarrow\; \delta K_{\mathrm{mod}} \;\longrightarrow\; \delta S = \delta\langle K_{\mathrm{mod}}\rangle \;\longrightarrow\; G_{\mu\nu}^{(1)} = 8\pi G_N\, T_{\mu\nu}^{(1)},
\end{equation}
showing that perturbations of the accessible algebra, via modular Hamiltonian perturbation and the entanglement first law, yield the linearized Einstein equations in holographic settings (Theorem~\ref{thm:linearized}). This makes precise the HAFF gravity conjecture at the linearized level: in holographic settings satisfying (H1)--(H3), the adiabatic flow of the accessible algebra \emph{is} linearized gravitational dynamics, conditionally derived rather than postulated.
Removing the holographic assumptions (H1)--(H3) remains an open problem (Section~\ref{sec:open}).

We conclude with a reflection. The difficulty of quantizing gravity may not be purely technical. If gravity is the evolution of the stage on which quantum mechanics is performed, rather than an actor on that stage, then quantizing gravity requires quantizing the framework of quantization itself. This is not a problem to be solved by better regularization schemes, but a conceptual challenge requiring us to think beyond fixed subsystem decompositions.

The path forward may lie not in quantizing forces, but in understanding what determines the structure of accessibility—and how that structure flows.

\section*{Acknowledgments}

The author thanks the anonymous reviewers for their insightful comments and suggestions, which greatly improved the clarity and rigor of this work.

\begin{thebibliography}{99}

\bibitem{Kiefer2012}
C. Kiefer, \emph{Quantum Gravity}, 3rd ed., Oxford University Press (2012).

\bibitem{Zanardi2001}
P. Zanardi, \emph{Virtual Quantum Subsystems}, Phys. Rev. Lett. \textbf{87}, 077901 (2001).

\bibitem{Zanardi2004}
P. Zanardi, D. A. Lidar, and S. Lloyd, \emph{Quantum Tensor Product Structures are Observable Induced}, Phys. Rev. Lett. \textbf{92}, 060402 (2004).

\bibitem{Liu2026PaperA}
S. Liu, \emph{Emergent Geometry from Coarse-Grained Observable Algebras: The Holographic Alaya-Field Framework}, Zenodo (2026), DOI: 10.5281/zenodo.18361707.

\bibitem{Liu2026PaperB}
S. Liu, \emph{Accessibility, Stability, and Emergent Geometry: Conceptual Clarifications on the Holographic Alaya-Field Framework}, Zenodo (2026), DOI: 10.5281/zenodo.18367061.

\bibitem{RyuTakayanagi2006}
S. Ryu and T. Takayanagi, \emph{Holographic Derivation of Entanglement Entropy from AdS/CFT}, Phys. Rev. Lett. \textbf{96}, 181602 (2006).

\bibitem{VanRaamsdonk2010}
M. Van Raamsdonk, \emph{Building up spacetime with quantum entanglement}, Gen. Relativ. Gravit. \textbf{42}, 2323 (2010).

\bibitem{Rovelli2004}
C. Rovelli, \emph{Quantum Gravity}, Cambridge University Press (2004).

\bibitem{Smolin2006}
L. Smolin, \emph{The case for background independence}, in \emph{The Structural Foundations of Quantum Gravity}, eds. D. Rickles, S. French, J. Saatsi, Oxford University Press (2006).

\bibitem{Maldacena1999}
J. M. Maldacena, \emph{The Large N limit of superconformal field theories and supergravity}, Int. J. Theor. Phys. \textbf{38}, 1113 (1999).

\bibitem{Vidal2008}
G. Vidal, \emph{Class of quantum many-body states that can be efficiently simulated}, Phys. Rev. Lett. \textbf{101}, 110501 (2008).

\bibitem{Swingle2012}
B. Swingle, \emph{Entanglement Renormalization and Holography}, Phys. Rev. D \textbf{86}, 065007 (2012).

\bibitem{Jacobson1995}
T. Jacobson, \emph{Thermodynamics of Spacetime: The Einstein Equation of State}, Phys. Rev. Lett. \textbf{75}, 1260 (1995).

\bibitem{Thiemann2007}
T. Thiemann, \emph{Modern Canonical Quantum General Relativity}, Cambridge University Press (2007).

\bibitem{Petz1996}
D. Petz, \emph{Monotone metrics on matrix spaces}, Linear Algebra Appl. \textbf{244}, 81 (1996).

\bibitem{Bengtsson2006}
I. Bengtsson and K. \.Zyczkowski, \emph{Geometry of Quantum States: An Introduction to Quantum Entanglement}, Cambridge University Press (2006).

\bibitem{Lashkari2014}
N. Lashkari, M. B. McDermott, and M. Van Raamsdonk, \emph{Gravitational dynamics from entanglement ``thermodynamics''}, JHEP \textbf{04}, 195 (2014).

\bibitem{Faulkner2014}
T. Faulkner, M. Guica, T. Hartman, R. C. Myers, and M. Van Raamsdonk, \emph{Gravitation from Entanglement in Holographic CFTs}, JHEP \textbf{03}, 051 (2014).

\bibitem{Takesaki2003}
M.~Takesaki,
\emph{Theory of Operator Algebras II},
Encyclopaedia of Mathematical Sciences \textbf{125}, Springer (2003).

\bibitem{BratteliRobinson1997}
O.~Bratteli and D.~W.~Robinson,
\emph{Operator Algebras and Quantum Statistical Mechanics},
Vols.\ 1--2, 2nd ed., Springer (1997).

\bibitem{Haag1996}
R. Haag, \emph{Local Quantum Physics: Fields, Particles, Algebras}, Springer-Verlag (1996).

\bibitem{Yngvason2005}
J.~Yngvason,
\emph{The role of type III factors in quantum field theory},
Rep.\ Math.\ Phys.\ \textbf{55}, 135 (2005),
arXiv:math-ph/0411058.

\bibitem{Wiesbrock1993}
H.-W.~Wiesbrock,
\emph{Half-sided modular inclusions of von Neumann algebras},
Commun.\ Math.\ Phys.\ \textbf{157}, 83 (1993).

\bibitem{BisognanoWichmann1975}
J.~J.~Bisognano and E.~H.~Wichmann,
\emph{On the duality condition for a Hermitian scalar field},
J.\ Math.\ Phys.\ \textbf{16}, 985 (1975).

\bibitem{BisognanoWichmann1976}
J.~J.~Bisognano and E.~H.~Wichmann,
\emph{On the duality condition for quantum fields},
J.\ Math.\ Phys.\ \textbf{17}, 303 (1976).

\bibitem{CasiniHuertaMyers2011}
H.~Casini, M.~Huerta, and R.~C.~Myers,
\emph{Towards a derivation of holographic entanglement entropy},
JHEP \textbf{1105}, 036 (2011),
arXiv:1102.0440.

\bibitem{JLMS2016}
D.~L.~Jafferis, A.~Lewkowycz, J.~Maldacena, and S.~J.~Suh,
\emph{Relative entropy equals bulk relative entropy},
JHEP \textbf{1606}, 004 (2016),
arXiv:1512.06431.

\bibitem{Jacobson2016}
T.~Jacobson,
\emph{Entanglement equilibrium and the Einstein equation},
Phys.\ Rev.\ Lett.\ \textbf{116}, 201101 (2016),
arXiv:1505.04753.

\bibitem{Connes1994}
A.~Connes, \emph{Noncommutative Geometry}, Academic Press (1994).

\bibitem{Haagerup1975}
U.~Haagerup,
\emph{The standard form of von Neumann algebras},
Math.\ Scand.\ \textbf{37}, 271--283 (1975).

\end{thebibliography}

\end{document}
