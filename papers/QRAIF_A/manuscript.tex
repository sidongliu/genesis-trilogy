\documentclass[12pt]{article}
\usepackage[margin=1in]{geometry}
\usepackage{amsmath,amssymb,amsthm}
\usepackage{graphicx}
\usepackage{hyperref}
\usepackage{physics}
\usepackage{booktabs}
\usepackage{array}
\newcolumntype{P}[1]{>{\raggedright\arraybackslash}p{#1}}

\newtheorem{theorem}{Theorem}
\newtheorem{lemma}[theorem]{Lemma}
\newtheorem{proposition}[theorem]{Proposition}
\newtheorem{definition}[theorem]{Definition}
\newtheorem{remark}[theorem]{Remark}
\newtheorem{corollary}[theorem]{Corollary}

\title{Algebraic Constraints on the Emergence of Lorentzian Metrics\\
in Entropic Gravity Frameworks}

\author{Sidong Liu, PhD \\
\small iBioStratix Ltd \\
\small \texttt{sidongliu@hotmail.com}}

\date{February 2026}

\begin{document}
\emergencystretch=2em
\raggedbottom
\hbadness=5000
\vbadness=5000
\maketitle

% ============================================================
\begin{abstract}
We investigate the algebraic conditions under which an emergent bulk geometry acquires a Lorentzian signature within the framework of entropic gravity.
While thermodynamic approaches to gravity~\cite{Jacobson1995,Verlinde2011} and the Ryu--Takayanagi formula~\cite{RyuTakayanagi2006} relate entanglement entropy to geometric data, the specific algebraic mechanism constraining the spacetime signature remains an open question.

We identify three independent constraints on the boundary algebra---associativity, metric compatibility, and causal channel encoding---and argue that their simultaneous satisfaction naturally selects a \textbf{Clifford algebra} $Cl(V,q)$ as the minimal compatible structure.
We support this claim by systematically examining alternative algebraic frameworks (von Neumann factors, $C^*$-algebras, Jordan algebras, Lie algebras) and demonstrating that each fails to satisfy at least one constraint.
A worked example using a qubit tensor network illustrates how the three constraints operate in a concrete setting.

The analysis complements the Holographic Alaya-Field Framework (HAFF)~\cite{Liu2026PaperA,Liu2026PaperB}, which establishes that geometry emerges from coarse-graining of observable algebras: the present work characterizes the algebraic constraints that any such emergent geometry must satisfy to be Lorentzian.

\medskip
\noindent\textbf{Keywords}: emergent geometry, Clifford algebra, entropic gravity, holographic principle, signature selection, algebraic quantum gravity
\end{abstract}

% ============================================================
\section{Introduction}
\label{sec:intro}

\subsection{Context and Motivation}

The AdS/CFT correspondence~\cite{Maldacena1999} and the Ryu--Takayanagi formula~\cite{RyuTakayanagi2006} have established that spacetime geometry can be viewed as an emergent property of quantum entanglement.
Thermodynamic approaches~\cite{Jacobson1995,Verlinde2011} further suggest that the Einstein equations arise as an equation of state.
Yet a critical question remains: \textit{What algebraic constraints ensure that the emergent geometry is Lorentzian?}

Most models assume the $(1,3)$ signature \textit{a priori}.
We argue that this assumption can be partially justified by examining the algebraic consistency conditions on the boundary degrees of freedom.

\subsection{Relation to HAFF}

The Holographic Alaya-Field Framework~\cite{Liu2026PaperA,Liu2026PaperB} demonstrates that inequivalent coarse-graining structures on a global quantum state induce inequivalent emergent geometries.
HAFF establishes \emph{that} geometry emerges from observable algebras; the present work addresses \emph{what algebraic constraints} such emergent geometry must satisfy to be Lorentzian.

\begin{center}
\begin{tabular}{ll}
\textbf{HAFF} & Geometry is coarse-graining-dependent (the ``ocean'') \\
\textbf{This paper} & Lorentzian signature is algebraically constrained (the ``water'')
\end{tabular}
\end{center}

\subsection{Scope and Disclaimers}

This work does not propose a new fundamental theory of gravity, nor does it claim to derive $Cl(1,3)$ from first principles alone.
Rather, it identifies a set of physically motivated algebraic constraints and argues that Clifford algebra is the minimal structure satisfying all of them simultaneously.
The argument is presented as a \emph{consistency analysis}, not a uniqueness proof.

The specific value $(1,3)$ for the signature requires additional input beyond the algebraic constraints developed here (e.g., observational dimensionality or anomaly cancellation arguments).
We do not address why spacetime has $3+1$ dimensions.

% ============================================================
\section{Candidate Algebraic Structures}
\label{sec:candidates}

Before deriving constraints, we survey the landscape of algebraic structures that could, in principle, describe boundary degrees of freedom in a holographic setting.
This survey serves as the basis for the exclusion argument in Section~\ref{sec:exclusion}.

\begin{definition}[Boundary Algebra]
Let $\mathcal{A}_\partial$ be the algebra of observables on a holographic boundary.
We require: (a)~$\mathcal{A}_\partial$ acts faithfully on $\mathcal{H}_\partial$; (b)~$\mathcal{A}_\partial$ admits a trace compatible with the holographic entropy bound; (c)~coarse-graining of $\mathcal{A}_\partial$ induces an effective bulk description.
\end{definition}

The following algebraic families are candidates:

\begin{enumerate}
\item \textbf{von Neumann algebras} (Type I, II, III): Associative, closed under adjoint, weakly closed. Standard in algebraic QFT~\cite{Haag1996}. Type~III$_1$ factors are generic in relativistic QFT.
\item \textbf{$C^*$-algebras}: Associative Banach algebras with involution. More general than von Neumann algebras. Standard framework for quantum observables.
\item \textbf{Lie algebras}: Antisymmetric bracket $[A,B] = -[B,A]$, satisfying the Jacobi identity. Encode infinitesimal symmetries. The universal enveloping algebra is associative.
\item \textbf{Jordan algebras}: Commutative but generally non-associative: $A \circ B = B \circ A$, satisfying the Jordan identity. Proposed for quantum mechanics by Jordan, von Neumann, and Wigner (1934).
\item \textbf{Octonion algebras}: Non-associative division algebra. Explored in the context of exceptional structures in string theory~\cite{Gunaydin1973}.
\item \textbf{Clifford algebras} $Cl(V,q)$: Associative, generated by a vector space $V$ with quadratic form $q$, subject to $v^2 = q(v)\mathbf{1}$. Encode both metric and algebraic structure~\cite{Hestenes1966,Doran2003}.
\end{enumerate}

% ============================================================
\section{Three Algebraic Constraints}
\label{sec:constraints}

We now derive three constraints from physically motivated requirements and examine which candidate algebras survive.

\subsection{Constraint I: Associativity}

\begin{lemma}[Associativity from Compositional Consistency]
\label{lem:assoc}
If the boundary algebra supports well-defined time evolution (evolution operators forming a semigroup), it must be associative.
\end{lemma}

\begin{proof}
The semigroup property requires $(U(t_1)U(t_2))U(t_3) = U(t_1)(U(t_2)U(t_3))$ for all $t_i \geq 0$.
In a non-associative algebra, different bracketings of $n$ sequential operations produce $C_n \sim 4^n / n^{3/2}$ distinct results (Catalan numbers), generating uncontrolled ambiguity that grows exponentially with the number of time steps.

We note that this constraint is automatically satisfied by operator algebras on Hilbert spaces, where composition of linear maps is inherently associative.
The constraint therefore functions as a \emph{structural boundary condition}: it delineates the algebraic regime in which consistent dynamics is possible, rather than excluding a plausible physical alternative.
For analysis of non-associative dynamics and their instabilities, see~\cite{Schafer1966,Gunaydin1973}.
\end{proof}

\begin{remark}[Physical pathologies of non-associative dynamics]
\label{rem:nonassoc}
The exclusion of non-associative algebras is not merely formal.
In octonionic quantum mechanics~\cite{Gunaydin1973}, the ambiguity of operator ordering leads to violations of the no-signaling condition: the outcome statistics of a measurement on subsystem $A$ can depend on the bracketing convention chosen for a distant operation on subsystem $B$.
In Jordan-algebraic quantum mechanics~\cite{AlfsenShultz2003}, the lack of associativity prevents the construction of tensor product state spaces with the standard entanglement structure, obstructing quantum error correction and the holographic encoding required by Constraint~II.
These pathologies are not hypothetical: they represent a quantifiable breakdown of information-processing primitives (teleportation fidelity, entanglement monogamy) that underpin the remainder of the framework.
\end{remark}

\textbf{Exclusions}: Jordan algebras and octonion algebras are non-associative and are excluded by Constraint~I.

\subsection{Constraint II: Metric Compatibility}
\label{sec:constraint2}

\begin{lemma}[Non-Degenerate Bilinear Form from Holographic Error Correction]
\label{lem:metric}
For the boundary algebra to support error correction compatible with holographic bulk reconstruction, a non-degenerate bilinear form must be available on the space of boundary operators.
\end{lemma}

\begin{proof}
Error correction in the holographic context requires quantifying the ``distance'' between the actual boundary state and the target code subspace.
This requires a Lyapunov-type function $V(\delta\rho) \geq 0$ with $\dot{V} < 0$ under the correction protocol, which in turn requires a gradient flow:
\begin{equation}
\dot{\lambda} = -\Gamma\, G^{-1} \nabla_\lambda V,
\end{equation}
where $G$ is a metric on the parameter manifold of boundary states.

\textbf{Important distinction}: The metric $G$ appearing here is an \emph{information-geometric} metric on the space of boundary states (analogous to the Fisher--Rao metric~\cite{Petz1996}), not the emergent spacetime metric $g_{\mu\nu}$.
However, recent results in holographic entanglement~\cite{Faulkner2014,Lashkari2014} establish that linearized perturbations of the bulk metric $\delta g_{\mu\nu}$ are encoded in the boundary modular Hamiltonian and its associated Fisher information.
Specifically, the quantum-corrected Ryu--Takayanagi formula~\cite{Faulkner2014} implies:
\begin{equation}
\delta S_A = \delta \langle K_A \rangle + \delta S_{\text{bulk}},
\end{equation}
where $K_A$ is the boundary modular Hamiltonian and $S_A$ is the boundary entanglement entropy.
The structure of $G_{\text{info}}$ on the boundary therefore constrains the structure of $g_{\mu\nu}$ in the bulk.

The requirement is thus that the boundary algebra carries a non-degenerate bilinear form compatible with this holographic encoding.
Standard quantum-state metrics (Bures, Fisher--Rao) are positive-definite and satisfy non-degeneracy, but they do not encode signature information (see Constraint~III).
\end{proof}

\textbf{Exclusions}: Lie algebras carry a Killing form, but it may be degenerate (for non-semisimple algebras) and does not naturally encode a quadratic form on the generating vector space.
General $C^*$-algebras and von Neumann algebras support multiple choices of metric (Bures, Hilbert--Schmidt, etc.) but none is canonically ``built in'' to the algebraic structure itself.

\begin{remark}[Quantitative cost of external metrics]
\label{rem:metric-cost}
The distinction between built-in and external metrics is not merely aesthetic; it has quantifiable information-theoretic consequences.
For an algebra generated by $n$ directions, an \emph{arbitrary} metric on the generating space requires $n(n+1)/2$ real parameters (a symmetric bilinear form).
In a Clifford algebra $Cl(V,q)$, the quadratic form $q$ is specified by $n$ values $\{q(e_i)\}_{i=1}^n$ (plus, in principle, the off-diagonal terms, which are fixed by the anticommutation relations $\{e_i, e_j\} = 2q(e_i, e_j)\mathbf{1}$).
However, by diagonalizability, the metric information reduces to $n$ eigenvalues and an $O(n)$ frame---a total of $n$ real parameters in the canonical (diagonal) basis.

For error correction with an external metric, the correction protocol must \emph{independently specify} $G$ at each step, adding $O(n^2)$ overhead per correction cycle.
For a Clifford algebra, the metric is encoded in the algebraic structure itself: no additional specification is needed, reducing the overhead to $O(1)$ (the algebra already ``knows'' its metric).
In the high-dimensional limit $n \to \infty$ relevant to holographic boundary theories, the ratio of built-in to external information cost scales as $n / [n(n+1)/2] = 2/(n+1) \to 0$, giving Clifford algebras an asymptotically infinite efficiency advantage.

This efficiency advantage becomes a \emph{necessity} when the error correction must operate within the thermodynamic bounds of T-DOME (Paper~I): the entropy production rate for maintaining NESS is bounded, and an $O(n^2)$ metric overhead can violate the Markovian ceiling for sufficiently large $n$, while the $O(1)$ Clifford overhead does not.
\end{remark}

\subsection{Constraint III: Causal Channel Encoding}

\begin{lemma}[Indefinite Signature from Causality]
\label{lem:signature}
For the emergent geometry to distinguish time-like from space-like separation, the bilinear form must have indefinite signature $(p,q)$ with $p \geq 1$, $q \geq 1$.
\end{lemma}

\begin{proof}
The argument proceeds from the algebraic characterization of causal structure in quantum field theory~\cite{Haag1996}.

\textbf{Step 1: Commutativity encodes spacelike separation.}
In algebraic QFT, two observables $A$ and $B$ are spacelike separated if and only if $[A, B] = 0$ (microcausality axiom).
This is not a convention but a physical requirement: spacelike-separated measurements must be jointly performable, which demands commutativity of the corresponding effects.

\textbf{Step 2: A definite-signature form produces trivial causal structure.}
Consider a quadratic form $q$ on the generating vector space $V$.
If $q$ is positive-definite, then all generators $v \in V$ satisfy $v^2 = q(v)\mathbf{1} > 0$, and they are algebraically indistinguishable with respect to norm type (all ``spacelike'').
In the corresponding Clifford algebra $Cl(n,0)$, the even subalgebra $Cl^0(n,0) \cong Cl(n{-}1,0)$ generates the compact group $\mathrm{Spin}(n)$---no non-compact (boost) generators exist.
The resulting automorphism group is compact, so there is no invariant cone structure that could distinguish timelike from spacelike directions: the causal structure is trivial (all directions are equivalent).

\textbf{Step 3: Non-trivial causal structure requires indefinite signature.}
A non-trivial causal structure requires that some pairs of generators commute (spacelike) while others do not (timelike or lightlike).
This demands generators $e_\mu$ with $e_\mu^2 = +\mathbf{1}$ (spacelike type) and $e_\nu$ with $e_\nu^2 = -\mathbf{1}$ (timelike type), hence $\mathrm{sig}(q) = (p,q)$ with $p, q \geq 1$.
The even subalgebra then contains both compact generators (spatial rotations) and non-compact generators (boosts), enabling a causal cone structure.

We emphasize that this argument does not determine the specific values of $p$ and $q$.
The identification $(p,q) = (1,3)$ requires additional input: the observational dimensionality of macroscopic spacetime.
The present argument establishes only that indefiniteness is \emph{necessary} for non-trivial causal structure.
\end{proof}

\textbf{Exclusions}: All positive-definite metrics (including standard Bures and Fisher--Rao on quantum state spaces) fail Constraint~III.
This is the constraint that separates Clifford algebras (which carry a built-in quadratic form of arbitrary signature) from generic associative algebras with positive-definite metrics.

% ============================================================
\subsection{Exclusion of Alternative Algebras}
\label{sec:exclusion}

We now systematically evaluate each candidate from Section~\ref{sec:candidates}:

\begin{table}[h]
\centering
\small
\begin{tabular}{@{}lcccl@{}}
\toprule
\textbf{Algebra} & \textbf{I: Assoc.} & \textbf{II: Metric} & \textbf{III: Indef.} & \textbf{Status} \\
\midrule
von Neumann (Type III$_1$) & \checkmark & $\sim$ & $\times$ & No built-in signature \\
$C^*$-algebra (general) & \checkmark & $\sim$ & $\times$ & Metric not canonical \\
Lie algebra & \checkmark$^*$ & $\times$ & --- & No quadratic form \\
Jordan algebra & $\times$ & \checkmark & --- & Non-associative \\
Octonion algebra & $\times$ & \checkmark & --- & Non-associative \\
\textbf{Clifford} $Cl(V,q)$ & \checkmark & \checkmark & \checkmark & \textbf{All satisfied} \\
\bottomrule
\end{tabular}
\caption{Evaluation of candidate algebras against three constraints.
$\checkmark$: satisfied; $\times$: violated; $\sim$: partially satisfied (metric exists but is not built-in or canonical).
$^*$Lie algebras are not associative, but their universal enveloping algebras are.}
\label{tab:exclusion}
\end{table}

The key observation is that Clifford algebras are distinguished by having the quadratic form $q$ \emph{built into} the algebraic structure via the defining relation $v^2 = q(v)\mathbf{1}$.
Other associative algebras (von Neumann, $C^*$) can be \emph{equipped with} metrics, but do not carry a canonical one; the metric is an additional choice external to the algebra.
In the holographic context, where the boundary algebra must encode bulk metric information, this built-in feature becomes a substantive advantage rather than a mere convenience.

% ============================================================
\subsection{Worked Example: Qubit Tensor Network}
\label{sec:toymodel}

To illustrate how the three constraints operate concretely, consider a tensor network model of holographic bulk reconstruction.

\paragraph{Setup.}
Take $N$ qubits arranged on a MERA (multiscale entanglement renormalization ansatz) tensor network~\cite{Swingle2012,Vidal2008}.
The boundary algebra is generated by tensor products of Pauli operators $\{\sigma_x, \sigma_y, \sigma_z\}$ acting on individual qubits.

\paragraph{Constraint I.}
The Pauli algebra is associative (it consists of $2 \times 2$ matrices).
If we were to replace the Pauli operators with elements of an octonion algebra (which is non-associative), the isometry conditions defining the MERA network---specifically, $V^\dagger V = \mathbf{1}$, which requires associative composition---would fail.

\paragraph{Constraint II.}
The Pauli operators satisfy $\{\sigma_i, \sigma_j\} = 2\delta_{ij}\mathbf{1}$, which defines a \emph{positive-definite} quadratic form.
This is the Clifford algebra $Cl(3,0)$ (since $\sigma_i^2 = +\mathbf{1}$, the quadratic form is positive-definite).
Note that $Cl(3,0) \cong M_2(\mathbb{C})$ while $Cl(0,3) \cong \mathbb{H} \oplus \mathbb{H}$ as real algebras; these are \emph{not} isomorphic.
The quadratic form is built into the anticommutation relation.

\paragraph{Constraint III.}
The Pauli algebra alone encodes a Euclidean signature $(3,0)$.
To obtain a Lorentzian signature, we must introduce a distinguished direction corresponding to the modular Hamiltonian $K$, which generates modular flow (the boundary analog of time evolution in the bulk).
The extended algebra $\{i K, \sigma_x, \sigma_y, \sigma_z\}$ then satisfies anticommutation relations encoding signature $(1,3)$:
\begin{equation}
(iK)^2 = -\mathbf{1}, \qquad \sigma_i^2 = +\mathbf{1}, \qquad \{iK, \sigma_i\} = 0.
\end{equation}
This is precisely $Cl(1,3)$.

\paragraph{Lesson.}
The tensor network example illustrates how associativity (isometry conditions), built-in metric (Pauli anticommutation), and indefinite signature (modular flow direction) naturally combine to produce Clifford structure in a concrete holographic model.

% ============================================================
\subsection{The Algebraic Compatibility Theorem}

\begin{theorem}[Clifford Compatibility]
\label{thm:selection}
Among finitely-generated associative algebras over a vector space $V$ equipped with a non-degenerate quadratic form $q$, the Clifford algebra $Cl(V,q)$ is the universal (and hence minimal) such structure, by its universal property.
\end{theorem}

\begin{proof}
Constraint~I requires associativity; Constraint~II requires a non-degenerate quadratic form on the generating space; Constraint~III requires indefinite signature.
The universal property of Clifford algebras~\cite{Hestenes1966,Doran2003} states that $Cl(V,q)$ is the unique (up to isomorphism) associative algebra generated by $V$ subject to $v^2 = q(v)\mathbf{1}$.
Any other associative algebra satisfying these constraints contains $Cl(V,q)$ as a subalgebra (or quotient), making Clifford the minimal compatible structure.
\end{proof}

\begin{remark}[Scope of the Claim]
\label{rem:scope}
Theorem~\ref{thm:selection} is a statement about \emph{algebraic compatibility}, not physical uniqueness.
It asserts that Clifford algebra is the natural minimal framework for encoding the three constraints simultaneously.
It does not exclude larger structures, nor does it claim that physics \emph{must} use the minimal option.
The theorem should be understood as identifying an algebraic bottleneck rather than deriving a unique physical theory.

We note that the final step of the proof---invoking the universal property of $Cl(V,q)$---is a standard algebraic fact (essentially the definition of Clifford algebras).
The substantive content of the argument resides in the three \emph{constraints}:
Lemma~\ref{lem:assoc} (associativity as a boundary condition, substantiated by the pathologies of non-associative alternatives),
Lemma~\ref{lem:metric} (non-degenerate bilinear form from holographic error correction, connected to the modular Hamiltonian and Fisher information),
and Lemma~\ref{lem:signature} (indefinite signature from the algebraic structure of causal order).
The theorem assembles these physical requirements into a single algebraic conclusion.
\end{remark}

% ============================================================
\section{Entropic Gravity from Algebraic Structure}
\label{sec:entropic}

\subsection{Holographic Screen and Einstein Equations}

Following Jacobson~\cite{Jacobson1995}, the entropic force $F = T \nabla S$ and the holographic entropy bound $S \leq A / 4G$ reproduce the Einstein field equations in the thermodynamic limit.
This derivation assumes local Lorentz invariance---a condition naturally satisfied when the boundary algebra is Clifford-compatible, since $Cl(1,3)$ contains $\mathrm{Spin}(1,3)$ (the double cover of the Lorentz group) as its even subalgebra.

\subsection{Relation to HAFF Emergence Chain}

Within HAFF, geometry emerges via:
\[
\text{Observable Algebra} \to \text{Representation} \to \text{Entanglement} \to \text{Connectivity} \to \text{Geometry}
\]
The present work adds a constraint on the final arrow: among geometrically admissible coarse-grainings~\cite{Liu2026PaperA}, those producing Lorentzian geometry must induce effective algebras compatible with $Cl(1,3)$.

\subsection{Relation to Algebraic QFT}
\label{sec:aqft}

In algebraic quantum field theory (AQFT)~\cite{Haag1996}, local observable algebras associated with spacetime regions are generically Type~III$_1$ von Neumann factors.
These algebras are associative and support rich mathematical structure, but they do not carry a canonical metric of indefinite signature.

The connection to Lorentzian structure emerges through the Tomita--Takesaki theorem: for any cyclic and separating state, the modular operator $\Delta$ generates a one-parameter group (modular flow) that, in the Bisognano--Wichmann theorem, coincides with the boost generator in Rindler spacetime.
This modular flow singles out a \emph{time-like direction} within the algebraic structure.

In the HAFF framework, the accessible algebra $\mathcal{A}_{\mathbf{c}}$ can be understood as a stable subalgebra of a Type~III$_1$ factor.
The present analysis suggests that when such a subalgebra supports a Lorentzian bulk description, it must admit a $Cl(1,3)$ representation---where the modular flow direction provides the time-like generator and spatial locality provides the space-like generators.

This perspective connects the present work to Witten's observation~\cite{Witten2018} that Type~III$_1$ algebras are essential in gravitational settings, and to Connes' noncommutative geometry program~\cite{Connes1994}, where Clifford algebras play a central role in the spectral characterization of Riemannian (and pseudo-Riemannian) manifolds.

% ============================================================
\section{Discussion}
\label{sec:discussion}

\subsection{What This Result Does and Does Not Show}

\textbf{Does show:}
Clifford algebra is the minimal algebraic structure simultaneously satisfying associativity, metric compatibility, and causal channel encoding.
The exclusion argument (Table~\ref{tab:exclusion}) demonstrates that alternative algebras fail at least one constraint.
The tensor network example (Section~\ref{sec:toymodel}) illustrates the constraints in a concrete model.

\textbf{Does not show:}
Why $3+1$ dimensions rather than some other $(p,q)$---the argument constrains to $Cl(p,q)$ for any $p \geq 1$; the value $(1,3)$ requires additional input.
That gravity \emph{is} entropic---we derive consistency conditions within the entropic gravity framework.
That $Cl(1,3)$ is the \emph{unique} boundary algebra---larger algebras containing $Cl(1,3)$ as a subalgebra are also compatible.
A complete theory of quantum gravity.

\subsection{Convergence with Paper B}

The companion paper~\cite{Liu2026QRAIF_B} arrives at $Cl(V,q)$ from a completely different direction: thermodynamic stability of persistent open quantum subsystems.

\begin{center}
\begin{tabular}{@{}lll@{}}
\toprule
& \textbf{Paper A (this work)} & \textbf{Paper B} \\
\midrule
Question & What algebra does geometry need? & What algebra does persistence need? \\
Method & Holographic consistency & Lyapunov stability \\
Perspective & The world (``ocean'') & The subsystem (``fish'') \\
Result & $Cl(1,3)$ from signature & $Cl(V,q)$ from error stability \\
\bottomrule
\end{tabular}
\end{center}

We note that this convergence is \emph{heuristic rather than deductive}: it suggests that Clifford algebra occupies a distinguished position in the landscape of emergent algebraic structures, but does not constitute a proof.
The convergence motivates further investigation, particularly through more elaborate models and deeper connections to established algebraic frameworks.

\subsection{Open Problems}

\begin{enumerate}
\item \textbf{Dimensionality}: What additional constraints (anomaly cancellation, stability of persistent subsystems, observational input) fix $(p,q) = (1,3)$?
\item \textbf{Constructive derivation}: Can Clifford generators be explicitly constructed from modular Hamiltonians or Tomita--Takesaki data in holographic models?
\item \textbf{Relation to noncommutative geometry}: How does the present analysis connect to Connes' spectral triples, where Clifford algebras characterize the Dirac operator?
\item \textbf{Tensor network realization}: Can the qubit toy model of Section~\ref{sec:toymodel} be made rigorous in the context of holographic error-correcting codes?
\end{enumerate}

% ============================================================
\section{Conclusion}

We have argued that the Lorentzian metric structure in entropic gravity frameworks is algebraically constrained by three independent requirements: associativity, metric compatibility (with holographic encoding), and indefinite signature.
A systematic exclusion of alternative algebras (von Neumann, $C^*$, Jordan, Lie, octonion) shows that Clifford algebra $Cl(V,q)$ is the minimal structure satisfying all three simultaneously.

This result connects the top-down perspective of HAFF (geometry emerges from observable algebras) with bottom-up algebraic constraints (any causal geometry must be Clifford-compatible).
It does not constitute a derivation of Lorentzian gravity from first principles, but identifies an algebraic bottleneck through which any emergent causal geometry must pass.

% ============================================================
\begin{thebibliography}{99}

\bibitem{Maldacena1999}
J.~M.~Maldacena, \emph{The Large N limit of superconformal field theories and supergravity}, Int.\ J.\ Theor.\ Phys.\ \textbf{38}, 1113 (1999).

\bibitem{RyuTakayanagi2006}
S.~Ryu and T.~Takayanagi, \emph{Holographic Derivation of Entanglement Entropy from AdS/CFT}, Phys.\ Rev.\ Lett.\ \textbf{96}, 181602 (2006).

\bibitem{Verlinde2011}
E.~Verlinde, \emph{On the Origin of Gravity and the Laws of Newton}, JHEP \textbf{04}, 029 (2011).

\bibitem{Jacobson1995}
T.~Jacobson, \emph{Thermodynamics of Spacetime: The Einstein Equation of State}, Phys.\ Rev.\ Lett.\ \textbf{75}, 1260 (1995).

\bibitem{Hestenes1966}
D.~Hestenes, \emph{Space-Time Algebra}, Gordon and Breach (1966).

\bibitem{Doran2003}
C.~Doran and A.~Lasenby, \emph{Geometric Algebra for Physicists}, Cambridge University Press (2003).

\bibitem{Haag1996}
R.~Haag, \emph{Local Quantum Physics: Fields, Particles, Algebras}, Springer-Verlag (1996).

\bibitem{Connes1994}
A.~Connes, \emph{Noncommutative Geometry}, Academic Press (1994).

\bibitem{Faulkner2014}
T.~Faulkner, M.~Guica, T.~Hartman, R.~C.~Myers, and M.~Van~Raamsdonk, \emph{Gravitation from Entanglement in Holographic CFTs}, JHEP \textbf{03}, 051 (2014).

\bibitem{Lashkari2014}
N.~Lashkari, M.~B.~McDermott, and M.~Van~Raamsdonk, \emph{Gravitational dynamics from entanglement ``thermodynamics''}, JHEP \textbf{04}, 195 (2014).

\bibitem{Witten2018}
E.~Witten, \emph{APS Medal for Exceptional Achievement in Research: Invited article on entanglement properties of quantum field theory}, Rev.\ Mod.\ Phys.\ \textbf{90}, 045003 (2018).

\bibitem{Petz1996}
D.~Petz, \emph{Monotone metrics on matrix spaces}, Linear Algebra Appl.\ \textbf{244}, 81 (1996).

\bibitem{Schafer1966}
R.~D.~Schafer, \emph{An Introduction to Nonassociative Algebras}, Academic Press (1966).

\bibitem{Gunaydin1973}
M.~G\"unaydin and F.~G\"ursey, \emph{Quark structure and octonions}, J.\ Math.\ Phys.\ \textbf{14}, 1651 (1973).

\bibitem{AlfsenShultz2003}
E.~M.~Alfsen and F.~W.~Shultz, \emph{Geometry of State Spaces of Operator Algebras}, Birkh\"auser (2003).

\bibitem{Swingle2012}
B.~Swingle, \emph{Entanglement Renormalization and Holography}, Phys.\ Rev.\ D \textbf{86}, 065007 (2012).

\bibitem{Vidal2008}
G.~Vidal, \emph{Class of quantum many-body states that can be efficiently simulated}, Phys.\ Rev.\ Lett.\ \textbf{101}, 110501 (2008).

\bibitem{Liu2026PaperA}
S.~Liu, \emph{Emergent Geometry from Coarse-Grained Observable Algebras: The Holographic Alaya-Field Framework}, Zenodo (2026), DOI: 10.5281/zenodo.18361706.

\bibitem{Liu2026PaperB}
S.~Liu, \emph{Accessibility, Stability, and Emergent Geometry: Conceptual Clarifications on the Holographic Alaya-Field Framework}, Zenodo (2026), DOI: 10.5281/zenodo.18367060.

\bibitem{Liu2026QRAIF_B}
S.~Liu, \emph{Thermodynamic Stability Constraints on the Operator Algebra of Persistent Open Quantum Subsystems}, Zenodo (2026), DOI: 10.5281/zenodo.18525890.

\end{thebibliography}

\end{document}
