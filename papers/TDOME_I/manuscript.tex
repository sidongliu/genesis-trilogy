% ============================================================
% T-DOME Paper I: The Seed
% T-DOME three-paper series, Paper I
% ============================================================

\documentclass[12pt]{article}
\usepackage[margin=1in]{geometry}
\usepackage{amsmath,amssymb,amsthm}
\usepackage[hidelinks]{hyperref}
\usepackage{booktabs}
\usepackage{physics}
\usepackage{graphicx}
\usepackage{caption}
\usepackage{array}
\newcolumntype{P}[1]{>{\raggedright\arraybackslash}p{#1}}

% Theorem environments — matching HAFF/Q-RAIF style
\newtheorem{theorem}{Theorem}
\newtheorem{lemma}[theorem]{Lemma}
\newtheorem{proposition}[theorem]{Proposition}
\newtheorem{definition}[theorem]{Definition}
\newtheorem{remark}[theorem]{Remark}
\newtheorem{corollary}[theorem]{Corollary}

\title{Non-Markovian Memory and the Thermodynamic\\
Necessity of Temporal Accumulation}

\author{Sidong Liu, PhD \\
\small iBioStratix Ltd \\
\small \texttt{sidongliu@hotmail.com}}

\date{February 2026}

\begin{document}
\emergencystretch=2em
\raggedbottom
\hbadness=10000
\vbadness=10000
\maketitle

% ============================================================
\begin{abstract}
We investigate the thermodynamic constraints on open quantum systems
that must persist far from equilibrium in stochastic environments.
Working within the framework of stochastic thermodynamics and
information thermodynamics (Sagawa--Ueda), we define a
\emph{survival functional} $\mathcal{S} := \Delta F - W$
measuring the difference between the non-equilibrium free energy
gained and the work invested by an agent.

We prove a \textbf{Markovian Ceiling}: for any open-loop
Markovian (GKSL) dynamics with no measurement or feedback,
$\mathcal{S} \leq 0$---the agent cannot thermodynamically
``profit.''
We then derive an exact identity---valid for
\emph{arbitrary} (possibly correlated) initial states under
autonomous evolution in the weak-coupling limit---expressing
the survival functional in terms of the change in
system--environment mutual information and bath displacement:
$\beta\,\mathcal{S} = -\Delta I(S{:}E)
- \Delta D_{\mathrm{KL}}(\rho_E \| \rho_E^{\mathrm{th}})$.
Pre-existing correlations $I(S{:}E;\, 0) > 0$, built during
prior interaction epochs, serve as a consumable thermodynamic
resource; their consumption during non-Markovian backflow
intervals yields $\mathcal{S} > 0$, bounded by the initial
correlation budget.

This establishes \textbf{memory as a thermodynamic necessity}
for sustained far-from-equilibrium persistence.
The memory kernel induces a causal partial order on system
trajectories that, when restricted to the classical sector
selected by decoherence (quantum Darwinism), is consistent
with the accessibility ordering of the Holographic Alaya-Field
Framework (HAFF).
A worked example---a spin-boson model with Lorentz--Drude
spectral density---illustrates how non-Markovian backflow
enables free-energy extraction unavailable to memoryless systems.

Finally, using the entropy rate and predictive information from
computational mechanics, we quantify the intrinsic cost of memory
and identify the \textbf{Memory Catastrophe}: unbounded memory
under finite energy leads to thermodynamic collapse, motivating
the symmetry-breaking mechanism of Paper~II\@.

\medskip
\noindent\textbf{Keywords}: non-Markovian dynamics, open quantum systems,
Nakajima--Zwanzig equation, memory kernel, thermodynamic arrow of time,
information backflow, entropy production, stochastic thermodynamics
\end{abstract}

% ============================================================
\section{Introduction}
\label{sec:intro}

% ------------------------------------------------------------
\subsection{Context: The Problem of Persistence}

A quantum system coupled to a thermal environment generically
relaxes toward equilibrium. This is the content of the
\emph{zeroth crisis}: absent special structure, every open
subsystem is eventually erased by thermal noise~\cite{BreuerPetruccione2002}.

Yet the physical world contains persistent far-from-equilibrium
structures---from molecular machines to living organisms---that
maintain themselves against the entropic tide for timescales
vastly exceeding their intrinsic relaxation times. What
structural feature of their dynamics makes this possible?

The standard answer invokes free-energy input: a persistent system
is one that continuously imports low-entropy energy and exports
high-entropy waste~\cite{Schrodinger1944}. This is correct but
incomplete. Two systems receiving \emph{identical} free-energy
flux from \emph{identical} environments may exhibit vastly
different persistence characteristics. The distinguishing factor,
we argue, is \emph{memory}---the capacity to condition present
dynamics on past environmental states.

% ------------------------------------------------------------
\subsection{Position within the Series}

This paper is the first of three constituting the
\textbf{T-DOME} (Thermodynamic Dynamics of Observer-Memory
Entanglement) framework, the third pillar of a three-paper
program.

\begin{center}
\small
\setlength{\tabcolsep}{4pt}%
\begin{tabular}{@{}lP{3.2cm}cP{3.4cm}c@{}}
\toprule
\textbf{Framework} & \textbf{Question} & & \textbf{Result} & \textbf{Status} \\
\midrule
HAFF~\cite{Liu2026HAFF_A,Liu2026HAFF_B}
  & How does geometry emerge?
  & Ocean
  & Algebra $\to$ Geometry
  & Complete \\[3pt]
Q-RAIF~\cite{Liu2026QRAIF_A,Liu2026QRAIF_B}
  & What algebra must an observer have?
  & Fish
  & $Cl(V,q) \hookrightarrow Cl(1,3)$
  & Complete \\[3pt]
\textbf{T-DOME I} (this work)
  & Why must agents carry memory?
  & Seed
  & Markovian ceiling; memory as necessity
  & \textbf{This paper} \\[3pt]
T-DOME II
  & Why must agents break symmetry?
  & Ego
  & Reference-frame selection
  & Planned \\[3pt]
T-DOME III
  & How does self-calibration arise?
  & Loop
  & Fisher self-referential bound
  & Planned \\
\bottomrule
\end{tabular}
\end{center}

The three T-DOME papers form an irreversible logical chain.
Each resolves a survival crisis created by its predecessor:
\begin{enumerate}
\item \textbf{Paper I (The Seed):} Without memory, a system
  is trapped in the \emph{Markovian present}---no accumulation,
  no temporal arrow, inevitable thermal death.
  Memory breaks this trap but floods the system with unbounded
  historical data.
\item \textbf{Paper II (The Ego):} Unbounded memory under finite
  computational resources causes processing collapse.
  Spontaneous symmetry breaking of the reference frame
  (establishing a ``self'') resolves the overload but introduces
  systematic bias.
\item \textbf{Paper III (The Loop):} Uncorrected bias diverges from a
  changing environment. A self-referential calibration loop
  (monitoring one's own prediction error) resolves the bias
  but requires the system to ``observe its own observation''---closing
  the self-calibration loop.
\end{enumerate}

\noindent
The present paper addresses only the first link in this chain.

% ------------------------------------------------------------
\subsection{Relation to HAFF Paper F}

HAFF Paper F~\cite{Liu2026HAFF_F} establishes the arrow of time
as the direction of \emph{accessibility propagation}:
informational redundancy $\mathcal{R}(\hat{O})$ generically expands,
inducing a partial order $\prec$ on observable algebras.
That analysis is purely algebraic---it characterizes temporal
asymmetry without invoking dynamics.

The present paper complements Paper F by identifying the
\emph{dynamical} origin of temporal asymmetry: the non-Markovian
memory kernel $\mathcal{K}(t,s)$. We show (Section~\ref{sec:arrow})
that the partial order induced by the kernel's temporal support
embeds into the HAFF accessibility ordering as a sub-structure.
The two descriptions are dual faces of the same phenomenon:
Paper F provides the algebraic skeleton; Paper I provides the
dynamical muscle.

% ------------------------------------------------------------
\subsection{Scope and Disclaimers}

\begin{enumerate}
\item This work does \emph{not} claim that non-Markovian dynamics
  is sufficient for persistence. Memory is identified as
  \emph{necessary} under the conditions specified; sufficiency
  requires additional structure (Papers II and III).
\item We do \emph{not} claim that all non-Markovian systems
  outperform all Markovian systems. The theorem establishes that
  the supremum of survival efficiency over non-Markovian dynamics
  strictly exceeds the Markovian supremum.
\item We do \emph{not} derive the specific form of the memory
  kernel from first principles. The kernel is treated as a
  structural feature of the system-environment coupling.
\item The term ``agent'' is used in the control-theoretic sense
  (a subsystem that acts on its environment to maintain a target
  state) and carries no implication of consciousness, intention,
  or subjective experience.
\item A broader structural analogy with classical philosophical
  concepts of temporal persistence exists but is outside the
  scope of this paper.
\end{enumerate}

% ============================================================
\section{Mathematical Preliminaries}
\label{sec:prelim}

% ------------------------------------------------------------
\subsection{Open Quantum Systems: The Markovian Baseline}

Consider a bipartite Hilbert space
$\mathcal{H} = \mathcal{H}_R \otimes \mathcal{H}_E$,
where $R$ denotes the ``agent'' (reduced system) and $E$ the
environment. The total Hamiltonian is
\begin{equation}
\label{eq:H_total}
H = H_R \otimes \mathbb{1}_E
  + \mathbb{1}_R \otimes H_E
  + \lambda\, H_{\mathrm{int}},
\end{equation}
where $\lambda$ parametrizes the coupling strength.

Under the Born--Markov and secular approximations, the reduced
dynamics of $\rho_R(t) = \mathrm{Tr}_E[\rho(t)]$ is governed by
the Gorini--Kossakowski--Sudarshan--Lindblad (GKSL) master
equation~\cite{Lindblad1976,GKS1976}:
\begin{equation}
\label{eq:GKSL}
\dot{\rho}_R(t)
= -i[H_{\mathrm{eff}},\, \rho_R(t)]
  + \sum_k \gamma_k \left(
    L_k \rho_R(t) L_k^\dagger
    - \tfrac{1}{2}\{L_k^\dagger L_k,\, \rho_R(t)\}
  \right),
\end{equation}
with $\gamma_k \geq 0$ and Lindblad operators $\{L_k\}$.

\begin{remark}[Markovian = Memoryless]
\label{rem:markov}
The GKSL equation is \emph{time-local}: $\dot{\rho}_R(t)$ depends
only on $\rho_R(t)$, never on $\rho_R(s)$ for $s < t$.
Physically, this corresponds to an environment with vanishing
correlation time ($\tau_E \to 0$): the bath ``forgets'' its
interaction with the system instantaneously.
The semigroup property $\Lambda(t+s) = \Lambda(t)\Lambda(s)$
ensures complete positivity at all times but precludes any
information backflow from environment to system~\cite{RivasHuelgaPlenio2014}.
\end{remark}

% ------------------------------------------------------------
\subsection{Beyond Markov: The Nakajima--Zwanzig Equation}

When the environmental correlation time $\tau_E$ is non-negligible,
the Born--Markov approximation fails. The exact reduced dynamics
is captured by the Nakajima--Zwanzig (NZ) integro-differential
equation~\cite{Nakajima1958,Zwanzig1960}:
\begin{equation}
\label{eq:NZ}
\dot{\rho}_R(t)
= -i[H_{\mathrm{eff}},\, \rho_R(t)]
  + \int_0^t ds\; \mathcal{K}(t,s)\, \rho_R(s),
\end{equation}
where $\mathcal{K}(t,s)$ is the \textbf{memory kernel}---a
superoperator encoding the influence of the system's entire
history on its present dynamics.

\begin{definition}[Memory Kernel]
\label{def:kernel}
The memory kernel $\mathcal{K}: [0,\infty)^2 \to
\mathcal{L}(\mathcal{B}(\mathcal{H}_R))$ is the superoperator
satisfying~\eqref{eq:NZ}. It encodes two types of information:
\begin{enumerate}
\item \textbf{Environmental structure:} the spectral density,
  correlation functions, and non-equilibrium features of the bath;
\item \textbf{Temporal reach:} the effective support
  $\tau_{\mathrm{mem}} := \inf\{\tau : \|\mathcal{K}(t,s)\| < \epsilon
  \;\forall\; t - s > \tau\}$, the ``memory depth.''
\end{enumerate}
The Markovian limit corresponds to
$\mathcal{K}(t,s) \to \mathcal{K}_0\, \delta(t-s)$,
recovering the GKSL generator.
\end{definition}

\begin{remark}[Information Backflow]
\label{rem:backflow}
Non-Markovian dynamics admits \emph{information backflow}:
the distinguishability of two initial states, as measured by
trace distance $D(\rho_1(t), \rho_2(t))$, can temporarily
increase~\cite{BreuerLainePiilo2009}. This is the operational
signature of memory---the environment returns previously
absorbed information to the system.
\end{remark}

% ------------------------------------------------------------
\subsection{Thermodynamic Framework}
\label{subsec:thermo}

We adopt the framework of stochastic thermodynamics for open
quantum systems~\cite{EspositoLindenbergVandenBroeck2010}.
The following conventions are fixed throughout.

\begin{definition}[Thermodynamic Setup]
\label{def:thermo_setup}
\leavevmode
\begin{enumerate}
\item \textbf{Hamiltonian decomposition.}
  The system Hamiltonian is
  $H_S(t) = H_R + H_{\mathrm{ctrl}}(t)$,
  where $H_R$ is the \emph{fixed} bare Hamiltonian and
  $H_{\mathrm{ctrl}}(t)$ is the agent's time-dependent
  control protocol.
  The bath Hamiltonian $H_E$ and coupling $H_{\mathrm{int}}$
  are as in~\eqref{eq:H_total}.

\item \textbf{Reference state.}
  The thermal equilibrium state of the bare Hamiltonian is
  \begin{equation}
  \label{eq:rho_eq}
  \rho_{\mathrm{eq}} := \frac{e^{-\beta H_R}}{Z_R},
  \qquad Z_R := \tr(e^{-\beta H_R}),
  \qquad \beta := (k_B T)^{-1}.
  \end{equation}
  Since $H_R$ is time-independent, $\rho_{\mathrm{eq}}$ is a
  well-defined, fixed reference throughout the protocol.

\item \textbf{Non-equilibrium free energy.}
  For any state $\rho$ of the reduced system,
  \begin{equation}
  \label{eq:F_neq}
  F(\rho) := \tr(\rho\, H_R) + \beta^{-1}\tr(\rho\ln\rho)
  = \langle H_R \rangle_\rho - \beta^{-1} S(\rho),
  \end{equation}
  where $S(\rho) = -\tr(\rho\ln\rho)$ is the von~Neumann entropy.
  The equilibrium value is $F_{\mathrm{eq}} = -\beta^{-1}\ln Z_R$.

\item \textbf{Free energy--relative entropy identity.}
  \begin{equation}
  \label{eq:DKL_F}
  D_{\mathrm{KL}}(\rho \| \rho_{\mathrm{eq}})
  = \beta\bigl(F(\rho) - F_{\mathrm{eq}}\bigr) \geq 0.
  \end{equation}
  Thus $D_{\mathrm{KL}}$ measures the free-energy surplus in
  units of $k_B T$.

\item \textbf{Work.}
  The work performed on the system by the control protocol
  over $[0,\tau]$ is
  \begin{equation}
  \label{eq:work}
  W[0,\tau] := \int_0^\tau
  \tr\!\left(\rho(t)\,
  \frac{\partial H_{\mathrm{ctrl}}}{\partial t}\right) dt.
  \end{equation}

\item \textbf{Entropy-production functional.}
  The \emph{generalised entropy production} over $[0,\tau]$ is
  \begin{equation}
  \label{eq:Sigma}
  \Sigma[0,\tau] := \beta\bigl(W[0,\tau] - \Delta F\bigr),
  \end{equation}
  where $\Delta F = F(\rho(\tau)) - F(\rho(0))$.
  For uncorrelated (product) initial states,
  $\Sigma \geq 0$ recovers the standard second-law bound.
  For initially correlated states, $\Sigma$ can be
  \emph{negative}, reflecting the consumption of
  pre-existing correlations
  (see Remark~\ref{rem:battery}).
\end{enumerate}
\end{definition}

\begin{remark}[Why $H_R$ is fixed]
\label{rem:H_fixed}
The bare Hamiltonian $H_R$ defines the system's energy scale and
hence the reference state $\rho_{\mathrm{eq}}$.
The agent acts on the world through $H_{\mathrm{ctrl}}(t)$,
which may be time-dependent.
This separation ensures that $\rho_{\mathrm{eq}}$ is
well-defined and time-independent, avoiding the ambiguity
that arises when the full $H_S(t)$ is used to define the
thermal reference.
\end{remark}

\begin{definition}[Standing Assumptions]
\label{def:assumptions}
The following minimal assumptions are in force throughout
Sections~\ref{sec:advantage}--\ref{sec:example} unless
stated otherwise.
Every main result
(Lemma~\ref{lem:info_thermo},
Theorem~\ref{thm:advantage},
Corollary~\ref{cor:three_regimes})
relies \emph{only} on items
\textup{(A1)--(A5)} below.
\begin{enumerate}
\item[\textup{(A1)}]
  \textbf{Finite-dimensional bipartite system.}
  $\mathcal{H} = \mathcal{H}_S \otimes \mathcal{H}_E$,
  with total Hamiltonian~\eqref{eq:H_total}
  and global unitary evolution
  $U(t) = \mathcal{T}\exp(-i\int_0^t H(s)\,ds)$.

\item[\textup{(A2)}]
  \textbf{Weak coupling.}
  The system--environment interaction satisfies
  $\lambda \ll 1$ in~\eqref{eq:H_total}, so that
  $\Delta\langle H_{\mathrm{int}} \rangle
  = O(\lambda)$~\cite{BreuerPetruccione2002}.
  Energy conservation is then
  $\Delta\langle H_R\rangle
  + \Delta\langle H_{\mathrm{ctrl}}\rangle
  + \Delta\langle H_E\rangle \approx 0$
  up to controlled $O(\lambda)$ corrections.

\item[\textup{(A3)}]
  \textbf{Fixed environmental reference.}
  $\rho_E^{\mathrm{th}} := e^{-\beta H_E}/Z_E$
  is a fixed \emph{bookkeeping} Gibbs state at inverse
  temperature $\beta$.
  The \emph{actual} initial bath state $\rho_E(0)$
  need not coincide with $\rho_E^{\mathrm{th}}$;
  when $\rho_E(0) \neq \rho_E^{\mathrm{th}}$, the quantity
  $D_{\mathrm{KL}}(\rho_E(t)\|\rho_E^{\mathrm{th}})$
  tracks the nonequilibrium free energy stored in the bath
  relative to this reference.
  The bath Hamiltonian $H_E$ is time-independent.

\item[\textup{(A4)}]
  \textbf{Arbitrary initial state.}
  The total initial state $\rho_{SE}(0)$ is \emph{not}
  required to be a product state.
  In particular, initial system--environment correlations
  $I(S{:}E;\,0) > 0$ and initial bath displacement
  $D_{\mathrm{KL}}(\rho_E(0)\|\rho_E^{\mathrm{th}}) > 0$
  are both permitted.
\item[\textup{(A5)}]
  \textbf{Regularity.}
  All quantum states appearing in the thermodynamic
  identities are assumed to have full rank (or are
  restricted to their support), so that all relative
  entropies $D_{\mathrm{KL}}(\rho\|\sigma)$ are finite.
\end{enumerate}
\end{definition}

\begin{remark}[Bookkeeping conventions]
\label{rem:bookkeeping}
The heat absorbed by the environment is
$Q := \Delta\langle H_E \rangle
= \mathrm{Tr}[\rho_E(\tau) H_E]
  - \mathrm{Tr}[\rho_E(0) H_E]$
(matching
Esposito \emph{et al.}~\cite{EspositoLindenbergVandenBroeck2010}).
We define $\Sigma := \beta(W - \Delta F)$ as a
\emph{generalised entropy-balance functional};
for correlated initial conditions $\Sigma$ need not be
nonnegative
(see Remark~\ref{rem:battery}).
\end{remark}

% ------------------------------------------------------------
\subsection{The Survival Functional}
\label{subsec:survival}

We now define the central quantity of this paper.

\begin{definition}[Survival Functional]
\label{def:survival}
For a reduced system $R$ evolving under dynamics $\Lambda$
over $[0,\tau]$, the \textbf{survival functional} is
\begin{equation}
\label{eq:survival}
\mathcal{S}[\Lambda, \tau]
:= \Delta F - W[0,\tau]
= \bigl[F(\rho(\tau)) - F(\rho(0))\bigr] - W[0,\tau].
\end{equation}
Equivalently, using~\eqref{eq:Sigma},
\begin{equation}
\label{eq:survival_Sigma}
\beta\,\mathcal{S}[\Lambda,\tau] = -\Sigma[0,\tau].
\end{equation}
\end{definition}

\noindent\textit{Note on nomenclature.}
We retain the term ``survival functional'' to emphasize
the biological interpretation of persistence far from
equilibrium; mathematically, $\mathcal{S}$ is strictly a
\emph{generalized entropy-balance functional} derived from
the first and second laws.

\begin{remark}[Interpretation]
\label{rem:survival_interp}
The survival functional has a transparent physical meaning:
\begin{itemize}
\item $\mathcal{S} > 0$: the system gained more free energy than
  was invested by the external protocol---a \emph{thermodynamic
  profit}. The agent has extracted usable work from environmental
  correlations.
\item $\mathcal{S} = 0$: the agent breaks even (reversible limit,
  $\Sigma = 0$).
\item $\mathcal{S} < 0$: the agent paid more than it gained
  (the generic irreversible case).
\end{itemize}
Under the standard second law ($\Sigma \geq 0$),
$\mathcal{S} \leq 0$ always.
As we show in Sections~\ref{sec:ceiling} and~\ref{sec:advantage},
achieving $\mathcal{S} > 0$ requires \emph{information}---and
the memory kernel provides exactly this.
\end{remark}

\begin{remark}[Connection to Information Thermodynamics]
\label{rem:connection_SU}
In the Sagawa--Ueda framework~\cite{SagawaUeda2010,SagawaUeda2012},
a system under feedback control satisfies the generalized second law
\begin{equation}
\label{eq:sagawa_ueda}
\Sigma \geq -I_{\mathrm{feedback}},
\end{equation}
where $I_{\mathrm{feedback}} \geq 0$ is the mutual information
gained through measurement of the system.
This permits $\Sigma < 0$ (and hence $\mathcal{S} > 0$) at
the expense of information.
The core thesis of this paper is that a non-Markovian memory
kernel provides \emph{implicit} feedback: the system's history
encodes correlations with the environment that play the same
thermodynamic role as explicit measurement outcomes.
\end{remark}

% ============================================================
\section{The Markovian Ceiling}
\label{sec:ceiling}

We now establish the fundamental thermodynamic limitation
of memoryless agents.
The result is elementary given the framework of
Section~\ref{subsec:thermo}, but its consequences are far-reaching:
under \emph{open-loop} control---where the agent's protocol
$H_{\mathrm{ctrl}}(t)$ is fixed in advance and receives no
information from the bath---the survival functional can never
be positive.

\subsection{Spohn's Inequality}

Throughout this section we assume that the GKSL generator
$\mathcal{L}$ is a \emph{thermal Lindbladian}: it is obtained
from the weak-coupling (Davies) limit of a system coupled to a
single thermal bath at inverse temperature $\beta$, and satisfies
\textbf{quantum detailed balance} (the KMS
condition)~\cite{Spohn1978,BreuerPetruccione2002}.
Under this assumption, the unique stationary state is the Gibbs
state $\rho_{\mathrm{ss}} = \rho_{\mathrm{eq}}$
of~\eqref{eq:rho_eq}, and the generator is self-adjoint with
respect to the KMS inner product.
This ensures that the entropy production rate below is
well-defined and non-negative.

\begin{definition}[Markovian Semigroup]
\label{def:markov_semigroup}
Throughout this paper, ``Markovian'' dynamics refers
strictly to a \textbf{dynamical semigroup} generated by a
time-independent GKSL generator $\mathcal{L}$ with
non-negative rates.
While time-dependent CP-divisible
maps~\cite{RivasHuelgaPlenio2014} are often called
Markovian in broader contexts, the ceiling theorem
(Theorem~\ref{thm:ceiling}) targets the semigroup case
$\Lambda(t) = e^{\mathcal{L}t}$, where no memory effects
or temporal correlations can be exploited.
\end{definition}

\begin{lemma}[Spohn~\cite{Spohn1978}]
\label{lem:spohn}
For any GKSL dynamical semigroup $\Lambda_t = e^{\mathcal{L}t}$
satisfying quantum detailed balance with unique invariant state
$\rho_{\mathrm{eq}}$, the entropy production
rate
\begin{equation}
\label{eq:spohn_sigma}
\sigma(t) := -\tr\!\bigl(\mathcal{L}[\rho(t)]\,
(\ln\rho(t) - \ln\rho_{\mathrm{eq}})\bigr)
\end{equation}
satisfies $\sigma(t) \geq 0$, with equality if and only if
$\rho(t) = \rho_{\mathrm{eq}}$.
\end{lemma}

\begin{proof}
This follows from the contractivity of CPTP maps under quantum
relative entropy~\cite{Spohn1978,BreuerPetruccione2002}:
$D_{\mathrm{KL}}(\Lambda_t\rho \| \Lambda_t\rho_{\mathrm{eq}})
\leq D_{\mathrm{KL}}(\rho \| \rho_{\mathrm{eq}})$
for all $t \geq 0$.
Differentiating at $t = 0$ yields $\sigma(t) \geq 0$.
\end{proof}

\subsection{The Markovian Ceiling Theorem}

\begin{definition}[Open-loop Markovian control class
$\mathcal{C}_{\mathrm{M}}$]
\label{def:control_class}
A protocol $H_{\mathrm{ctrl}}(t)$ belongs to the
\textbf{open-loop Markovian control class}
$\mathcal{C}_{\mathrm{M}}$ if and only if:
\begin{enumerate}
\item[\textup{(C1)}] $H_{\mathrm{ctrl}}(t)$ is a
  \emph{predetermined} function of $t$ alone, fixed before the
  protocol begins.
\item[\textup{(C2)}] No measurement of the system or
  environment is performed during $[0,\tau]$, and
  $H_{\mathrm{ctrl}}(t)$ receives no feedback from measurement
  outcomes.
\item[\textup{(C3)}] $H_{\mathrm{ctrl}}(t)$ is statistically
  independent of the bath realization
  $\{\xi_E(s) : s \in [0,\tau]\}$.
\end{enumerate}
Protocols involving adaptive measurement-based feedback
(Sagawa--Ueda~\cite{SagawaUeda2010}) are \emph{excluded} from
$\mathcal{C}_{\mathrm{M}}$.
\end{definition}

\begin{theorem}[Markovian Ceiling]
\label{thm:ceiling}
Let $\Lambda^{\mathrm{M}}$ denote GKSL dynamics~\eqref{eq:GKSL}
satisfying quantum detailed balance
(Lemma~\ref{lem:spohn}), coupled to a stationary thermal bath at
inverse temperature $\beta$, under a control protocol
$H_{\mathrm{ctrl}}(t) \in \mathcal{C}_{\mathrm{M}}$
(Definition~\ref{def:control_class}).
Then the survival functional satisfies
\begin{equation}
\label{eq:ceiling}
\mathcal{S}[\Lambda^{\mathrm{M}}, \tau] \leq 0
\qquad\text{for all } \tau \geq 0.
\end{equation}
Equality holds in the quasi-static limit ($\Sigma \to 0$),
where the protocol varies slowly enough that the state
remains close to the instantaneous Gibbs state
$\rho_{\mathrm{eq}}(t)$ at all times.
\end{theorem}

\begin{proof}
The proof proceeds in two steps.

\noindent\textbf{Step 1: Free-energy balance.}
Differentiating~\eqref{eq:DKL_F}, the relative entropy
evolves as
\begin{equation}
\label{eq:balance}
\frac{d}{dt}\, D_{\mathrm{KL}}(\rho(t) \| \rho_{\mathrm{eq}})
= \beta\,\dot{W}(t) - \sigma(t),
\end{equation}
where $\dot{W}(t) = \tr(\rho(t)\,\partial_t H_{\mathrm{ctrl}})$
is the instantaneous power and $\sigma(t)$ is Spohn's entropy
production rate~\eqref{eq:spohn_sigma}.
Integrating over $[0,\tau]$:
\begin{equation}
\label{eq:balance_int}
\Delta D_{\mathrm{KL}} = \beta\, W[0,\tau]
  - \underbrace{\int_0^\tau \sigma(t)\, dt}_{\displaystyle
    = \Sigma \;\geq\; 0}.
\end{equation}

\noindent\textbf{Step 2: Applying Spohn.}
By Lemma~\ref{lem:spohn}, $\sigma(t) \geq 0$ for all $t$,
so $\Sigma \geq 0$.
From~\eqref{eq:balance_int}:
\begin{equation}
\Delta D_{\mathrm{KL}} \leq \beta\, W[0,\tau].
\end{equation}
Converting via~\eqref{eq:DKL_F}:
$\Delta F \leq W[0,\tau]$,
whence $\mathcal{S} = \Delta F - W \leq 0$.

The ceiling $\mathcal{S} = 0$ is achieved in the reversible
limit where the protocol is infinitely slow and $\sigma(t) \to 0$
pointwise.
\end{proof}

\begin{remark}[The ``Open-Loop'' Qualifier]
\label{rem:open_loop}
The restriction to the control class
$\mathcal{C}_{\mathrm{M}}$
(Definition~\ref{def:control_class}) is essential.
If the agent can perform \emph{measurements} on the bath and
condition its protocol on the outcomes---i.e., violate
condition~\textup{(C2)}---the Sagawa--Ueda
generalized second law~\eqref{eq:sagawa_ueda} permits
$\Sigma < 0$ (and hence $\mathcal{S} > 0$) at the expense
of mutual information.
The Markovian ceiling is therefore not a universal bound on
all Markovian agents, but on agents whose protocols satisfy
\textup{(C1)--(C3)}.

This qualifier is precisely the point:
the memory kernel of non-Markovian dynamics provides implicit
access to bath correlations, playing the role of implicit
measurement---the subject of Section~\ref{sec:advantage}.
\end{remark}

\begin{corollary}[Temporal Blindness]
\label{cor:blindness}
Under the Born--Markov approximation, the bath correlation
function is replaced by its white-noise limit
$C(t,s) \to C_0\,\delta(t-s)$, and the GKSL dissipator depends
only on the spectral density $J(\omega)$ evaluated at the
system's Bohr frequencies.
The agent interacts with the environment's
\emph{power spectrum} but is structurally blind to its
\emph{temporal correlations}---the off-diagonal elements
$C(t,s)$ for $t \neq s$.

Consequently, the spectral gap
$\lambda_{\min} \propto \sum_k J(\omega_k)$ of the Liouvillian
sets the rate of irreversible decay.
Maintaining $D_{\mathrm{KL}} > 0$ requires continuous work at
rate $\dot{W} \geq \beta^{-1}\sigma(t) > 0$, and the integrated
cost always meets or exceeds the integrated gain.
\end{corollary}

\begin{remark}[Dissipative vs.\ Self-Nourishing Structures]
\label{rem:ceiling_physics}
The Markovian ceiling partitions far-from-equilibrium structures
into two classes:
\begin{itemize}
\item \textbf{Dissipative structures} ($\mathcal{S} \leq 0$):
  sustained by continuous external free-energy input.
  Every unit of order is paid for in full.
  (Prigogine's sense~\cite{Schrodinger1944}.)
\item \textbf{Self-nourishing structures} ($\mathcal{S} > 0$):
  extract structured advantage from environmental correlations,
  gaining more free energy than they consume.
  These require information flow, and hence memory.
\end{itemize}
The ceiling is not a limitation of the agent's control
strategy but a \emph{structural consequence} of temporal
blindness: without memory, the environment's temporal
correlations are thermodynamically invisible.
\end{remark}

% ============================================================
\section{The Non-Markovian Advantage}
\label{sec:advantage}

Having established that open-loop Markovian agents are
thermodynamically capped at $\mathcal{S} \leq 0$, we now
demonstrate how non-Markovian dynamics breaks this ceiling.
The mechanism is grounded entirely in standard quantities:
the quantum mutual information $I(S{:}E)$ between system
and environment serves as a consumable thermodynamic resource.
Non-Markovian backflow intervals are precisely those during
which pre-existing correlations are consumed, enabling the
system to extract free energy beyond what open-loop work
provides.

\subsection{System--Environment Mutual Information}

We work with the total system--environment state
$\rho_{SE}(t)$, evolving unitarily under the total
Hamiltonian~\eqref{eq:H_total}.
The quantum mutual information
\begin{equation}
\label{eq:mutual_info}
I(S{:}E;\, t) := S(\rho_S(t)) + S(\rho_E(t))
  - S(\rho_{SE}(t))
= D_{\mathrm{KL}}\!\bigl(\rho_{SE}(t) \,\big\|\,
  \rho_S(t) \otimes \rho_E(t)\bigr)
\geq 0
\end{equation}
quantifies the total correlations (classical and quantum)
between the system $S$ and the environment $E$ at time $t$.

\begin{remark}[Role of Initial Correlations]
\label{rem:initial_corr}
Under the Born approximation, the initial state is taken as a
product $\rho_{SE}(0) = \rho_S(0) \otimes \rho_E^{\mathrm{th}}$,
so $I(S{:}E;\, 0) = 0$.
For a system that has already been interacting with its
environment (the physically generic situation for a
``persistent agent''), the effective initial state at any
restart time $t_0 > 0$ is \emph{not} a product state:
the preceding evolution has established correlations
$I(S{:}E;\, t_0) > 0$.
These pre-existing correlations---the system's ``memory''
of past interactions---are the thermodynamic resource
that the memory kernel can exploit.
\end{remark}

\subsection{The Information--Thermodynamic Identity}

The following identity is the central technical tool of this
section.
It holds for \textbf{any} initial state---product or
correlated---and relies only on unitarity and the definitions
of mutual information and relative entropy.

\begin{remark}[Relative-entropy chain rule]
\label{rem:chain_rule}
We repeatedly use the identity
\begin{equation}
\label{eq:chain_general}
D_{\mathrm{KL}}\!\bigl(\rho_{SE}\,\big\|\,
  \rho_S\otimes\sigma_E\bigr)
= I(S{:}E)_{\rho_{SE}}
  + D_{\mathrm{KL}}(\rho_E\|\sigma_E),
\end{equation}
valid for \emph{arbitrary} (possibly correlated)
$\rho_{SE}$ and any full-rank reference state
$\sigma_E$.\footnote{This follows from the definition of
quantum relative entropy and
$\ln(\rho_S\otimes\sigma_E)
= \ln\rho_S\otimes\mathbb{1}_E
  +\mathbb{1}_S\otimes\ln\sigma_E$.
See, e.g., M.~M.~Wilde, \emph{Quantum Information
Theory}, 2nd ed., Cambridge University Press (2017),
Sec.~11; and M.~A.~Nielsen and I.~L.~Chuang,
\emph{Quantum Computation and Quantum Information},
Cambridge University Press (2000), Ch.~11.}
Importantly, this is a \emph{pure algebraic identity} and
does not assume product initial conditions.
\end{remark}

\begin{lemma}[Information--Thermodynamic Identity]
\label{lem:info_thermo}
Let $\rho_{SE}(t)$ evolve unitarily under the total
Hamiltonian.
Then, for \textbf{any} initial state $\rho_{SE}(0)$
(product or correlated):
\begin{equation}
\label{eq:info_thermo}
\Delta I(S{:}E)
+ \Delta D_{\mathrm{KL}}\!\bigl(\rho_E \,\big\|\,
  \rho_E^{\mathrm{th}}\bigr)
= \Delta S_S + \beta\,\Delta\langle H_E \rangle,
\end{equation}
where $\Delta S_S = S(\rho_S(\tau)) - S(\rho_S(0))$ is the
change in the system's von~Neumann entropy and
$\Delta\langle H_E \rangle
= \mathrm{Tr}[\rho_E(\tau)\, H_E]
  - \mathrm{Tr}[\rho_E(0)\, H_E]$
is the energy absorbed by the environment.
\end{lemma}

\begin{proof}
Applying the chain rule~\eqref{eq:chain_general}
(Remark~\ref{rem:chain_rule}) with
$\sigma_E = \rho_E^{\mathrm{th}}$:
\begin{equation}
\label{eq:chain_step}
D_{\mathrm{KL}}\!\bigl(\rho_{SE}(t) \,\big\|\,
  \rho_S(t) \otimes \rho_E^{\mathrm{th}}\bigr)
= I(S{:}E;\, t)
  + D_{\mathrm{KL}}\!\bigl(\rho_E(t) \,\big\|\,
    \rho_E^{\mathrm{th}}\bigr).
\end{equation}
Expanding the left side using
$\ln\rho_E^{\mathrm{th}} = -\beta H_E - \ln Z_E$:
\begin{equation}
\label{eq:phi_expand}
D_{\mathrm{KL}}\!\bigl(\rho_{SE}(t) \,\big\|\,
  \rho_S(t) \otimes \rho_E^{\mathrm{th}}\bigr)
= -S(\rho_{SE}(t)) + S(\rho_S(t))
  + \beta\langle H_E \rangle_t + \ln Z_E.
\end{equation}
Since the total evolution is unitary,
$S(\rho_{SE}(t)) = S(\rho_{SE}(0))$ for all $t$.
Taking the difference between times $\tau$ and $0$ cancels
both $S(\rho_{SE})$ and $\ln Z_E$, yielding
\begin{equation}
\Delta\!\left[I(S{:}E)
  + D_{\mathrm{KL}}(\rho_E \| \rho_E^{\mathrm{th}})\right]
= \Delta S_S + \beta\,\Delta\langle H_E \rangle.
\end{equation}
\end{proof}

\begin{remark}[No assumption on the initial state]
\label{rem:no_product_lemma}
The proof of Lemma~\ref{lem:info_thermo} uses \emph{only}
unitarity ($\Delta S(\rho_{SE}) = 0$) and the algebraic
structure of the KL divergence.
No assumption is made about the initial state $\rho_{SE}(0)$,
the coupling strength, or the character (Markovian or
non-Markovian) of the reduced dynamics.
When the initial state is a product state with the environment
in thermal equilibrium, all initial-time terms vanish and the
identity reduces to the Esposito
decomposition~\cite{EspositoLindenbergVandenBroeck2010}:
$\Sigma = I(S{:}E;\,\tau)
+ D_{\mathrm{KL}}(\rho_E(\tau) \| \rho_E^{\mathrm{th}})$.
\end{remark}

\subsection{The Survival Identity}

We now connect the information--thermodynamic
identity~\eqref{eq:info_thermo} to the survival
functional $\mathcal{S}$ defined in
Section~\ref{subsec:survival}.

\begin{theorem}[Survival Functional: General Form]
\label{thm:advantage}
Under Assumptions \textup{(A1)--(A5)} of
Definition~\ref{def:assumptions},
let $\rho_{SE}(t)$ evolve unitarily from an \emph{arbitrary}
(possibly correlated) initial state $\rho_{SE}(0)$.
The survival functional
satisfies
\begin{equation}
\label{eq:advantage}
\boxed{\;
\beta\,\mathcal{S}[\Lambda,\tau]
= -\Delta I(S{:}E)
  - \Delta D_{\mathrm{KL}}\!\bigl(\rho_E \,\big\|\,
    \rho_E^{\mathrm{th}}\bigr)
  - \beta\,\Delta\langle H_{\mathrm{ctrl}} \rangle,
\;}
\end{equation}
where $\Delta\langle H_{\mathrm{ctrl}} \rangle
= \mathrm{Tr}[\rho_S(\tau)\, H_{\mathrm{ctrl}}(\tau)]
  - \mathrm{Tr}[\rho_S(0)\, H_{\mathrm{ctrl}}(0)]$
is the change in the control-field energy.

For \textbf{autonomous evolution}
($H_{\mathrm{ctrl}} = 0$ throughout $[0,\tau]$),
the control term vanishes:
\begin{equation}
\label{eq:advantage_auto}
\beta\,\mathcal{S}[\Lambda,\tau]
= -\Delta I(S{:}E)
  - \Delta D_{\mathrm{KL}}\!\bigl(\rho_E \,\big\|\,
    \rho_E^{\mathrm{th}}\bigr).
\end{equation}
\end{theorem}

\begin{proof}
The proof uses three ingredients: the definition of
$\mathcal{S}$, the first law, and
Lemma~\ref{lem:info_thermo}.

\noindent\textbf{Step 1 (First law in weak coupling).}
Since $H_{\mathrm{ctrl}}(t)$ is the only time-dependent
component of $H$, the work satisfies
$W = \Delta\langle H \rangle
\approx \Delta\langle H_R \rangle
  + \Delta\langle H_{\mathrm{ctrl}} \rangle
  + \Delta\langle H_E \rangle$
by Assumption~\textup{(A2)}.

\noindent\textbf{Step 2 (Connecting $\Sigma$ to the
identity).}
From Definition~\ref{def:survival} and~\eqref{eq:Sigma},
using $\Delta F = \Delta\langle H_R \rangle
- \beta^{-1}\Delta S_S$:
\begin{align}
\Sigma &= \beta(W - \Delta F)
= \beta\bigl(W - \Delta\langle H_R \rangle\bigr)
  + \Delta S_S \nonumber \\
&= \beta\bigl(\Delta\langle H_{\mathrm{ctrl}} \rangle
  + \Delta\langle H_E \rangle\bigr) + \Delta S_S
\nonumber \\
&= \bigl(\Delta S_S
  + \beta\,\Delta\langle H_E \rangle\bigr)
  + \beta\,\Delta\langle H_{\mathrm{ctrl}} \rangle.
\label{eq:Sigma_decomp}
\end{align}
By Lemma~\ref{lem:info_thermo}, the parenthesized term
equals
$\Delta I(S{:}E)
+ \Delta D_{\mathrm{KL}}(\rho_E \| \rho_E^{\mathrm{th}})$.
Hence
\begin{equation}
\label{eq:Sigma_general}
\Sigma = \Delta I(S{:}E)
  + \Delta D_{\mathrm{KL}}\!\bigl(\rho_E \,\big\|\,
    \rho_E^{\mathrm{th}}\bigr)
  + \beta\,\Delta\langle H_{\mathrm{ctrl}} \rangle.
\end{equation}

\noindent\textbf{Step 3 (Survival functional).}
$\beta\,\mathcal{S} = -\Sigma$ by~\eqref{eq:survival_Sigma},
yielding~\eqref{eq:advantage}.
For $H_{\mathrm{ctrl}} = 0$:
$\Delta\langle H_{\mathrm{ctrl}} \rangle = 0$,
recovering~\eqref{eq:advantage_auto}.
\end{proof}

\begin{remark}[Nature of the result]
\label{rem:identity_not_bound}
Equation~\eqref{eq:advantage} is an exact
\emph{accounting identity}, not an inequality or
optimality bound.
It establishes that any thermodynamic profit
($\mathcal{S} > 0$) in the autonomous regime must be
perfectly balanced by the consumption of
system--environment correlations ($\Delta I < 0$) or
the relaxation of the bath
($\Delta D_{\mathrm{KL}} < 0$).
The ``non-Markovian advantage'' arises because memory
kernels allow access to regimes where
$\Delta I(S{:}E)$ is negative and dominant---a channel
that memoryless (Born--Markov) dynamics resets to zero
at every time step
(Remark~\ref{rem:born_kills}).
\end{remark}

\begin{remark}[Scope of the theorem]
\label{rem:no_product}
Theorem~\ref{thm:advantage} holds for \emph{any} initial
state $\rho_{SE}(0)$---product or correlated.
The proof requires only
Assumptions~\textup{(A1)--(A5)} of
Definition~\ref{def:assumptions} and the definitions of
$\mathcal{S}$, $I(S{:}E)$, and $D_{\mathrm{KL}}$.
No assumption about the reduced dynamics (Markovian,
non-Markovian, or otherwise) is needed.
This generality is essential: a persistent agent that has
already been interacting with its environment necessarily
carries correlations ($I(S{:}E;\,0) > 0$), and it is
precisely these correlations that constitute the thermodynamic
resource for survival.
\end{remark}

\subsection{Three Regimes of Survival}

We specialize to the autonomous case
($H_{\mathrm{ctrl}} = 0$), which is the natural setting for
the ``memory as a resource'' argument: the agent benefits from
pre-existing correlations without external driving.

\begin{corollary}[Three Regimes]
\label{cor:three_regimes}
Under autonomous evolution, identity~\eqref{eq:advantage_auto}
identifies three regimes:
\begin{enumerate}
\item \textbf{Product initial state}
  ($I(S{:}E;\, 0) = 0$,
  $D_{\mathrm{KL}}(\rho_E(0) \| \rho_E^{\mathrm{th}}) = 0$):
  Both $\Delta I$ and $\Delta D_{\mathrm{KL}}$ are increases
  from zero to non-negative final values, so
  \[
  \beta\,\mathcal{S}
  = -\bigl(I(S{:}E;\,\tau)
  + D_{\mathrm{KL}}(\rho_E(\tau) \| \rho_E^{\mathrm{th}})\bigr)
  \leq 0.
  \]
  This recovers the Markovian ceiling
  (Theorem~\ref{thm:ceiling}), now with a precise accounting
  of \emph{where} the entropy goes: into system--environment
  correlations and bath displacement.

\item \textbf{Correlated initial state}
  ($I(S{:}E;\, 0) > 0$):
  If the dynamics \emph{consumes} pre-existing correlations
  ($\Delta I < 0$, i.e., $I(S{:}E;\, \tau) < I(S{:}E;\, 0)$),
  the first term contributes \emph{positively} to
  $\mathcal{S}$.
  Provided
  \begin{equation}
  \label{eq:ceiling_breach}
  |\Delta I(S{:}E)| > \Delta D_{\mathrm{KL}}\!\bigl(\rho_E
  \,\big\|\, \rho_E^{\mathrm{th}}\bigr),
  \end{equation}
  the survival functional is strictly positive:
  $\mathcal{S} > 0$.
  The agent has converted pre-existing correlations into
  usable free energy.

\item \textbf{Upper bound:}
  Since $I(S{:}E;\, \tau) \geq 0$ and
  $D_{\mathrm{KL}}(\rho_E(\tau) \| \rho_E^{\mathrm{th}})
  \geq 0$, the maximum survival gain is bounded by
  \begin{equation}
  \label{eq:S_upper}
  \beta\,\mathcal{S}
  \leq I(S{:}E;\, 0)
    + D_{\mathrm{KL}}\!\bigl(\rho_E(0) \,\big\|\,
      \rho_E^{\mathrm{th}}\bigr).
  \end{equation}
  The thermodynamic profit cannot exceed the total initial
  ``resource budget''---the pre-existing correlations plus
  the initial displacement of the bath from equilibrium.
\end{enumerate}
\end{corollary}

\subsection{The Correlation Battery}

The three regimes of Corollary~\ref{cor:three_regimes} raise a
natural question: \emph{where do the initial correlations
$I(S{:}E;\, 0) > 0$ come from?}

\begin{remark}[The Correlation Battery]
\label{rem:battery}
The answer is: from \textbf{prior interaction epochs}.
A persistent agent does not begin its existence in a product
state.
Over any interaction interval, unitary evolution generically
builds system--environment correlations ($\Delta I > 0$),
at a thermodynamic cost ($\mathcal{S} < 0$ during this
phase by Corollary~\ref{cor:three_regimes}(i)).
The non-Markovian agent's advantage is that these correlations
\emph{persist} and can be consumed during later intervals
($\Delta I < 0$, $\mathcal{S} > 0$).

The process is analogous to a \textbf{battery}:
\begin{itemize}
\item \textbf{Charging phase}
  (correlation building, $\Delta I > 0$):
  the agent ``pays'' free energy to build
  system--environment correlations.
  $\mathcal{S} < 0$.
\item \textbf{Discharging phase}
  (correlation consumption, $\Delta I < 0$):
  the agent extracts free energy from the stored
  correlations.
  $\mathcal{S} > 0$.
\end{itemize}
A Markovian agent cannot operate this battery.
The Born approximation resets $I(S{:}E) = 0$ at every
infinitesimal time step, destroying the stored correlations
before they can be used.
The semigroup property $\Lambda(t+s) = \Lambda(t)\Lambda(s)$
is precisely the statement that no inter-epoch correlations
survive.
The memory kernel $\mathcal{K}(t,s)$ is what allows the
non-Markovian agent to carry charge across epochs.

Crucially, global thermodynamics remains respected.
For any full cycle starting from an uncorrelated thermal
state ($I(S{:}E;\,0) = 0$,
$D_{\mathrm{KL}}(\rho_E(0)\|\rho_E^{\mathrm{th}}) = 0$),
the total survival functional satisfies
\begin{equation}
\label{eq:battery_closure}
\beta\,\mathcal{S}[0, t]
= -\Sigma[0, t] \;\leq\; 0
\qquad\text{(second law).}
\end{equation}
The local positivity $\mathcal{S}[t^*, t] > 0$ during the
discharging phase is strictly funded by the free energy
dissipated during the earlier charging phase
(see Proposition~\ref{prop:full_cycle} for the formal
decomposition).
\end{remark}

\begin{proposition}[Full-cycle closure]
\label{prop:full_cycle}
Under the conditions of
Theorem~\textup{\ref{thm:advantage}} with autonomous evolution
($H_{\mathrm{ctrl}} = 0$), partition $[0,\tau]$ at any
intermediate time $t^*$ into a charging phase $[0,t^*]$ and
a discharging phase $[t^*,\tau]$.
\begin{enumerate}
\item[\textup{(i)}]
  \textbf{Charging} (product initial state, $I(S{:}E;\,0) = 0$).
  By Corollary~\textup{\ref{cor:three_regimes}(i)},
  \begin{equation}
  \label{eq:charge}
  \beta\,\mathcal{S}[0,t^*]
  = -I(S{:}E;\,t^*)
    - D_{\mathrm{KL}}\!\bigl(\rho_E(t^*) \,\big\|\,
      \rho_E^{\mathrm{th}}\bigr)
  \;\leq\; 0.
  \end{equation}
\item[\textup{(ii)}]
  \textbf{Discharging} (correlated initial state at $t^*$).
  Applying~\eqref{eq:advantage_auto} to $[t^*,\tau]$:
  \begin{equation}
  \label{eq:discharge}
  \beta\,\mathcal{S}[t^*,\tau]
  = -\bigl(I(S{:}E;\,\tau) - I(S{:}E;\,t^*)\bigr)
    - \bigl(D_{\mathrm{KL}}(\rho_E(\tau)\|\rho_E^{\mathrm{th}})
      - D_{\mathrm{KL}}(\rho_E(t^*)\|\rho_E^{\mathrm{th}})\bigr),
  \end{equation}
  which is positive whenever the decrease in correlations
  dominates the change in bath displacement
  (Corollary~\textup{\ref{cor:three_regimes}(ii)}).
\item[\textup{(iii)}]
  \textbf{Full cycle.}
  Since $\mathcal{S}$ is additive over concatenated intervals,
  $\beta\,\mathcal{S}[0,\tau]
  = \beta\,\mathcal{S}[0,t^*]
    + \beta\,\mathcal{S}[t^*,\tau]$.
  Equivalently, applying~\eqref{eq:advantage_auto} directly to
  $[0,\tau]$ with $I(S{:}E;\,0) = 0$:
  \begin{equation}
  \label{eq:full_cycle}
  \beta\,\mathcal{S}[0,\tau]
  = -I(S{:}E;\,\tau)
    - D_{\mathrm{KL}}\!\bigl(\rho_E(\tau) \,\big\|\,
      \rho_E^{\mathrm{th}}\bigr)
  \;\leq\; 0.
  \end{equation}
\end{enumerate}
The net thermodynamic profit over the full cycle is
non-positive---the ``interest'' paid during charging
meets or exceeds the ``dividend'' collected during
discharging.
But the \emph{local} positivity of $\mathcal{S}$ during
discharge~\eqref{eq:discharge} is what enables the agent
to survive through intervals that would kill a memoryless
system.
\end{proposition}

\begin{proof}
The survival functional is additive over concatenated
intervals:
\[
\mathcal{S}[0,\tau]
= \underbrace{(\Delta F[0,t^*] - W[0,t^*])}_{\mathcal{S}[0,t^*]}
  +\; \underbrace{(\Delta F[t^*,\tau] - W[t^*,\tau])}_{\mathcal{S}[t^*,\tau]},
\]
since both $\Delta F$ and $W$ decompose additively.
Items (i) and (iii) then follow from
Theorem~\ref{thm:advantage} (autonomous case) applied to
$[0,t^*]$ and $[0,\tau]$ respectively, each starting from
a product state.
Item (ii) follows from Theorem~\ref{thm:advantage} applied
to $[t^*,\tau]$ with correlated initial state
$\rho_{SE}(t^*)$.
Inequality~\eqref{eq:full_cycle} holds because
$I(S{:}E;\,\tau) \geq 0$ and
$D_{\mathrm{KL}}(\rho_E(\tau)\|\rho_E^{\mathrm{th}}) \geq 0$.
\end{proof}

\subsection{Connection to Non-Markovianity Measures}

\begin{remark}[The Born Approximation Destroys the Resource]
\label{rem:born_kills}
Under the Born (product-state) approximation, every
infinitesimal time step begins from
$\rho_{SE} \approx \rho_S \otimes \rho_E^{\mathrm{th}}$,
enforcing $I(S{:}E) = 0$ at all times.
Corollary~\ref{cor:three_regimes}(i) then guarantees
$\mathcal{S} \leq 0$ for \emph{every} finite interval.
The Born approximation does not merely simplify the
dynamics---it \emph{eliminates the thermodynamic resource}
(system--environment correlations) that would otherwise be
available.
\end{remark}

\begin{remark}[Connection to BLP Non-Markovianity]
\label{rem:BLP}
The Breuer--Laine--Piilo (BLP) measure of
non-Markovianity~\cite{BreuerLainePiilo2009} is defined via
the temporary increase of trace distance between pairs of
initial states:
$\mathcal{N}_{\mathrm{BLP}} := \max_{\rho_{1,2}}
\int_{\dot{D}>0} \frac{d}{dt}
D(\rho_1(t),\rho_2(t))\, dt$.
The intervals where trace distance increases are precisely the
``discharging'' intervals of
Remark~\ref{rem:battery}~\cite{RivasHuelgaPlenio2014}:
correlations previously deposited in the bath flow back to the
system, restoring distinguishability.
The BLP measure thus witnesses the thermodynamic resource
that drives $\mathcal{S} > 0$ in
Corollary~\ref{cor:three_regimes}(ii).
\end{remark}

\begin{remark}[Consistency with the Sagawa--Ueda Framework]
\label{rem:SU_consistency}
In the Sagawa--Ueda framework~\cite{SagawaUeda2010,SagawaUeda2012},
measurement-based feedback permits
$\Sigma \geq -I_{\mathrm{feedback}}$, where
$I_{\mathrm{feedback}}$ is the mutual information gained
through measurement.
The memory kernel plays an analogous role:
the pre-existing correlations $I(S{:}E;\, 0)$ are the
non-Markovian analogue of $I_{\mathrm{feedback}}$.
The total system (agent + bath) still satisfies
$\Sigma_{\mathrm{total}} \geq 0$; the apparent ``profit'' for
the agent is paid for by the correlations consumed from the
system--environment entanglement.
The bound~\eqref{eq:S_upper} is the non-Markovian analogue of
the Sagawa--Ueda bound $\beta\,\mathcal{S} \leq
I_{\mathrm{feedback}}$.
\end{remark}

\subsection{Mechanism: The Surfer Analogy}

The physical mechanism admits an intuitive picture.

\begin{itemize}
\item \textbf{The Markovian Agent (The Stone):}
A stone thrown into the ocean sinks.
It interacts with the water only at the instant of contact,
dissipates its kinetic energy, and thermalizes
($\mathcal{S} \leq 0$).
Each collision builds system--environment correlations that are
immediately discarded (Born approximation), so
$I(S{:}E) = 0$ at all times.
The wave structure is invisible to it.

\item \textbf{The Non-Markovian Agent (The Surfer):}
A surfer carries \emph{memory} of past wave patterns---encoded
in the correlations $I(S{:}E;\, t_0) > 0$ built up over
previous interactions (the ``charging phase'' of
Remark~\ref{rem:battery}).
During backflow intervals ($\Delta I < 0$), the surfer
\emph{spends} these stored correlations to extract free energy
from the wave itself.
The surfer remains far from equilibrium not by fighting the
environment, but by converting temporal correlations into
thermodynamic profit.
\end{itemize}

\begin{remark}[Thermodynamic Rectification]
\label{rem:rectification}
The ``surfing'' mechanism is \textbf{thermodynamic
rectification}: the memory kernel $\mathcal{K}(t,s)$ functions
as a temporal filter that enables the system to accumulate
correlations during one phase and consume them during another.
Formally, the kernel enables access to the resource
$I(S{:}E;\, 0)$ accumulated during previous interaction
epochs---converting the environment's temporal correlations
into the system's structural persistence via the
$\Delta I$ term in Theorem~\ref{thm:advantage}.
\end{remark}

\begin{remark}[Memory as Implicit Maxwell's Demon]
\label{rem:demon}
The memory kernel functions as an \emph{implicit Maxwell's demon}.
A Markovian system interacts with each environmental fluctuation
exactly once, at the moment of contact; the Born approximation
resets $I(S{:}E) = 0$ after each step.
A non-Markovian system retains a trace of past fluctuations
(via $\mathcal{K}(t,s)$ with $s < t$) and can exploit
correlations between past and present environmental states.
This is not a violation of the second law but an instance of
the Sagawa--Ueda generalization: the demon's cost is paid in
the currency of memory maintenance (Landauer erasure), a point
we quantify in Section~\ref{sec:cost}.
The total budget for ``demonic profit'' is capped by the
bound~\eqref{eq:S_upper}.
\end{remark}

% ============================================================
\section{Emergent Temporal Arrow}
\label{sec:arrow}

We have shown that survival requires memory.
This requirement yields a corollary: the emergence of a
thermodynamic arrow of time.
In this framework, time is not an external parameter;
rather, \emph{the direction of time is the direction of memory
accumulation}.

We formalize this by defining a dynamical partial order induced
by the memory kernel and connecting it to the algebraic
accessibility structure of HAFF Paper~F~\cite{Liu2026HAFF_F}.

\subsection{The Causal Memory Order}

A non-Markovian memory kernel $\mathcal{K}(t,s)$ defines a
causal link between a past state at $s$ and the present dynamics
at $t$.
We define a partial order based on the effective support of this
influence.

\begin{definition}[Causal Memory Order]
\label{def:memory_order}
Let $\mathcal{T} = \{\rho(t) \mid t \in \mathbb{R}^+\}$ be
a state trajectory.
We define the binary relation $\prec_K$ on $\mathcal{T}$ by
\begin{equation}
\label{eq:memory_order}
\rho(s) \prec_K \rho(t) \quad \iff \quad
\exists\, \tau \in [s, t] \text{ such that }
\| \mathcal{K}(t, \tau)[\rho(s)] \| > \epsilon,
\end{equation}
where $\epsilon > 0$ is a physical distinguishability threshold
set by the thermal noise floor
$\epsilon \sim e^{-\beta\, \Delta E_{\min}}$.
Physically, $\rho(s) \prec_K \rho(t)$ means ``the dynamics at
$t$ retains operationally distinguishable information about the
state at $s$.''
\end{definition}

For a Markovian agent, $\mathcal{K}(t,s) \propto \delta(t-s)$,
so $\rho(s) \nprec_K \rho(t)$ for any $s < t$.
The Markovian agent has no dynamical past---it lives in an
eternal ``now.''
A non-Markovian agent carries its history within its dynamics;
the depth of the order $\prec_K$ is set by the memory time
$\tau_{\mathrm{mem}}$ (Definition~\ref{def:kernel}).

\subsection{Unidirectionality from Survival Optimization}

Why does the order $\prec_K$ point ``forward''?
While the microscopic laws are time-reversible, the
\emph{survival imperative} (maximizing $\mathcal{S}$) creates
a statistical irreversibility.

\begin{proposition}[Fisher Information Accretion]
\label{prop:fisher_accretion}
Let $\mathcal{I}_F(\theta;\, \rho(t))$ denote the Fisher
information contained in the system state $\rho(t)$ regarding
a parameter $\theta$ encoded in the environment at time $s < t$.
For an agent whose dynamics maximize the survival
functional~\eqref{eq:survival}, the time-averaged Fisher
information satisfies
\begin{equation}
\label{eq:fisher_accretion}
\overline{\frac{d}{dt}\,
\mathcal{I}_F(\theta;\, \rho(t))} \geq 0,
\end{equation}
where the overbar denotes a time average over scales larger
than the bath correlation time $\tau_B$.
\end{proposition}

\begin{proof}
By Theorem~\ref{thm:advantage}, the survival functional is
maximized when $I(S{:}E;\,0)$ is large and can be consumed
($\Delta I < 0$) during subsequent evolution.
Maintaining a large correlation budget $I(S{:}E)$ requires
the system state to retain correlations with environmental
degrees of freedom; this is precisely the content of
$\mathcal{I}_F(\theta;\, \rho(t)) > 0$.
An agent that discards useful correlations (decreasing
$\mathcal{I}_F$) without thermodynamic necessity depletes the
resource $I(S{:}E)$ and hence its survival functional.
Since the environment's correlations decay on a timescale
$\tau_B$, the agent must continuously build new correlations
to replace decaying ones.
The net effect is a time-averaged accretion of Fisher
information, whose gradient defines the dynamical arrow of time.
\end{proof}

\subsection{The Bridge to HAFF}

We now connect this dynamical picture to the algebraic picture of
HAFF Paper~F~\cite{Liu2026HAFF_F}, where the arrow of time was
defined by the expansion of the redundancy subalgebra
$\mathcal{R}$.

The connection requires care: quantum information cannot be
cloned (the no-cloning theorem), so the ``redundancy expansion''
of HAFF must be interpreted through the lens of
\emph{quantum Darwinism}~\cite{Zurek2009}.
In this framework, the environment acquires not copies of the
quantum state $\rho(s)$ itself, but rather \emph{coarse-grained
classical records} of pointer-state outcomes---precisely the
information that survives decoherence and can be redundantly
encoded in many environmental fragments.

\begin{proposition}[Dynamical--Algebraic Correspondence]
\label{prop:HAFF_bridge}
Let $\prec_K$ be the causal memory order
(Definition~\ref{def:memory_order}) and let
$\prec_{\mathrm{HAFF}}$ be the accessibility order of
HAFF Paper~F, defined by the inclusion of redundancy
subalgebras $\mathcal{R}$.
Under the additional assumption that the system--environment
interaction produces decoherence in a preferred pointer
basis~\cite{Zurek2009}, there exists a coarse-graining map
$\Phi: \rho(t) \mapsto \hat{p}(t)$ (projecting onto the
diagonal in the pointer basis) such that:
\begin{equation}
\label{eq:HAFF_bridge}
\rho(s) \prec_K \rho(t)
\quad\Longrightarrow\quad
\mathcal{R}(\Phi[\rho(s)])
\subseteq \mathcal{R}(\Phi[\rho(t)]).
\end{equation}
That is, the dynamical partial order maps into the algebraic
accessibility order when restricted to the classical sector
selected by decoherence.
\end{proposition}

\begin{proof}
The argument has three steps.

\noindent\textbf{Step 1 (Dynamical side):}
$\rho(s) \prec_K \rho(t)$ implies that the memory kernel
$\mathcal{K}$ transduces information about the state at $s$
into the dynamics at $t$, via system--environment correlations
built up over $[s,t]$.

\noindent\textbf{Step 2 (Quantum Darwinism):}
The system--environment interaction selects pointer states
$\{|i\rangle\}$ that are robust under
decoherence~\cite{Zurek2009}.
The diagonal populations
$p_i(t) = \langle i | \rho(t) | i \rangle$
constitute \emph{classical} information.
Quantum Darwinism~\cite{Zurek2009} establishes that this
classical information---and \emph{only} this information---is
redundantly imprinted in many environmental fragments $E_k$
through the decoherence interaction.
Each fragment that acquires a record of
$\hat{p}(t) = \{p_i(t)\}$ contributes to the growth of the
redundancy subalgebra $\mathcal{R}$.
Crucially, no quantum cloning is involved: the no-cloning
theorem forbids copying of arbitrary quantum states, but does
not constrain the classical pointer-state probabilities, which
are freely duplicable.
The expansion of $\mathcal{R}$ reflects the proliferation of
these classical records, not the copying of quantum coherences.

\noindent\textbf{Step 3 (Correspondence):}
The coarse-graining map $\Phi$ projects onto the
\emph{commuting} subalgebra generated by the pointer
observables $\{|i\rangle\langle i|\}$.
The resulting probability distributions $\hat{p}(t)$ are
classical and lie in a simplex.
If $\rho(s) \prec_K \rho(t)$, then the dynamics at $t$
retains information about the state at $s$
(Definition~\ref{def:memory_order}); in the pointer basis,
this means $\hat{p}(s)$ is statistically reconstructible
from the environmental records available at $t$.
Since each environmental fragment carrying a record of
$\hat{p}$ contributes to the HAFF redundancy subalgebra
$\mathcal{R}$, and the number of such fragments grows
monotonically with the accumulation of decoherence records,
the inclusion
$\mathcal{R}(\Phi[\rho(s)]) \subseteq
\mathcal{R}(\Phi[\rho(t)])$ follows.
\end{proof}

\begin{remark}[Scope of the Correspondence]
\label{rem:scope}
Proposition~\ref{prop:HAFF_bridge} is a \emph{consistency
result}, not a derivation of HAFF from T-DOME or vice versa.
It shows that the dynamical arrow (memory accumulation) and the
algebraic arrow (redundancy expansion) are compatible when
restricted to the decoherence-selected classical sector.
The quantum coherences---which are not redundantly
recorded---lie outside this correspondence and are handled by
the full non-Markovian dynamics.
\end{remark}

\begin{remark}[Dynamical and Algebraic Time]
\label{rem:dual_time}
The correspondence links two independently motivated notions of
temporal direction:
\begin{center}
\begin{tabular}{@{}lll@{}}
\toprule
& \textbf{Paper F (HAFF)} & \textbf{Paper I (T-DOME)} \\
\midrule
Nature & Algebraic & Dynamical \\
Mechanism & Redundancy expansion
  & Information backflow from memory \\
Formalism & Partial order on $\mathcal{A}_{\mathbf{c}}$
  & Partial order $\prec_K$ on $\rho_R(t)$ \\
Asymmetry source & Phase-space measure
  & Bath correlation structure \\
Domain & Classical (pointer) sector
  & Full quantum dynamics \\
\bottomrule
\end{tabular}
\end{center}
Paper~F provides the structural \emph{skeleton} of temporal
asymmetry; Paper~I provides the dynamical \emph{muscle}.
\end{remark}

\begin{remark}[The Seed and the Tree]
\label{rem:seed_tree}
The correspondence justifies the title of this paper.
In HAFF, the geometry of spacetime is the static ``tree.''
In T-DOME, the memory kernel is the ``seed'' containing the
generative algorithm for growth.
Time is not the space in which the tree grows;
time is the \emph{act of growing} itself.
\end{remark}

% ============================================================
\section{Worked Example: The Quantum Predictive Agent}
\label{sec:example}

To illustrate the Markovian ceiling and the memory advantage
\emph{quantitatively}, we employ the archetypal open quantum
system model: the spin-boson model with Lorentz--Drude spectral
density, which admits an exact analytic solution for the
decoherence dynamics~\cite{BreuerPetruccione2002}.

\subsection{Model Setup}

The total Hamiltonian is $H = H_S + H_B + H_I$.
The agent is a two-level system with energy gap $\omega_0$:
$H_S = \tfrac{\omega_0}{2}\sigma_z$.
The environment is a bosonic bath:
$H_B = \sum_k \omega_k b_k^\dagger b_k$.
The interaction is of the pure-dephasing form
$H_I = \sigma_z \otimes \sum_k (g_k\, b_k
+ g_k^*\, b_k^\dagger)$.

The spectral density
$J(\omega) = \sum_k |g_k|^2 \delta(\omega - \omega_k)$
characterizes the environment. We choose the Lorentz--Drude form:
\begin{equation}
\label{eq:lorentz_drude}
J(\omega) = \frac{2\lambda\, \gamma\, \omega}
  {\omega^2 + \gamma^2},
\end{equation}
where $\lambda$ is the reorganization energy and $\gamma$ is the
bath memory rate (inverse correlation time
$\tau_B = 1/\gamma$).
We place the system in the low-temperature regime
$\beta\omega_0 \gg 1$ (i.e., $k_BT \ll \omega_0$).
The bath correlation function in the $T \to 0$ limit is
$C(t) = \lambda\gamma\, e^{-\gamma|t|}$, so the parameter
$\gamma$ directly controls the bath memory depth.
For $\beta\omega_0 \geq 10$ the finite-temperature corrections
to all quantities below are of order
$O(e^{-\beta\omega_0}) \lesssim 5 \times 10^{-5}$ and
are neglected throughout.\footnote{All plots and numerical values
use the standard $T \to 0$ analytic expression for the
decoherence function~\eqref{eq:decoherence}
(see Breuer and Petruccione~\cite{BreuerPetruccione2002},
Sec.~12.3, for the Lorentz--Drude pure-dephasing solution),
which provides an accurate proxy in the low-temperature
regime;
$\beta$ is a well-defined bookkeeping parameter and
$\beta^{-1}$ a finite energy scale.}

\subsection{Exact Decoherence Function}

For the pure-dephasing spin-boson model in the
$T \to 0$ limit,
the off-diagonal element of the reduced density matrix
$\rho_{01}(t) = \rho_{01}(0)\, p(t)$ is governed by the
\textbf{decoherence function}~\cite{BreuerPetruccione2002}:
\begin{equation}
\label{eq:decoherence}
p(t) = e^{-\gamma t / 2}\left[
  \cos(\Omega t) + \frac{\gamma}{2\Omega}\,\sin(\Omega t)
\right],
\end{equation}
where $\Omega := \frac{1}{2}\sqrt{4\lambda\gamma - \gamma^2}$.
This solution is exact for the Lorentz--Drude spectral density.

\begin{remark}[Non-Markovian Regime]
\label{rem:NM_regime}
The character of the dynamics is controlled by the discriminant
$\Delta := 4\lambda\gamma - \gamma^2 = \gamma(4\lambda - \gamma)$:
\begin{itemize}
\item $\gamma > 4\lambda$ ($\Delta < 0$):
  $\Omega$ is imaginary, $p(t)$ decays monotonically.
  The dynamics is Markovian (no backflow).
\item $\gamma = 4\lambda$ ($\Delta = 0$):
  Critical damping. $p(t) = (1 + \gamma t/2)\, e^{-\gamma t/2}$.
\item $\gamma < 4\lambda$ ($\Delta > 0$):
  $\Omega$ is real and positive.
  $p(t)$ oscillates with envelope $e^{-\gamma t/2}$.
  The dynamics is \emph{non-Markovian}: intervals where
  $|p(t)|$ increases correspond to information
  backflow~\cite{BreuerLainePiilo2009}.
\end{itemize}
The non-Markovian regime $\gamma < 4\lambda$ is thus the regime
of structured, long-memory baths.
\end{remark}

\subsection{Quantitative Evaluation}

We now evaluate the survival functional explicitly.
For the pure-dephasing model, populations are conserved
($p_0(t) = p_0(0)$, $p_1(t) = p_1(0)$), and the non-equilibrium
free energy depends only on the coherence:
\begin{equation}
\label{eq:F_qubit}
F(\rho(t)) - F(\rho_{\mathrm{eq}})
= \beta^{-1}\, D_{\mathrm{KL}}(\rho(t) \| \rho_{\mathrm{eq}}).
\end{equation}
For a qubit with initial state
$\rho(0) = \tfrac{1}{2}(\mathbb{1} + \vec{r}\cdot\vec{\sigma})$
and $r_z = 0$ (maximal coherence in the $x$--$y$ plane),
the relative entropy reduces to
$D_{\mathrm{KL}}(\rho(t) \| \rho_{\mathrm{eq}})
\approx |p(t)|^2\, |\rho_{01}(0)|^2$
to leading order in the coherence (see,
e.g.,~\cite{BreuerPetruccione2002}).
Since there is no external driving ($H_{\mathrm{ctrl}} = 0$,
$W = 0$), the survival functional is simply
\begin{equation}
\label{eq:S_qubit}
\beta\,\mathcal{S}(t)
= D_{\mathrm{KL}}(\rho(t) \| \rho_{\mathrm{eq}})
  - D_{\mathrm{KL}}(\rho(0) \| \rho_{\mathrm{eq}})
\propto |p(t)|^2 - 1.
\end{equation}
The proportionality in~\eqref{eq:S_qubit} is specific to
the \textbf{pure-dephasing model} with the chosen maximally
coherent initial state ($r_z = 0$) and measurement in the
pointer basis ($\sigma_z$).
Under these conditions, the exact
solution~\cite{BreuerPetruccione2002} ensures that population
terms vanish from the free energy
($\Delta\langle H_S \rangle = 0$), leaving only the
coherence contribution:
$\beta\,\mathcal{S} = -\Delta S_S$ depends only on the
coherence trajectory $|p(t)|$.
The proxy $|p(t)|^2$ thus rigorously captures the sign and
monotone behaviour of $\beta\,\mathcal{S}$; the exact
numerical prefactor depends on the initial state and
on $\beta$, but the qualitative conclusion---$\mathcal{S} > 0$
during backflow intervals---is robust and does not depend on
the proxy normalization.

For a Markovian evolution, $|p(t)|$ decreases monotonically,
so $|p(t)|^2 - 1 \leq 0$ for all $t$: $\mathcal{S} \leq 0$
always (consistent with Theorem~\ref{thm:ceiling}).
For non-Markovian evolution with $\gamma < 4\lambda$, the
oscillations in $p(t)$ produce intervals where $|p(t)|$
\emph{increases after a previous decrease}, i.e., the system
\emph{re-coheres}.

\textbf{Concrete parameters.}
Set $\omega_0 = 1$ (energy units), $\lambda = 1$,
$\gamma = 0.5$ (deep non-Markovian regime: $\gamma / 4\lambda = 0.125 \ll 1$).
Then:
\begin{equation}
\Omega = \tfrac{1}{2}\sqrt{4 \cdot 1 \cdot 0.5 - 0.25}
= \tfrac{1}{2}\sqrt{1.75} \approx 0.661.
\end{equation}

The decoherence function~\eqref{eq:decoherence} first reaches
$p(t^*) = 0$ at $t^* \approx 2.00/\Omega \approx 3.03$
(in units of $\omega_0^{-1}$), where the system has fully
decohered.
Subsequently, the environment \emph{returns} coherence:
$|p(t)|$ increases, reaching a local maximum
$|p(t_1)| \approx 0.31$ at $t_1 \approx 4.75/\omega_0$.

Over the backflow interval $[t^*, t_1]$,
for the pure-dephasing qubit with the chosen initial state
($r_z = 0$, maximal coherence) and in the autonomous setting
($H_{\mathrm{ctrl}} = 0$, $W = 0$),
the survival proxy~\eqref{eq:S_qubit} gives
\begin{equation}
\label{eq:S_numeric}
\beta\,\mathcal{S}[t^*, t_1]
\propto |p(t_1)|^2 - |p(t^*)|^2
\approx 0.093 - 0 = 0.093 > 0.
\end{equation}
Equivalently,
$\mathcal{S} \approx 0.093\,\beta^{-1}$
in the bookkeeping units set by $\beta$.
The agent has gained a dimensionless survival advantage
$\beta\,\mathcal{S} \approx +0.093$
\emph{with zero work input}
(autonomous evolution, $H_{\mathrm{ctrl}} = 0$),
solely by exploiting the non-Markovian backflow.
Figure~\ref{fig:survival} illustrates the contrast between
Markovian and non-Markovian evolution.

\begin{figure}[t]
\centering
\includegraphics[width=\textwidth]{fig_survival.pdf}
\caption{%
Pure-dephasing spin-boson model
(Section~\ref{sec:example}) with Lorentz--Drude spectral
density~\eqref{eq:lorentz_drude}.
\textbf{Parameters:}
$\omega_0 = 1$ (energy unit),
$\lambda = 1$ (reorganization energy).
\textbf{Units:}
all times in $\omega_0^{-1}$;
energies in $\hbar\omega_0$.
\textbf{Regime:}
low temperature ($\beta\omega_0 \gg 1$);
the standard $T \to 0$ analytic
expression~\eqref{eq:decoherence}~\cite{BreuerPetruccione2002}
is used as an accurate proxy.
\textbf{(a)}~Decoherence amplitude $|p(t)|$
(eq.~\eqref{eq:decoherence}).
Blue: non-Markovian
($\gamma = 0.5$, $\gamma/4\lambda = 0.125$).
Orange: Markovian
($\gamma = 5.0$, $\gamma/4\lambda = 1.25$).
Dashed: exponential envelope $e^{-\gamma t/2}$.
Green bands indicate backflow
(Remark~\ref{rem:NM_regime}:
$d|p|/dt > 0$, $\Gamma(t) < 0$
per~\eqref{eq:Gamma_inst}).
\textbf{(b)}~Survival proxy
$|p(t)|^2 \propto \beta\,\mathcal{S}$
(eq.~\eqref{eq:S_qubit}).
At the first revival
($t_1 = \pi/\Omega \approx 4.75\,\omega_0^{-1}$),
the non-Markovian agent achieves
$\beta\,\mathcal{S}[t^*,t_1] \approx +0.093$
(eq.~\eqref{eq:S_numeric}), consistent with the
closed-form prediction~\eqref{eq:revival_exact},
funded by the consumption of pre-existing
correlations
(Proposition~\ref{prop:full_cycle}).
The Markovian agent decays monotonically:
$\mathcal{S} \leq 0$ always
(Theorem~\ref{thm:ceiling}).%
}
\label{fig:survival}
\end{figure}

\textbf{Consistency with
Theorem~\ref{thm:advantage} and
Proposition~\ref{prop:full_cycle}.}
Since this is autonomous evolution ($H_{\mathrm{ctrl}} = 0$),
identity~\eqref{eq:advantage_auto} applies exactly:
$\beta\,\mathcal{S} = -\Delta I(S{:}E)
- \Delta D_{\mathrm{KL}}(\rho_E \| \rho_E^{\mathrm{th}})$.
The example realizes the \emph{correlation battery}
of Remark~\ref{rem:battery}, with the charge--discharge
decomposition of
Proposition~\ref{prop:full_cycle}:
\begin{itemize}
\item \textbf{Charging} ($[0, t^*]$, eq.~\eqref{eq:charge}):
  the system decoheres, building correlations
  $I(S{:}E;\, t^*) > 0$ at the cost of
  $\mathcal{S} < 0$.
\item \textbf{Discharging} ($[t^*, t_1]$,
  eq.~\eqref{eq:discharge}):
  the correlations are consumed
  ($\Delta I < 0$ over this interval),
  returning $\beta\,\mathcal{S} \approx +0.093 > 0$.
\end{itemize}
The bound~\eqref{eq:S_upper} is satisfied:
$\beta\,\mathcal{S}[t^*, t_1] = 0.093
\leq I(S{:}E;\, t^*)$.
Full-cycle closure~\eqref{eq:full_cycle} is confirmed:
$\beta\,\mathcal{S}[0, t_1] < 0$.

\textbf{Instantaneous decoherence rate.}
The rate of coherence loss is
\begin{equation}
\label{eq:Gamma_inst}
\Gamma(t) := -\frac{d}{dt}\ln|p(t)|
= \frac{\gamma}{2}
  - \frac{\Omega\sin(\Omega t)
    + \tfrac{\gamma}{2}\cos(\Omega t)}
  {\cos(\Omega t)
    + \tfrac{\gamma}{2\Omega}\sin(\Omega t)}\,.
\end{equation}
In the Markovian limit $\gamma \gg 4\lambda$,
$\Gamma(t) \to \gamma/2 > 0$ for all $t$
(monotone decoherence).
In the non-Markovian regime $\gamma < 4\lambda$,
$\Gamma(t)$ oscillates and becomes \emph{negative}
during the backflow intervals where $|p(t)|$ increases.
These are precisely the intervals where
$\mathcal{S} > 0$.

\textbf{Closed-form revival amplitude.}
The decoherence function~\eqref{eq:decoherence} can be
written as $p(t) = R\, e^{-\gamma t/2}\cos(\Omega t - \phi)$,
where $R = \sqrt{1 + (\gamma/2\Omega)^2}$ and
$\phi = \arctan(\gamma/2\Omega)$, with $R\cos\phi = 1$.
The extrema of $|p(t)|$ occur at $t_n = n\pi/\Omega$
($n = 0, 1, 2, \ldots$), and the first revival peak
after the first zero is at $t_1 = \pi/\Omega$.
Its amplitude is \emph{exactly}
\begin{equation}
\label{eq:revival_exact}
|p(t_1)| = e^{-\gamma\pi/(2\Omega)},
\qquad
\beta\,\mathcal{S}[t^*, t_1]
\approx |p(t_1)|^2
= e^{-\gamma\pi/\Omega}.
\end{equation}
This is the paper's central computable prediction: the
survival gain at first backflow is determined by a single
dimensionless ratio $\gamma/\Omega$.

\begin{remark}[Parameter Survey]
\label{rem:parameter_survey}
Table~\ref{tab:survey} demonstrates the transition from
the Markovian regime ($\mathcal{S} \leq 0$) to the
non-Markovian regime ($\mathcal{S} > 0$) as the bath
memory rate $\gamma$ decreases below the critical value
$4\lambda$.
All entries use $\omega_0 = 1$, $\lambda = 1$,
$W = 0$ (autonomous evolution), with revival amplitudes
computed from~\eqref{eq:revival_exact}.
\begin{center}
\begin{tabular}{@{}cccccc@{}}
\toprule
$\gamma$ & $\gamma/4\lambda$ & Regime
& $|p(t_1)|$ & $\beta\,\mathcal{S}(t_1)$
& $\Gamma_{\min}$ \\
\midrule
$20.0$ & $5.0$ & Markov & --- & $\leq 0$ & $>0$ \\
$4.0$ & $1.0$ & Critical & --- & $\leq 0$ & $= 0$ \\
$2.0$ & $0.50$ & Non-Markov & $0.043$
  & $+0.002$ & $< 0$ \\
$1.0$ & $0.25$ & Non-Markov & $0.163$
  & $+0.027$ & $< 0$ \\
$0.5$ & $0.125$ & Deep NM & $0.305$
  & $+0.093$ & $< 0$ \\
$0.1$ & $0.025$ & Deep NM & $0.605$
  & $+0.37\phantom{0}$ & $< 0$ \\
\bottomrule
\end{tabular}
\end{center}
\captionsetup{hypcap=false}%
\captionof{table}{%
Survival functional at first backflow revival
as a function of the bath memory rate $\gamma$,
for the pure-dephasing spin-boson model
with Lorentz--Drude spectral density.
$|p(t_1)|$ is computed
from~\eqref{eq:revival_exact};
$\Gamma_{\min}$ is the sign of the minimum of the
instantaneous decoherence rate~\eqref{eq:Gamma_inst}.
The transition $\mathcal{S} \leq 0 \to \mathcal{S} > 0$
occurs precisely at the non-Markovian threshold
$\gamma = 4\lambda$.
For $\gamma = 0.1$ (deep non-Markovian),
the agent achieves $\beta\,\mathcal{S} \approx +0.37$
per backflow cycle in the autonomous setting
($H_{\mathrm{ctrl}} = 0$).%
}
\label{tab:survey}
\end{remark}

\begin{remark}[The Two Regimes: Summary]
\label{rem:two_regimes}
\begin{center}
\begin{tabular}{@{}lll@{}}
\toprule
& \textbf{Markovian ($\gamma = 20$)}
& \textbf{Non-Markovian ($\gamma = 0.5$)} \\
\midrule
$\gamma / 4\lambda$ & $5.0$ (overdamped)
  & $0.125$ (underdamped) \\
$\tau_B$ & $0.05\,\omega_0^{-1}$
  & $2.0\,\omega_0^{-1}$ \\
$p(t)$ & Monotone decay
  & Oscillatory with envelope \\
$|p(t_1)|$ at first revival & $0$ (no revival)
  & $\approx 0.31$ \\
$\beta\,\mathcal{S}$ at revival & $\leq 0$
  & $\approx +0.093$ \\
$\Gamma(t)$ & $> 0$ always
  & Oscillates, $< 0$ during backflow \\
Interpretation & Stone (sinks)
  & Surfer (rides backflow) \\
\bottomrule
\end{tabular}
\end{center}
The non-Markovian agent achieves
$\beta\,\mathcal{S} \approx +0.093$
per backflow cycle (autonomous, $H_{\mathrm{ctrl}} = 0$),
while the Markovian agent can only lose free energy.
As the coupling deepens ($\gamma/4\lambda \to 0$), the
revival amplitude grows and
$\mathcal{S}$ increases
(Table~\ref{tab:survey}), bounded above by
$\beta\,\mathcal{S} \leq I(S{:}E;\, t^*)$
(Corollary~\ref{cor:three_regimes}(iii)).
\end{remark}

% ============================================================
\section{The Cost of Memory}
\label{sec:cost}

We have shown that memory allows an agent to breach the
Markovian ceiling.
However, every advantage carries a thermodynamic shadow.
We now quantify the cost of memory and identify the survival
crisis that sets the stage for Paper~II\@.

\subsection{The Landauer Debt}

To exploit the memory kernel $\mathcal{K}(t,s)$, the physical
substrate of the agent must maintain correlations with its own
past.
This is equivalent to storing information.
By Landauer's principle, erasing or overwriting this information
dissipates heat; if the agent does not erase, it must pay an
entropic cost to store.

\begin{proposition}[Landauer Cost of Memory]
\label{prop:cost}
Let $\mathcal{I}_{\mathrm{stored}}(\tau_{\mathrm{mem}})$ be the
mutual information between the agent's state trajectory over
$[t - \tau_{\mathrm{mem}},\, t]$ and its current control
protocol $H_{\mathrm{ctrl}}(t)$.
The free-energy cost of maintaining this memory satisfies
\begin{equation}
\label{eq:landauer}
\Delta F_{\mathrm{mem}}
\geq k_B T \ln 2 \cdot
\mathcal{I}_{\mathrm{stored}}(\tau_{\mathrm{mem}}).
\end{equation}
\end{proposition}

\subsection{The Memory Catastrophe}

The crisis arises from the scaling of
$\mathcal{I}_{\mathrm{stored}}$ with time.
To quantify this, we borrow two quantities from computational
mechanics~\cite{CrutchfieldYoung1989,ShaliziCrutchfield2001}:

\begin{definition}[Entropy Rate and Predictive Information]
\label{def:entropy_rate}
Let $\{X_t\}$ be the stochastic process describing the
environment's influence on the agent (e.g., the sequence of
bath correlation values).
\begin{enumerate}
\item The \textbf{entropy rate} of the environment is
  \begin{equation}
  \label{eq:entropy_rate}
  h_\mu := \lim_{n \to \infty}
  H(X_n \mid X_{n-1}, \ldots, X_1),
  \end{equation}
  measuring the intrinsic unpredictability per time step.
\item The \textbf{predictive information} (excess entropy) is
  \begin{equation}
  \label{eq:pred_info}
  I_{\mathrm{pred}} := I(\overleftarrow{X};\,
  \overrightarrow{X})
  = \sum_{k=1}^{\infty}
  \bigl[H(X_k) - h_\mu\bigr],
  \end{equation}
  where $\overleftarrow{X}$ and $\overrightarrow{X}$ denote
  the past and future half-chains.
  This is the total amount of information about the future
  that is encoded in the past---the \emph{useful} memory.
\end{enumerate}
\end{definition}

For an environment with finite predictive information
($I_{\mathrm{pred}} < \infty$), an optimal agent needs only
finite memory to capture all exploitable correlations.
However, for environments with \emph{divergent} predictive
information (e.g., processes with long-range temporal
correlations, $1/f$ noise, or non-stationary statistics),
the required memory grows without bound.

\begin{proposition}[The Memory Catastrophe]
\label{thm:catastrophe}
\textbf{Assumptions.}
Let the environment be a \emph{stationary, mixing} stochastic
process with positive entropy rate $h_\mu > 0$
\cite{CrutchfieldYoung1989,ShaliziCrutchfield2001}.
Consider an agent that maintains a memory kernel
$\mathcal{K}(t,s)$ with support on $[t - \tau_{\mathrm{mem}},\, t]$.
Let $\dot{W}_{\mathrm{budget}}$ be the agent's available
free-energy flux (constant).
\begin{enumerate}
\item The minimum memory required to exploit correlations up
  to depth $\tau_{\mathrm{mem}}$ satisfies
  \begin{equation}
  \label{eq:mem_lower_bound}
  \mathcal{I}_{\mathrm{stored}}(\tau_{\mathrm{mem}})
  \geq \min\!\big(I_{\mathrm{pred}},\;
  h_\mu\, \tau_{\mathrm{mem}}\big).
  \end{equation}
\item The Landauer cost of maintaining this memory is
  \begin{equation}
  \label{eq:mem_cost_rate}
  \dot{W}_{\mathrm{mem}} \geq
  k_B T \ln 2 \cdot h_\mu,
  \end{equation}
  since the agent must erase (or overwrite) at least $h_\mu$
  bits per unit time to prevent memory overflow.
\item There exists a critical time
  $t_{\mathrm{crit}}$ beyond which the memory maintenance cost
  exceeds the survival gain:
  \begin{equation}
  \label{eq:catastrophe}
  t > t_{\mathrm{crit}}
  \quad\Longrightarrow\quad
  \dot{W}_{\mathrm{mem}}(t) > \dot{W}_{\mathrm{budget}},
  \end{equation}
  unless the agent compresses its memory.
\end{enumerate}
The agent dies not from entropy (disorder) but from
\emph{hypermnesia}: the thermodynamic cost of perfect memory
exceeds the benefit it provides.
\end{proposition}

\begin{proof}
Part~(1): an agent exploiting temporal correlations to depth
$\tau_{\mathrm{mem}}$ must store at least the mutual information
between the past $\tau_{\mathrm{mem}}$ time steps and the
present.
For a stationary ergodic process, this mutual information is
bounded below by $\min(I_{\mathrm{pred}},\,
h_\mu\, \tau_{\mathrm{mem}})$~\cite{CrutchfieldYoung1989,
BialekNemenmanTishby2001}.

Part~(2): each time step, the agent receives $\sim h_\mu$
bits of genuinely new information.
To maintain a fixed-capacity memory, it must erase at least
this many bits, incurring Landauer cost
$k_B T \ln 2 \cdot h_\mu$ per time step.

Part~(3): if $I_{\mathrm{pred}} = \infty$ (as for environments
with long-range correlations), the stored information grows
as $\mathcal{I}_{\mathrm{stored}} \sim h_\mu\,
\tau_{\mathrm{mem}}$.
Combined with part~(2), the memory cost grows linearly in the
effective memory depth.
For any finite budget $\dot{W}_{\mathrm{budget}}$, there exists
$t_{\mathrm{crit}}$ such that the cost exceeds the budget.
\end{proof}

\subsection{Resolution: The Necessity of Forgetting}

To survive beyond $t_{\mathrm{crit}}$, the agent must introduce
a \emph{lossy compression} scheme: it must discard the vast
majority of stored correlations and retain only the
thermodynamically salient features.

\begin{itemize}
\item \textbf{Compression requires a criterion.}
  To decide what to keep and what to erase, the agent needs a
  \emph{relevance function}---a mapping from stored correlations
  to survival value.
  This is a reference frame that ranks information by its
  contribution to $\mathcal{S}$.

\item \textbf{A reference frame requires symmetry breaking.}
  An ``unbiased'' agent that treats all correlations as equally
  valuable cannot compress: it must keep everything.
  The act of preferring one subset of information over another
  is a spontaneous breaking of the informational symmetry.
  This is the thermodynamic definition of a ``perspective''---or,
  more precisely, a \emph{privileged basis}.
\end{itemize}

\begin{remark}[The Origin of Paper II]
\label{rem:paper_II}
Proposition~\ref{thm:catastrophe} reveals the \emph{poison} embedded
in Paper~I's medicine.
Memory enables survival beyond the Markovian ceiling, but
unbounded memory under finite energy resources leads to
\emph{computational explosion}: the agent must process an
ever-growing archive with bounded free energy.

This is the precise thermodynamic origin of the crisis
addressed in Paper~II\@.
The resolution---spontaneous symmetry breaking of the
agent's reference frame---is not an additional hypothesis
but a \emph{thermodynamic necessity}: the agent must
compress its infinite history into a finite, biased
representation.
The ``self'' (a privileged computational basis) emerges as
the minimal structure that makes memory computationally
tractable.

In the structural parallel noted in HAFF
Essay~C~\cite{Liu2026HAFF_C}: the accumulation mechanism of
Paper~I provides the raw material for survival, but without the
discriminative compression of Paper~II, the system collapses
under the weight of its own stored correlations.
\end{remark}

% ============================================================
\section{Numerical Demonstration}
\label{sec:numerical}

The preceding sections establish analytic bounds and a
worked example in the spin-boson model.  We now provide a numerical illustration showing that the
Markovian ceiling signature predicted by
Theorem~\ref{thm:ceiling} and the memory advantage of
Theorem~\ref{thm:advantage} are reproduced in a minimal
partially observed environment.
Full code and parameters are provided for reproducibility.

% ------------------------------------------------------------
\subsection{Model}
\label{subsec:demo_model}

\paragraph{Environment.}
A two-hidden-state HMM with aliased observations.
The hidden state $s_t \in \{0,1\}$ evolves as a persistent
Markov chain with $\Pr(s_{t+1} = s_t) = 1 - \varepsilon$;
the parameter $\varepsilon \in [10^{-3}, 10^{-1}]$ controls
the correlation length $\ell \sim 1/\varepsilon$.
Observations $o_t \in \{A, B\}$ are aliased:
$\Pr(o_t = A \mid s_t = 0) = 0.5 + \delta$,
$\Pr(o_t = A \mid s_t = 1) = 0.5 - \delta$,
with $\delta = 0.05$ (mutual information
$I(O; S) \approx 0.007$~bits).
Reward: $r_t = 1$ if $a_t = s_t$, $0$ otherwise.

\paragraph{Agents.}
All agents use the true model parameters and compute
exact Bayesian posteriors; the only difference is how many
observations each agent retains.
\begin{itemize}
\item \textbf{Markov-$k$} ($k \in \{1,2,4,8\}$): runs
  an exact Bayes filter over the most recent $k$
  observations (sliding window, uniform prior at each
  window start); acts by MAP.
\item \textbf{Memory (Bayes filter)}: maintains the full
  belief state $b_t = \Pr(s_t = 1 \mid o_{1:t})$ via the
  exact predict--update cycle over all past observations;
  acts by MAP.
\end{itemize}

\paragraph{Parameters.}
\begin{center}
\small
\begin{tabular}{@{}lcl@{}}
\toprule
\textbf{Quantity} & \textbf{Value} & \textbf{Role} \\
\midrule
$T$ & $100{,}000$ & horizon per trial \\
Seeds & 10 & independent replications \\
$\delta$ & 0.05 & observation asymmetry \\
$k$ & $\{1, 2, 4, 8\}$ & Markov window sizes \\
$\varepsilon$ & logspace($10^{-3}$, $10^{-1}$, 15)
  & transition noise grid \\
Burn-in & $5{,}000$ & discarded steps \\
\bottomrule
\end{tabular}
\end{center}

% ------------------------------------------------------------
\subsection{Results}
\label{subsec:demo_results}

Figure~\ref{fig:markov_ceiling} shows the two key signatures.

\paragraph{Result 1: Markov ceiling
(Figure~\ref{fig:markov_ceiling}a).}
The average reward $\bar{R}$ of the Bayes filter (memory
agent) increases monotonically with correlation length
$\ell = 1/\varepsilon$, while each Markov-$k$ agent
saturates at a distinct ceiling.  The ceilings are
ordered: $k = 1$ (lowest) through $k = 8$ (highest),
and all fall below the memory agent for
$\ell \gtrsim 20$.  This is consistent with the qualitative prediction of
Theorem~\ref{thm:ceiling}: finite-order Markov
representations have a performance upper bound that
the memory-carrying agent surpasses.

\paragraph{Result 2: Memory advantage
(Figure~\ref{fig:markov_ceiling}b).}
The gap $\Delta\bar{R} = \bar{R}_{\mathrm{mem}}
- \bar{R}_{\mathrm{Markov}\text{-}k}$ increases
monotonically with $\ell$, and is larger for smaller $k$.
Shaded bands show 95\% confidence intervals across
10 seeds.  The Markov-1 and Markov-2 curves nearly
overlap at small $\ell$, reflecting the fact that
short observation windows provide negligible additional
information in this aliasing regime---a consistency
check, not a deficiency.

\begin{figure}[t]
\centering
\includegraphics[width=\textwidth]%
  {fig_paper1_markov_ceiling.pdf}
\caption{%
\textbf{Markov ceiling and memory advantage.}
$T = 100{,}000$, 10 seeds, 95\% CI bands.
\textbf{(a)}~Average reward $\bar{R}$ vs correlation
length $\ell = 1/\varepsilon$.  The Bayes filter
(blue, bold) rises monotonically; Markov-$k$ agents
saturate at $k$-dependent ceilings.
\textbf{(b)}~Performance gap
$\Delta\bar{R} = \bar{R}_{\mathrm{mem}}
- \bar{R}_{\mathrm{Markov}\text{-}k}$
increases with $\ell$; smaller $k$ yields a larger gap.}
\label{fig:markov_ceiling}
\end{figure}

% ------------------------------------------------------------
\subsection{Scope of This Demonstration}
\label{subsec:demo_scope}

These simulations illustrate the ceiling phenomenon
predicted by Theorem~\ref{thm:ceiling} under the stated
model class; they do not constitute a proof beyond this
class.

\medskip
This demonstration \textbf{does} show:
\begin{enumerate}
\item A reproducible regime in which finite-order Markov
  agents exhibit a performance ceiling while a
  memory-carrying (Bayes filter) agent improves---the
  Markov ceiling signature predicted by
  Theorem~\ref{thm:ceiling}.
\item The memory advantage (Theorem~\ref{thm:advantage})
  manifests as a monotonically growing gap that widens
  with correlation length and tightens with window size.
\end{enumerate}

\noindent
This demonstration does \textbf{not} show:
\begin{enumerate}
\item Universality across environments, observation models,
  or agent architectures.  The model uses a two-state
  HMM with binary aliased observations.
\item Tight constants or the functional form of the
  ceiling boundary $\ell_c(k)$.
\item That the Bayes filter is optimal among all possible
  memory-carrying agents.
\end{enumerate}

\paragraph{Reproducibility.}
\begin{sloppypar}
The complete simulation is a self-contained Python script
(\texttt{paper1\_markov\_ceiling\_demo.py}, ${\sim}\,560$
lines, requiring only NumPy and Matplotlib) with fixed
random seeds.  All figures in this section can be
reproduced by executing the script.  The following files
are included in the supplementary archive:
\begin{itemize}
\item \texttt{paper1\_markov\_ceiling\_demo.py} --- simulation script
\item \texttt{fig\_paper1\_markov\_ceiling.pdf} --- Figure~\ref{fig:markov_ceiling}
\item \texttt{markov\_ceiling\_data.csv} --- raw sweep data
\item \texttt{markov\_ceiling\_boundary.csv} --- extracted
  ceiling boundaries $\ell_c(k)$
\end{itemize}
\end{sloppypar}

% ============================================================
\section{Discussion}
\label{sec:discussion}

\subsection{Summary of Results}

\begin{center}
\small
\setlength{\tabcolsep}{4pt}%
\begin{tabular}{@{}lP{7.5cm}c@{}}
\toprule
\textbf{Result} & \textbf{Statement} & \textbf{Sec.} \\
\midrule
Markovian Ceiling &
  $\mathcal{S} \leq 0$ for open-loop GKSL (no feedback)
  & \ref{sec:ceiling} \\[3pt]
Memory Advantage &
  $\beta\mathcal{S} = -\Delta I
  - \Delta D_{\mathrm{KL}}
  - \beta\Delta\langle H_{\mathrm{ctrl}}\rangle$;\;
  $\mathcal{S} > 0$ when correlations consumed
  (any initial state)
  & \ref{sec:advantage} \\[3pt]
Quantitative demo &
  Spin-boson: $\beta\mathcal{S} \approx +0.093 > 0$
  at first backflow revival (Fig.~\ref{fig:survival},
  Table~\ref{tab:survey})
  & \ref{sec:example} \\[3pt]
Temporal Arrow &
  $\prec_K \to \prec_{\mathrm{HAFF}}$ via quantum
  Darwinism
  & \ref{sec:arrow} \\[3pt]
Memory Catastrophe &
  $\dot{W}_{\mathrm{mem}} \geq k_BT\ln 2 \cdot h_\mu$;\;
  exceeds budget at $t_{\mathrm{crit}}$
  & \ref{sec:cost} \\[3pt]
Numerical demo &
  Markov ceiling reproduced in HMM
  (Fig.~\ref{fig:markov_ceiling})
  & \ref{sec:numerical} \\
\bottomrule
\end{tabular}
\end{center}

\subsection{What This Paper Does and Does Not Show}

\textbf{This paper shows:}
\begin{enumerate}
\item Under open-loop GKSL dynamics (no measurement or feedback),
  the survival functional $\mathcal{S} \leq 0$
  (Theorem~\ref{thm:ceiling}).
\item For any initial state (product or correlated), the
  survival functional satisfies the exact identity
  $\beta\,\mathcal{S} = -\Delta I(S{:}E)
  - \Delta D_{\mathrm{KL}}(\rho_E \| \rho_E^{\mathrm{th}})
  - \beta\,\Delta\langle H_{\mathrm{ctrl}}\rangle$
  (Theorem~\ref{thm:advantage}).
  Under autonomous evolution, when pre-existing
  system--environment correlations are consumed
  ($\Delta I < 0$), $\mathcal{S} > 0$ is achievable,
  bounded by the initial correlation budget
  (Corollary~\ref{cor:three_regimes}).
\item A quantitative spin-boson example illustrates:
  $\beta\,\mathcal{S} \approx +0.093 > 0$ at the first
  non-Markovian revival (Section~\ref{sec:example}).
\item The causal memory order $\prec_K$ is consistent with the
  HAFF accessibility order when restricted to the classical
  (pointer-state) sector
  (Proposition~\ref{prop:HAFF_bridge}).
\item The thermodynamic cost of memory, quantified by the
  environment's entropy rate $h_\mu$, creates a survival
  crisis for agents with finite energy budgets
  (Proposition~\ref{thm:catastrophe}).
\item A minimal computational demonstration reproduces the
  Markov ceiling and memory advantage signatures in a
  two-state HMM with aliased observations
  (Section~\ref{sec:numerical},
  Figure~\ref{fig:markov_ceiling}).
\end{enumerate}

\textbf{This paper does not show:}
\begin{enumerate}
\item That non-Markovian dynamics is \emph{sufficient} for
  persistence (it is necessary but not sufficient;
  Paper~II addresses the additional requirements).
\item That \emph{all} non-Markovian systems outperform all
  Markovian systems (the comparison is between suprema under
  specified constraints).
\item That Markovian agents with explicit measurement-feedback
  are bounded by the ceiling (the Sagawa--Ueda framework shows
  they are not; Remark~\ref{rem:open_loop}).
\item That the specific form of the optimal memory kernel can be
  derived from first principles without specifying the
  environment.
\item That memory implies or requires consciousness.
\end{enumerate}

% ============================================================
% REFERENCES
% ============================================================
\begin{thebibliography}{99}

\bibitem{BreuerPetruccione2002}
H.-P.~Breuer and F.~Petruccione,
\emph{The Theory of Open Quantum Systems},
Oxford University Press (2002).

\bibitem{Schrodinger1944}
E.~Schr\"odinger,
\emph{What is Life?},
Cambridge University Press (1944).

\bibitem{Lindblad1976}
G.~Lindblad,
\emph{On the generators of quantum dynamical semigroups},
Commun.\ Math.\ Phys.\ \textbf{48}, 119 (1976).

\bibitem{GKS1976}
V.~Gorini, A.~Kossakowski, and E.~C.~G.~Sudarshan,
\emph{Completely positive dynamical semigroups of $N$-level systems},
J.\ Math.\ Phys.\ \textbf{17}, 821 (1976).

\bibitem{Nakajima1958}
S.~Nakajima,
\emph{On quantum theory of transport phenomena},
Prog.\ Theor.\ Phys.\ \textbf{20}, 948 (1958).

\bibitem{Zwanzig1960}
R.~Zwanzig,
\emph{Ensemble method in the theory of irreversibility},
J.\ Chem.\ Phys.\ \textbf{33}, 1338 (1960).

\bibitem{SagawaUeda2010}
T.~Sagawa and M.~Ueda,
\emph{Generalized Jarzynski Equality under Nonequilibrium Feedback Control},
Phys.\ Rev.\ Lett.\ \textbf{104}, 090602 (2010).

\bibitem{SagawaUeda2012}
T.~Sagawa and M.~Ueda,
\emph{Fluctuation Theorem with Information Exchange},
Phys.\ Rev.\ Lett.\ \textbf{109}, 180602 (2012).

\bibitem{RivasHuelgaPlenio2014}
\'{A}.~Rivas, S.~F.~Huelga, and M.~B.~Plenio,
\emph{Quantum non-Markovianity: characterization, quantification and detection},
Rep.\ Prog.\ Phys.\ \textbf{77}, 094001 (2014).

\bibitem{Spohn1978}
H.~Spohn,
\emph{Entropy production for quantum dynamical semigroups},
J.\ Math.\ Phys.\ \textbf{19}, 1227 (1978).

\bibitem{BreuerLainePiilo2009}
H.-P.~Breuer, E.-M.~Laine, and J.~Piilo,
\emph{Measure for the Degree of Non-Markovian Behavior of Quantum Processes in Open Systems},
Phys.\ Rev.\ Lett.\ \textbf{103}, 210401 (2009).

\bibitem{EspositoLindenbergVandenBroeck2010}
M.~Esposito, K.~Lindenberg, and C.~Van~den~Broeck,
\emph{Entropy production as correlation between system and reservoir},
New J.\ Phys.\ \textbf{12}, 013013 (2010).

\bibitem{Zurek2009}
W.~H.~Zurek,
\emph{Quantum Darwinism},
Nature Physics \textbf{5}, 181 (2009).

\bibitem{CrutchfieldYoung1989}
J.~P.~Crutchfield and K.~Young,
\emph{Inferring statistical complexity},
Phys.\ Rev.\ Lett.\ \textbf{63}, 105 (1989).

\bibitem{ShaliziCrutchfield2001}
C.~R.~Shalizi and J.~P.~Crutchfield,
\emph{Computational mechanics: Pattern and prediction, structure and simplicity},
J.\ Stat.\ Phys.\ \textbf{104}, 817 (2001).

\bibitem{BialekNemenmanTishby2001}
W.~Bialek, I.~Nemenman, and N.~Tishby,
\emph{Predictability, complexity, and learning},
Neural Computation \textbf{13}, 2409 (2001).

\bibitem{Liu2026HAFF_A}
S.~Liu,
\emph{Emergent Geometry from Coarse-Grained Observable Algebras},
Zenodo (2026), DOI: 10.5281/zenodo.18361707.

\bibitem{Liu2026HAFF_B}
S.~Liu,
\emph{Accessibility, Stability, and Emergent Geometry},
Zenodo (2026), DOI: 10.5281/zenodo.18367061.

\bibitem{Liu2026HAFF_C}
S.~Liu,
\emph{Causation, Agency, and Existence},
Zenodo (2026), DOI: 10.5281/zenodo.18391651.

\bibitem{Liu2026HAFF_F}
S.~Liu,
\emph{Temporal Asymmetry as Accessibility Propagation},
Zenodo (2026), DOI: 10.5281/zenodo.18417099.

\bibitem{Liu2026QRAIF_A}
S.~Liu,
\emph{Algebraic Constraints on the Emergence of Lorentzian Metrics in Entropic Gravity Frameworks},
Zenodo (2026), DOI: 10.5281/zenodo.18525877.

\bibitem{Liu2026QRAIF_B}
S.~Liu,
\emph{Thermodynamic Stability Constraints on the Operator Algebra of Persistent Open Quantum Subsystems},
Zenodo (2026), DOI: 10.5281/zenodo.18525891.

\end{thebibliography}

\end{document}
