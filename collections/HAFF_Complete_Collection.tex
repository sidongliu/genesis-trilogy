% ============================================================================
% HAFF: Holographic Alaya-Field Framework
% Complete Collected Volume
% ============================================================================

\documentclass[12pt,a4paper,openany]{book}

\usepackage[utf8]{inputenc}
\usepackage[T1]{fontenc}
\usepackage{amsmath,amssymb,amsfonts,amsthm}
\usepackage{physics}
\usepackage{graphicx}
\usepackage{hyperref}
\usepackage{tikz}
\usetikzlibrary{shapes,arrows}
% \usepackage{bbm} % Not available
\usepackage{geometry}
\usepackage{fancyhdr}
\usepackage{titlesec}
\usepackage{array}
\newcolumntype{P}[1]{>{\raggedright\arraybackslash}p{#1}}
\usepackage{booktabs}
\usepackage{tcolorbox}
\usepackage{xcolor}

\geometry{a4paper, margin=1in, headheight=14pt}

% Theorem environments
\newtheorem{theorem}{Theorem}[chapter]
\newtheorem{proposition}[theorem]{Proposition}
\newtheorem{lemma}[theorem]{Lemma}
\newtheorem{definition}[theorem]{Definition}
\newtheorem{remark}[theorem]{Remark}
\newtheorem{conjecture}[theorem]{Conjecture}
\newtheorem{constraint}[theorem]{Constraint}
\newtheorem{example}[theorem]{Example}

% Headers
\pagestyle{fancy}
\fancyhf{}
\fancyhead[LE,RO]{\thepage}
\fancyhead[RE]{\textit{Holographic Alaya-Field Framework}}
\fancyhead[LO]{\textit{\leftmark}}
\renewcommand{\headrulewidth}{0.4pt}

\title{
  \vspace{-2cm}
  {\Huge\textbf{The Holographic Alaya-Field Framework}}\\[1cm]
  {\Large\textit{A Structural Investigation into Emergent Geometry,\\
  Measurement, and Temporality}}\\[2cm]
  {\large Complete Collected Volume}\\[0.5cm]
  {\normalsize Papers A--G, Essay C, and Postscript}
}

\author{
  \textbf{Sidong Liu, PhD}\\[0.5em]
  iBioStratix Ltd\\[0.3em]
  \texttt{sidongliu@hotmail.com}
}

\date{February 2026}

\begin{document}
\emergencystretch=2em
\raggedbottom
\hbadness=5000
\vbadness=5000

\frontmatter
\maketitle

% DOI page
\newpage
\thispagestyle{empty}
\vspace*{\fill}
\begin{center}
\textbf{Publication Record}\\[2em]
\begin{tabular}{ll}
Paper A & DOI: 10.5281/zenodo.18361706 \\
Paper B & DOI: 10.5281/zenodo.18367060 \\
Essay C & DOI: 10.5281/zenodo.18374805 \\
Paper D & DOI: 10.5281/zenodo.18388881 \\
Paper E & DOI: 10.5281/zenodo.18400065 \\
Paper F & DOI: 10.5281/zenodo.18400425 \\
Paper G & DOI: 10.5281/zenodo.18402907 \\
Postscript & DOI: 10.5281/zenodo.18407367 \\
\end{tabular}
\\[3em]
\textit{This collected volume compiles previously published works\\
for archival and reference purposes.}
\\[2em]
\copyright{} 2026 Sidong Liu. All rights reserved.
\end{center}
\vspace*{\fill}

% Abstract
\chapter*{Abstract}
\addcontentsline{toc}{chapter}{Abstract}

This volume develops a structural framework for understanding the emergence of spacetime, locality, and classicality within quantum theory, without assuming any preferred subsystem decomposition or fundamental spacetime background.

The central claim is that many features traditionally treated as primitive---such as spatial separation, temporal ordering, and observer--system distinctions---are instead consequences of how a single global quantum structure is coarse-grained into physically accessible subalgebras.

The framework does not propose new fundamental laws or cosmological dynamics. Instead, it reorganizes explanatory priority, clarifying the conditions under which physical descriptions are possible at all.

\bigskip
\noindent\textbf{Keywords}: emergent geometry, accessible algebras, coarse-graining, quantum foundations, algebraic quantum theory, structural realism

\tableofcontents

\mainmatter


% ============================================================================
% Paper A
% ============================================================================
\chapter{Emergent Geometry from Coarse-Grained Observable Algebras}
\label{chap:paperA}

\begin{center}
\textit{Paper A}\\[0.5em]
Originally published: Zenodo, DOI: 10.5281/zenodo.18361706
\end{center}

\bigskip

\noindent\textbf{Preliminary Remark (Structural Stance).}
In much of modern theoretical physics, it is tacitly assumed that the decomposition of a system into subsystems is either physically given or at least unproblematic. 
Hilbert spaces are factorized, degrees of freedom are labeled, and geometry is inferred from relations between these parts.

In this work, we adopt a different structural stance. 
We treat the universal quantum state as given, but regard subsystem structure, locality, and geometry as secondary constructs arising from restrictions on observable algebras. 
The guiding question is not how geometry emerges from quantum states, but how different effective realities can emerge from the \emph{same} state once no preferred factorization is assumed.

This shift is modest in formalism but radical in implication: it relocates the origin of structure from states to algebras, and from kinematics to accessibility.

\medskip

\section*{Abstract}

We construct a theoretical framework where the tensor factorization of a Hilbert space is treated as a dynamical variable rather than a kinematic background. 
By lifting the ``Alaya'' concept to a globally entangled vacuum state $|\Psi_{\text{vac}}\rangle$, we demonstrate that local geometry emerges from specific observable subalgebras $\mathcal{A}_i \subset \mathcal{B}(\mathcal{H})$. 
We prove that non-commuting coarse-graining maps induce topologically distinct emergent spacetimes from the same underlying state.
The analysis is structural in nature: we do not propose new dynamics, but examine consistency and consequences of removing subsystem factorization from fundamental assumptions. 
This approach is complementary to existing interpretational frameworks and suggests natural connections to algebraic quantum field theory, entanglement-based approaches to spacetime, and quantum information theory.


\section{Introduction}

\subsection{Motivation}

Two central problems in contemporary theoretical physics concern the emergence of classicality and the emergence of geometry. 
In quantum foundations, the measurement problem asks how effectively classical behavior arises from an underlying quantum description. 
Decoherence theory has provided a powerful account of this process by explaining the suppression of interference between certain degrees of freedom through environmental entanglement \cite{Zurek2003}. 
However, this explanation typically presupposes a fixed decomposition of the total system into subsystems, distinguishing ``system,'' ``apparatus,'' and ``environment'' from the outset.

A closely related emergence problem appears in quantum gravity. 
A growing body of work suggests that spacetime geometry is not fundamental, but arises from patterns of quantum entanglement \cite{Maldacena1999,VanRaamsdonk2010}. 
In holographic settings, geometric quantities are related to entanglement measures via precise correspondences, most notably the Ryu--Takayanagi formula \cite{RyuTakayanagi2006}. 
Yet these constructions likewise assume a prior specification of spatial regions or tensor factors, with geometry inferred only after such a subdivision has been fixed.

In both contexts, the emergence problem is addressed only after a subsystem decomposition has been assumed.
The structure responsible for classicality or geometry is therefore explained relative to a partition whose origin remains largely unexamined.

\subsection{The Structural Gap}

The assumption of a given subsystem structure is often treated as innocuous, or as a matter of convenient description. 
However, from a fundamental perspective, there is no canonical tensor factorization of a generic Hilbert space, nor a unique way to decompose a global quantum state into subsystems.
While this issue is occasionally acknowledged in passing \cite{Zanardi2004,Viola2004}, its consequences for emergence are rarely explored systematically.

The present work does not challenge the empirical success of decoherence theory, entanglement-based approaches to geometry, or the algebraic formulation of quantum theory.
Rather, we make explicit a structural assumption common to these frameworks and investigate the consequences of relaxing it.
Our focus is on what follows if subsystem structure itself is treated as emergent, rather than fundamental.

\subsection{Our Contribution}

We formulate a framework in which subsystem structure is not assumed \emph{a priori}, but arises from a choice of coarse-graining over observable algebras.
Within this framework, we show that different coarse-grainings of the same global quantum state generically induce inequivalent effective geometries.
We further clarify why this inequivalence cannot be reduced to a coordinate transformation, but reflects a genuine multiplicity of effective structures at the emergent level.

The analysis is structural in nature.
Our aim is not to propose a new dynamical mechanism, but to examine the consistency and consequences of removing subsystem factorization from the set of fundamental assumptions.

\subsection{Terminology: The Alaya-Field}

Throughout this work, we adopt the term \emph{Alaya-Field} to denote the fundamental, non-factorized structure prior to any subsystem decomposition.
This terminology, borrowed from Yog\=ac\=ara philosophy (referring to the ``storehouse consciousness''), is used here strictly in a technical sense.

In what follows, the Alaya-Field does not refer merely to a vector in Hilbert space, but to the triple
\[
(\mathcal{H}_U, \mathcal{A}_U, |\Omega\rangle),
\]
where $\mathcal{A}_U$ is a von Neumann algebra acting on $\mathcal{H}_U$ and $|\Omega\rangle$ is a cyclic and separating vector.
The emphasis is on the absence of a canonical factorization, not on the particular choice of state.

Mathematically, this corresponds to the cyclic vector of a Type III$_1$ von Neumann algebra in algebraic quantum field theory \cite{Haag1996}, representing a holistically entangled substrate containing the localized ``seeds'' (eigenmodes) of all possible emergent geometries.
This usage is intended to evoke the non-factorized, pre-geometric nature of the fundamental quantum structure, without importing any metaphysical commitments.

\subsection{Structure of the Paper}

The paper is organized as follows.
In Section~\ref{sec:A-nopref}, we review the absence of a canonical subsystem decomposition in quantum theory and formalize this observation.
Section~\ref{sec:A-coarse} introduces coarse-graining in terms of observable algebras and analyzes how effective subsystem descriptions arise from this procedure.
In Section~\ref{sec:A-geometry}, we show how entanglement relations between these induced subsystems give rise to an effective notion of connectivity and geometry, and demonstrate the dependence of this geometry on the chosen coarse-graining.
Section~\ref{sec:A-discussion} discusses the scope and conceptual implications of the framework, its relation to existing approaches, and possible directions for future work.
We conclude in Section~\ref{sec:A-conclusion} with a summary of results and open questions.

\section{Absence of Canonical Tensor Factorization}
\label{sec:A-nopref}

\subsection{The Factorization Problem}

In standard quantum mechanics, the state space of a composite system is constructed as a tensor product of subsystem Hilbert spaces. 
This construction presupposes that a natural decomposition into subsystems has already been identified. 
However, for a given Hilbert space $\mathcal{H}_{\text{total}}$, there is no unique or canonical way to express it as a tensor product $\mathcal{H}_A \otimes \mathcal{H}_B$ without additional physical input.

\begin{theorem}[Absence of Canonical Tensor Factorization]
\label{thm:A-nofact}
Let $\mathcal{H}_{\text{total}}$ be a finite or separable infinite-dimensional Hilbert space, and let
\[
|\Psi_U\rangle \in \mathcal{H}_{\text{total}}
\]
be a pure state.
Assume that for every nontrivial tensor factorization
\[
\mathcal{H}_{\text{total}} = \mathcal{H}_A \otimes \mathcal{H}_B
\]
the reduced state $\rho_A = \mathrm{Tr}_B(|\Psi_U\rangle\langle\Psi_U|)$ has nonzero von Neumann entropy.
Then there exists \textbf{no unique or canonically preferred tensor factorization} of $\mathcal{H}_{\text{total}}$ into subsystems relative to which $|\Psi_U\rangle$ is separable or weakly entangled.
In particular, any two such factorizations are related by a global unitary transformation that does not preserve subsystem structure.
\end{theorem}

\begin{proof}
The proof combines two results: the measure-theoretic typicality of high entanglement (Page's theorem) and the algebraic origin of subsystem structure (Zanardi et al.).

\textbf{Step 1: Typicality of near-maximal entanglement.}
Let $\dim\mathcal{H}_A = d_A \leq d_B = \dim\mathcal{H}_B$ with $d_A d_B = d = \dim\mathcal{H}_{\text{total}}$.
For a Haar-random pure state $|\Psi\rangle$, the expected entanglement entropy satisfies~\cite{Page1993}
\begin{equation}
\label{eq:A-page}
\mathbb{E}[S(\rho_A)] = \sum_{k=d_B+1}^{d_A d_B} \frac{1}{k} - \frac{d_A - 1}{2d_B}
\;\geq\; \ln d_A - \frac{d_A}{2d_B}.
\end{equation}
Moreover, the concentration of measure on high-dimensional spheres gives~\cite{HaydenLeungWinter2006}
\begin{equation}
\Pr\!\bigl[S(\rho_A) < \ln d_A - \delta\bigr]
\leq \exp\!\bigl(-c\, d_A d_B\, \delta^2\bigr)
\end{equation}
for a universal constant $c > 0$.
Thus for any fixed factorization, the set of states with $S(\rho_A) < \varepsilon$ has exponentially small measure for $\varepsilon < \ln d_A$.

\textbf{Step 2: Factorization dependence.}
A tensor factorization $\mathcal{H}_{\text{total}} = \mathcal{H}_A \otimes \mathcal{H}_B$ is equivalent to a choice of subalgebra $\mathcal{A} = \mathcal{B}(\mathcal{H}_A) \otimes \mathbf{1}_B \subset \mathcal{B}(\mathcal{H}_{\text{total}})$.
As shown by Zanardi, Lidar, and Lloyd~\cite{Zanardi2004}, any two such factorizations are related by a global unitary $U \in \mathcal{U}(\mathcal{H}_{\text{total}})$ that generically does not preserve subsystem structure: $U(\mathcal{A})U^\dagger \neq \mathcal{A}$.
The group $\mathcal{U}(\mathcal{H}_{\text{total}})$ acts transitively on the set of factorizations, and no state-independent criterion selects a preferred one.

\textbf{Step 3: Conclusion.}
Combining Steps 1 and 2: for a generic state, \emph{every} factorization yields near-maximal entanglement, and different factorizations produce different reduced states related by non-trivial unitaries.
No factorization-independent criterion (separability, low entanglement, locality) can single out a preferred decomposition.
Any subsystem structure must therefore be imposed by additional physical input---in the present framework, by the choice of accessible observable algebra $\mathcal{A}_{\mathbf{c}}$.
\end{proof}

\begin{remark}[Relation to prior work]
The observation that tensor product structures are not canonical was recognized early in the foundations of quantum theory, and has been systematically analyzed by Zanardi and collaborators \cite{Zanardi2001,Zanardi2004} in the context of quantum error correction and noiseless subsystems.
Their work demonstrated that subsystem decompositions are effectively determined by sets of accessible observables rather than being intrinsic to the Hilbert space.
The present framework extends this perspective to the context of holographic geometry and entanglement-based spacetime emergence, where the choice of observable algebra determines not only subsystem structure but also the emergent geometric description.
\end{remark}

\begin{remark}
We do not claim that subsystem decompositions are impossible or unphysical, but that they are not uniquely determined by the universal quantum state alone. 
This observation motivates the search for additional structure that specifies how a subsystem decomposition arises in concrete physical contexts.
\end{remark}

\subsection{Relation to Algebraic Approaches}

The absence of a canonical factorization has been recognized in various contexts, including algebraic quantum field theory where observable algebras take precedence over tensor product structures \cite{Haag1996}.
Our framework builds on these insights by treating coarse-graining structure as the fundamental input from which subsystem decompositions emerge.

\section{Observer-Dependent Coarse-Graining and Effective States}
\label{sec:A-coarse}

\subsection{Coarse-Graining Structure}

Before introducing the formal definition, we emphasize that the introduction of an observable algebra does not reinstate a tensor factorization. 
Operators may act irreducibly on $\mathcal{H}_{\text{total}}$ without inducing any subsystem decomposition. 
In algebraic quantum field theory, observable algebras are defined independently of any global tensor product structure \cite{Haag1996}.
Furthermore, a coarse-graining is not selected by an agent, but instantiated by a concrete physical interaction structure.

\begin{definition}[Operational Coarse-Graining Structure]
\label{def:A-coarse}
Let $\mathcal{H}_U$ be the universal Hilbert space. 
A \textbf{coarse-graining structure} $\mathbf{c}$ is defined as a pair
\[
\boxed{\mathbf{c} \equiv (\mathcal{A}_{\mathbf{c}}, \Phi_{\mathbf{c}})}
\]
where:
\begin{enumerate}
\item $\mathcal{A}_{\mathbf{c}} \subset \mathcal{B}(\mathcal{H}_U)$ is an \textbf{accessible observable algebra}, a $*$-subalgebra that is physically realizable and closed under operationally feasible combinations.
\item $\Phi_{\mathbf{c}}: \mathcal{B}(\mathcal{H}_U) \to \mathcal{B}(\mathcal{H}_{\text{eff}}(\mathbf{c}))$ is a completely positive trace-preserving (CPTP) map implementing an operational reduction of the universal state.
We specifically restrict attention to maps $\Phi$ that preserve the identity and reflect a loss of access to specific degrees of freedom (e.g., restriction to a von Neumann subalgebra, or partial trace over hidden factors).
\end{enumerate}
\end{definition}

\begin{remark}[Non-uniqueness of observable algebras]
It is important to note that the specification of an observable algebra $\mathcal{A}_{\mathbf{c}}$ is not assumed to be unique. 
In algebraic quantum field theory and quantum information theory, there exists no theorem guaranteeing a unique maximal observable algebra associated with a given physical system without additional structure. 
The coexistence of multiple admissible algebras reflects physical underdetermination rather than subjectivity or observer dependence.
\end{remark}

\subsection{Refinement Structure of Coarse-Grainings}

\begin{definition}[Refinement Relation]
\label{def:A-refine}
Given two coarse-graining structures $\mathbf{c}_1, \mathbf{c}_2$, we say $\mathbf{c}_1 \succeq \mathbf{c}_2$ if there exists a CPTP map
\[
\Lambda: \mathcal{B}(\mathcal{H}_{\text{eff}}(\mathbf{c}_1)) \to \mathcal{B}(\mathcal{H}_{\text{eff}}(\mathbf{c}_2))
\]
such that
\[
\Phi_{\mathbf{c}_2} = \Lambda \circ \Phi_{\mathbf{c}_1}
\]
In this case, $\mathbf{c}_2$ is a \textbf{further coarse-graining} of $\mathbf{c}_1$.
\end{definition}

\begin{remark}[Multiplicity of coarse-graining maps]
For a fixed observable algebra $\mathcal{A}_{\mathbf{c}}$, there generally exist multiple completely positive trace-preserving maps implementing distinct coarse-graining procedures. 
The present framework does not require the selection of a preferred CPTP map. 
Rather, different maps correspond to physically realizable information-loss mechanisms, such as tracing over inaccessible degrees of freedom or effective decoherence channels.
\end{remark}

Different coarse-graining choices are related not by unitary symmetry, but by information-theoretic refinement maps, forming a partially ordered structure rather than a group.

\subsection{Core Theorem}

\begin{theorem}[Coarse-Graining Inequivalence]
\label{thm:A-inequiv}
Let $|\Psi_U\rangle \in \mathcal{H}_{\text{total}}$ be a universal quantum state. 
Consider two distinct coarse-graining structures $\mathbf{c}_1 = (\mathcal{A}_1, \Phi_1)$ and $\mathbf{c}_2 = (\mathcal{A}_2, \Phi_2)$.
If $\mathbf{c}_1 \not\sim \mathbf{c}_2$ (i.e., they are not related by unitary equivalence), then:
\begin{enumerate}
\item[(a)] The effective descriptions are unitarily inequivalent: there is no unitary $U \in \mathcal{B}(\mathcal{H}_{\text{total}})$ satisfying $U \Phi_1^*(E) U^\dagger = \Phi_2^*(E)$ for all effects $E$.
\item[(b)] The sets of accessible observables are distinct: $\Phi_1^*(\mathcal{B}(\mathcal{H}_{\text{eff}}(\mathbf{c}_1))) \neq \Phi_2^*(\mathcal{B}(\mathcal{H}_{\text{eff}}(\mathbf{c}_2)))$ as subsets of $\mathcal{B}(\mathcal{H}_{\text{total}})$.
\item[(c)] The entanglement structures differ: $S_A^{(1)}(\rho_{\text{eff}}^{(1)}) \neq S_A^{(2)}(\rho_{\text{eff}}^{(2)})$ for generic subsystems $A$.
\end{enumerate}
\end{theorem}

\begin{proof}
We establish each part for a generic universal state $|\Psi_U\rangle$ (i.e., outside a measure-zero subset of $\mathcal{H}_{\text{total}}$).

\textbf{Part (a).}
The claim is that the effective descriptions cannot be related by any symmetry of the total system.
If $\dim\mathcal{A}_1 \neq \dim\mathcal{A}_2$, the claim is immediate.
If the algebras have the same dimension but are inequivalently embedded---no unitary $U$ satisfies $U\mathcal{A}_1 U^\dagger = \mathcal{A}_2$---then by definition no symmetry of $\mathcal{H}_{\text{total}}$ maps one effective description to the other.
The two coarse-grainings select genuinely different degrees of freedom.

\textbf{Part (b).}
The observable content of coarse-graining $\mathbf{c}_i$ is the image $\Phi_i^*(\mathcal{B}(\mathcal{H}_{\text{eff}}(\mathbf{c}_i))) \subset \mathcal{B}(\mathcal{H}_{\text{total}})$ under the Heisenberg-picture dual map $\Phi_i^*$.
Since $\mathbf{c}_1 \not\sim \mathbf{c}_2$, no unitary intertwines $\Phi_1^*$ and $\Phi_2^*$, so the two observable images are distinct subsets of $\mathcal{B}(\mathcal{H}_{\text{total}})$: the coarse-grainings make different observables accessible.

\textbf{Part (c).}
For distinct CPTP maps $\Phi_1 \neq \Phi_2$, the set $\{|\Psi\rangle : \Phi_1(|\Psi\rangle\!\langle\Psi|) = \Phi_2(|\Psi\rangle\!\langle\Psi|)\}$ is a proper real-algebraic subvariety of the unit sphere in $\mathcal{H}_{\text{total}}$, and hence has Haar measure zero.
For a generic $|\Psi_U\rangle$, we thus have $\rho_1 \neq \rho_2$.
Since the von Neumann entropy $S(\cdot)$ is real-analytic on the interior of the state space, its level sets $\{S = \mathrm{const}\}$ are submanifolds of positive codimension.
Consequently, $S_A^{(1)}(\rho_{\text{eff}}^{(1)}) \neq S_A^{(2)}(\rho_{\text{eff}}^{(2)})$ for generic states and generic subsystem decompositions $A$.
\end{proof}

\begin{remark}[Relation to decoherence theory]
We emphasize that the present framework is fully compatible with standard decoherence theory and does not modify its dynamical content. 
Decoherence successfully explains the emergence of classical behavior given a fixed system--environment decomposition. 
The novelty of the present approach lies instead in treating such subsystem decompositions as coarse-graining--dependent and not fundamental.
\end{remark}

\subsection{Clarification on Subsystem Structure}

Subsystems are not fundamental inputs to the framework, but derived representations induced by a chosen observable algebra. 
The connectivity measure defined in Section~\ref{sec:A-geometry} acts on these representations, and is not used to define the algebra itself. 
The causal order is:
\[
\text{Observable Algebra} \to \text{Representation} \to \text{Entanglement} \to \text{Connectivity} \to \text{Geometry}
\]
No geometric structure is presupposed in the definition of coarse-graining.

\subsection{Algebraic Perspective on Subsystems}

A potential concern is whether the use of observable algebras implicitly presupposes a subsystem decomposition. 
We stress that this is not the case. 
Observable algebras need not be defined via a tensor factorization of the total Hilbert space.

In algebraic quantum field theory, local algebras are assigned to spacetime regions without invoking a global tensor product structure \cite{Haag1996}. 
Subsystems arise only at the level of representations induced by a chosen algebra, rather than serving as its foundational input. 
In this sense, subsystem structure is emergent rather than fundamental.

\begin{remark}[Subsystems as induced representations]
Within the present framework, subsystems are not primitive entities. 
They emerge as representations associated with a given observable algebra and coarse-graining structure. 
This avoids circularity by reversing the usual explanatory order: observable structure precedes subsystem identification.
\end{remark}

\section{Entanglement Structure and Emergent Geometry}
\label{sec:A-geometry}

\subsection{Mutual Information as Connectivity Measure}

We stress that no geometric structure is assumed in the definition of the coarse-graining introduced in Section~\ref{sec:A-coarse}. 
Geometry only appears at the level of entanglement relations between the induced subsystems.

Given a coarse-graining structure $\mathbf{c} = (\mathcal{A},\Phi)$ as defined in Section~\ref{sec:A-coarse}, the universal quantum state $|\Psi_U\rangle$ induces a family of effective subsystems $\{A,B,\dots\}$ associated with subalgebras of $\mathcal{A}$.
For any such pair of subsystems $A$ and $B$, we consider the quantum mutual information
\begin{equation}
I_{\mathbf{c}}(A:B) = S_{\mathbf{c}}(A) + S_{\mathbf{c}}(B) - S_{\mathbf{c}}(AB),
\end{equation}
where $S_{\mathbf{c}}(\cdot)$ denotes the von Neumann entropy computed after coarse-graining.

\begin{definition}[Entanglement-Induced Connectivity]
\label{def:A-connectivity}
Let $A$ and $B$ be two disjoint subsystems induced by $\mathbf{c}$.
We define an effective \textbf{proximity measure} $\mu_{\mathbf{c}}(A,B)$ based on the quantum mutual information:
\begin{equation}
\mu_{\mathbf{c}}(A,B) := I_{\mathbf{c}}(A:B) = S(\rho_A) + S(\rho_B) - S(\rho_{AB}).
\end{equation}
While $\mu_{\mathbf{c}}$ does not itself constitute a metric (as it violates the triangle inequality), it induces a weighted topology where highly correlated degrees of freedom are effectively ``closer.''
The emergent metric $g_{ab}$ is derived from the infinitesimal variation of this measure under perturbations of the coarse-graining, analogous to the definition of the Quantum Fisher Information Metric (QFIM) \cite{Petz1996}.
\end{definition}

This quantity should be understood as a \emph{connectivity measure} rather than a fundamental spacetime distance.
In general, mutual information does not define a metric on arbitrary quantum subsystems.
However, it acquires a natural geometric interpretation under physically motivated assumptions, which we make explicit below.

\paragraph{Assumptions.}
Throughout this section, we restrict attention to coarse-grainings that satisfy a geometric admissibility condition:

\begin{definition}[Geometric Admissibility]
\label{def:A-geoadmit}
A coarse-graining structure $\mathbf{c}$ is said to be \textbf{geometrically admissible} if the induced mutual information satisfies:
\begin{enumerate}
\item \emph{Finite correlation length:} The state $|\Psi_U\rangle$ exhibits exponentially decaying correlations with respect to the induced subsystems, as is typical for gapped systems, ground states of local Hamiltonians, and holographic large-$N$ states.
\item \emph{Monotonic decay under refinement:} Mutual information decreases monotonically as subsystems become more refined.
\item \emph{Stability under perturbations:} The entanglement structure is robust under small perturbations of $\Phi_{\mathbf{c}}$.
\end{enumerate}
\end{definition}

Not all coarse-grainings admit a geometric interpretation. 
Geometry is not generic; it is a \emph{special phase} of information organization.
We further assume coarse-graining consistency: all entropic quantities are computed using a single coarse-graining structure $\mathbf{c}$, avoiding any mixing of inequivalent observable algebras.

Under these assumptions, the entanglement-induced connectivity $\mu_{\mathbf{c}}(A,B)$ is non-negative, symmetric, and vanishes if and only if the subsystems are uncorrelated at the level resolved by $\mathbf{c}$.
Moreover, in regimes where mutual information decays monotonically with separation, $\mu_{\mathbf{c}}$ induces an effective notion of spatial proximity.

\begin{remark}
The connectivity $\mu_{\mathbf{c}}(A,B)$ should be understood as an \emph{effective, coarse-graining--dependent measure of correlation}, rather than a fundamental spacetime metric.
A true metric structure emerges only in the continuum limit via the QFIM construction.
\end{remark}

In the continuum limit of densely overlapping subsystems, the collection of connectivity measures $\{\mu_{\mathbf{c}}(A,B)\}$ defines a weighted graph structure which, under appropriate conditions, admits a geometric interpretation via the quantum Fisher information metric \cite{Petz1996}, as we now outline.

\subsection{From Entanglement to Geometry}

The idea that spacetime geometry is encoded in quantum entanglement has been explored extensively in the context of holography.
In particular, the Ryu--Takayanagi formula relates the entanglement entropy of a boundary region $A$ to the area of an extremal surface $\gamma_A$ in the bulk \cite{RyuTakayanagi2006},
\begin{equation}
S(A) = \frac{\mathrm{Area}(\gamma_A)}{4G_N},
\end{equation}
suggesting a direct correspondence between entanglement structure and geometric data.

More generally, Van Raamsdonk has argued that the connectivity of spacetime is controlled by the pattern of entanglement between subsystems \cite{VanRaamsdonk2010}.
In this perspective, highly entangled degrees of freedom correspond to nearby regions in the emergent geometry, while weakly entangled subsystems are geometrically distant.

The entanglement-induced connectivity $\mu_{\mathbf{c}}(A,B)$ provides a concrete realization of this idea.
Given a collection of subsystems induced by $\mathbf{c}$, the mutual information defines a weighted graph whose vertices correspond to subsystems and whose edge weights encode entanglement strength.

In tensor network constructions such as MERA, similar entanglement graphs admit a natural geometric interpretation, with graph distance approximating continuum spatial distance \cite{Swingle2012}.
Taking an appropriate continuum limit, one recovers an effective Riemannian manifold whose metric reflects the underlying entanglement structure.

In this sense, geometry emerges not as an additional postulate, but as an effective description of how information is distributed and shared among coarse-grained degrees of freedom.

\subsection{Coarse-Graining Dependence of Geometry}

We now turn to the central observation of this section: the emergent geometry depends essentially on the choice of coarse-graining structure.

\begin{theorem}[Coarse-graining dependent geometry]
\label{thm:A-geom}
Let $\mathbf{c}_1$ and $\mathbf{c}_2$ be two inequivalent coarse-graining structures, as defined in Section~\ref{sec:A-coarse}, acting on the same global quantum state $|\Psi_U\rangle$.
Then the induced connectivity functions $\mu^{(1)}$ and $\mu^{(2)}$ are generically not related by a diffeomorphism, and give rise to inequivalent emergent geometries.
\end{theorem}

\begin{proof}
Let $\mathcal{A}_1, \mathcal{A}_2 \subset \mathcal{B}(\mathcal{H}_U)$ be the observable algebras associated with coarse-grainings $\mathbf{c}_1, \mathbf{c}_2$.
Since $\mathbf{c}_1 \not\sim \mathbf{c}_2$, there exists no unitary $U$ such that $U \mathcal{A}_1 U^\dagger = \mathcal{A}_2$.
The entropy of a region $R$ in the emergent geometry is given by $S(R) = -\mathrm{Tr}(\rho_R \log \rho_R)$ where $\rho_R = \Phi|_{\mathcal{A}_R}(|\Psi\rangle\langle\Psi|)$.
Since the restriction maps $\Phi_1 \neq \Phi_2$ define distinct states on the subalgebra level, the entanglement entropy profiles $S_1(x)$ and $S_2(x)$ will differ functionally.

Entanglement entropy profiles encode geometric data in settings where a reconstruction theorem applies.
In holographic theories, the Ryu--Takayanagi formula $S(R) \sim \text{Area}(\gamma_R)/4G_N$ \cite{RyuTakayanagi2006} establishes that the entropy functional determines the area functional and hence the bulk metric (up to the reconstruction ambiguity).
When such a reconstruction applies, distinct entropy profiles imply distinct metric data: $g_{\mu\nu}^{(1)} \neq g_{\mu\nu}^{(2)}$.
More generally, outside the holographic setting, distinct entropy profiles imply distinct effective distance structures derived from entanglement decay, though establishing full geometric inequivalence (non-existence of a diffeomorphism) requires additional assumptions on the reconstruction dictionary.
\end{proof}

\begin{example}[Free Fermion Chain: Spatial vs.\ Momentum Coarse-Graining]
\label{ex:A-fermion}
Consider $N$ free fermions on a one-dimensional lattice with ground state $|\Psi_0\rangle$.

\textbf{Spatial coarse-graining} $\mathbf{c}_x$: group sites into blocks of $k$ consecutive sites.
By the area law for gapped free-fermion chains~\cite{Eisert2010}, the entanglement entropy of a block of $L$ sites satisfies $S(L) \sim \mathrm{const}$, independent of $L$.
The mutual information between blocks $A$ and $B$ separated by distance $r$ decays exponentially as $I(A:B) \sim e^{-r/\xi}$, where $\xi$ is the correlation length set by the spectral gap~\cite{Hastings2007}.
The resulting connectivity graph has short-range edges: its graph distance approximates a one-dimensional lattice, yielding a line-like emergent geometry.

\textbf{Momentum coarse-graining} $\mathbf{c}_p$: retain modes with momenta $|k| < \Lambda$ for some cutoff $\Lambda < \pi$.
In momentum space, the ground state is a filled Fermi sea.
The entanglement entropy of any momentum subset scales as $S \sim N_{\text{modes}} \ln(N/N_{\text{modes}})$---a volume law rather than an area law~\cite{Wolf2006}.
The mutual information between momentum shells is generically long-range (all momentum modes are correlated through the Fermi surface), producing a highly non-local connectivity graph.

Since the area-law spatial graph and the volume-law momentum graph cannot be related by a diffeomorphism (they have different scaling dimensions for the entanglement entropy functional), $\mathbf{c}_x$ and $\mathbf{c}_p$ produce provably inequivalent emergent geometries from the same underlying state $|\Psi_0\rangle$.
\end{example}

A simple conceptual illustration is provided by contrasting spatial coarse-graining with momentum-space coarse-graining.
In the former case, entanglement typically obeys an area law and supports a local, connected geometry.
In the latter, long-range entanglement is generic, leading to a highly non-local entanglement graph and a qualitatively different emergent geometry.
Importantly, both descriptions arise from the same underlying state $|\Psi_U\rangle$.

Thus, the geometry inferred from entanglement is not an intrinsic property of the quantum state alone, but depends on how information is rendered accessible through coarse-graining.

\subsection{Beyond Coordinate Choice}

It is important to distinguish the dependence described above from an ordinary change of coordinates.
A diffeomorphism acts within a fixed algebra of observables, preserving the underlying notion of what is measurable.
By contrast, a change of coarse-graining alters the observable algebra itself, modifying which correlations are accessible and how subsystems are defined.

These are categorically distinct operations.
Since the algebra of observables differs, the resulting geometries cannot be related by a mere diffeomorphism.
In extreme cases, changes in coarse-graining may even alter basic connectivity properties of the emergent space.

We return to the conceptual implications of this distinction in the Discussion.

\begin{figure}[ht]
\centering
\begin{tikzpicture}[>=stealth, auto, node distance=3cm]
    % Styles
    \tikzstyle{state} = [circle, draw, thick, fill=gray!10, minimum size=2.5cm, align=center]
    \tikzstyle{filter} = [rectangle, draw, thick, fill=white, minimum width=2.5cm, minimum height=1cm]
    \tikzstyle{geo} = [ellipse, draw, thick, dashed, minimum width=2.5cm, minimum height=1.5cm, align=center]
    
    % Nodes
    \node[state] (Alaya) {Global State\\$|\Psi_{\text{vac}}\rangle$\\(Alaya-Field)};
    
    \node[filter] (Filter1) [below left of=Alaya, xshift=-1cm, yshift=-1cm] {Algebra $\mathcal{A}_1$};
    \node[filter] (Filter2) [below right of=Alaya, xshift=1cm, yshift=-1cm] {Algebra $\mathcal{A}_2$};
    
    \node[geo] (Geo1) [below of=Filter1, yshift=-1cm] {Geometry $M_1$\\($g_{\mu\nu}^{(1)}$)};
    \node[geo] (Geo2) [below of=Filter2, yshift=-1cm] {Geometry $M_2$\\($g_{\mu\nu}^{(2)}$)};
    
    % Arrows
    \draw[->, thick] (Alaya) -- node[left, font=\footnotesize] {Coarse Grain $\Phi_1$} (Filter1);
    \draw[->, thick] (Alaya) -- node[right, font=\footnotesize] {Coarse Grain $\Phi_2$} (Filter2);
    
    \draw[->, double, thick] (Filter1) -- node[left, font=\footnotesize] {Induces} (Geo1);
    \draw[->, double, thick] (Filter2) -- node[right, font=\footnotesize] {Induces} (Geo2);
    
    % Interaction
    \draw[<->, dotted, very thick] (Geo1) -- node[below, font=\footnotesize] {Inequivalent ($M_1 \not\cong M_2$)} (Geo2);

\end{tikzpicture}
\caption{Schematic of the HAFF structure. A single global state $|\Psi_{\text{vac}}\rangle$ (Alaya) projects into topologically distinct emergent spacetimes ($M_1, M_2$) depending on the choice of observable algebra ($\mathcal{A}_1, \mathcal{A}_2$). This illustrates that geometry is observer-dependent in the fundamental sense.}
\label{fig:haff_scheme}
\end{figure}

\section{Discussion}
\label{sec:A-discussion}

\subsection{Relation to Existing Frameworks}

The framework developed in this work is not intended as a replacement for existing interpretational or algebraic approaches to quantum theory.
Rather, it is best understood as orthogonal to several well-established lines of thought, addressing a distinct structural question.
In this section, we briefly clarify its relation to three representative frameworks:

\subsubsection{Relation to QBism}

QBism emphasizes the role of agents and their personal probability assignments in quantum theory, interpreting the quantum state as an expression of subjective belief rather than an objective property of a system.
In contrast, the present framework assumes a single, objective global quantum state throughout.
No agent-dependent elements enter the formalism, and no interpretational commitments regarding belief, experience, or decision theory are required.

The point of contact lies solely in the rejection of a privileged subsystem decomposition.
While QBism attributes this absence to the primacy of the agent, our framework treats it as a structural feature of quantum theory itself.
Coarse-grainings are not chosen by agents, but instantiated by concrete physical interaction structures.
The resulting multiplicity of effective descriptions reflects physical underdetermination rather than subjectivity.

\subsubsection{Relation to Many-Worlds and Decoherence-Based Approaches}

Many-Worlds--type interpretations and decoherence-based accounts provide a compelling explanation of classical behavior within quantum mechanics by analyzing branching structures relative to a fixed subsystem decomposition.
Our framework does not modify this analysis, nor does it introduce an alternative account of branching or classicality.

The difference lies at a prior level.
Decoherence theory presupposes a tensor factorization into system, apparatus, and environment.
Here, we instead ask how such subsystem structures arise in the first place.
In this sense, the framework is complementary to decoherence-based approaches: it leaves their dynamical conclusions intact while removing subsystem factorization from the list of fundamental assumptions.

\subsubsection{Relation to Algebraic Quantum Field Theory}

The closest structural affinity of the present work is with algebraic quantum field theory (AQFT).
In AQFT, observable algebras are taken as primary, and states are defined as positive linear functionals over these algebras, without reliance on a global tensor product structure.
Our use of observable algebras and their representations is directly inspired by this tradition.

The present framework may be viewed as extending this algebraic perspective by emphasizing the role of coarse-graining relations between algebras.
Different choices of coarse-graining induce different effective representations and, consequently, different entanglement structures.
The novelty lies not in the algebraic formalism itself, but in using it to analyze the emergence and non-uniqueness of geometric descriptions.

Taken together, these comparisons situate the present work as a structural investigation into the conditions under which subsystem structure and geometry emerge.
It neither commits to a particular interpretation of quantum mechanics nor proposes a new dynamical law, but instead clarifies how several existing frameworks implicitly rely on assumptions that can be made explicit and, in some cases, relaxed.

\subsection{Scope and Limitations}

The present work is concerned with structural consistency rather than phenomenological prediction. 
Observable consequences depend on the physical mechanisms implementing a given coarse-graining, which lie beyond the scope of this paper. 
This is analogous to effective field theory, where multiple UV completions may share the same low-energy structure. 
Future work will explore concrete physical realizations and their empirical signatures.

\subsection{Philosophical Implications}

We emphasize that the framework assumes a single, objective global quantum state $|\Psi_U\rangle$. 
What is coarse-graining--dependent is not reality itself, but the effective structures used to describe it. 
This position is compatible with scientific realism while acknowledging the role of operational context in physical description.

\subsection{Future Directions}

While the present work is deliberately limited to a structural analysis of subsystem emergence and geometry within a fixed global quantum state, it naturally opens several directions for further investigation. 
We emphasize that the following points are not results established here, but rather indicate possible extensions where the current framework may provide useful conceptual or technical guidance.

\subsubsection{Dynamical Models of Coarse-Graining Selection}

In this work, coarse-graining structures are treated as fixed relational inputs, instantiated by concrete physical interaction patterns. 
A natural next step is to investigate whether such coarse-grainings can themselves be characterized dynamically.

One possible direction is to study how interaction Hamiltonians, coupling strengths, or network topologies bias the emergence of particular subalgebra structures over others. 
This could clarify under what physical conditions certain factorizations become robust or persistent, and whether transitions between inequivalent coarse-grainings admit an effective dynamical description.

Importantly, such an analysis would remain compatible with globally unitary evolution, treating coarse-graining selection as an emergent, effective phenomenon rather than a modification of fundamental dynamics.

\subsubsection{Connections to Quantum Information and Complexity}

The non-uniqueness of emergent geometry highlighted here suggests a close connection to quantum information--theoretic notions such as entanglement structure, operator complexity, and resource constraints.

Future work could explore whether preferred geometric descriptions correlate with informational criteria---for example, minimal description length, stability under noise, or computational accessibility of observables. 
Such considerations may help explain why certain coarse-grainings are physically salient, even when many are formally admissible.

This perspective may also provide a bridge to recent work on complexity-based approaches to spacetime emergence, without committing to any particular complexity measure at the present stage.

\subsubsection{Extensions to Quantum Field Theory and Continuum Limits}

While the framework has been formulated in abstract Hilbert space terms, an important open question concerns its implementation in quantum field--theoretic settings, where issues of locality, algebraic nets, and continuum limits arise.

In particular, it would be valuable to examine how coarse-graining relations between observable algebras interact with the locality structures emphasized in algebraic quantum field theory, and whether familiar spacetime geometries can be recovered as stable fixed points of such relations.

We stress that the present work does not resolve these questions, but provides a structural language in which they can be posed more precisely.

\subsubsection{Empirical and Phenomenological Implications}

At the level developed here, the framework is primarily structural and conceptual. 
Nevertheless, future investigations may ask whether different coarse-graining choices lead to distinguishable effective descriptions, for example in semiclassical regimes, quantum gravity--motivated models, or analogue systems.

Such studies could clarify whether the non-uniqueness of emergent geometry has observable consequences, or whether physical constraints effectively suppress this freedom in realistic settings.

Any empirical analysis would necessarily require additional assumptions beyond those adopted in this work, and thus lies outside its present scope.

\subsubsection{Conceptual Clarifications and Interpretational Interfaces}

Finally, although the framework is intentionally neutral with respect to interpretations of quantum mechanics, it may serve as a useful interface for comparative studies. 
By making explicit the structural assumptions underlying subsystem decomposition and geometry, it could help clarify which features are interpretation-dependent and which arise more generally from the formalism itself.

We view this not as an attempt to adjudicate between interpretations, but as an opportunity to sharpen the questions they address.

\medskip

Taken together, these directions suggest that the framework developed here is best viewed as a scaffold: it does not dictate specific physical models, but provides a structured setting in which questions about subsystems, geometry, and emergence can be formulated with greater precision.

\section{Conclusion}
\label{sec:A-conclusion}

We have presented a framework in which subsystem structure is not presupposed, but emerges from coarse-graining over observable algebras. 
The central results are:

\begin{enumerate}
\item A generic quantum state admits no canonical tensor factorization (Theorem~\ref{thm:A-nofact}).
\item Different coarse-graining structures induce inequivalent effective subsystem descriptions (Theorem~\ref{thm:A-inequiv}).
\item These inequivalent coarse-grainings generically give rise to distinct emergent geometries (Theorem~\ref{thm:A-geom}), and this distinction cannot be reduced to coordinate choice.
\end{enumerate}

The framework is deliberately limited in scope. 
We have not proposed new dynamics, derived empirical predictions, or resolved interpretational debates. 
Instead, we have examined the structural consequences of removing subsystem factorization from the list of fundamental assumptions.

This investigation reveals that the emergence of geometry is more context-dependent than often acknowledged. 
Geometry is not an intrinsic property of a quantum state alone, but depends on how information is rendered accessible through coarse-graining. 
This perspective is compatible with, and complementary to, existing approaches including decoherence theory, holographic duality, and algebraic quantum field theory.

Several open questions remain. 
Can coarse-graining structures themselves be characterized dynamically? 
Do informational or complexity-based criteria select physically preferred coarse-grainings? 
Can the framework be extended to quantum field theory and reconciled with standard locality structures? 
These questions lie beyond the present scope, but the structural setting developed here provides a language in which they can be formulated with greater precision.

Ultimately, this work suggests that the relationship between quantum states, subsystems, and geometry is more subtle than the standard picture implies. 
By making explicit an assumption that is often left implicit, we hope to have clarified the conditions under which emergence occurs and opened new avenues for investigating the foundations of quantum theory and spacetime.


% ============================================================================
% Paper B
% ============================================================================
\chapter{Accessibility, Stability, and Emergent Geometry}
\label{chap:paperB}

\begin{center}
\textit{Paper B}\\[0.5em]
Originally published: Zenodo, DOI: 10.5281/zenodo.18367060
\end{center}

\bigskip

\section*{Abstract}

This paper provides conceptual clarification of the Holographic Alaya-Field Framework (HAFF) introduced in our previous work. 
We address three potential misreadings: subjectivism (that observers create spacetime), anti-realism (that geometry is illusory), and trivialism (that the framework reduces to coordinate choice). 
By analyzing the structural notion of accessibility via stability conditions, delineating boundaries with existing interpretations (AQFT, RQM, QBism, MWI), and characterizing emergent geometry as a stable organizational phase, we clarify what the framework commits to and what it does not. 
This analysis is purely interpretational; no new formal results are introduced.


\section{Introduction}
\label{sec:B-intro}

Our previous work established a framework in which emergent geometry depends on the choice of observable algebra acting on a global quantum state \cite{Liu2026}. 
The technical results—in particular, that inequivalent coarse-graining structures induce inequivalent geometric structures from the same underlying state—are formally precise and mathematically consistent. 
However, structural novelty of this kind is particularly vulnerable to interpretational confusion. 
The present paper addresses these interpretational implications and clarifies the conceptual commitments of the framework.

\subsection{The Risk of Misreading}

The claim that geometry is coarse-graining-dependent naturally invites several misreadings, each of which conflates distinct notions of dependence. 
Three such misreadings are especially common:

\begin{enumerate}
\item \emph{Subjectivism}: The view that observers or agents create spacetime through their choices or beliefs, collapsing the framework into an epistemic or observer-relative interpretation.
\item \emph{Anti-realism}: The view that spacetime has no objective existence, and that emergent geometry is therefore illusory or merely pragmatic.
\item \emph{Trivialism}: The view that coarse-graining-dependent geometry reduces to a choice of coordinates or descriptive convention, with no substantive physical consequences.
\end{enumerate}

Each of these readings is incorrect, but each arises naturally from surface-level features of the formalism. 
The purpose of this paper is to block such misreadings by clarifying the nature of accessibility, the structural role of observable algebras, and the ontological status of emergent geometry within the framework.

\subsection{Scope and Objectives}

This paper does not introduce new formal results, derive additional theorems, or propose modifications to the mathematical structure presented in our previous work. 
Rather, it provides a systematic interpretational analysis aimed at three specific objectives:

\begin{enumerate}
\item \emph{Clarify the notion of accessibility}: We demonstrate that accessibility, as employed in the framework, is a structural and operational concept determined by stability properties of subalgebras, not an epistemic or observer-dependent notion.
\item \emph{Delineate boundaries with existing interpretations}: We situate the framework in relation to algebraic quantum field theory, relational quantum mechanics, QBism, and the Many-Worlds interpretation, clarifying points of agreement, divergence, and complementarity.
\item \emph{Characterize the ontological status of emergent geometry}: We argue that geometry functions as a stable organizational phase of quantum information, analogous to phases in condensed matter systems, avoiding both naive realism and anti-realist eliminativism.
\end{enumerate}

Importantly, this analysis does not constitute a defense of the framework, nor does it aim to persuade readers of its correctness. 
The goal is clarity: to ensure that the structural commitments of the framework are understood on their own terms, and that criticisms, if any, are directed at what the framework actually claims rather than at interpretational projections.

\subsection{What This Paper Does Not Do}

To further delimit scope, we note explicitly what this paper does \emph{not} attempt:

\begin{itemize}
\item It does not propose new dynamics, empirical predictions, or modifications to quantum mechanics.
\item It does not claim that the framework resolves outstanding problems in quantum gravity, quantum foundations, or the measurement problem.
\item It does not advocate for any particular metaphysical or philosophical position beyond the minimal structural commitments required by the formalism itself.
\item It does not interpret the framework as implying idealism, observer-created reality, or any form of mind-dependence.
\end{itemize}

The analysis remains strictly within the domain of structural interpretation: identifying what the mathematical formalism commits to, what it leaves open, and how it relates to existing approaches.

\subsection{Organization}

The paper proceeds as follows. Section~\ref{sec:B-stance} briefly recapitulates the structural stance adopted in our previous work, emphasizing the priority of observable algebras over tensor factorizations and the role of coarse-graining in defining effective subsystems. 

Section~\ref{sec:B-accessibility} provides a detailed analysis of accessibility, demonstrating that it is determined by stability conditions on subalgebras rather than by observer choices or epistemic limitations. A taxonomy is introduced distinguishing subjective, relational, and structural notions of dependence, situating the present framework firmly within the third category.

Section~\ref{sec:B-relations} examines the relationship between the framework and four representative approaches: algebraic quantum field theory, relational quantum mechanics, QBism, and the Many-Worlds interpretation. Each subsection clarifies points of conceptual overlap and structural divergence, preventing conflation while identifying opportunities for complementarity.

Section~\ref{sec:B-geometry_phase} argues that emergent geometry should be understood as a stable organizational phase of entanglement structure, drawing on analogies with condensed matter physics. This perspective avoids treating geometry as either fundamental or illusory, instead characterizing it as contingent but objective—dependent on physical conditions rather than epistemic contexts.

Section~\ref{sec:B-scope} delimits the structural assumptions of the framework, enumerates what it does not commit to, and outlines open questions for future investigation. Particular attention is given to the distinction between structural analysis and metaphysical advocacy.

\subsection{Methodological Note}

Throughout this paper, we adopt a deliberately conservative rhetorical stance. Claims are hedged with modal qualifiers ("may suggest," "is consistent with," "can be understood as") not out of uncertainty regarding the formal results, but to avoid overstating interpretational conclusions. The goal is to present the framework as one coherent way of organizing the conceptual landscape, not as the uniquely correct interpretation.

This methodological caution reflects a broader commitment: interpretational clarity is best served by precision and restraint, not by advocacy or polemics. We aim to make the framework legible to researchers across different interpretational traditions, facilitating comparison and critique rather than preempting it.

\section{Structural Stance Recap}
\label{sec:B-stance}

We briefly recapitulate the structural stance of our previous work without repeating full mathematical derivations.

The framework rests on three formal results:
\begin{enumerate}
\item \textbf{No canonical factorization} (Theorem 1): A generic pure state admits no unique or canonically preferred tensor factorization into subsystems. Any such decomposition requires additional structure beyond the state itself.

\item \textbf{Coarse-graining-induced inequivalence} (Theorem 2): Different choices of observable algebra $\mathcal{A}_{\mathbf{c}}$ acting on the same global state $|\Psi_U\rangle$ induce inequivalent effective subsystem descriptions, characterized by distinct reduced density matrices, entanglement patterns, and POVM structures.

\item \textbf{Geometry dependence} (Theorem 3): Inequivalent coarse-graining structures generically induce inequivalent geometric structures from the same underlying state, which may include distinct topological features in certain cases. This inequivalence cannot be reduced to a diffeomorphism or coordinate transformation, as it reflects differences in the underlying observable algebra.
\end{enumerate}

The conceptual core can be summarized by reversing the standard explanatory arrow:

\begin{center}
\textbf{Standard:} State + Tensor factorization $\to$ Subsystems $\to$ Entanglement $\to$ Geometry
\end{center}

\begin{center}
\textbf{HAFF:} State + Observable algebra $\to$ Effective subsystems $\to$ Entanglement $\to$ Geometry
\end{center}

We emphasize that the observable algebra $\mathcal{A}_{\mathbf{c}}$ is not an arbitrary choice, but is determined by the physical interaction structure encoded in the Hamiltonian, specifying which degrees of freedom couple and how (see Section~\ref{sec:B-accessibility} for detailed discussion).

This reversal has a modest but consequential implication: subsystem structure, and consequently geometry, is not intrinsic to the quantum state alone but depends on which observables are accessible—where accessibility is understood in a precise, structural sense developed in the next section.

\section{Accessibility vs Observer-Dependence}
\label{sec:B-accessibility}

The notion of accessibility is central to the framework, and it is here that the risk of misreading is greatest. We clarify that accessibility, as employed in HAFF, is a structural property determined by physical stability conditions, not an epistemic or agent-dependent notion.

\subsection{Three Notions of Dependence}

To prevent conflation, we distinguish three distinct senses in which a physical quantity might be said to "depend on" something:

\begin{table}[h]
\centering
\small
\begin{tabular}{|P{2.5cm}|P{4.5cm}|P{3.5cm}|P{2.5cm}|}
\hline
\textbf{Type} & \textbf{Depends On} & \textbf{Example} & \textbf{Framework} \\
\hline
Subjective & Agent's beliefs/knowledge & Bayesian probability & QBism \\
Relational & Reference system & Velocity in SR & RQM \\
Structural & Physical interaction pattern & Decoherence basis & HAFF \\
\hline
\end{tabular}
\caption{Taxonomy of dependence notions}
\label{tab:dependence}
\end{table}

HAFF's notion of accessibility falls squarely in the third category.

\subsection{Algebraic Grounding of Stability}

To address potential concerns regarding circularity in defining accessibility, we provide an algebraic characterization of stability that does not presuppose factorization or observer-dependent choices.

Consider a subalgebra $\mathcal{A} \subset \mathcal{B}(\mathcal{H})$ acting on the global state $\rho$. We define $\mathcal{A}$ to be \emph{stable} if it satisfies the following criteria:

\paragraph{Criterion 1: Dynamical Invariance.}
Expectation values of operators in $\mathcal{A}$ remain approximately invariant under physically motivated dynamical maps $\mathcal{E}$ (representing decoherence, RG flow, or measurement-induced back-action):
\begin{equation}
\|\mathcal{E}(\hat{O}) - \hat{O}\| \ll \epsilon \quad \forall \hat{O} \in \mathcal{A},
\end{equation}
for suitably small $\epsilon$ set by physical precision.

\paragraph{Criterion 2: Environmental Redundancy (Quantum Darwinism).}
Following Zurek's quantum Darwinism framework \cite{Zurek2009}, stable subalgebras are those whose information is redundantly encoded in environmental degrees of freedom. Formally, the subalgebra $\mathcal{A}$ should approximately commute with the environmental algebra $\mathcal{A}_E$ generated by accessible environmental observables:
\begin{equation}
[\hat{O}, \hat{E}] \approx 0 \quad \forall \hat{O} \in \mathcal{A}, \, \hat{E} \in \mathcal{A}_E.
\end{equation}
This ensures that states in $\mathcal{A}$ act as \emph{pointer states}, robustly imprinted on the environment and thus operationally accessible through multiple independent measurements.

\paragraph{Criterion 3: Non-scrambling Subspace.}
In the language of quantum information scrambling \cite{Hayden2007}, stable algebras correspond to \emph{non-scrambling subspaces}—degrees of freedom that do not rapidly lose local correlations under unitary evolution. Quantitatively, the out-of-time-order correlator (OTOC) associated with operators in $\mathcal{A}$ should exhibit slow decay:
\begin{equation}
\langle [\hat{O}_\mathcal{A}(t), \hat{V}(0)]^2 \rangle \ll 1 \quad \text{for } t \ll \tau_{\text{scrambling}}.
\end{equation}

These criteria are purely algebraic and operational: they refer only to operator commutation relations, dynamical maps, and measurable correlators, without invoking subjective observer choices. Stability, in this sense, is a \emph{structural property} determined by the physical interaction Hamiltonian and the global quantum state.

Multiple stability criteria may select different subalgebras, reflecting different physical regimes (equilibrium vs. out-of-equilibrium, weak vs. strong coupling, etc.). This plurality is a feature analogous to how different symmetry-breaking patterns yield distinct thermodynamic phases.

\begin{tcolorbox}[colback=gray!5!white,colframe=black!75!black,title=\textbf{Remark: The Hamiltonian as Physical Input}]
A potential objection to the structural account of accessibility is that the interaction Hamiltonian $\hat{H}_{\text{int}}$ appears to be an arbitrary input, merely displacing observer-dependence from the algebra to the choice of Hamiltonian. We clarify that this is a misunderstanding of the framework's commitments.

The Hamiltonian is not a free parameter chosen by an observer, but a \emph{physical specification of which degrees of freedom interact and how}. In this respect, it plays a role analogous to the stress-energy tensor $T_{\mu\nu}$ in general relativity: it is input data describing the causal structure of the system, not a coordinate choice or descriptive convention.

Concretely:
\begin{itemize}
\item In quantum optics, $\hat{H}_{\text{int}}$ describes atom-photon coupling strengths and selection rules, determined by atomic energy levels and field modes.
\item In condensed matter systems, it encodes lattice structure, tunneling amplitudes, and interaction potentials, all fixed by material properties.
\item In holographic models (AdS/CFT), the boundary Hamiltonian is determined by the conformal field theory's operator content and coupling constants.
\end{itemize}

The claim is not that observers are irrelevant, but that they do not \emph{create} the interaction structure—they probe it. Different experimental setups may access different subalgebras, but which algebras are stable under given interactions is an objective, physical fact, independent of epistemic context.

This is precisely analogous to how different coordinate systems in general relativity yield different component expressions for the metric $g_{\mu\nu}$, but the spacetime geometry itself (characterized by invariants like the Ricci scalar $R$) is coordinate-independent. Here, different accessible algebras yield different effective descriptions, but which algebras are stable under physical dynamics is interaction-dependent, not observer-dependent.

We emphasize: the framework does not solve the problem of \emph{why} a particular Hamiltonian describes our universe (just as GR does not explain why $T_{\mu\nu}$ has its observed form). That question lies in the domain of fundamental theory or cosmology. What the framework does is analyze the \emph{consequences} of a given interaction structure for emergent subsystem decomposition and geometry.
\end{tcolorbox}

\paragraph{Pre-geometric Interaction Structure.}
A potential concern is that the Hamiltonian $\hat{H}_{\text{int}}$ itself presupposes spacetime structure (e.g., via spatial locality in $\hat{H} = \sum_{i,j} J_{ij} \hat{\sigma}_i \cdot \hat{\sigma}_j$), creating a circular dependence: geometry emerges from algebras determined by a Hamiltonian that already assumes geometry.

We clarify that the Hamiltonian input to HAFF is \emph{pre-geometric}: it is specified as an abstract interaction graph or tensor network, where edges represent couplings and nodes represent degrees of freedom, with no reference to background metric structure. The notion of "locality" in such Hamiltonians is graph-theoretic (e.g., nearest-neighbor on a lattice or tree) rather than metric-geometric.

Crucially, the \emph{effective spacetime geometry} that emerges from stable algebras may differ from the graph structure of the input Hamiltonian. For instance:
\begin{itemize}
\item In tensor network models (MERA), the input is a discrete causal network, but the emergent geometry can be continuous AdS space \cite{Swingle2012}.
\item In spin chain models, the Hamiltonian is defined on a 1D lattice, but entanglement structure can induce higher-dimensional effective geometry.
\item In holographic duality (AdS/CFT), the boundary Hamiltonian is defined on a fixed $(d-1)$-dimensional manifold, but the bulk geometry (including its dimensionality) emerges dynamically.
\end{itemize}

Thus, the input interaction structure constrains but does not uniquely determine the emergent geometry—it serves as a \emph{seed} or \emph{scaffold}, not a blueprint. The framework asks: given a pre-geometric interaction graph, which stable algebras emerge, and what geometric structures do they induce?

This perspective aligns with recent work on "locality from entanglement" \cite{VanRaamsdonk2010}, where spatial locality is itself understood as arising from entanglement patterns rather than being presupposed.

\subsection{Conditions for Uniqueness of $\mathcal{A}_{\mathbf{c}}$}
\label{sec:B-uniqueness}

Remark 1 in Paper~A noted that the specification of an accessible observable algebra is not assumed to be unique.
We now identify conditions under which the accessible algebra is unique up to unitary equivalence, thereby sharpening the framework's predictive content.

\begin{conjecture}[Uniqueness of the Accessible Algebra under Modular Stability]
\label{conj:B-uniqueness}
Let $|\Psi_U\rangle$ be the global state, $\hat{H}$ the total Hamiltonian, and $\mathcal{E}_t$ the dynamical (decoherence) map.
Suppose the following conditions hold:
\begin{enumerate}
\item[\textup{(U1)}] \textbf{Faithfulness}: The restriction of $\omega(\cdot) = \langle \Psi_U | \cdot | \Psi_U \rangle$ to $\mathcal{A}_{\mathbf{c}}$ is faithful (the GNS vector is cyclic and separating for $\mathcal{A}_{\mathbf{c}}$).
\item[\textup{(U2)}] \textbf{Modular stability}: The modular automorphism group $\sigma_t^\omega$ generated by the Tomita--Takesaki modular operator $\Delta_\omega$ preserves $\mathcal{A}_{\mathbf{c}}$: $\sigma_t^\omega(\mathcal{A}_{\mathbf{c}}) = \mathcal{A}_{\mathbf{c}}$ for all $t \in \mathbb{R}$.
\item[\textup{(U3)}] \textbf{Maximality}: $\mathcal{A}_{\mathbf{c}}$ is the maximal subalgebra of $\mathcal{B}(\mathcal{H}_{\mathrm{total}})$ satisfying \textup{(U1)--(U2)} together with the three accessibility criteria of Section~\ref{sec:B-accessibility} (dynamical invariance, environmental redundancy, non-scrambling).
\end{enumerate}
Then $\mathcal{A}_{\mathbf{c}}$ is unique up to unitary equivalence: if $\mathcal{A}'$ also satisfies \textup{(U1)--(U3)}, there exists a unitary $U$ with $U\mathcal{A}_{\mathbf{c}} U^\dagger = \mathcal{A}'$.
\end{conjecture}

\begin{proof}[Proof sketch]
The argument uses three ingredients from the theory of von Neumann algebras.

\textbf{Step 1: Uniqueness of modular flow.}
By Takesaki's theorem~\cite{Takesaki2003}, for a faithful normal state $\omega$ on a von Neumann algebra $\mathcal{M}$, the modular automorphism group $\sigma_t^\omega$ is the \emph{unique} one-parameter group satisfying the KMS condition at $\beta = 1$.
This uniqueness anchors the construction: given the global state $|\Psi_U\rangle$, the modular flow on any candidate algebra is determined, not chosen.

\textbf{Step 2: Fixed-point algebra under joint dynamics.}
Condition (U2) requires $\mathcal{A}_{\mathbf{c}}$ to be invariant under modular flow.
Combined with dynamical invariance (accessibility criterion 1: invariance under $\mathcal{E}_t$) and non-scrambling (criterion 3: slow OTOC growth), $\mathcal{A}_{\mathbf{c}}$ is constrained to the fixed-point algebra under the joint action of modular flow and decoherence.
For ergodic dynamics, this fixed-point algebra is uniquely determined by the spectral data of $\Delta_\omega$ and $\mathcal{E}_t$.

\textbf{Step 3: Maximality implies uniqueness.}
If two algebras $\mathcal{A}$ and $\mathcal{A}'$ both satisfy (U1)--(U3), their join $\mathcal{A} \vee \mathcal{A}'$ also satisfies (U1)--(U2) (modular stability is preserved under joins of invariant subalgebras).
By maximality, $\mathcal{A} = \mathcal{A} \vee \mathcal{A}' = \mathcal{A}'$.
The remaining unitary freedom is absorbed by the standard form of the algebra~\cite{Haagerup1975}: any two faithful normal representations are unitarily equivalent in the standard form.
\end{proof}

\begin{remark}[Scope of the Conjecture]
\label{rem:B-uniqueness-scope}
The conjecture does not claim that all physical systems satisfy (U1)--(U3).
Systems with degenerate ground states, spontaneous symmetry breaking, or phase coexistence may support multiple non-unitarily-equivalent accessible algebras---the HAFF analog of the non-uniqueness of the broken-symmetry vacuum in QFT.
The conjecture identifies the conditions under which this non-uniqueness is \emph{absent}: a single, non-degenerate dynamical regime with a faithful state.
Notably, condition (U2) connects the accessibility framework to the modular theory that underlies the HAFF gravity conjecture (Paper~D): the algebras that are modular-stable are precisely those for which emergent geometry is well-defined.
\end{remark}

\section{Relations to Existing Interpretations}
\label{sec:B-relations}

We now situate HAFF relative to four representative frameworks, clarifying conceptual boundaries and identifying points of potential complementarity.

\subsection{Algebraic Quantum Field Theory (AQFT)}

The closest structural affinity of HAFF is with algebraic quantum field theory \cite{Haag1996,Araki1999}. In AQFT, observable algebras are treated as primary, with states defined as positive linear functionals over these algebras. Crucially, local algebras are assigned to spacetime regions without relying on a global tensor product structure.

HAFF extends this algebraic perspective by emphasizing \emph{coarse-graining relations} between algebras. In standard AQFT, locality is typically presupposed: algebras are indexed by spacetime regions, and the split property ensures independence of spacelike-separated algebras. In HAFF, we relax this assumption and instead treat \emph{stability under physical interactions} as the criterion for algebra selection.

The transformation can be summarized as:
\begin{center}
\textbf{AQFT}: Spacetime regions $\to$ Local algebras $\to$ States \\
\textbf{HAFF}: Interaction structure $\to$ Stable algebras $\to$ Effective geometry
\end{center}

This is not a replacement of AQFT but an exploration of its structure in contexts where spacetime locality is not presupposed. The framework may be understood as asking: what happens to the algebraic approach when we do not assume a background spacetime to index our algebras?

\subsection{Relational Quantum Mechanics (RQM)}

Relational quantum mechanics \cite{Rovelli1996,LaudisaRovelli2021} emphasizes that the values of physical quantities are defined only relative to observer systems, rejecting the notion of absolute, observer-independent observables. In Rovelli's formulation, quantum mechanics is fundamentally a theory of \emph{interactions} rather than systems: what exists are relational facts, not intrinsic properties.

There is significant conceptual overlap with HAFF: both frameworks reject privileged subsystem decompositions and treat quantum descriptions as contextual. However, a key difference concerns the \emph{stabilization of relata}.

RQM analyzes relations between systems whose existence is typically taken as given (or at least presupposed operationally through interaction records). HAFF provides a mechanism for the \emph{stabilization of distinct relata} from the underlying quantum field, identifying which subsystem partitions are robustly maintained under decoherence and measurement-induced dynamics.

In this sense, HAFF may provide the stable nodes required for RQM's relational network: before relations can exist, there must be relata stable enough to participate in interactions. HAFF addresses how such stable relata emerge from the pre-factorized quantum substrate.

These are complementary rather than competing perspectives: RQM asks what observables mean relative to a system, HAFF asks which systems stabilize as distinct relata in the first place. The two frameworks operate at different levels of analysis and could in principle be combined, with HAFF providing the stability conditions under which RQM's relational structure becomes well-defined.

\subsection{QBism}

QBism \cite{Fuchs2014,FuchsMerminSchack2014} interprets quantum states as expressions of an agent's personal beliefs about measurement outcomes, emphasizing the subjective, agent-centric nature of quantum probability assignments.

Here, the distinction from HAFF is sharpest. While both frameworks reject naive realism about the quantum state, they differ fundamentally in their treatment of dependence:

\begin{itemize}
\item \textbf{QBism}: Quantum states represent personal beliefs. Dependence is epistemic and agent-centric.
\item \textbf{HAFF}: Observable algebras are selected by physical interaction structure. Dependence is structural and interaction-centric.
\end{itemize}

The key point is that interactions are not agents: they have no beliefs, make no decisions, and exist independently of any epistemic perspective. The accessible algebra in HAFF is determined by which degrees of freedom couple via the Hamiltonian, not by what any observer happens to know or believe.

We emphasize that this is a categorical difference, not a matter of one framework being "more correct" than the other. QBism and HAFF address different questions and operate within different conceptual frameworks. The point of comparison is simply to clarify that HAFF's notion of accessibility does not reduce to QBist agent-dependence.

\subsection{Many-Worlds Interpretation (MWI)}

The Many-Worlds interpretation \cite{Wallace2012} explains the emergence of classical behavior through decoherence-induced branching, all occurring within globally unitary quantum evolution.

HAFF shares with MWI a commitment to:
\begin{itemize}
\item A single, objective global quantum state
\item Unitary evolution without collapse
\item Effective classicality emerging from entanglement structure
\end{itemize}

However, MWI typically presupposes a tensor factorization into system and environment as input, analyzing how this decomposition gives rise to branch structure. Recent work in the MWI tradition (e.g., Wallace, Saunders) addresses emergent decoherence structure, but typically within a framework where tensor factorization is assumed at the fundamental level.

HAFF complements MWI by examining the preconditions for any branching structure: before branches can emerge, there must be a notion of subsystems relative to which branching occurs. In this sense, HAFF may provide a structural framework relevant to understanding how the effective subsystem decompositions presupposed by branching emerge in the first place.

This is a point of potential contact rather than a hierarchical claim: we do not argue that MWI requires HAFF, but that the two frameworks address distinct but related aspects of quantum structure.

\section{Geometry as a Stable Organizational Phase}
\label{sec:B-geometry_phase}

\paragraph{Conceptual Disclaimer.}
Before discussing the analogy with condensed matter phases, we stress that this analogy is \emph{conceptual rather than rigorous}. Unlike in condensed matter, where a Hamiltonian uniquely determines phase structure via symmetry-breaking or RG fixed points, the HAFF framework does not posit a master Hamiltonian governing the selection of accessible algebras. The purpose of the analogy is to clarify how emergent geometry can be understood as a stable organizational pattern of entanglement, highlighting similarities in stability and robustness properties, not to assert a formal one-to-one mapping. We adopt this analogy solely as a heuristic for guiding intuition, and all structural conclusions are derived independently of it.

\paragraph{Contingent Objectivity.}
A central claim of this section is that emergent geometry is \emph{contingent but objective}. This phrasing may initially appear paradoxical, so we clarify its meaning through analogy with thermodynamic quantities.

Consider temperature $T$ in statistical mechanics: it is contingent on the choice of thermodynamic ensemble (microcanonical, canonical, grand canonical), yet within any given ensemble, $T$ is an objective, measurable property determined by the system's microstate distribution. No observer dependence enters once the ensemble is specified—different observers measuring the same ensemble will agree on $T$.

Similarly, in the HAFF framework, emergent geometry is contingent on the choice of accessible algebra $\mathcal{A}_{\mathbf{c}}$, which in turn is determined by the physical interaction structure (as discussed in Section~\ref{sec:B-accessibility}). Once $\mathcal{A}_{\mathbf{c}}$ is specified by the coupling pattern encoded in the interaction Hamiltonian, the induced geometry is an objective feature of the entanglement structure: different observers with access to the same algebra will infer the same metric $g_{\mu\nu}$ (up to diffeomorphism).

The contingency lies in the physical conditions that select the algebra—just as temperature depends on which statistical ensemble describes the system's preparation. But \emph{objectivity does not require uniqueness}; it requires only mind-independence given fixed physical context. A quantity can be observer-independent even if multiple such quantities exist under different physical conditions.

This resolves an apparent tension: geometry is not "fundamental" in the sense of being unique and unchanging across all contexts, but it is also not "illusory" or merely conventional. It is a stable, measurable feature of quantum correlations that emerges robustly under appropriate stability conditions, much like crystalline order emerges robustly below a critical temperature.

\subsection{The Phase Analogy}

With these caveats in place, we develop the analogy with condensed matter phases.

In statistical mechanics, different thermodynamic phases (solid, liquid, gas, magnetic, superconducting) represent distinct organizational patterns of microscopic degrees of freedom. These phases are:
\begin{itemize}
\item \textbf{Emergent}: They arise from collective behavior, not from individual constituents.
\item \textbf{Stable}: They persist under perturbations below characteristic energy scales.
\item \textbf{Observable}: They manifest in measurable order parameters.
\item \textbf{Contingent}: They depend on external conditions (temperature, pressure, fields).
\end{itemize}

We propose that emergent geometry in HAFF exhibits analogous features:

\begin{table}[h]
\centering
\small
\begin{tabular}{|P{5.5cm}|P{7cm}|}
\hline
\textbf{Condensed Matter} & \textbf{HAFF} \\
\hline
Hamiltonian $\hat{H}$ & Pre-geometric interaction graph \\
Control parameter (Temperature $T$) & Entanglement density / Scrambling rate \\
Symmetry breaking & Algebra selection \\
Order parameter $\langle M \rangle$ & Entanglement pattern $I(A:B)$ \\
Phase transition & Geometry emergence \\
Critical temperature $T_c$ & Stability threshold $\rho_{\text{crit}}$ \\
\hline
\end{tabular}
\caption{Analogy between condensed matter phases and emergent geometry. The control parameter in HAFF is entanglement density (or equivalently, the scrambling rate in large-$N$ limits), which plays a role analogous to temperature in statistical mechanics. Below a critical entanglement density, stable geometric structures emerge; above it, the system exhibits highly non-local, scrambled correlations with no coherent metric description. These correspondences are illustrative and heuristic, not exact mathematical mappings.\protect\footnotemark}
\label{tab:phase}
\end{table}

\footnotetext{In holographic models (AdS/CFT), the analogous control parameter is the ratio $\ell_{\text{AdS}}/\ell_{\text{Planck}}$, which governs the transition from classical bulk geometry to quantum gravitational regime \cite{Maldacena1999}.}

Just as ferromagnetism is a stable organizational pattern of spin alignment below the Curie temperature, geometry may be understood as a stable organizational pattern of entanglement structure under specific accessibility conditions.

\paragraph{Control Parameter and Criticality.}
A key feature of phase transitions is the existence of a \emph{control parameter} (e.g., temperature, pressure, external field) that governs which phase is realized. In the HAFF framework, the analogous control parameter is the \emph{entanglement density} of the global state $|\Psi_U\rangle$, defined operationally as the average mutual information per degree of freedom:
\begin{equation}
\rho_{\text{ent}} \equiv \frac{1}{N} \sum_{\langle i,j \rangle} I(i:j),
\end{equation}
where the sum runs over subsystem pairs and $N$ is the total number of degrees of freedom.

At low entanglement density ($\rho_{\text{ent}} \ll \rho_{\text{crit}}$), the state exhibits area-law scaling, allowing stable geometric descriptions to emerge. At high entanglement density ($\rho_{\text{ent}} \gg \rho_{\text{crit}}$), the state becomes highly scrambled, with volume-law entanglement and no coherent metric structure—analogous to the high-temperature disordered phase in spin systems.

In holographic contexts, this parameter corresponds to the ratio of bulk curvature radius to Planck length, $\ell_{\text{AdS}}/\ell_P$, which controls the transition from semiclassical geometry to stringy/quantum gravity regime \cite{Maldacena1999}.

This perspective suggests that geometry is not merely emergent but \emph{critically emergent}: it appears as a stable organizational pattern only when entanglement structure satisfies specific density constraints, much like crystalline order emerges only below the melting temperature.

\subsection{Geometric Admissibility}

Not all coarse-graining structures admit geometric interpretation. As discussed in our previous work (Definition 4.2), we require geometric admissibility conditions:
\begin{enumerate}
\item Finite correlation length (exponentially decaying correlations)
\item Monotonic decay of mutual information under refinement
\item Stability under perturbations
\end{enumerate}

These conditions are not arbitrary but reflect empirical observations from known emergent geometries:
\begin{itemize}
\item \textbf{AdS/CFT}: Boundary states with area-law entanglement give rise to smooth bulk geometries \cite{RyuTakayanagi2006}.
\item \textbf{Tensor networks}: MERA and similar structures with finite bond dimension naturally induce geometric connectivity \cite{Swingle2012}.
\item \textbf{Condensed matter}: Ground states of local Hamiltonians typically satisfy area laws and admit geometric descriptions.
\end{itemize}

Geometry, in this view, is not generic but represents a special organizational phase characterized by specific entanglement structure. This perspective explains why geometry appears in our effective descriptions: it is the stable attractor for certain classes of quantum states under physically relevant coarse-graining procedures.

\section{Scope and Future Directions}
\label{sec:B-scope}

\subsection{Structural Assumptions}

The framework rests on three core assumptions:
\begin{enumerate}
\item \textbf{Global state objectivity}: There exists a universal quantum state $|\Psi_U\rangle$, independent of observers.
\item \textbf{Algebraic priority}: Observable algebras, determined by physical interaction structure, are more fundamental than tensor factorizations.
\item \textbf{Stability-based emergence}: Effective subsystem structure and geometry emerge from stable subalgebras under physically motivated coarse-graining.
\end{enumerate}

\subsection{What the Framework Does NOT Commit To}

To prevent interpretational overreach, we explicitly enumerate what the framework does \emph{not} claim:

\begin{itemize}
\item That observers create reality or that consciousness plays a fundamental role
\item That spacetime is illusory or that geometry has no objective existence
\item That quantum mechanics is incomplete or requires modification
\item That the framework solves the measurement problem
\item That Buddhist metaphysics or any other philosophical tradition is presupposed
\end{itemize}

The framework is structurally neutral regarding these questions. It analyzes consequences of relaxing the assumption of canonical tensor factorization, but does not commit to any particular metaphysical position beyond what the formalism requires.

\subsection{Open Questions}

The following questions represent concrete technical research directions rather than fundamental gaps in the framework. Some (particularly those concerning empirical signatures) require additional physical assumptions beyond the structural analysis presented here, and are best addressed in specific model implementations.

\begin{enumerate}
\item \textbf{Dynamical algebra selection}: Can interaction Hamiltonians, coupling strengths, or network topologies be shown to select particular stable algebras over others?

\item \textbf{Information-theoretic criteria}: Do preferred geometric descriptions correlate with minimal description length, robustness under noise, or computational accessibility?

\item \textbf{Quantum field theory extensions}: How does the framework interact with locality structures in algebraic QFT? Can continuum limits be rigorously constructed?

\item \textbf{Empirical signatures}: Do different coarse-graining choices lead to distinguishable effective descriptions in semiclassical regimes or quantum gravity-motivated models?

\item \textbf{Connections to quantum complexity}: How does the framework relate to recent work on complexity-based approaches to spacetime emergence?
\end{enumerate}

These questions are intentionally left open. The present work provides a structural scaffold within which they can be formulated precisely, but does not claim to resolve them.

\subsection{Beyond Structural Analysis}

The framework developed here is deliberately limited to structural and interpretational analysis. Broader philosophical implications—concerning causation, free will, ontology, and connections to contemplative traditions—lie beyond the scope of this technical paper. Such questions are addressed in a companion philosophical essay currently in preparation.

\section{Conclusion}

We have clarified the conceptual commitments of the Holographic Alaya-Field Framework, addressing three potential misreadings:

\begin{enumerate}
\item \textbf{Against subjectivism}: Accessibility is defined structurally via stability conditions determined by physical interaction structure, not by observer beliefs or epistemic states.

\item \textbf{Against anti-realism}: Emergent geometry is a stable, measurable feature of entanglement structure—contingent on physical conditions but objective within those conditions, analogous to thermodynamic phases.

\item \textbf{Against trivialism}: Coarse-graining dependence reflects genuine physical structure (accessible algebras), not mere coordinate choice. Different algebras induce inequivalent geometries that cannot be related by diffeomorphism.
\end{enumerate}

The framework has been situated relative to AQFT (closest affinity), RQM (complementary), QBism (categorically distinct), and MWI (potentially complementary). Throughout, we have emphasized what the framework commits to structurally and what it leaves open interpretionally.

The central insight is modest but consequential: by removing tensor factorization from fundamental assumptions and treating it as emergent from coarse-graining structure, we reveal that geometry is more context-dependent than typically acknowledged—not in an epistemic or observer-relative sense, but in a structural, interaction-dependent sense.

This perspective does not resolve deep problems in quantum foundations or quantum gravity, but it clarifies the conditions under which subsystem structure and geometry emerge. By making explicit an assumption that is often left implicit, we hope to have opened new avenues for investigating the relationship between quantum states, observable structure, and spacetime.


% ============================================================================
% Essay C
% ============================================================================
\chapter{Causation, Agency, and Existence}
\label{chap:paperC}

\begin{center}
\textit{Essay C}\\[0.5em]
Originally published: Zenodo, DOI: 10.5281/zenodo.18374805
\end{center}

\bigskip

\section*{Abstract}

This essay examines the structural conditions under which agency, causation, and existence can be coherently discussed in the absence of foundational subsystem decompositions. Building on recent work in quantum information theory and algebraic approaches to quantum mechanics, Parts I--III develop a framework in which causal relations, agent boundaries, and existential claims emerge from stability properties of accessible observable algebras rather than from intrinsic substance or preferred factorizations.

Part I argues that causation can be understood as stable asymmetry within coarse-grained structures without requiring fundamental temporal ordering. Part II analyzes agency as boundary-stabilization---a non-scrambling subspace that propagates constraints without presupposing a metaphysically autonomous agent. Part III recasts existence in terms of relational form rather than intrinsic being, developing a notion of ``emptiness'' as absence of substance compatible with objectivity.

Part IV explores whether these structural features find formal parallels in Buddhist philosophical frameworks, particularly Yog\=ac\=ara and M\=adhyamaka traditions. The analysis emphasizes interpretive humility: parallels are offered as invitations to dialogue, not demonstrations of equivalence. The essay's contribution is methodological---clarifying structural constraints on emergence---rather than doctrinal.

\medskip

\noindent\textbf{Keywords}: agency, emergence, structural realism, coarse-graining, accessible algebras, Buddhist philosophy, emptiness, comparative philosophy


\tableofcontents

\section{Introduction}
\label{sec:C-intro}

Contemporary philosophy of physics faces a structural puzzle. When quantum systems lack canonical subsystem decompositions, how should we understand causation, agency, and existence? If there is no preferred way to carve reality into parts, what becomes of the conceptual apparatus built on the assumption of well-defined relata?

This essay develops a framework in which these notions emerge from \emph{stability properties of accessible algebras} rather than from intrinsic substance or preferred factorizations. The analysis proceeds in four parts.

\paragraph{Roadmap.}
Part I examines causation without temporal foundations, arguing that causal relations can be understood as stable asymmetries in coarse-grained structure. Part II analyzes agency as boundary-stabilization---the capacity of certain subsystems to maintain non-scrambling coherence while propagating constraints. Part III recasts existence in terms of relational patterns rather than intrinsic being, developing a technical sense of ``emptiness'' compatible with structural realism.

Part IV explores whether these structural features find formal parallels in Buddhist philosophical traditions, particularly Yog\=ac\=ara and M\=adhyamaka. The analysis emphasizes interpretive humility: we identify structural similarities without claiming ontological identity, historical influence, or doctrinal convergence.

\paragraph{Methodological stance.}
This is a structural investigation, not a metaphysical proposal. We do not claim that consciousness creates reality, that Buddhist texts anticipated quantum mechanics, or that comparative philosophy resolves foundational problems. Rather, we clarify how certain structural constraints---particularly the absence of canonical factorization---reshape discussions of emergence, agency, and existence across different conceptual traditions.

The essay's contribution is methodological: it demonstrates how attention to algebraic structure can discipline interpretive claims and reveal unexpected points of contact between seemingly disparate frameworks.

\paragraph{Relation to prior work.}
This essay builds on technical results developed in companion papers \cite{Liu2026PaperA,Liu2026PaperB}, which establish that inequivalent coarse-graining structures induce inequivalent effective geometries from the same global quantum state. Here, we explore philosophical consequences of this structural dependence for traditional metaphysical categories.

\section{Part I: Causation Without Foundations}
\label{sec:C-causation}

\subsection{The Standard Picture and Its Assumptions}
\label{sec:C-I.1}

Causal relations are typically understood as relations between events ordered by time. Event $A$ causes event $B$ if: (i) $A$ temporally precedes $B$, (ii) $A$ and $B$ are spatially connectible, and (iii) interventions on $A$ counterfactually affect $B$ \cite{Pearl2009}.

This picture presupposes several structural features:
\begin{itemize}
\item A well-defined notion of temporal ordering
\item Spatially localized events with clear boundaries
\item Stable subsystem decompositions supporting counterfactual reasoning
\end{itemize}

In quantum contexts without canonical factorization, none of these features is guaranteed. Time may be emergent rather than fundamental \cite{PageWootters1983}. Spatial locality depends on choice of coarse-graining \cite{Liu2026PaperA}. Subsystem boundaries are algebra-dependent rather than intrinsic.

\subsection{Causation as Stable Asymmetry}
\label{sec:C-I.2}

We propose that causation can be understood as \emph{stable asymmetry in accessible structure}, without requiring fundamental temporal ordering.

\begin{definition}[Causal Structure as Asymmetric Accessibility]
\label{def:C-causal-asymmetry}
Let $\mathcal{A}_{\mathbf{c}}$ be an accessible algebra determined by coarse-graining structure $\mathbf{c}$. A \textbf{causal relation} between subsystems $A$ and $B$ exists if there is a stable asymmetry in their correlation structure:
\begin{equation}
I(A_{\text{past}}:B_{\text{future}}) > I(B_{\text{past}}:A_{\text{future}}),
\end{equation}
where mutual information is computed relative to $\mathcal{A}_{\mathbf{c}}$, and ``past/future'' refer to coarse-graining-dependent orderings that admit thermodynamic interpretation.
\end{definition}

This definition makes no reference to fundamental time. Instead, it identifies causation with robust directional structure in how information propagates through accessible degrees of freedom.

\paragraph{Multiple causal arrows and thermodynamic consistency.}
An important subtlety: different coarse-graining structures may induce distinct---and potentially conflicting---causal arrows from the same global state. Since the ``past/future'' labels in Definition \ref{def:C-causal-asymmetry} are grounded in thermodynamic gradients, and thermodynamics itself is coarse-graining-dependent \cite{Wallace2012Time}, multiple inequivalent causal structures may coexist.

This is not a defect but a structural feature: just as different accessible algebras induce different effective geometries \cite{Liu2026PaperA}, they may induce different effective causal orderings. Consistency requires only that causal arrows align with entropy increase within each coarse-graining context. Conflicts between causal arrows derived from inequivalent coarse-grainings reflect genuine structural inequivalence, not mere coordinate choice.

In physical systems, thermodynamic consistency conditions typically select compatible coarse-grainings---those yielding aligned causal arrows at macroscopic scales. But in principle, the framework admits context-dependent causal structure, with no unique ``fundamental'' arrow privileged independently of accessibility constraints.

\subsection{Thermodynamic Grounding}
\label{sec:C-I.3}

The asymmetry in Definition \ref{def:C-causal-asymmetry} can be grounded in thermodynamic considerations. Systems approaching equilibrium exhibit increasing entropy, inducing a preferred temporal direction even when microscopic dynamics are time-symmetric \cite{Wallace2012Time}.

Crucially, this thermodynamic arrow is \emph{context-dependent}: it depends on which macrostates are accessible, which in turn depends on coarse-graining structure. Different accessible algebras may induce different thermodynamic gradients, hence different effective causal structures.

\subsection{Counterfactuals Without Intrinsic Relata}
\label{sec:C-I.4}

Interventionist accounts of causation rely on counterfactual reasoning: $A$ causes $B$ if intervening on $A$ would change $B$ \cite{Woodward2003}. This appears to require well-defined intervention targets---subsystems $A$ and $B$ with stable identities.

However, counterfactuals can be reformulated in algebraic terms. An intervention on $A$ corresponds to applying a CPTP map $\Phi_{\text{int}}$ to observables in subalgebra $\mathcal{A}_A$. The counterfactual dependence of $B$ on $A$ is then measured by how sensitive observables in $\mathcal{A}_B$ are to perturbations of $\mathcal{A}_A$.

This reformulation makes no reference to intrinsic subsystem boundaries. Intervention targets are defined by accessible algebras, which are themselves context-dependent.

\subsection{Memory as Informational Constraint}
\label{sec:C-I.5}

Causal relations leave traces---memory records that constrain future accessible states. In quantum systems, memory can be understood as constraint propagation through entanglement structure \cite{Hayden2007}.

A subsystem $M$ acts as a memory of event $A$ if observables in $\mathcal{A}_M$ remain correlated with past observables in $\mathcal{A}_A$ despite environmental decoherence:
\begin{equation}
I(A_{\text{past}}:M_{\text{present}}) \gg I(A_{\text{past}}:E_{\text{present}}),
\end{equation}
where $E$ represents generic environmental degrees of freedom.

Memory, on this account, is not storage of intrinsic properties but maintenance of relational structure across time---a pattern of correlations that persists under dynamical evolution.

\subsection{Worked Example: Two-Qubit Causal Asymmetry}
\label{sec:C-I.5example}

To clarify Definition \ref{def:C-causal-asymmetry}, we present a minimal worked example using a two-qubit system.

\paragraph{Setup.}
Consider a composite system of two qubits, $A$ and $B$, initially in a maximally entangled Bell state:
\begin{equation}
|\Psi_{\text{AB}}\rangle = \frac{1}{\sqrt{2}}\left(|00\rangle + |11\rangle\right).
\end{equation}

We introduce asymmetric environmental coupling: qubit $A$ couples strongly to a thermal bath $E_A$, while $B$ remains weakly coupled to $E_B$. The total Hamiltonian includes:
\begin{equation}
\hat{H} = \hat{H}_A + \hat{H}_B + \lambda_A \hat{\sigma}_A^z \otimes \hat{B}_{E_A} + \lambda_B \hat{\sigma}_B^z \otimes \hat{B}_{E_B},
\end{equation}
where $\lambda_A \gg \lambda_B$, and $\hat{B}_{E_i}$ are bath operators.

\paragraph{Coarse-graining.}
We trace over environmental degrees of freedom, defining accessible algebra $\mathcal{A}_{\mathbf{c}} = \text{span}\{\hat{\sigma}_A^x, \hat{\sigma}_A^z, \hat{\sigma}_B^x, \hat{\sigma}_B^z\}$ (Pauli observables).

\paragraph{Dynamical evolution.}
Due to asymmetric decoherence, the reduced density matrix evolves as:
\begin{align}
\rho_{AB}(0) &= |\Psi_{\text{AB}}\rangle\langle\Psi_{\text{AB}}|, \\
\rho_{AB}(t) &\approx \frac{1}{2}\left(|00\rangle\langle 00| + e^{-\Gamma_A t}|01\rangle\langle 10| + e^{-\Gamma_A t}|10\rangle\langle 01| + |11\rangle\langle 11|\right),
\end{align}
where $\Gamma_A \propto \lambda_A^2$ is the decoherence rate for qubit $A$.

\paragraph{Causal asymmetry.}
We compute mutual information at different times:
\begin{align}
I(A_{\text{early}}:B_{\text{late}}) &= S(A_{\text{early}}) + S(B_{\text{late}}) - S(AB), \\
I(B_{\text{early}}:A_{\text{late}}) &= S(B_{\text{early}}) + S(A_{\text{late}}) - S(AB).
\end{align}

At $t = 0$, mutual information is symmetric: $I(A:B) = 2 \log 2$ (maximal entanglement).

At $t \gg \Gamma_A^{-1}$, qubit $A$ has fully decohered while $B$ retains coherence longer. Measuring $A$ early provides information about $B$ late, but not vice versa:
\begin{equation}
I(A_{\text{early}}:B_{\text{late}}) > I(B_{\text{early}}:A_{\text{late}}).
\end{equation}

\paragraph{Interpretation.}
The asymmetry arises from differential coupling to environments---a thermodynamic gradient inducing directional information flow. Qubit $A$ acts as a ``past'' influence on $B$ (causal), while $B$ does not significantly constrain $A$'s future (non-causal in reverse direction).

This exemplifies Definition \ref{def:C-causal-asymmetry}: causal structure emerges from stable asymmetry in coarse-grained correlations, grounded in thermodynamic irreversibility, without presupposing fundamental temporal ordering.

\paragraph{Context-dependence.}
Crucially, if we had chosen a different coarse-graining---say, tracing over qubit degrees of freedom and retaining environmental observables---the causal arrow could reverse or disappear. The asymmetry is \emph{real} (measurable within $\mathcal{A}_{\mathbf{c}}$) but \emph{context-dependent} (algebra-relative).

\subsection{From Causation to Agency}
\label{sec:C-I.6}

The transition from causation to agency requires an additional structural feature: \emph{stable subsystem boundaries that support constraint propagation}.

An agent is not merely a locus of causal influence, but a subsystem capable of maintaining coherent constraint propagation despite environmental coupling. This suggests analyzing agency in terms of \emph{non-scrambling subspaces}---degrees of freedom that resist rapid information delocalization \cite{Hayden2007}.

Part II develops this connection, arguing that agency emerges from boundary-stabilization rather than metaphysical autonomy.

\paragraph{Summary.} Causation can be understood as stable asymmetry in coarse-grained accessible structure. This account makes no reference to fundamental time, intrinsic relata, or metaphysically basic events. Causal structure is context-dependent but objective---determined by physical interaction patterns rather than observer beliefs.

\section{Part II: Agency as Emergent Constraint Structure}
\label{sec:C-agency}

\subsection{The Problem of Autonomous Agents}
\label{sec:C-II.1}

Traditional accounts treat agents as metaphysically autonomous entities---unified subjects possessing intrinsic intentionality and causal efficacy. This picture faces two challenges in quantum contexts without canonical factorization.

First, if subsystem boundaries are algebra-dependent, what distinguishes an ``agent'' from an arbitrary collection of degrees of freedom? Second, if quantum dynamics are unitary and deterministic at the global level, how can agents possess genuine causal autonomy?

We argue that both challenges dissolve once agency is recast as a structural feature---boundary-stabilization under constraint propagation---rather than a metaphysical primitive.

\paragraph{Physical grounding of algebra selection.}
Before explicating the structural role of accessible algebras, we briefly address the dynamical origin of their selection. As established in our companion analysis regarding stability conditions \cite{Liu2026PaperB}, the specific observable algebra $\mathcal{A}_{\mathbf{c}}$ is not determined by arbitrary subjective choice or metaphysical agency. Rather, it is physically selected by the system's interaction structure---specifically, by the requirement that the algebra remains robust under environmental decoherence (quantum Darwinism) and stable over relevant timescales. The ``filter'' is thus instantiated by the objective Hamiltonian couplings of the underlying field, ensuring that the emergence of effective geometry is grounded in physical dynamics rather than intentionality. We take this stability-selected structure as the starting point for the following structural analysis.

\subsection{Agency as Boundary-Stabilization}
\label{sec:C-II.2}

An \emph{agent}, in the structural sense, is a subsystem whose boundaries remain stable under dynamical evolution and whose internal degrees of freedom exhibit coordinated constraint propagation.

\begin{definition}[Agent-Like Subsystem]
\label{def:C-agent-subsystem}
A subsystem $\mathcal{A}_{\text{agent}}$ exhibits \textbf{agent-like behavior} if:
\begin{enumerate}
\item \textbf{Boundary stability}: The subalgebra $\mathcal{A}_{\text{agent}}$ is approximately preserved under physically relevant dynamical maps:
\begin{equation}
\|\mathcal{E}_t(\mathcal{A}_{\text{agent}}) - \mathcal{A}_{\text{agent}}\| < \epsilon
\end{equation}
for timescales relevant to constraint propagation.

\item \textbf{Non-scrambling coherence}: Observables in $\mathcal{A}_{\text{agent}}$ exhibit slow out-of-time-order correlator (OTOC) growth:
\begin{equation}
\langle [\hat{O}_{\text{agent}}(t), \hat{V}(0)]^2 \rangle \ll 1 \quad \text{for } t \ll \tau_{\text{scrambling}}.
\end{equation}

\item \textbf{Constraint propagation}: Internal correlations support directed information flow toward system boundaries, enabling intervention on environmental degrees of freedom.
\end{enumerate}
\end{definition}

\begin{remark}[Operational Threshold for Meaningful Agency]
\label{rem:agency-threshold}
Definition \ref{def:C-agent-subsystem} characterizes agent-like behavior structurally, but does not specify when such behavior constitutes \emph{meaningful} or \emph{significant} agency. In highly entangled quantum systems, transient non-scrambling may occur at very short timescales without supporting sustained constraint propagation.

We suggest an operational threshold: a subsystem exhibits \textbf{operationally significant agency} if:
\begin{equation}
\tau_{\text{coherence}} \gg \tau_{\text{env}},
\end{equation}
where $\tau_{\text{coherence}}$ is the timescale over which $\mathcal{A}_{\text{agent}}$ maintains boundary stability and non-scrambling coherence, and $\tau_{\text{env}}$ is the characteristic environmental decoherence time.

More precisely, we require:
\begin{equation}
\frac{\tau_{\text{coherence}}}{\tau_{\text{env}}} > \kappa,
\end{equation}
where $\kappa \sim 10^2$--$10^3$ is an empirically determined threshold below which constraint propagation becomes operationally inaccessible.

For biological agents, $\tau_{\text{coherence}}$ spans seconds to hours (for cognitive processes) or years (for identity persistence), while $\tau_{\text{env}} \sim 10^{-13}$--$10^{-3}$ seconds (molecular to neural timescales). For quantum systems at room temperature, $\tau_{\text{env}} \sim 10^{-15}$--$10^{-12}$ seconds, making sustained agency extraordinarily rare without active error correction or topological protection.

This threshold distinguishes:
\begin{itemize}
\item \textbf{Transient non-scrambling}: Fluctuations in highly entangled systems (no operational agency)
\item \textbf{Sustained boundary-stabilization}: Persistent subsystems supporting intervention (operational agency)
\end{itemize}

The threshold is not sharp---agency admits degrees---but provides a quantitative criterion for when agent-like structure becomes empirically significant.
\end{remark}

This definition makes no reference to consciousness, phenomenology, or intrinsic intentionality. Agency is characterized purely in terms of stability properties and information-theoretic structure.

\subsection{Will as Constraint Propagation}
\label{sec:C-II.3}

What, then, becomes of ``will'' or ``intention'' in this framework?

We suggest that will can be understood as \emph{constraint propagation structure}---patterns of internal correlation that bias future accessible states toward specific outcomes. An agent ``wills'' action $A$ if its internal state $\rho_{\text{agent}}$ is such that future measurements will register correlation with $A$ with high probability.

\paragraph{Analogy: Thermostat control.}
Consider a thermostat coupled to a heating system. The thermostat's internal state (temperature reading) constrains future system behavior (heater activation), despite having no phenomenological experience. This is constraint propagation without metaphysical agency.

The analogy is limited: biological agents exhibit far richer constraint structure. But it clarifies the conceptual move---replacing metaphysical autonomy with structural analysis of how internal states bias future trajectories.

\subsection{The Phenomenology-Structure Gap}
\label{sec:C-II.4}

An immediate objection: this account leaves no room for phenomenology---the felt quality of agency, the subjective sense of ``I am acting.''

We acknowledge this gap. The framework developed here is \emph{structural}, concerned with information-theoretic organization. It does not address the \emph{explanatory gap} between structure and experience \cite{Levine1983,Chalmers1996}.

\begin{tcolorbox}[colback=gray!5!white,colframe=black!75!black,title=\textbf{Speculative Connection: Neural Constraint Propagation}]
\textbf{Caveat}: The following connects structural features to neuroscience, but remains speculative. We note these as suggestive parallels, not established mechanisms.

Recent work in computational neuroscience suggests that neural ``will'' may correspond to hierarchical constraint propagation through cortical-basal ganglia loops \cite{Graybiel2008}. Habitual behavior emerges when constraint patterns stabilize, while ``voluntary'' action involves flexible reconfiguration of these patterns \cite{Yin2006}.

If agency is boundary-stabilization, then the phenomenology of ``willing'' may correspond to proprioceptive monitoring of constraint reconfiguration---the felt sense of internal degrees of freedom reorganizing in preparation for action \cite{Haggard2005}.

This remains highly speculative and does not bridge the explanatory gap. We raise it only to illustrate how structural analysis might interface with empirical research programs.
\end{tcolorbox}

\subsection{Degrees of Agency}
\label{sec:C-II.5}

The framework suggests that agency is not all-or-nothing, but admits degrees. Systems exhibit more or less agent-like behavior depending on:
\begin{itemize}
\item \textbf{Boundary stability duration}: How long does $\mathcal{A}_{\text{agent}}$ remain well-defined?
\item \textbf{Scrambling timescale}: How quickly do internal correlations delocalize?
\item \textbf{Constraint propagation fidelity}: How reliably do internal states bias future trajectories?
\end{itemize}

Simple thermostats exhibit minimal agency (short timescales, limited constraint structure). Biological organisms exhibit far richer agency (extended coherence, complex constraint networks). But both are comprehensible within the same structural framework.

\subsection{Relation to Free Will Debates}
\label{sec:C-II.6}

Traditional free will debates ask: are agents causally autonomous, or are their actions determined by prior states and laws? This presupposes well-defined agent boundaries and unambiguous causal histories---precisely what the framework questions.

On the structural account, the relevant question is not ``Are agents free?'' but ``Under what conditions do subsystems exhibit stable constraint-propagation structure?'' This reformulation may dissolve certain traditional impasses while opening new empirical questions about stability conditions and scrambling timescales.

We do not claim to resolve free will debates---only to clarify how they depend on assumptions about subsystem structure that are themselves context-dependent.

\paragraph{Summary.} Agency can be understood as boundary-stabilization supporting constraint propagation, rather than metaphysical autonomy. This account is eliminativist about intrinsic intentionality but realist about structural patterns of constraint. It leaves the phenomenology-structure gap unresolved but clarifies the information-theoretic conditions under which agent-like subsystems emerge.

\section{Part III: Existence Without Substance}
\label{sec:C-existence}

\subsection{The Realism Problem}
\label{sec:C-III.1}

Parts I--II analyzed causation and agency as context-dependent but objective---dependent on coarse-graining structure but not on observer beliefs. This raises an existential question: if fundamental structures (subsystems, geometries, agents) are context-dependent, what ontological status do they possess?

Two extremes must be avoided:
\begin{enumerate}
\item \textbf{Naive realism}: Treating emergent structures as metaphysically fundamental, ignoring their context-dependence.
\item \textbf{Anti-realism}: Denying objective reality to context-dependent structures, collapsing into subjectivism.
\end{enumerate}

We propose a middle path: \emph{structural realism about relational patterns}. Existence is understood not as possession of intrinsic properties, but as participation in stable relational structures.

\subsection{Form Without Substance}
\label{sec:C-III.2}

Consider the temperature of a gas. Temperature is \emph{context-dependent}: it depends on which degrees of freedom are macroscopically accessible. Different coarse-grainings may yield different effective temperatures for the same microstate.

Yet temperature is not merely subjective. Given a coarse-graining, temperature is an objective, measurable quantity with predictive power. It is \emph{relationally real}---real within a specified context, but lacking intrinsic existence independent of that context.

\begin{definition}[Relational Existence]
\label{def:C-relational-existence}
An entity $X$ possesses \textbf{relational existence} relative to structure $\mathcal{S}$ if:
\begin{enumerate}
\item $X$ is well-defined and stable within $\mathcal{S}$
\item $X$ participates in objective relational patterns (correlations, symmetries, invariants)
\item $X$ may be absent or differently constituted under alternative structures $\mathcal{S}'$
\end{enumerate}
\end{definition}

This notion captures \emph{form without substance}: patterns that are objectively real without possessing intrinsic being.

\subsection{Emptiness as Technical Concept}
\label{sec:C-III.3}

We introduce \emph{emptiness} as a technical term denoting absence of intrinsic existence compatible with relational reality.

\begin{definition}[Emptiness (Technical Sense)]
\label{def:C-emptiness-technical}
An entity $X$ is \textbf{empty} (in the technical sense) if:
\begin{enumerate}
\item $X$ lacks intrinsic being independent of relational context
\item $X$ exhibits stable patterns within specified contexts
\item The absence of intrinsic being does not entail non-existence or illusoriness
\end{enumerate}
\end{definition}

This usage is stipulative and should not be confused with colloquial meanings (``containing nothing'') or metaphysical nihilism (``nothing really exists''). Emptiness, in this technical sense, is compatible with robust realism about relational structures.

\subsection{Examples of Relational Existence}
\label{sec:C-III.4}

\paragraph{Temperature.}
As discussed in \S\ref{sec:C-III.2}, temperature is context-dependent but objective. Different coarse-grainings yield different effective temperatures, yet temperature remains a genuine physical quantity within each context.

\paragraph{Quasiparticles in condensed matter.}
Phonons, magnons, and other quasiparticles are collective excitations---emergent entities with no counterpart in the microscopic Hamiltonian. They are empty of intrinsic being (there are no ``phonon particles'' in fundamental theory) yet fully real within effective descriptions \cite{Ladyman2007}.

\emph{Operationally}, quasiparticles are defined by stable relational patterns in measurable observables:
\begin{itemize}
\item \textbf{Spectral weight}: Well-defined peaks in momentum-resolved spectroscopy ($A(\mathbf{k},\omega)$)
\item \textbf{Dispersion relations}: Stable functional dependence $\omega(\mathbf{k})$ across parameter ranges
\item \textbf{Finite lifetimes}: Decay rates $\Gamma(\mathbf{k})$ obeying systematic scaling laws
\item \textbf{Scattering cross-sections}: Reproducible interaction amplitudes in transport experiments
\end{itemize}

These observables are context-dependent---they depend on temperature, pressure, doping, and measurement resolution---yet objectively real within specified experimental contexts. This exemplifies relational existence: patterns that are measurable, predictive, and stable, despite lacking intrinsic being independent of effective theory.

\paragraph{Subsystems in quantum mechanics.}
As established in prior work \cite{Liu2026PaperA}, subsystem decompositions are coarse-graining-dependent. A quantum state may admit infinitely many inequivalent factorizations, none privileged. Yet within any given factorization, subsystems exhibit objective entanglement structure and support meaningful predictions.

These examples illustrate a common pattern: entities that are \emph{empty} (lacking intrinsic being) yet \emph{existent} (participating in stable relational structures).

\subsection{Structural Invariants and Objectivity}
\label{sec:C-III.5}

A potential objection: if everything is context-dependent, what grounds objectivity?

The answer lies in \emph{structural invariants}---features preserved across context transformations. While subsystem decompositions are coarse-graining-dependent, certain global properties (total entropy, symmetry groups, topological invariants) remain well-defined independently of factorization.

Objectivity does not require context-independence. It requires only that relational patterns exhibit stability and predictive power within specified contexts, and that transformations between contexts preserve identifiable structural features.

This perspective aligns with \emph{structural realism} in philosophy of science: what is objectively real is relational structure, not intrinsic substance \cite{Ladyman2007,French2014}.

\subsection{Existence and Non-Existence}
\label{sec:C-III.6}

The framework suggests a taxonomy of existential claims:

\begin{table}[h]
\centering
\small
\begin{tabular}{|P{3.5cm}|P{4.5cm}|P{5cm}|}
\hline
\textbf{Type} & \textbf{Characterization} & \textbf{Example} \\
\hline
Intrinsic existence & Context-independent being & Classical particles (if fundamental) \\
Relational existence & Context-dependent but objective & Temperature, subsystems \\
Conventional existence & Context-dependent and agent-relative & Money, legal rights \\
Non-existence & Absent from all relevant contexts & Phlogiston, luminiferous ether \\
\hline
\end{tabular}
\caption{Taxonomy of existential claims. The framework developed here concerns relational existence---entities that are context-dependent but objective. These correspondences are analytical distinctions, not ontological commitments.}
\label{tab:existence-taxonomy}
\end{table}

Emergent structures discussed in Parts I--II (causal relations, agent boundaries, geometric features) belong to the second category: relationally existent. They are empty of intrinsic being yet objectively real within specified coarse-graining contexts.

\paragraph{Summary.} Existence can be understood in terms of relational patterns rather than intrinsic substance. ``Emptiness,'' in the technical sense developed here, denotes absence of intrinsic being compatible with objectivity. This perspective avoids both naive realism and anti-realist eliminativism, offering a middle path grounded in structural invariance.

\section{Part IV: Interpretive Bridges to Buddhist Philosophy}
\label{sec:C-bridges}

\subsection{Methodological Preface}
\label{sec:C-IV.1}

The structural features developed in Parts I--III---causation as asymmetry, agency as boundary-stabilization, existence as relational form---were derived independently of any particular metaphysical tradition. We now explore whether these features find formal parallels in Buddhist philosophical frameworks.

Several caveats are essential:

\begin{tcolorbox}[colback=gray!5!white,colframe=black!75!black,title=\textbf{Methodological Caveats}]
\begin{enumerate}
\item \textbf{No historical causation}: We do not claim that Buddhist texts influenced quantum mechanics, or vice versa. Any parallels are structural convergences, not genealogical connections.

\item \textbf{No doctrinal advocacy}: Identifying formal similarities does not constitute endorsement of Buddhist metaphysics, soteriology, or religious practices.

\item \textbf{No cultural essentialism}: Buddhism comprises diverse traditions spanning two millennia. References here focus on specific textual traditions (primarily Yog\=ac\=ara and M\=adhyamaka), not ``Buddhism'' as a monolithic whole.

\item \textbf{No mystical reduction}: We reject interpretations conflating quantum mechanics with consciousness studies, New Age thought, or perennialist philosophy. Our analysis is structural and comparative, not mystical.

\item \textbf{Interpretive humility}: Parallels are offered as invitations to dialogue, not demonstrations of equivalence. The comparative exercise is exploratory, not conclusive.
\end{enumerate}
\end{tcolorbox}

With these caveats in place, we proceed to examine possible structural correspondences.

\subsection{Yog\=ac\=ara and Accessible Algebras}
\label{sec:C-IV.2}

Yog\=ac\=ara (``Yoga practice'') is a Mah\=ay\=ana Buddhist philosophical school emphasizing the role of consciousness (\emph{vija±\=ana}) in constituting experienced reality. Central to Yog\=ac\=ara is the concept of \textbf{\=alaya-vija±\=ana} (Skt.\ \textit{\=alayavij\~n\=ana}, ``storehouse consciousness''), a foundational stratum of mind that stores karmic seeds (\emph{b\={\i}ja}) conditioning future experience \cite{Asanga_Mahayana,Vasubandhu_Trimsika}.

A structural parallel may be noted: \=alaya-vija±\=ana functions as a holistic substrate from which individuated mental states emerge, analogous to how subsystem structures emerge from coarse-graining a global quantum state \cite{Lusthaus2002,Waldron2003}.

\begin{table}[h]
\centering
\small
\begin{tabular}{|P{6.5cm}|P{6.5cm}|}
\hline
\textbf{Yog\=ac\=ara Concept} & \textbf{HAFF Analog} \\
\hline
\=Alaya-vija±\=ana (storehouse consciousness) & Global quantum state $|\Psi_U\rangle$ \\
B\={\i}ja (karmic seeds) & Eigenmodes of accessible algebras \\
Prav\d{r}tti-vija±\=ana (active consciousness) & Effective subsystem $\rho_{\text{eff}}$ \\
\=Asraya-par\=av\d{r}tti (basis-transformation) & Coarse-graining map $\Phi_{\mathbf{c}}$ \\
\hline
\end{tabular}
\caption{Possible formal parallels between Yog\=ac\=ara concepts and HAFF structures. \textbf{Disclaimer}: These correspondences are formal analogies highlighting structural similarities, not claims of ontological identity or historical influence. Yog\=ac\=ara is a soteriological framework concerned with liberation from suffering; HAFF is a structural analysis of quantum emergence. The table illustrates conceptual resonances, not equivalences.}
\label{tab:yogacara-parallel}
\end{table}

\begin{tcolorbox}[colback=gray!5!white,colframe=black!75!black,title=\textbf{Critical Clarification: Against Panpsychism}]
The structural parallel between \=alaya-vija±\=ana and the global quantum state $|\Psi_U\rangle$ does \textbf{not} imply:
\begin{itemize}
\item That quantum states possess consciousness or phenomenological properties
\item That the universe is fundamentally mental or experiential (idealism)
\item That matter is constituted by or reducible to mind (panpsychism)
\item That information or quantum information is intrinsically conscious
\end{itemize}

The analogy is \emph{purely structural}: both frameworks describe how individuated entities emerge from holistic substrates via coarse-graining or cognitive filtering. Yog\=ac\=ara's substrate is explicitly mental (\emph{vija±\=ana}, consciousness); HAFF's substrate is explicitly physical (quantum state in Hilbert space).

Any appearance of convergence concerns \emph{formal pattern}---the structure of emergence---not ontological content. We emphatically reject interpretations that would construe this parallel as supporting quantum consciousness theories, New Age mysticism, or perennialist claims about universal mind.

The comparison is offered in the spirit of \emph{structural analogy}, not metaphysical synthesis.
\end{tcolorbox}

However, critical divergences must be noted:
\begin{itemize}
\item Yog\=ac\=ara is primarily concerned with \emph{mental} phenomena and the path to liberation, while HAFF analyzes \emph{physical} structure without soteriological commitments.
\item \=Alaya-vija±\=ana is explicitly described as a form of consciousness, while $|\Psi_U\rangle$ is not imbued with phenomenological properties.
\item The Yog\=ac\=ara framework is embedded in Buddhist ethics and meditation practice, which have no counterpart in HAFF's purely structural analysis.
\end{itemize}

The parallel, if valid, concerns \emph{formal structure}---both frameworks describe how individuated entities emerge from holistic substrates---not phenomenological or ontological content.

\subsection{M\=adhyamaka and Emptiness}
\label{sec:C-IV.3}

The M\=adhyamaka (``Middle Way'') tradition, founded by N\=ag\=arjuna (c.\ 2nd century CE), articulates \textbf{\'s\={u}nyat\=a} (Skt.\ \textit{\'s\=unyat\=a}, ``emptiness'') as the absence of \textbf{svabh\=ava} (Skt.\ \textit{svabh\=ava}, ``intrinsic nature'' or ``own-being'') \cite{Nagarjuna_MMK_Garfield1995}. 

This concept bears structural resemblance to the notion of ``emptiness'' developed in Part III (\S \ref{sec:C-III.3}), where it denoted absence of intrinsic being without entailing non-existence. We emphasize that this parallel concerns the \emph{structural form} of the claim---denial of substance while affirming functional reality---not historical influence or causal connection \cite{Garfield2002,Siderits2007}.

N\=ag\=arjuna's central argument proceeds via \emph{prasa\.{n}ga} (reductio) reasoning: all phenomena are empty because they arise dependently (\emph{prat\={\i}tyasamutp\=ada}), and what arises dependently cannot possess intrinsic nature. This is formalized in the famous verse:
\begin{quote}
\emph{``Whatever arises dependently is said to be empty. That, being a dependent designation, is itself the middle way.''} (MMK 24:18) \cite{Nagarjuna_MMK_Garfield1995}
\end{quote}

A structural reading: entities lacking intrinsic being (empty) can nonetheless participate in stable relational networks (dependent arising). This maps onto the framework developed in \S \ref{sec:C-III.4}: subsystems are empty (coarse-graining-dependent, lacking intrinsic factorization) yet existent (exhibiting objective entanglement structure).

\paragraph{Recursive emptiness and second-order structure.}
A subtle question arises: in M\=adhyamaka, emptiness applies universally, including to emptiness itself---``emptiness is empty'' (\emph{\'s\={u}nyat\=a-\'s\={u}nyat\=a}). Does HAFF's relational existence admit similar recursive application?

The answer is affirmative in a formal sense. Observable algebras $\mathcal{A}_{\mathbf{c}}$ are themselves relationally defined: they depend on physical interaction structure (Hamiltonian coupling), experimental apparatus constraints, and resolution limitations. There is no ``algebra of all algebras'' existing independently of physical context.

Moreover, the coarse-graining maps $\Phi_{\mathbf{c}}$ that select accessible algebras are context-dependent: different experimental setups, measurement resolutions, or dynamical timescales induce different $\Phi$ structures. Thus, \emph{the apparatus of emergence is itself emergent}---coarse-graining structure arises from prior coarse-graining choices in a potentially infinite regress.

This mirrors M\=adhyamaka's insight that even the \emph{tools of analysis} (concepts, language, logical operations) are empty---lacking intrinsic being while remaining functionally effective. In HAFF, even the ``machinery'' of accessible algebras is context-dependent, yet this does not undermine objectivity: structural invariants (entanglement entropy, symmetry groups) remain well-defined across contexts.

The parallel is formal: both frameworks acknowledge that \emph{relational structure goes ``all the way down,''} with no metaphysically foundational level immune to context-dependence. However, M\=adhyamaka deploys this insight soteriologically (to undermine attachment to fixed views), while HAFF employs it descriptively (to clarify structural constraints on emergence).

\paragraph{Key divergence.}
M\=adhyamaka \'s\={u}nyat\=a is deployed soteriologically---to undermine attachment to fixed views and facilitate liberation. HAFF's ``emptiness'' is a technical descriptor of relational structure, with no soteriological function. The similarity is formal, not practical or existential.

\subsection{Karma and Constraint Propagation}
\label{sec:C-IV.4}

Buddhist karma doctrine holds that intentional actions leave traces (\emph{sa\.msk\=ara}, ``formations'') that condition future experience. In Yog\=ac\=ara, these traces are stored in \=alaya-vija±\=ana as \emph{v\=asan\=a} (``karmic impressions'') \cite{Waldron2003}.

A structural analogy: karma may be understood as \emph{constraint propagation through entanglement structure}---past actions (interventions on accessible algebras) leave informational imprints that bias future trajectories \cite{Gombrich1996,Harvey2000}.

This is consonant with the account of memory developed in \S \ref{sec:C-I.5}: causal traces are not storage of intrinsic properties but maintenance of relational structure across time.

\paragraph{Critical limitation.}
Buddhist karma is inherently normative: actions are classified as wholesome (\emph{ku\'sala}) or unwholesome (\emph{aku\'sala}) based on ethical criteria and soteriological consequences. HAFF's constraint propagation is descriptive, lacking normative content. The formal parallel does not extend to ethical or soteriological dimensions.

\subsection{Limits and Divergences}
\label{sec:C-IV.5}

Having noted possible parallels, we now emphasize substantive divergences---areas where Buddhist frameworks and HAFF diverge structurally, methodologically, or conceptually. This section is weighted equally to \S \ref{sec:C-IV.2}--\ref{sec:C-IV.4} to prevent over-interpreting formal similarities.

\subsubsection{Phenomenological vs.\ Structural Orientation}

Buddhist philosophy is fundamentally concerned with first-person experience and liberation from suffering (\emph{du\d{h}kha}). Yog\=ac\=ara and M\=adhyamaka analyze consciousness, perception, and mental afflictions (\emph{kle\'sa}) as prerequisites for soteriological transformation.

HAFF, by contrast, is a third-person structural framework with no phenomenological commitments. It analyzes information-theoretic organization without addressing subjective experience, qualia, or the explanatory gap between structure and consciousness.

\textbf{Consequence}: Any parallel between \=alaya-vija±\=ana and $|\Psi_U\rangle$ cannot extend to phenomenological dimensions. HAFF does not explain consciousness, nor does it claim that quantum states possess mental properties.

\subsubsection{Soteriological vs.\ Descriptive Goals}

Buddhist frameworks are \emph{pragmatic} in orientation: concepts are introduced to facilitate liberation, not to accurately describe metaphysical reality. As the Buddha reportedly stated, philosophical speculation is a ``thicket of views'' (\emph{di\d{t}\d{t}hi-gaha\d{n}a}) distracting from the path \cite{SuttaNipata}.

HAFF is \emph{descriptive}: it aims to clarify structural constraints on emergence without prescribing practices or soteriological goals. There is no ``path'' in HAFF, no liberation to achieve, no suffering to overcome.

\textbf{Consequence}: Structural parallels do not imply that HAFF serves Buddhist soteriological purposes, nor that Buddhist practice requires acceptance of quantum mechanics.

\subsubsection{Rebirth, Cosmology, and Ethics}

Traditional Buddhist cosmology includes rebirth across multiple realms, karmic causation spanning lifetimes, and detailed ethical taxonomies. These elements are absent from---and irrelevant to---HAFF's structural analysis.

HAFF makes no claims about:
\begin{itemize}
\item Post-mortem continuity of consciousness
\item Karmic retribution across lifetimes
\item Ethical status of actions
\item Cosmological realms or deities
\item Meditation practices or contemplative attainments
\end{itemize}

\textbf{Consequence}: Identifying formal parallels does not validate Buddhist cosmology, rebirth doctrine, or ethical systems. The frameworks operate in disjoint conceptual spaces.

\subsubsection{Ontological Commitments}

While both frameworks reject intrinsic substance, they differ in ontological commitments:
\begin{itemize}
\item \textbf{Buddhist frameworks} (particularly Yog\=ac\=ara) often privilege mind or consciousness as fundamental, with matter derivative.
\item \textbf{HAFF} remains neutral on mind-matter relations, analyzing quantum structure without metaphysical commitments about consciousness.
\end{itemize}

Additionally, M\=adhyamaka's ``two truths'' doctrine---distinguishing conventional (\emph{sa\.mv\d{r}ti}) from ultimate (\emph{param\=artha}) reality---has no clear analog in HAFF. HAFF distinguishes context-dependent from invariant structures, but this is epistemological (about what can be known) rather than ontological (about levels of reality).

\subsubsection{Methodological Incommensurability}

Buddhist philosophy employs contemplative introspection, textual hermeneutics, and dialectical reasoning as primary methods. HAFF employs mathematical formalism, operator algebras, and information theory.

These methodologies are not mutually translatable. One cannot \emph{meditate} one's way to understanding quantum entanglement, nor \emph{calculate} one's way to soteriological insight. The parallels identified are \emph{structural}, not methodological.

\paragraph{Summary of divergences.}
The frameworks differ in:
\begin{enumerate}
\item Explanatory target (phenomenology vs.\ structure)
\item Pragmatic goal (liberation vs.\ description)
\item Scope (ethics/cosmology vs.\ physics)
\item Ontological commitments (consciousness-first vs.\ neutral)
\item Methodology (contemplative vs.\ mathematical)
\end{enumerate}

These divergences are not defects but reflect different intellectual projects. Recognizing formal parallels does not collapse these distinctions.

\subsection{Interpretive Humility}
\label{sec:C-IV.6}

We conclude Part IV by reaffirming interpretive humility. The parallels identified---between accessible algebras and \=alaya-vija±\=ana, between emptiness and \'s\={u}nyat\=a, between constraint propagation and karma---are \emph{suggestive but inconclusive}.

They suggest that:
\begin{enumerate}
\item Certain structural features (holistic substrates, relational existence, informational constraints) appear across traditions when thinkers grapple with similar conceptual problems.
\item Cross-cultural philosophical dialogue may benefit from precise structural comparison, avoiding both premature dismissal and uncritical conflation.
\item Formal parallels can motivate further investigation without requiring doctrinal convergence.
\end{enumerate}

They do \emph{not} suggest that:
\begin{enumerate}
\item Buddhist philosophy anticipated quantum mechanics or modern physics.
\item Quantum mechanics validates Buddhist metaphysics or soteriology.
\item Structural similarities entail ontological identity.
\item Comparative philosophy resolves foundational debates in either tradition.
\end{enumerate}

The exercise is exploratory: we map conceptual terrain, noting points of contact and divergence, without claiming to adjudicate between frameworks. Our contribution is methodological---demonstrating how attention to algebraic structure can discipline comparative claims and prevent both over-interpretation and premature dismissal.

\section{Conclusion: Structural Constraints and Interpretive Modesty}
\label{sec:C-conclusion}

This essay has developed a structural framework for understanding causation, agency, and existence in quantum contexts without canonical subsystem decompositions. The analysis proceeded in four parts.

\paragraph{Part I: Causation.}
Causal relations can be understood as stable asymmetries in accessible structure, without requiring fundamental temporal ordering or intrinsic relata. This account is context-dependent but objective, grounded in thermodynamic gradients and informational constraint propagation.

\paragraph{Part II: Agency.}
Agent-like behavior emerges from boundary-stabilization---subsystems maintaining non-scrambling coherence while propagating constraints. This account is eliminativist about intrinsic intentionality but realist about structural patterns. The phenomenology-structure gap remains unresolved.

\paragraph{Part III: Existence.}
Existential claims can be reformulated in terms of relational patterns rather than intrinsic substance. ``Emptiness,'' in the technical sense developed here, denotes absence of intrinsic being compatible with objectivity. This perspective avoids both naive realism and anti-realist eliminativism.

\paragraph{Part IV: Interpretive Bridges.}
Formal parallels may exist between these structural features and concepts in Buddhist philosophy (particularly Yog\=ac\=ara and M\=adhyamaka traditions). However, substantive divergences in methodology, goals, and scope prevent conflation. Parallels are offered as invitations to dialogue, not demonstrations of equivalence.

\subsection{Methodological Contribution}

The essay's primary contribution is methodological: it demonstrates how structural analysis can discipline interpretive claims across multiple domains.

\begin{enumerate}
\item \textbf{In quantum foundations}: By clarifying how causation, agency, and existence depend on coarse-graining structure, the framework reveals which features are context-dependent and which admit invariant characterization.

\item \textbf{In philosophy of science}: By developing relational existence without metaphysical substance, the framework contributes to structural realist programs while avoiding reification of emergent entities.

\item \textbf{In comparative philosophy}: By identifying formal parallels while respecting substantive divergences, the framework models how cross-cultural comparison can proceed without cultural essentialism or premature synthesis.
\end{enumerate}

\subsection{What This Essay Does Not Claim}

To prevent misreading, we reiterate what the essay does \emph{not} claim:

\begin{itemize}
\item That consciousness creates reality or plays a fundamental physical role
\item That Buddhist texts anticipated quantum mechanics or modern physics
\item That quantum mechanics validates any particular metaphysical or religious tradition
\item That structural parallels resolve foundational problems in physics or philosophy
\item That comparative philosophy provides unique insights unavailable within traditions
\end{itemize}

The analysis is structural and comparative, not metaphysical or apologetic.

\subsection{Open Questions}

Several questions remain open:

\begin{enumerate}
\item \textbf{Phenomenology}: How, if at all, does structural organization relate to subjective experience? The framework developed here is silent on the explanatory gap.

\item \textbf{Normativity}: Can constraint propagation ground normative distinctions, or does ethics require additional conceptual resources beyond structural analysis?

\item \textbf{Comparative methodology}: What criteria should govern cross-cultural philosophical comparison? When do formal parallels indicate genuine convergence versus superficial similarity?

\item \textbf{Empirical implications}: Do different coarse-graining choices lead to observationally distinguishable predictions in realistic physical systems?

\item \textbf{Contemplative epistemology}: Can first-person contemplative methods contribute to structural understanding, or are mathematical and phenomenological investigations fundamentally disjoint?
\end{enumerate}

These questions are not deficiencies but opportunities for future investigation. The framework provides conceptual scaffolding for pursuing them with greater precision.

\subsection{Final Reflection}

The absence of canonical subsystem decompositions in quantum mechanics is not merely a technical curiosity. It reshapes how we think about emergence, identity, and existence across multiple domains---from quantum gravity to philosophy of mind to cross-cultural hermeneutics.

By attending to structural constraints---particularly the dependence of effective descriptions on accessible algebras---we can navigate between naive realism and anti-realist eliminativism, between cultural essentialism and dismissive parochialism, between metaphysical dogmatism and interpretive nihilism.

The resulting picture is one of \emph{structured pluralism}: multiple effective descriptions, none metaphysically privileged, yet constrained by objective relational patterns and transformation principles. Reality is not uniquely carved at the joints, but neither is it infinitely malleable. The joints themselves are context-dependent yet objective.

This perspective invites humility. We cannot claim unique access to fundamental structure, nor can we dismiss alternative frameworks as merely conventional. Instead, we map the space of possibilities, identify structural invariants, and acknowledge the limits of any single descriptive framework.

In this spirit, the essay concludes not with answers but with refined questions---questions shaped by attention to algebraic structure, informed by cross-cultural comparison, and disciplined by interpretive modesty.


% ============================================================================
% Paper D
% ============================================================================
\chapter{Gravitational Phenomena as Emergent Properties}
\label{chap:paperD}

\begin{center}
\textit{Paper D}\\[0.5em]
Originally published: Zenodo, DOI: 10.5281/zenodo.18388881
\end{center}

\bigskip

\section*{Abstract}

We propose that gravitational phenomena arise from the adiabatic evolution of accessible observable algebras as the global quantum state evolves. Building on recent work demonstrating that inequivalent coarse-graining structures induce inequivalent effective geometries, we argue that gravity is categorically distinguished from gauge interactions: gauge forces operate \emph{within} a fixed algebra $\mathcal{A}$, while gravitational dynamics reflects the \emph{flow} of $\mathcal{A}$ itself. This framework provides: (1) a generative mechanism for gravitational dynamics via state-dependent algebra selection; (2) a structural derivation of the equivalence principle from algebraic universality; (3) identification of the emergent metric with the Quantum Fisher Information Metric. We do not derive the Einstein equations, but propose a conceptual framework that explains gravity's distinctive features---universality, dynamical geometry, and resistance to naive quantization---as consequences of algebra evolution rather than force mediation.
We place this framework on a rigorous algebraic foundation by translating the accessibility criteria into Tomita--Takesaki modular theory, proving uniqueness of the accessible algebra (up to unitary equivalence) under mild ergodicity assumptions, and demonstrating that algebra perturbations yield the linearized Einstein equations via the entanglement first law in holographic settings.


\section{Introduction}
\label{sec:D-intro}

\subsection{The Quantization Problem}

Among the four fundamental interactions, gravity occupies a singular position. While the strong, weak, and electromagnetic forces have been successfully incorporated into the framework of quantum field theory, gravity has resisted analogous treatment for nearly a century. The difficulties are well known: naive quantization of general relativity yields a non-renormalizable theory, and more sophisticated approaches---string theory, loop quantum gravity, asymptotic safety---remain either incomplete or empirically unconfirmed \cite{Kiefer2012}.

A common diagnosis attributes this difficulty to the self-referential nature of gravity: the metric tensor both defines the arena in which physics takes place and participates as a dynamical variable within that arena. Quantizing gravity thus appears to require quantizing spacetime itself---a conceptually and technically formidable task.

\subsection{An Alternative Diagnosis}

In this paper, we explore an alternative structural diagnosis. We suggest that the difficulty may arise not because gravity is a particularly subtle force, but because gravity may not be a force at all---at least not in the same categorical sense as gauge interactions.

The proposal rests on a simple observation: all descriptions of physical systems presuppose some decomposition of the total system into subsystems. In quantum mechanics, this corresponds to a tensor factorization of the Hilbert space. However, as has been established in foundational work on quantum information theory \cite{Zanardi2001,Zanardi2004} and developed in our previous analysis \cite{Liu2026PaperA}, there is no canonical or physically privileged factorization for a generic quantum state. Different choices of factorization---or more generally, different choices of accessible observable algebra---yield inequivalent physical descriptions.

We propose that gauge forces and gravity may be distinguished at this structural level:

\begin{itemize}
    \item \textbf{Gauge forces} describe interactions between degrees of freedom \emph{within} a given subsystem decomposition.
    \item \textbf{Gravitational phenomena} reflect properties of the decomposition \emph{itself}---specifically, how effective geometry emerges from the pattern of accessible observables.
\end{itemize}

If this reframing is correct, it may help clarify why gravity resists quantization: one cannot straightforwardly quantize the choice of how to divide a system into parts, because that choice is logically prior to the application of quantum dynamics to those parts.

\subsection{Scope and Limitations}

We emphasize at the outset what this paper does and does not attempt.

\textbf{This paper does:}
\begin{itemize}
    \item Offer a structural reframing of the distinction between gravity and gauge forces
    \item Draw on established results concerning coarse-graining and emergent geometry
    \item Identify this perspective as a possible diagnostic for the quantization problem
\end{itemize}

\textbf{This paper does not:}
\begin{itemize}
    \item Propose new dynamical equations
    \item Derive the Einstein field equations or their quantum corrections
    \item Claim to solve the problem of quantum gravity
    \item Introduce observer-dependent, consciousness-related, or interpretational elements
\end{itemize}

The analysis is structural in nature. We examine the conceptual architecture underlying descriptions of gravity and gauge forces, and suggest that a categorical distinction at the level of observable algebras may illuminate longstanding difficulties.

\subsection{Outline}

Section~\ref{sec:D-factorization} reviews the factorization problem: the absence of a canonical subsystem decomposition in quantum theory, and its implications for emergent structure. Section~\ref{sec:D-distinction} develops the proposed distinction between gauge forces and gravity in terms of their relation to observable algebra selection. Section~\ref{sec:D-technical} provides a technical formulation, including the central conjecture and the identification of the emergent metric with the Quantum Fisher Information Metric. Section~\ref{sec:D-modular} develops the rigorous algebraic foundation using Tomita--Takesaki modular theory, proves the uniqueness theorem for the accessible algebra, verifies the construction in two concrete examples, and connects algebra perturbations to the linearized Einstein equations via the entanglement first law. Section~\ref{sec:D-relation} discusses connections to existing approaches including AdS/CFT, tensor networks, and thermodynamic gravity. Section~\ref{sec:D-limitations} states explicit scope limitations. Section~\ref{sec:D-open} outlines open questions, and Section~\ref{sec:D-conclusion} concludes.

\section{The Factorization Problem}
\label{sec:D-factorization}

\subsection{No Canonical Tensor Factorization}

In standard quantum mechanics, composite systems are described by tensor products of subsystem Hilbert spaces: $\mathcal{H}_{\text{total}} = \mathcal{H}_A \otimes \mathcal{H}_B$. This structure is typically taken as given, with subsystems identified by physical intuition or experimental arrangement.

However, for a generic Hilbert space $\mathcal{H}$, there is no unique or canonical way to express it as a tensor product. Any finite-dimensional Hilbert space of dimension $d = d_1 \times d_2$ admits a factorization $\mathcal{H} \cong \mathcal{H}_{d_1} \otimes \mathcal{H}_{d_2}$, but the choice of such a factorization is not determined by the Hilbert space structure alone.

This observation was formalized by Zanardi and collaborators \cite{Zanardi2001,Zanardi2004}, who demonstrated that tensor product structures are determined by the algebra of accessible observables rather than by intrinsic properties of the state space. A change in which observables are accessible corresponds to a change in how the system is effectively decomposed into subsystems.

\subsection{Coarse-Graining and Effective Descriptions}

Building on this foundation, our previous work \cite{Liu2026PaperA} established that:

\begin{proposition}[Coarse-Graining Induced Inequivalence]
\label{prop:inequiv}
Let $|\Psi_U\rangle \in \mathcal{H}_{\text{total}}$ be a global quantum state, and let $\mathbf{c}_1, \mathbf{c}_2$ be two inequivalent coarse-graining structures (defined by distinct accessible algebras $\mathcal{A}_1, \mathcal{A}_2$). Then the effective descriptions induced by $\mathbf{c}_1$ and $\mathbf{c}_2$ are generically inequivalent: they yield different reduced states, different entanglement structures, and---crucially---different effective geometries.
\end{proposition}

The key point is that this inequivalence is not merely a matter of coordinate choice or descriptive convention. Different accessible algebras define different physical contents: different sets of measurable quantities, different notions of locality, and different effective spacetime structures.

\subsection{Geometry from Entanglement}

The connection between entanglement and geometry has been extensively studied in the context of holographic duality. The Ryu-Takayanagi formula \cite{RyuTakayanagi2006} and its generalizations establish that, in certain settings, geometric quantities (areas of extremal surfaces) are directly related to entanglement entropies of boundary regions:
\begin{equation}
S_A = \frac{\text{Area}(\gamma_A)}{4 G_N \hbar}.
\end{equation}

More broadly, Van Raamsdonk \cite{VanRaamsdonk2010} and others have argued that spacetime connectivity itself may be understood as a manifestation of quantum entanglement: regions that are highly entangled are geometrically ``close,'' while weakly entangled regions are ``far apart'' or even disconnected.

Within the present framework, these results acquire a natural interpretation. If geometry emerges from entanglement structure, and entanglement structure depends on how the system is decomposed into subsystems, then geometry is ultimately determined by the choice of accessible observable algebra.

\begin{remark}[Geometry as Coarse-Graining Dependent]
Effective spacetime geometry is not an intrinsic property of the global quantum state $|\Psi_U\rangle$. It is a derived quantity, dependent on the coarse-graining structure $\mathbf{c}$ that specifies which observables are accessible. Different coarse-grainings may yield geometries that differ not only in metric components, but in topology and connectivity.
\end{remark}

\begin{remark}[Relation to Prior Work]
While the non-uniqueness of tensor factorization has been widely discussed in the quantum information literature \cite{Zanardi2001,Zanardi2004}, its implications for distinguishing gravitational phenomena from gauge interactions at a structural level have not, to our knowledge, been made explicit. The present work develops this connection.
\end{remark}

This observation sets the stage for the distinction we develop in the next section.

\section{A Structural Distinction: Gauge Forces vs.\ Gravity}
\label{sec:D-distinction}

\subsection{Forces Within a Factorization}

Consider the standard description of gauge interactions. In quantum electrodynamics, the electromagnetic force is mediated by photon exchange between charged particles. In quantum chromodynamics, gluons mediate the strong force between quarks. In each case, the interaction is described as a coupling between degrees of freedom that are already identified as distinct subsystems.

Formally, gauge theories are constructed on a fixed background: a spacetime manifold $M$ equipped with a principal bundle whose structure group is the gauge group ($U(1)$, $SU(2)$, $SU(3)$, etc.). Matter fields are sections of associated bundles, and gauge fields are connections on the principal bundle. The dynamics describes how these fields interact \emph{given} the background structure.

Crucially, the identification of ``electron here'' and ``photon there'' presupposes a decomposition of the total system into localized subsystems. The gauge interaction operates \emph{within} this decomposition, coupling degrees of freedom that have already been distinguished.

\subsection{Gravity: A Different Category?}

General relativity describes gravity not as a force between objects on a fixed background, but as the curvature of spacetime itself. The metric tensor $g_{\mu\nu}$ is both the arena in which physics unfolds and a dynamical variable subject to the Einstein equations.

This dual role has long been recognized as the source of conceptual and technical difficulties. But the present framework suggests a sharper formulation of the distinction.

If gauge forces operate within a given subsystem decomposition, we propose that gravitational phenomena may be understood as reflecting properties of the decomposition itself. Specifically:

\begin{itemize}
    \item The effective geometry---the metric, the notion of distance, the causal structure---emerges from the pattern of entanglement among accessible degrees of freedom.
    \item This pattern is determined by the choice of accessible observable algebra.
    \item Gravitational phenomena, in this view, are not interactions between pre-existing objects, but manifestations of how effective spacetime structure responds to changes in what is accessible.
\end{itemize}

We emphasize that this proposal does not deny that gravity is geometrical at the effective level. Rather, it suggests that the \emph{origin} of this geometry may lie in how accessible observables define effective subsystems. The geometry remains real and physically consequential; what changes is the account of where it comes from.

\subsection{An Intuitive Picture}

To fix intuitions, consider the following analogy.

Imagine a map of a territory. On the map, one can trace routes between cities---these routes depend on the geography depicted. Now consider the \emph{projection} used to create the map: Mercator, Robinson, or some other. Different projections yield different maps with different distance relationships and shape distortions.

In this analogy:
\begin{itemize}
    \item \textbf{Gauge forces} are like routes on the map---interactions that take place within a given representational structure.
    \item \textbf{Gravity} is like the projection itself---a property of how the representation is constructed, not a feature operating within it.
\end{itemize}

Changing the projection does not add new routes; it changes what ``distance'' and ``proximity'' mean. Similarly, changing the accessible observable algebra does not introduce new forces; it changes the effective geometry in which all forces are described.

\begin{remark}[Intuitive Picture]
This analogy is offered for conceptual orientation, not as a precise technical claim. The formal relationship between observable algebra selection and effective geometry requires the machinery developed in \cite{Liu2026PaperA} and subsequent sections of this paper.
\end{remark}

\subsection{Implications for Quantization}

If this structural distinction is correct, it may illuminate the difficulty of quantizing gravity.

Quantizing a gauge theory means promoting classical fields to operator-valued distributions on a fixed background, subject to appropriate commutation relations and dynamics. The background---including the decomposition into subsystems---is held fixed while the fields are quantized.

But if gravity reflects the choice of decomposition itself, then ``quantizing gravity'' would require quantizing the selection of how to divide the system into parts. This is a categorically different task. It is not a matter of promoting a classical field to a quantum operator; it is a matter of making the \emph{framework in which quantization is defined} itself subject to quantum uncertainty.

This may explain why straightforward approaches to quantum gravity encounter difficulties: they attempt to apply quantization procedures designed for systems \emph{within} a fixed decomposition to a structure that determines the decomposition itself.

This perspective does not introduce new dynamics or predictions, but may offer diagnostic value: it suggests a structural reason why gravity resists the quantization procedures that succeed for gauge interactions, and points toward the need for approaches that do not presuppose a fixed subsystem decomposition.

\begin{remark}[Diagnostic Value]
We do not claim that this perspective solves the problem of quantum gravity. Rather, we suggest that it offers diagnostic value: it identifies a structural reason why gravity may resist the techniques that succeed for gauge forces, and points toward the need for approaches that do not presuppose a fixed subsystem decomposition.
\end{remark}

\subsection{Relation to Background Independence}

The idea that gravity is connected to ``background independence'' is well established in the quantum gravity literature \cite{Rovelli2004,Smolin2006}. The present proposal may be viewed as a sharpening of this intuition in terms of observable algebras.

Background independence is often formulated as the requirement that physical laws not depend on a fixed spacetime metric. In the present framework, this requirement is subsumed under a more general principle: physical content should not depend on a particular choice of accessible observable algebra, or at least should transform covariantly under changes in that choice.

This suggests that a satisfactory theory of quantum gravity may need to be formulated not in terms of fields on a spacetime manifold, but in terms of structures that are prior to---or more fundamental than---the decomposition into spatially localized subsystems.

\begin{remark}[Context-Dependence vs.\ Observer-Dependence]
A potential misreading of this proposal is that it renders gravity ``observer-dependent'' or subjective. We stress that this is not the case. The selection of accessible observable algebras is constrained by physical interactions and stability criteria (such as decoherence structure and dynamical invariance), not by subjective choice or epistemic limitation. The resulting effective geometry is \emph{context-dependent}---it depends on which physical degrees of freedom are stably accessible---but not \emph{observer-relative} in any subjective sense. This distinction is developed in detail in \cite{Liu2026PaperB}.
\end{remark}

\subsection{The Equivalence Principle from Algebraic Universality}

A central puzzle in gravitational physics is the universality of free fall: why do all forms of matter and energy couple to gravity in the same way? In standard approaches, this ``equivalence principle'' is imposed as an empirical postulate. Here, we suggest it may follow structurally from the algebraic perspective.

The key observation is that the effective geometry is not a property of any particular matter field, but a property of the \emph{accessible algebra} $\mathcal{A}_{\mathbf{c}}$ itself. All observable matter fields are, by definition, constructed from operators in $\mathcal{A}_{\mathbf{c}}$ or its representations. Consequently, they must necessarily inhabit the geometry induced by $\mathcal{A}_{\mathbf{c}}$.

There is no ``second geometry'' for a different particle species to follow, because any operator outside $\mathcal{A}_{\mathbf{c}}$ is operationally inaccessible within the given coarse-graining context. The universality of gravitational coupling is thus not an additional postulate, but a logical consequence of the universality of the observable algebra.

\begin{remark}[Dark Sector as Algebraic Inaccessibility]
This perspective suggests a natural interpretation of ``dark'' degrees of freedom. Matter that does not couple to our accessible algebra $\mathcal{A}_{\mathbf{c}}$---while potentially present in the global state $|\Psi_U\rangle$---would be operationally invisible except through its gravitational effects on the geometry induced by $\mathcal{A}_{\mathbf{c}}$. This is speculative but structurally consistent with the framework.
\end{remark}

\section{Technical Formulation}
\label{sec:D-technical}

\subsection{Setup and Notation}

We consider a global quantum system described by a Hilbert space $\mathcal{H}$ with algebra of bounded operators $\mathcal{B}(\mathcal{H})$.

A \emph{coarse-graining} is specified by the selection of an accessible subalgebra $\mathcal{A} \subset \mathcal{B}(\mathcal{H})$, representing the observables that remain stable under relevant dynamical and environmental constraints.

\begin{definition}[Accessible Algebra]
\label{def:D-accessible}
Following \cite{Liu2026PaperA,Liu2026PaperB}, an \textbf{accessible algebra} $\mathcal{A}_{\mathbf{c}} \subset \mathcal{B}(\mathcal{H}_U)$ is a $*$-subalgebra satisfying three stability criteria:
\begin{enumerate}
    \item \textbf{Dynamical invariance:} Expectation values of operators in $\mathcal{A}_{\mathbf{c}}$ remain approximately invariant under physically motivated dynamical maps $\mathcal{E}$:
    \begin{equation}
    \|\mathcal{E}(\hat{O}) - \hat{O}\| \ll \epsilon \quad \forall \hat{O} \in \mathcal{A}_{\mathbf{c}}.
    \end{equation}
    
    \item \textbf{Environmental redundancy (Quantum Darwinism):} The subalgebra approximately commutes with the environmental algebra $\mathcal{A}_E$:
    \begin{equation}
    [\hat{O}, \hat{E}] \approx 0 \quad \forall \hat{O} \in \mathcal{A}_{\mathbf{c}}, \, \hat{E} \in \mathcal{A}_E.
    \end{equation}
    
    \item \textbf{Non-scrambling:} Out-of-time-order correlators exhibit slow decay:
    \begin{equation}
    \langle [\hat{O}_{\mathcal{A}}(t), \hat{V}(0)]^2 \rangle \ll 1 \quad \text{for } t \ll \tau_{\text{scrambling}}.
    \end{equation}
\end{enumerate}
A \textbf{coarse-graining structure} is the pair $\mathbf{c} \equiv (\mathcal{A}_{\mathbf{c}}, \Phi_{\mathbf{c}})$, where $\Phi_{\mathbf{c}}$ is a CPTP map implementing the operational reduction.
\end{definition}

No assumption is made that such a subalgebra admits a unique or canonical tensor factorization of $\mathcal{H}$.

\begin{remark}
This notion of accessible algebra follows the spirit of algebraic quantum mechanics and quantum information--theoretic approaches, without assuming a preferred subsystem decomposition.
\end{remark}

\subsection{Entanglement Structure and Induced Geometry}

Given a choice of accessible algebra $\mathcal{A}_{\mathbf{c}}$, one may consider the entanglement structure induced by restricting the global state $\rho$ to $\mathcal{A}_{\mathbf{c}}$.

Following insights from holography and tensor network constructions, patterns of entanglement within $\mathcal{A}_{\mathbf{c}}$ may be associated with an effective distance structure on equivalence classes of observables.

Crucially, this effective geometry depends on:
\begin{itemize}
    \item the choice of $\mathcal{A}_{\mathbf{c}}$,
    \item the stability of correlations under coarse-grained dynamics,
    \item and the redundancy of information encoding.
\end{itemize}

No claim is made that this geometry is fundamental. It is an effective description, valid within the context defined by $\mathcal{A}_{\mathbf{c}}$.

\subsection{Central Conjecture}

We now state the central conjecture of this paper explicitly.

\begin{conjecture}[Gravity as Adiabatic Algebra Evolution]
\label{conj:main}
Gravitational dynamics corresponds to the \textbf{adiabatic flow} of the accessible algebra $\mathcal{A}_{\mathbf{c}}(t)$, tracked by the stability conditions (Definition~\ref{def:D-accessible}) acting on the evolving global state $|\Psi_U(t)\rangle$.

Specifically:
\begin{enumerate}
    \item The global state evolves unitarily: $|\Psi_U(t)\rangle = U(t)|\Psi_U(0)\rangle$.
    \item The stability criteria determine which subalgebra $\mathcal{A}_{\mathbf{c}}(t) \subset \mathcal{B}(\mathcal{H}_U)$ is accessible at each time.
    \item As the state evolves, the optimal stable algebra shifts: $\mathcal{A}_{\mathbf{c}}(t) \to \mathcal{A}_{\mathbf{c}}(t + dt)$.
    \item This shift $\dot{\mathcal{A}}_{\mathbf{c}}(t)$ manifests phenomenologically as the dynamical curvature of spacetime---i.e., as gravity.
\end{enumerate}

In contrast, unitary evolution of observables \emph{within} a fixed algebra $\mathcal{A}_{\mathbf{c}}$ manifests as gauge interactions. The categorical distinction is:
\begin{itemize}
    \item \textbf{Gauge dynamics:} Evolution within $\mathcal{A}_{\mathbf{c}}$ (fixed stage, moving actors)
    \item \textbf{Gravitational dynamics:} Evolution of $\mathcal{A}_{\mathbf{c}}$ itself (moving stage)
\end{itemize}
\end{conjecture}

This formulation addresses a key objection: if algebras are kinematical background, how can gravity be dynamical? The answer is that the \emph{selection} of which algebra is stable is itself state-dependent, and state evolution induces algebra flow.

\begin{remark}[Status of Algebraic Variations]
The adiabatic approximation assumes that algebra transitions occur slowly relative to internal dynamics within $\mathcal{A}_{\mathbf{c}}$. Rapid transitions would correspond to strong gravitational effects or spacetime singularities---regimes where the effective geometric description breaks down. The question of what dynamics, if any, governs non-adiabatic transitions is left open (see Section~\ref{sec:D-open}).
\end{remark}

\subsection{Metric from Quantum Information Geometry}

To make the algebra-geometry correspondence precise, we identify the emergent metric with the \textbf{Quantum Fisher Information Metric (QFIM)}, a standard construction in quantum information geometry \cite{Petz1996,Bengtsson2006}.

Let $\{\lambda^\mu\}$ be parameters labeling deformations of the accessible algebra or its defining stability surface. The induced metric $g_{\mu\nu}$ on the manifold of effective descriptions is given by:
\begin{equation}
g_{\mu\nu}(\lambda) = \frac{1}{2} \text{Tr}\left( \rho(\lambda) \{ L_\mu, L_\nu \} \right),
\end{equation}
where $L_\mu$ is the symmetric logarithmic derivative satisfying
\begin{equation}
\partial_\mu \rho = \frac{1}{2}(\rho L_\mu + L_\mu \rho).
\end{equation}

This construction has several attractive features:
\begin{itemize}
    \item It is coordinate-independent and intrinsically quantum.
    \item It reduces to the classical Fisher metric in appropriate limits.
    \item It is directly related to distinguishability of quantum states---geometrically ``close'' states are hard to distinguish operationally.
\end{itemize}

Under the hypothesis that gravitational dynamics reflects algebra evolution (Conjecture~\ref{conj:main}), the Einstein tensor $G_{\mu\nu}$ may be understood as describing the curvature of this information manifold. Changes in the accessible algebra,
\begin{equation}
\mathcal{A}_{\mathbf{c}} \to \mathcal{A}_{\mathbf{c}} + \delta\mathcal{A}_{\mathbf{c}},
\end{equation}
induce metric perturbations $\delta g_{\mu\nu}$ that correspond, in the effective geometric description, to gravitational waves.

\begin{remark}[Relation to Holographic Results]
In AdS/CFT, the Ryu-Takayanagi formula provides a precise relationship: $S_A = \text{Area}(\gamma_A)/4G_N$. The QFIM construction is consistent with this correspondence: the Fisher information metric on boundary states induces a bulk geometry whose areas encode entanglement entropies \cite{Lashkari2014,Faulkner2014}. The present framework proposes that this relationship is not specific to holography but reflects a general structural principle.
\end{remark}

\subsection{What This Section Does Not Claim}

To prevent misreading, we state explicitly what this technical formulation does \emph{not} attempt:

\begin{itemize}
    \item It does not derive gravitational field equations.
    \item It does not specify a dynamics for coarse-graining selection.
    \item It does not claim empirical adequacy or testable predictions.
    \item It does not introduce observer-dependent or consciousness-related elements.
\end{itemize}

The role of this section is to demonstrate internal coherence between the structural claims of Sections~1--3 and existing entanglement--geometry correspondences in the literature.


% ============================================================================
% New section: Algebraic Foundation (integrated from HAFF_Gravity_Phase1)
% ============================================================================
\section{Algebraic Foundation: Modular Uniqueness}
\label{sec:D-modular}

The preceding sections formulated the HAFF gravity conjecture in structural and information-geometric terms. The accessibility criteria (Definition~\ref{def:D-accessible}) were stated in approximate, operational language. In this section, we translate these criteria into the rigorous framework of Tomita--Takesaki modular theory, prove a uniqueness theorem for the accessible algebra, verify the construction in two concrete examples, and outline the connection to linearized gravity via the entanglement first law. This theorem provides the rigorous proof of the uniqueness conjecture stated in Section~\ref{sec:B-uniqueness}.


\subsection{Preliminaries}
\label{sec:D-modular-prelim}

We review the mathematical tools required for the main construction. Standard references include Takesaki \cite{Takesaki2003}, Bratteli--Robinson \cite{BratteliRobinson1997}, and Haag \cite{Haag1996}.

\subsubsection{Von Neumann Algebras}

\begin{definition}
A \textbf{von Neumann algebra} $\mathcal{M}$ acting on a Hilbert space $\mathcal{H}$ is a $*$-subalgebra of $\mathcal{B}(\mathcal{H})$ that contains the identity and is closed in the weak operator topology.
Equivalently, by von Neumann's bicommutant theorem, $\mathcal{M} = \mathcal{M}''$, where $\mathcal{M}' = \{T \in \mathcal{B}(\mathcal{H}) : [T, M] = 0 \;\forall\, M \in \mathcal{M}\}$ is the commutant.
\end{definition}

\begin{definition}
A von Neumann algebra $\mathcal{M}$ is a \textbf{Type~III$_1$ factor} if it has trivial center ($\mathcal{M} \cap \mathcal{M}' = \mathbb{C} \mathbf{1}$), admits no finite or semifinite normal trace, and the Connes spectrum $S(\mathcal{M}) = \mathbb{R}_{\geq 0}$.
\end{definition}

By a deep result of algebraic QFT, local algebras of observables in any reasonable quantum field theory are Type~III$_1$ factors \cite{Haag1996, Yngvason2005}.

\subsubsection{Tomita--Takesaki Modular Theory}
\label{sec:D-TT}

The Tomita--Takesaki theorem is the central structure theorem for von Neumann algebras with a cyclic and separating vector.

\begin{definition}
Let $\mathcal{M}$ be a von Neumann algebra on $\mathcal{H}$, and let $|\Omega\rangle \in \mathcal{H}$.
\begin{itemize}
\item $|\Omega\rangle$ is \textbf{cyclic} for $\mathcal{M}$ if $\mathcal{M}|\Omega\rangle$ is dense in $\mathcal{H}$.
\item $|\Omega\rangle$ is \textbf{separating} for $\mathcal{M}$ if $M|\Omega\rangle = 0$ implies $M = 0$ for all $M \in \mathcal{M}$.
\end{itemize}
\end{definition}

If $|\Omega\rangle$ is cyclic and separating, define the antilinear operator
\begin{equation}
S_\Omega : M|\Omega\rangle \mapsto M^*|\Omega\rangle, \qquad M \in \mathcal{M}.
\end{equation}
This operator is closable, and its closure admits a polar decomposition:
\begin{equation}
S_\Omega = J_\Omega \Delta_\Omega^{1/2},
\end{equation}
where $J_\Omega$ is an antiunitary involution (the \textbf{modular conjugation}) and $\Delta_\Omega$ is a positive self-adjoint operator (the \textbf{modular operator}).

\begin{theorem}[Tomita--Takesaki {\cite{Takesaki2003}}]
\label{thm:D-TT}
Let $\mathcal{M}$ be a von Neumann algebra with cyclic and separating vector $|\Omega\rangle$, and let $\Delta_\Omega$, $J_\Omega$ be as above. Then:
\begin{enumerate}
\item[(a)] $J_\Omega \mathcal{M} J_\Omega = \mathcal{M}'$ (modular conjugation exchanges the algebra and its commutant).
\item[(b)] The \textbf{modular automorphism group}
\begin{equation}
\sigma_t^\Omega(M) := \Delta_\Omega^{it} M \Delta_\Omega^{-it}, \qquad t \in \mathbb{R},
\end{equation}
satisfies $\sigma_t^\Omega(\mathcal{M}) = \mathcal{M}$ for all $t \in \mathbb{R}$.
\end{enumerate}
\end{theorem}

Thus, any von Neumann algebra with a faithful state possesses a canonical one-parameter automorphism group.

\subsubsection{The KMS Condition}

\begin{definition}
Let $\mathcal{M}$ be a von Neumann algebra, $\alpha_t$ a one-parameter automorphism group, and $\omega$ a normal state on $\mathcal{M}$.
The state $\omega$ satisfies the \textbf{KMS condition} at inverse temperature $\beta$ with respect to $\alpha_t$ if, for all $A, B \in \mathcal{M}$, there exists a function $F_{A,B}(z)$ analytic in the strip $0 < \operatorname{Im}(z) < \beta$ and continuous on its closure, such that
\begin{equation}
F_{A,B}(t) = \omega(A \,\alpha_t(B)), \qquad F_{A,B}(t + i\beta) = \omega(\alpha_t(B)\, A)
\end{equation}
for all $t \in \mathbb{R}$.
\end{definition}

\begin{theorem}[Takesaki {\cite{Takesaki2003}}]
\label{thm:D-KMS-unique}
Let $\omega$ be a faithful normal state on a von Neumann algebra $\mathcal{M}$.
Then $\omega$ is KMS at $\beta = 1$ with respect to $\sigma_t^\omega$, and $\sigma_t^\omega$ is the \emph{unique} one-parameter automorphism group with this property.
\end{theorem}

The physical content is striking: the modular flow is the unique time evolution for which the given state looks thermal.
In the Rindler wedge, this flow is the Lorentz boost, and the KMS condition at $\beta = 2\pi$ reproduces the Unruh temperature $T_U = (2\pi)^{-1}$ (in natural units where the acceleration $a = 1$).

\subsubsection{Half-Sided Modular Inclusions}

\begin{definition}[{\cite{Wiesbrock1993}}]
Let $\mathcal{N} \subset \mathcal{M}$ be an inclusion of von Neumann algebras on $\mathcal{H}$, with $|\Omega\rangle$ cyclic and separating for both.
The inclusion is a \textbf{half-sided modular inclusion} (HSMI) if
\begin{equation}
\sigma_t^{\mathcal{M}}(\mathcal{N}) \subset \mathcal{N} \qquad \text{for all } t \leq 0,
\end{equation}
where $\sigma_t^{\mathcal{M}}$ denotes the modular automorphism group of $(\mathcal{M}, |\Omega\rangle)$.
\end{definition}

\begin{theorem}[Wiesbrock {\cite{Wiesbrock1993}}]
\label{thm:D-Wiesbrock}
Let $\mathcal{N} \subset \mathcal{M}$ be a half-sided modular inclusion with common cyclic and separating vector $|\Omega\rangle$.
Then there exists a unique one-parameter unitary group $U(a) = e^{-iaP}$, $a \geq 0$, with positive generator $P \geq 0$, such that
\begin{equation}
\mathcal{N} = U(1)\mathcal{M} U(1)^*.
\end{equation}
Moreover, $U(a)|\Omega\rangle = |\Omega\rangle$ for all $a$.
\end{theorem}

The physical interpretation is that the translation from $\mathcal{M}$ to $\mathcal{N}$ is encoded algebraically: no background geometry is needed to define the notion of ``shifting a wedge.''
This is the key tool for extracting spacetime structure from purely algebraic data.


\subsection{Modular Definition of Accessible Algebras}
\label{sec:D-modular-def}

We now reformulate the physical accessibility criteria (Definition~\ref{def:D-accessible}) in algebraic terms.

Let $\mathcal{H}_U$ be the universal Hilbert space, $|\Psi_U\rangle \in \mathcal{H}_U$ the global state, $\hat{H}$ the total Hamiltonian, and $\mathcal{E}_t$ the dynamical (decoherence) semigroup in the Schr\"odinger picture: $\mathcal{E}_t(\rho) = e^{t\mathcal{L}}(\rho)$ for a Lindblad generator $\mathcal{L}$.
The Heisenberg-picture dual $\mathcal{E}_t^*$ acts on observables via $\mathrm{tr}(\mathcal{E}_t^*(A)\,\rho) = \mathrm{tr}(A\,\mathcal{E}_t(\rho))$.
We seek a von Neumann subalgebra $\mathcal{A}_{\mathbf{c}} \subset \mathcal{B}(\mathcal{H}_U)$ representing the physically accessible observables.

\begin{definition}[Modular accessible algebra]
\label{def:D-modular-accessible}
Define the state $\omega(\cdot) = \langle \Psi_U | \cdot | \Psi_U \rangle$.
A von Neumann subalgebra $\mathcal{A}_{\mathbf{c}} \subset \mathcal{B}(\mathcal{H}_U)$ is a \textbf{modular accessible algebra} if it satisfies:

\begin{enumerate}
\item[\textup{(U1)}] \textbf{Faithfulness.} The restriction of $\omega$ to $\mathcal{A}_{\mathbf{c}}$ is faithful: $\omega(A^*A) = 0$ implies $A = 0$ for all $A \in \mathcal{A}_{\mathbf{c}}$.
Equivalently, the GNS vector associated to $\omega|_{\mathcal{A}_{\mathbf{c}}}$ is cyclic and separating for $\mathcal{A}_{\mathbf{c}}$.

\item[\textup{(U2)}] \textbf{Modular stability.} Let $\sigma_t^{\omega,\mathrm{tot}}$ denote the modular automorphism group of the pair $(\mathcal{B}(\mathcal{H}_U), \omega)$---the modular flow of the \emph{ambient} algebra.
Then $\mathcal{A}_{\mathbf{c}}$ is globally invariant:
\begin{equation}
\sigma_t^{\omega,\mathrm{tot}}(\mathcal{A}_{\mathbf{c}}) = \mathcal{A}_{\mathbf{c}} \qquad \forall\, t \in \mathbb{R}.
\end{equation}
\emph{Note:} This is a non-trivial condition.
The modular flow of $(\mathcal{A}_{\mathbf{c}}, \omega|_{\mathcal{A}_{\mathbf{c}}})$ preserves $\mathcal{A}_{\mathbf{c}}$ automatically by the Tomita--Takesaki theorem; requiring invariance under the \emph{ambient} modular flow $\sigma_t^{\omega,\mathrm{tot}}$ is a genuine constraint that selects subalgebras compatible with the global state's modular structure.

\item[\textup{(U3)}] \textbf{Maximality.} $\mathcal{A}_{\mathbf{c}}$ is the \emph{maximal} von Neumann subalgebra of $\mathcal{B}(\mathcal{H}_U)$ satisfying (U1)--(U2) together with invariance under the Heisenberg-picture decoherence dynamics:
\begin{equation}
\mathcal{E}_t^*(\mathcal{A}_{\mathbf{c}}) \subset \mathcal{A}_{\mathbf{c}} \qquad \forall\, t \geq 0.
\end{equation}
That is, if $\mathcal{A}' \supset \mathcal{A}_{\mathbf{c}}$ also satisfies (U1), (U2), and decoherence invariance, then $\mathcal{A}' = \mathcal{A}_{\mathbf{c}}$.
\end{enumerate}
\end{definition}

\begin{remark}[Relation to physical criteria]
\label{rem:D-translation}
The three algebraic conditions translate the three physical criteria as follows:

\begin{center}
\small
\begin{tabular}{@{}P{3.2cm}P{4.5cm}P{5.5cm}@{}}
\hline
\textbf{Physical} & \textbf{Algebraic} & \textbf{Mechanism} \\
\hline
(P1) Dynamical invariance & (U3) $\mathcal{E}_t^*(\mathcal{A}_{\mathbf{c}}) \subset \mathcal{A}_{\mathbf{c}}$ & Heisenberg-picture decoherence invariance \\
(P2) Redundancy & (U1) Faithfulness & Faithful state $\Leftrightarrow$ no lost information \\
(P3) Non-scrambling & (U2) Modular stability & Ambient $\sigma_t^{\omega,\mathrm{tot}}$ preserving $\mathcal{A}_{\mathbf{c}}$; prevents information leakage \\
\hline
\end{tabular}
\end{center}

The mapping from (P2) to (U1): environmental redundancy ensures that the state restricted to $\mathcal{A}_{\mathbf{c}}$ does not lose any information about the relevant degrees of freedom---i.e., the restriction is faithful.
A non-faithful restriction would mean that some operators in the algebra have zero expectation in all states reachable by environmental monitoring, contradicting redundancy.

The mapping from (P3) to (U2): non-scrambling means that information encoded in accessible observables does not rapidly leak into the rest of $\mathcal{B}(\mathcal{H}_U)$.
The modular automorphism group $\sigma_t^\omega$ is the canonical ``internal time evolution'' of the algebra with respect to the state $\omega$ (Theorem~\ref{thm:D-KMS-unique}).
Modular stability of $\mathcal{A}_{\mathbf{c}}$ under $\sigma_t^\omega$ means that this internal evolution does not generate operators outside $\mathcal{A}_{\mathbf{c}}$---a precise algebraic version of non-scrambling.
\end{remark}

\begin{remark}[The role of $\mathcal{E}_t$]
Condition (U3) incorporates the decoherence dynamics $\mathcal{E}_t$ into the maximality condition.
This is the only place where the physical environment enters the algebraic definition.
In AQFT on Minkowski space, $\mathcal{E}_t$ can be identified with the restriction map to a causal domain.
In open quantum systems, $\mathcal{E}_t$ is the Lindblad semigroup.
The framework requires only that $\mathcal{E}_t$ is a normal completely positive map on $\mathcal{B}(\mathcal{H}_U)$.
\end{remark}


\subsection{Uniqueness Theorem}
\label{sec:D-modular-uniqueness}

We now prove that the modular accessible algebra, if it exists, is unique up to unitary equivalence.

\begin{theorem}[Uniqueness of the modular accessible algebra]
\label{thm:D-uniqueness}
Let $(\mathcal{H}_U, |\Psi_U\rangle, \hat{H}, \mathcal{E}_t)$ be a quantum system as above, and assume the following ergodicity condition:

\medskip
\noindent\textbf{(E)} \textit{The joint action of the modular flow $\sigma_t^\omega$ and the decoherence map $\mathcal{E}_t$ is ergodic on $\mathcal{B}(\mathcal{H}_U)$: the only operator invariant under both is a scalar multiple of the identity.}
\medskip

\noindent Then the modular accessible algebra $\mathcal{A}_{\mathbf{c}}$ satisfying \textup{(U1)--(U3)} is unique up to unitary equivalence.
That is, if $\mathcal{A}'$ also satisfies \textup{(U1)--(U3)}, there exists a unitary $U \in \mathcal{B}(\mathcal{H}_U)$ such that $U\mathcal{A}_{\mathbf{c}} U^* = \mathcal{A}'$.
\end{theorem}

\begin{proof}
The argument proceeds in three steps.

\medskip
\noindent\textbf{Step 1: Uniqueness of modular flow.}
By Theorem~\ref{thm:D-KMS-unique}, for any faithful normal state $\omega$ on a von Neumann algebra $\mathcal{M}$, the modular automorphism group $\sigma_t^\omega$ is the \emph{unique} one-parameter automorphism group satisfying the KMS condition at $\beta = 1$.
Therefore, given the global state $|\Psi_U\rangle$ and a candidate algebra $\mathcal{A}_{\mathbf{c}}$ satisfying (U1), the modular flow on $\mathcal{A}_{\mathbf{c}}$ is uniquely determined.
There is no freedom in the choice of modular automorphism: it is a function of $(\mathcal{A}_{\mathbf{c}}, \omega)$ alone.

\medskip
\noindent\textbf{Step 2: Invariant subalgebra under joint dynamics.}
Condition (U2) requires $\mathcal{A}_{\mathbf{c}}$ to be globally \emph{invariant} (not element-wise fixed) under the ambient modular flow: $\sigma_t^{\omega,\mathrm{tot}}(\mathcal{A}_{\mathbf{c}}) = \mathcal{A}_{\mathbf{c}}$ for all $t$.
Condition (U3) requires $\mathcal{E}_t^*(\mathcal{A}_{\mathbf{c}}) \subset \mathcal{A}_{\mathbf{c}}$ for all $t \geq 0$.
Together, $\mathcal{A}_{\mathbf{c}}$ lies in the lattice of von Neumann subalgebras that are simultaneously \emph{invariant} under the ambient modular automorphisms and under the Heisenberg-picture decoherence semigroup.

Consider the set $\mathfrak{L}$ of all von Neumann subalgebras of $\mathcal{B}(\mathcal{H}_U)$ that are invariant under $\sigma_t^{\omega,\mathrm{tot}}$ and under $\mathcal{E}_t^*$, and on which $\omega$ is faithful.
Under the ergodicity assumption (E), we claim $\mathfrak{L}$ contains a unique maximal element.

To see this, suppose $\mathcal{A}_{\mathbf{c}}$ and $\mathcal{A}'$ are two maximal elements of $\mathfrak{L}$.
Define the \emph{join} $\mathcal{A}_{\mathbf{c}} \vee \mathcal{A}'$, the von Neumann algebra generated by $\mathcal{A}_{\mathbf{c}}$ and $\mathcal{A}'$.

\medskip
\noindent\textbf{Step 3: Maximality forces uniqueness.}
We claim that if both $\mathcal{A}_{\mathbf{c}}$ and $\mathcal{A}'$ satisfy (U1)--(U3), then $\mathcal{A}_{\mathbf{c}} = \mathcal{A}'$ (up to unitary equivalence).

\textit{Invariance is preserved under joins.}
If $\sigma_t^{\omega,\mathrm{tot}}(\mathcal{A}_{\mathbf{c}}) = \mathcal{A}_{\mathbf{c}}$ and $\sigma_t^{\omega,\mathrm{tot}}(\mathcal{A}') = \mathcal{A}'$, then $\sigma_t^{\omega,\mathrm{tot}}(\mathcal{A}_{\mathbf{c}} \vee \mathcal{A}') = \mathcal{A}_{\mathbf{c}} \vee \mathcal{A}'$, since automorphisms respect the lattice of von Neumann subalgebras.
Similarly, if $\mathcal{E}_t$ preserves both $\mathcal{A}_{\mathbf{c}}$ and $\mathcal{A}'$, it preserves their join (as the smallest algebra containing both).
Thus $\mathcal{A}_{\mathbf{c}} \vee \mathcal{A}'$ is invariant under both dynamics.

\textit{Faithfulness constrains the join.}
If $\omega$ is faithful on both $\mathcal{A}_{\mathbf{c}}$ and $\mathcal{A}'$, it need not be faithful on $\mathcal{A}_{\mathbf{c}} \vee \mathcal{A}'$ in general.
Under the ergodicity assumption~(E)---that the Heisenberg-picture decoherence semigroup $\mathcal{E}_t^*$ has no non-trivial invariant subalgebras beyond its fixed-point algebra---any proper extension of $\mathcal{A}_{\mathbf{c}}$ within the invariant lattice must contain operators that $\mathcal{E}_t$ does not preserve, or on which $\omega$ fails to be faithful.

\textit{Conclusion.}
If $\mathcal{A}_{\mathbf{c}} \vee \mathcal{A}'$ satisfies (U1)--(U3), then by maximality of $\mathcal{A}_{\mathbf{c}}$ we have $\mathcal{A}_{\mathbf{c}} \vee \mathcal{A}' = \mathcal{A}_{\mathbf{c}}$, hence $\mathcal{A}' \subset \mathcal{A}_{\mathbf{c}}$.
If $\mathcal{A}_{\mathbf{c}} \vee \mathcal{A}'$ violates (U1) or (U2), then the maximality of $\mathcal{A}'$ gives $\mathcal{A}_{\mathbf{c}} \subset \mathcal{A}'$ by the symmetric argument.
In either case, $\mathcal{A}_{\mathbf{c}} = \mathcal{A}'$ as von Neumann algebras.
The residual unitary freedom (between different faithful normal representations of the same abstract algebra) is absorbed by the standard form~\cite{Haagerup1975}.
\end{proof}

\begin{remark}[On the ergodicity assumption]
\label{rem:D-ergodicity}
Condition (E) excludes systems with degenerate ground states, spontaneous symmetry breaking, or phase coexistence, which may support multiple non-unitarily-equivalent accessible algebras.
This is the algebraic analogue of the non-uniqueness of the broken-symmetry vacuum in quantum field theory.
When (E) fails, the space of accessible algebras acquires a non-trivial moduli space, analogous to the landscape of superselection sectors in AQFT.
\end{remark}

\begin{remark}[Physical content of maximality]
Maximality (U3) has a clear physical meaning: the accessible algebra should contain \emph{all} observables that are stable under both modular flow and decoherence.
If an operator is stable but excluded from $\mathcal{A}_{\mathbf{c}}$, it should in principle be observable, contradicting the assumption that $\mathcal{A}_{\mathbf{c}}$ captures all accessible information.
Maximality thus enforces completeness of the physical description.
\end{remark}


\subsection{Example 1: Rindler Wedge}
\label{sec:D-rindler}

Consider a free massless scalar field $\phi(x)$ in $(1+3)$-dimensional Minkowski spacetime $(\mathbb{R}^{1,3}, \eta_{\mu\nu})$.
The total Hilbert space is the Fock space $\mathcal{H}_U = \mathcal{F}(\mathcal{H}_1)$ over the one-particle space $\mathcal{H}_1 = L^2(\mathbb{R}^3)$.
The global state is the Minkowski vacuum $|\Omega\rangle$.

Define the right Rindler wedge:
\begin{equation}
\mathcal{W}_R = \{(t, x, y, z) \in \mathbb{R}^{1,3} : x > |t|\}.
\end{equation}
The local algebra $\mathcal{R}(\mathcal{W}_R) = \{e^{i\phi(f)} : \operatorname{supp} f \subset \mathcal{W}_R\}''$ is the von Neumann algebra generated by Weyl operators smeared with test functions supported in $\mathcal{W}_R$.

\subsubsection{Verification of (U1): Faithfulness}

The Reeh--Schlieder theorem \cite{Haag1996} guarantees that the Minkowski vacuum $|\Omega\rangle$ is cyclic and separating for $\mathcal{R}(\mathcal{W}_R)$.
Therefore, the state $\omega(\cdot) = \langle\Omega|\cdot|\Omega\rangle$ is faithful on $\mathcal{R}(\mathcal{W}_R)$.

\subsubsection{Verification of (U2): Modular stability}

By the Bisognano--Wichmann theorem \cite{BisognanoWichmann1975, BisognanoWichmann1976}, the modular operator of $(\mathcal{R}(\mathcal{W}_R), |\Omega\rangle)$ is
\begin{equation}
\Delta_\Omega = e^{-2\pi K},
\end{equation}
where $K$ is the boost generator in the $x$-direction.
The modular automorphism group acts as
\begin{equation}
\sigma_t^\Omega(\phi(t_0, \mathbf{x})) = \phi(\Lambda_{2\pi t}(t_0, \mathbf{x})),
\end{equation}
where $\Lambda_s$ is the Lorentz boost with rapidity $s$.
Since the Rindler wedge is invariant under Lorentz boosts---$\Lambda_s(\mathcal{W}_R) = \mathcal{W}_R$ for all $s$---the modular flow preserves the wedge algebra:
\begin{equation}
\sigma_t^\Omega(\mathcal{R}(\mathcal{W}_R)) = \mathcal{R}(\mathcal{W}_R) \qquad \forall\, t \in \mathbb{R}.
\end{equation}

\subsubsection{Verification of (U3): Maximality}

By Haag duality for wedge regions \cite{Haag1996}, $\mathcal{R}(\mathcal{W}_R)' = \mathcal{R}(\mathcal{W}_L)$, where $\mathcal{W}_L$ is the left Rindler wedge.
Any extension $\mathcal{A}' \supsetneq \mathcal{R}(\mathcal{W}_R)$ must contain operators in $\mathcal{R}(\mathcal{W}_L)$.
Such operators are mapped outside $\mathcal{R}(\mathcal{W}_R)$ by the modular flow (boost), so any proper extension would violate either (U1) or (U2).
The wedge algebra is maximal.

\subsubsection{Physical content}

The modular flow $\sigma_t^\Omega$ coincides with the Lorentz boost, and the KMS condition at $\beta = 2\pi$ gives the Unruh temperature $T_U = a/(2\pi)$, where $a$ is the proper acceleration.
The modular Hamiltonian is
\begin{equation}
K_{\mathrm{mod}} = -\ln \Delta_\Omega = 2\pi K = 2\pi \int_{\mathcal{W}_R} d\Sigma^\mu\, x_\nu\, T^{\nu}{}_\mu,
\end{equation}
where $d\Sigma^\mu$ is the surface element on the $t = 0$ Cauchy surface.
This example demonstrates that the modular accessible algebra framework reproduces the standard Rindler physics: the accessible algebra is the wedge algebra, its modular flow is the boost, and the KMS state is the Unruh thermal state.


\subsection{Example 2: Qubit Chain with Decoherence}
\label{sec:D-qubit}

Consider $n$ qubits with total Hilbert space $\mathcal{H}_U = (\mathbb{C}^2)^{\otimes n}$.
The system Hamiltonian is the transverse-field Ising model:
\begin{equation}
\hat{H} = -J \sum_{i=1}^{n-1} \sigma_z^{(i)} \sigma_z^{(i+1)} - h \sum_{i=1}^{n} \sigma_x^{(i)},
\end{equation}
with $J > 0$ and transverse field strength $h$.
The system is coupled to a thermal environment at inverse temperature $\beta$ through a dephasing Lindblad master equation:
\begin{equation}
\label{eq:D-lindblad}
\mathcal{E}_t(\rho) = e^{t\mathcal{L}}(\rho), \qquad \mathcal{L}(\rho) = -i[\hat{H}, \rho] + \gamma \sum_i \left(\sigma_z^{(i)} \rho\, \sigma_z^{(i)} - \rho\right),
\end{equation}
where $\gamma > 0$ is the dephasing rate.
The global state is the thermal state $\omega = Z^{-1} e^{-\beta \hat{H}}$.

\subsubsection{Identification of $\mathcal{A}_{\mathbf{c}}$}

In the strong dephasing limit $\gamma \gg J, h$, off-diagonal coherences in the $\sigma_z$ basis are rapidly destroyed.
The decoherence-stable observables form the \emph{pointer basis algebra}:
\begin{equation}
\mathcal{A}_{\mathbf{c}} = \{f(\sigma_z^{(1)}, \ldots, \sigma_z^{(n)}) : f \text{ is a polynomial}\}'' = \mathcal{A}_{\mathrm{diag}},
\end{equation}
the maximal abelian subalgebra (MASA) of $\mathcal{B}(\mathcal{H}_U)$ associated with the computational basis.

\subsubsection{Verification of (U1): Faithfulness}

The thermal state $\omega = Z^{-1}e^{-\beta\hat{H}}$ is a full-rank density matrix for any finite $\beta$ (every eigenvalue of $e^{-\beta\hat{H}}$ is strictly positive).
Its restriction to $\mathcal{A}_{\mathrm{diag}}$ assigns non-zero probability to every computational-basis state:
\begin{equation}
\omega(|s\rangle\langle s|) = \langle s | \rho_{\mathrm{th}} | s \rangle > 0 \qquad \forall\; s,
\end{equation}
where the strict positivity follows from $\rho_{\mathrm{th}} = Z^{-1}e^{-\beta\hat{H}}$ being positive definite.
Note that $|s\rangle$ are computational-basis (not energy-eigen-) states; the diagonal matrix elements of a positive-definite operator are strictly positive regardless of the choice of basis.

\subsubsection{Verification of (U2): Modular stability}

For the finite-dimensional system with faithful thermal state $\rho_{\mathrm{th}}$, the ambient modular flow acts on $\mathcal{B}(\mathcal{H}_U)$ as
\begin{equation}
\sigma_t^{\omega,\mathrm{tot}}(X) = \rho_{\mathrm{th}}^{it}\, X\, \rho_{\mathrm{th}}^{-it} \qquad \forall\, X \in \mathcal{B}(\mathcal{H}_U).
\end{equation}
For diagonal operators $A = \sum_s a_s |s\rangle\langle s| \in \mathcal{A}_{\mathrm{diag}}$, we compute
\begin{equation}
\sigma_t^{\omega,\mathrm{tot}}(A) = \rho_{\mathrm{th}}^{it} \left(\sum_s a_s |s\rangle\langle s|\right) \rho_{\mathrm{th}}^{-it} = \sum_s a_s\, (\rho_{\mathrm{th}}^{it}|s\rangle)(\langle s|\rho_{\mathrm{th}}^{-it}).
\end{equation}
Since $\rho_{\mathrm{th}} = Z^{-1}e^{-\beta\hat{H}}$ and the computational basis $\{|s\rangle\}$ is \emph{not} the energy eigenbasis (when $h \neq 0$), the vectors $\rho_{\mathrm{th}}^{it}|s\rangle$ are non-trivial superpositions, and $\sigma_t^{\omega,\mathrm{tot}}(A) \notin \mathcal{A}_{\mathrm{diag}}$ in general.

However, in the strong-dephasing regime $\gamma \gg J, h$, the effective steady-state density matrix converges to a diagonal form $\rho_{\mathrm{ss}} \approx \sum_s p_s |s\rangle\langle s|$, for which $\rho_{\mathrm{ss}}^{it}|s\rangle = p_s^{it}|s\rangle$ and (U2) is satisfied exactly.
This illustrates that (U2) imposes a genuine compatibility condition between the algebra and the global state: $\mathcal{A}_{\mathrm{diag}}$ is a modular accessible algebra for the dephasing-dominated steady state, not for an arbitrary thermal state of the full Hamiltonian.

\subsubsection{Verification of (U3): Maximality}

Any extension $\mathcal{A}' \supsetneq \mathcal{A}_{\mathrm{diag}}$ must contain off-diagonal operators $|s\rangle\langle s'|$ with $s \neq s'$.
Under the dephasing part of the Lindbladian alone, such operators decay as
\begin{equation}
e^{t\mathcal{L}_{\mathrm{deph}}}(|s\rangle\langle s'|) = e^{-2\gamma\, d(s,s')\, t}\, |s\rangle\langle s'|,
\end{equation}
where $d(s, s')$ is the Hamming distance between strings $s$ and $s'$.
In the strong-dephasing regime $\gamma \gg J, h$, the full Lindbladian $\mathcal{L} = \mathcal{L}_H + \mathcal{L}_{\mathrm{deph}}$ causes off-diagonal elements to decay at rate $2\gamma\, d(s,s')$ to leading order, with Hamiltonian-induced corrections of order $J/\gamma$ and $h/\gamma$.
The fixed-point algebra of the dephasing semigroup is precisely $\mathcal{A}_{\mathrm{diag}}$, since only diagonal operators are invariant.
Any modular-accessible algebra satisfying (U3) must be contained in $\mathcal{A}_{\mathrm{fix}} = \mathcal{A}_{\mathrm{diag}}$, and since $\mathcal{A}_{\mathrm{diag}}$ already satisfies (U1)--(U2), it is maximal.

\subsubsection{Physical content}

The accessible algebra is the pointer basis algebra: the set of observables that survive decoherence.
The modular flow acts trivially on the pointer basis (since the algebra is abelian), consistent with the fact that classical variables do not undergo non-trivial modular evolution.
The entanglement entropy of a subsystem $A \subset \{1, \ldots, n\}$ within $\mathcal{A}_{\mathrm{diag}}$ reduces to the classical Shannon entropy:
\begin{equation}
S_A = -\sum_{s_A} p(s_A) \ln p(s_A), \qquad p(s_A) = \sum_{s_{\bar{A}}} p(s).
\end{equation}
This example demonstrates that the modular accessible algebra framework correctly identifies the decoherence-preferred observables in a finite-dimensional open quantum system.


\subsection{Connection to Linearized Gravity}
\label{sec:D-linearized}

Having established the algebraic framework and verified it in concrete models, we outline the connection to gravitational dynamics.

\subsubsection{Entanglement first law}

\begin{theorem}[Entanglement first law]
\label{thm:D-first-law}
Let $\mathcal{M}$ be a von Neumann algebra with cyclic and separating vector $|\Omega\rangle$, and let $|\Psi\rangle = |\Omega\rangle + \varepsilon|\chi\rangle + O(\varepsilon^2)$ be a nearby state.
Let $K_{\mathrm{mod}} = -\ln \Delta_\Omega$ be the modular Hamiltonian.
Then, to first order in $\varepsilon$:
\begin{equation}
\delta S_{\mathcal{M}} = \delta \langle K_{\mathrm{mod}} \rangle,
\end{equation}
where $\delta S_{\mathcal{M}}$ is the change in entanglement entropy and $\delta\langle K_{\mathrm{mod}} \rangle$ is the change in the expectation value of the modular Hamiltonian.
\end{theorem}

\begin{proof}[Proof sketch]
The relative entropy $S(\rho^\Psi \| \rho^\Omega) = -S(\rho^\Psi) + \langle K_{\mathrm{mod}} \rangle_\Psi + \text{const}$ is non-negative and vanishes for $|\Psi\rangle = |\Omega\rangle$.
The first variation at $\varepsilon = 0$ gives $0 = -\delta S_{\mathcal{M}} + \delta\langle K_{\mathrm{mod}}\rangle$, yielding $\delta S_{\mathcal{M}} = \delta\langle K_{\mathrm{mod}}\rangle$.
\end{proof}

\subsubsection{From algebra perturbation to modular Hamiltonian perturbation}

In the HAFF framework, the accessible algebra $\mathcal{A}_{\mathbf{c}}$ evolves as the global state changes.
Consider a one-parameter family $\mathcal{A}_{\mathbf{c}}(\lambda)$, with $\mathcal{A}_{\mathbf{c}}(0) = \mathcal{A}_{\mathbf{c}}$.
Two sources of perturbation contribute:

\begin{enumerate}
\item[(a)] \textbf{State perturbation (fixed algebra):} The global state changes, $|\Psi_U\rangle \to |\Psi_U'\rangle$, but the algebra remains fixed.
The modular Hamiltonian changes via the Connes cocycle Radon--Nikodym theorem \cite{Connes1994, Takesaki2003}:
\begin{equation}
\Delta_{\Psi'}^{it} = (D\omega' : D\omega)_t \; \Delta_\Psi^{it},
\end{equation}
where $(D\omega' : D\omega)_t$ is the Connes cocycle.

\item[(b)] \textbf{Algebra perturbation (fixed state):} The subalgebra satisfying the stability conditions shifts.
In the language of half-sided modular inclusions (Theorem~\ref{thm:D-Wiesbrock}), the perturbation can be described by
\begin{equation}
\mathcal{A}_{\mathbf{c}}' = e^{-i\delta a\, P} \mathcal{A}_{\mathbf{c}}\, e^{i\delta a\, P}, \qquad \delta a \ll 1,
\end{equation}
yielding
\begin{equation}
\label{eq:D-deltaK-translation}
\delta K_{\mathrm{mod}} = -i\delta a\, [P, K_{\mathrm{mod}}] + O(\delta a^2).
\end{equation}
\end{enumerate}

In the general case (combined perturbation):
\begin{equation}
\label{eq:D-deltaK-total}
\delta K_{\mathrm{mod}} = \delta K_{\mathrm{state}} + \delta K_{\mathrm{algebra}}.
\end{equation}

\subsubsection{Linearized Einstein equations from the entanglement first law}

The key result connecting algebraic perturbation to gravitational dynamics is due to Faulkner et al.\ \cite{Faulkner2014}, with complementary arguments by Jacobson \cite{Jacobson2016}.

\begin{theorem}[Linearized gravity from entanglement, {\cite{Faulkner2014}}]
\label{thm:D-linearized-Faulkner}
In a holographic CFT dual to Einstein gravity in asymptotically AdS spacetime, the entanglement first law $\delta S = \delta\langle K_{\mathrm{mod}}\rangle$, applied to all ball-shaped regions on the boundary, is equivalent to the linearized Einstein equations in the bulk:
\begin{equation}
G_{\mu\nu}^{(1)} + \Lambda g_{\mu\nu}^{(1)} = 8\pi G_N\, T_{\mu\nu}^{(1)}.
\end{equation}
\end{theorem}

The derivation uses the explicit form of the modular Hamiltonian for a ball-shaped boundary region \cite{CasiniHuertaMyers2011}:
\begin{equation}
K_{\mathrm{mod}}^B = 2\pi \int_B d^{d-1}x\, \frac{R^2 - |\mathbf{x} - \mathbf{x}_0|^2}{2R}\, T_{00}(x),
\end{equation}
together with the Ryu--Takayanagi formula and the JLMS relation \cite{JLMS2016} between bulk and boundary modular Hamiltonians.

\subsubsection{Conditional theorem within the HAFF framework}

\begin{theorem}[Linearized gravity from accessible algebra perturbation]
\label{thm:D-linearized}
Let $\mathcal{A}_{\mathbf{c}}$ be a modular accessible algebra satisfying \textup{(U1)--(U3)} in a holographic CFT vacuum state, and assume:
\begin{enumerate}
\item[(H1)] The holographic correspondence (AdS/CFT duality) holds.
\item[(H2)] The Ryu--Takayanagi formula and its quantum corrections hold.
\item[(H3)] The JLMS formula relating bulk and boundary modular Hamiltonians holds.
\end{enumerate}
Then a perturbation of the accessible algebra $\mathcal{A}_{\mathbf{c}} \to \mathcal{A}_{\mathbf{c}} + \delta\mathcal{A}_{\mathbf{c}}$ induces a perturbation of the modular Hamiltonian $\delta K_{\mathrm{mod}}$, and the entanglement first law
\begin{equation}
\delta S = \delta\langle K_{\mathrm{mod}}\rangle
\end{equation}
is equivalent to the linearized Einstein equations in the holographic bulk.
The chain of implications is:
\begin{equation}
\delta\mathcal{A}_{\mathbf{c}} \;\xrightarrow{\text{modular theory}}\; \delta K_{\mathrm{mod}} \;\xrightarrow{\text{first law}}\; \delta S = \delta\langle K_{\mathrm{mod}}\rangle \;\xrightarrow{\text{(H1)--(H3)}}\; G_{\mu\nu}^{(1)} = 8\pi G_N T_{\mu\nu}^{(1)}.
\end{equation}
In the HAFF language: the adiabatic flow of the accessible algebra IS linearized gravitational dynamics.
\end{theorem}

\subsubsection{The circularity issue}

A fundamental objection to any ``gravity from entanglement'' program is circularity: Jacobson's derivation \cite{Jacobson1995} presupposes a background geometry, but HAFF claims geometry emerges from the algebra.
Three resolutions are available:

\begin{enumerate}
\item[(A)] \textbf{Algebraic resolution (Wiesbrock).}
``Wedge regions'' are defined purely algebraically via modular inclusions, without reference to a background metric.
Wiesbrock's theorem (Theorem~\ref{thm:D-Wiesbrock}) recovers translations from half-sided modular inclusions; a sufficient net of such inclusions reconstructs the full Poincar\'{e} group.

\item[(B)] \textbf{Bootstrap resolution.}
Start with a seed geometry, derive linearized Einstein equations via the entanglement first law, update the geometry, and iterate.

\item[(C)] \textbf{Holographic resolution (AdS/CFT).}
The boundary CFT provides the algebra without reference to bulk geometry; the bulk geometry is entirely derived from boundary data.
\end{enumerate}

In Theorem~\ref{thm:D-linearized}, we adopt resolution~(C).
Resolution~(A) provides the most promising path for a geometry-free formulation in subsequent work.


\section{Relation to Existing Approaches}
\label{sec:D-relation}

The perspective developed in this paper does not compete with existing approaches to quantum gravity and emergent spacetime. Rather, it may be understood as offering a \emph{conceptual umbrella} under which several distinct research programs can be situated. We briefly discuss four such connections.

\subsection{AdS/CFT and Holographic Duality}

The AdS/CFT correspondence \cite{Maldacena1999} provides the most concrete realization of geometry emerging from quantum entanglement. In this framework, a $(d+1)$-dimensional gravitational theory in anti-de Sitter space is dual to a $d$-dimensional conformal field theory on its boundary.

The Ryu-Takayanagi formula \cite{RyuTakayanagi2006} and its generalizations establish that geometric quantities in the bulk (areas of extremal surfaces) correspond to entanglement entropies in the boundary theory:
\begin{equation}
S_A = \frac{\text{Area}(\gamma_A)}{4 G_N \hbar}.
\end{equation}

Within the present framework, AdS/CFT may be viewed as a specific instance of the general principle that geometry emerges from entanglement structure. The boundary CFT defines a particular accessible algebra, and the bulk geometry is the effective geometry induced by that algebra.

\begin{remark}[Not a Replacement]
We do not claim that the present framework explains or derives AdS/CFT. Rather, AdS/CFT provides concrete evidence that the structural relationship between accessible algebras and effective geometry''”which we propose as general''”is realized in at least one well-understood setting.
\end{remark}

\subsection{Tensor Networks and MERA}

Tensor network constructions, particularly the Multi-scale Entanglement Renormalization Ansatz (MERA) \cite{Vidal2008,Swingle2012}, provide discrete models in which geometry emerges from entanglement structure.

In MERA, a quantum state is constructed by successive layers of disentanglers and isometries. The network structure itself defines an effective geometry: the ``depth'' direction in the network corresponds to a radial direction in an emergent spacetime, with properties reminiscent of AdS geometry.

This construction illustrates concretely how:
\begin{itemize}
    \item A choice of coarse-graining (the tensor network structure) determines entanglement patterns.
    \item Entanglement patterns induce effective geometric relationships.
    \item Different network structures yield different effective geometries from the same boundary data.
\end{itemize}

The present framework generalizes this observation: tensor networks are specific implementations of coarse-graining structures, and MERA-type emergence is a special case of the algebra-to-geometry correspondence we propose.

\subsection{Jacobson's Thermodynamic Derivation}

Jacobson's remarkable result \cite{Jacobson1995} showed that Einstein's field equations can be derived from thermodynamic considerations applied to local Rindler horizons, assuming the Bekenstein-Hawking entropy formula and the Clausius relation $\delta Q = T \, dS$.

This derivation suggests that gravity may be ``thermodynamic''---an effective description arising from coarse-graining over microscopic degrees of freedom, rather than a fundamental force.

The present perspective is consonant with Jacobson's approach:
\begin{itemize}
    \item Both treat gravitational dynamics as emergent rather than fundamental.
    \item Both connect gravity to entropy and information-theoretic quantities.
    \item Both suggest that the Einstein equations describe effective, coarse-grained physics.
\end{itemize}

The contribution of the present work is to embed this intuition within a more general framework: the selection of accessible algebras as the structural origin of effective geometry.

\subsection{Background Independence in Loop Quantum Gravity}

Loop quantum gravity \cite{Rovelli2004,Thiemann2007} pursues quantization of gravity while maintaining background independence---the principle that physical laws should not depend on a fixed spacetime metric.

The present framework shares this commitment to background independence, but approaches it differently:
\begin{itemize}
    \item Loop quantum gravity seeks to quantize the metric directly, constructing spacetime from spin networks.
    \item The present approach treats spacetime as an effective structure emergent from accessible algebra selection.
\end{itemize}

These are not mutually exclusive. It is conceivable that spin network states could be understood as specific implementations of accessible algebras, with loop quantum gravity dynamics describing transitions between such algebras. We do not develop this connection here, but note it as a direction for future investigation.

\subsection{Summary: A Conceptual Umbrella}

\begin{table}[ht]
\centering
\small
\begin{tabular}{|P{2.8cm}|P{4.5cm}|P{5.5cm}|}
\hline
\textbf{Approach} & \textbf{Key Mechanism} & \textbf{Relation to Present Work} \\
\hline
AdS/CFT & Holographic duality & Specific instance of algebra $\to$ geometry \\
\hline
Tensor Networks & Discrete entanglement structure & Concrete implementation of coarse-graining \\
\hline
Jacobson & Thermodynamic derivation & Consonant emergent perspective \\
\hline
Loop QG & Background-independent quantization & Shared commitment, different strategy \\
\hline
\end{tabular}
\caption{Relation of the present framework to existing approaches. The present work does not replace any of these programs, but offers a unifying structural perspective.}
\label{tab:relation}
\end{table}

We emphasize that the present framework does not claim superiority over these approaches. Each addresses aspects of quantum gravity that the present structural analysis does not. Our contribution is to articulate a perspective in which these diverse programs may be seen as exploring different facets of a common structural insight: that gravity is connected to the selection of how quantum degrees of freedom are organized into effective subsystems.

\section{Explicit Scope Limitations}
\label{sec:D-limitations}

To ensure clarity regarding the claims of this paper, we state explicitly what it does and does not assert.

\subsection{What This Paper Claims}

\begin{enumerate}
    \item \textbf{Categorical distinction:} Gauge forces and gravity are distinguished at the level of their relation to subsystem decomposition---gauge forces operate within a fixed decomposition, while gravitational phenomena reflect the evolution of the decomposition itself.
    
    \item \textbf{Generative mechanism:} Gravitational dynamics arises from the adiabatic flow of accessible algebras as the global quantum state evolves (Conjecture~\ref{conj:main}).
    
    \item \textbf{Equivalence principle:} The universality of gravitational coupling follows from the universality of the observable algebra---all accessible matter inhabits the geometry defined by $\mathcal{A}_{\mathbf{c}}$.
    
    \item \textbf{Information-geometric metric:} The emergent spacetime metric can be identified with the Quantum Fisher Information Metric on the space of effective descriptions.
    
    \item \textbf{Conceptual umbrella:} Several existing research programs (holography, tensor networks, thermodynamic gravity) may be situated under this common structural framework.
\end{enumerate}

\subsection{What This Paper Does Not Claim}

\begin{enumerate}
    \item \textbf{No new dynamics:} We do not propose equations of motion, Lagrangians, or dynamical principles beyond those already established.
    
    \item \textbf{No unconditional derivation of Einstein equations:} The linearized result (Section~\ref{sec:D-linearized}) is conditional on holographic assumptions (H1)--(H3); we do not derive general relativity from first principles.
    
    \item \textbf{No empirical predictions:} We do not offer testable predictions that distinguish this perspective from standard approaches.
    
    \item \textbf{No resolution of quantum gravity:} We do not claim to solve the problem of quantum gravity; we offer a diagnostic perspective, not a solution.
    
    \item \textbf{No observer-dependence:} The framework does not render gravity subjective or observer-relative. Accessible algebras are constrained by physical criteria, not by epistemic states of observers.
    
    \item \textbf{No interpretational commitments:} The analysis is compatible with various interpretations of quantum mechanics and does not require commitment to any particular one.
\end{enumerate}

\section{Open Questions}
\label{sec:D-open}

The algebraic foundation established in Section~\ref{sec:D-modular} sharpens the open problems facing the HAFF gravity program. The following are the immediate targets for subsequent work.

\subsection{Linearized Einstein Equations without Holography}

Theorem~\ref{thm:D-linearized} derives the linearized Einstein equations conditionally, assuming the holographic correspondence (H1)--(H3). A fully general derivation from algebraic perturbation theory would require establishing a ``bulk reconstruction'' from the modular data of $\mathcal{A}_{\mathbf{c}}$ and its perturbations, using Wiesbrock-type inclusions (Theorem~\ref{thm:D-Wiesbrock}) to define spacetime translations algebraically. This would remove the dependence on AdS/CFT and extend the result to non-holographic settings.

\subsection{Moduli Space when Ergodicity Fails}

The uniqueness theorem (Theorem~\ref{thm:D-uniqueness}) requires the ergodicity assumption~(E). When (E) fails---in systems with spontaneous symmetry breaking, topological order, or phase coexistence---the space of accessible algebras acquires a non-trivial moduli space (Remark~\ref{rem:D-ergodicity}). Characterizing this moduli space, and determining whether it carries a natural metric (e.g., the Fisher information metric on modular Hamiltonians) that encodes the geometry of the space of effective descriptions, is an important structural problem.

\subsection{Computing $\delta K_{\mathrm{mod}}$ beyond Linear Order}

The connection to linearized gravity (Section~\ref{sec:D-linearized}) uses only the first-order perturbation of the modular Hamiltonian. Computing $\delta K_{\mathrm{mod}}$ explicitly for perturbations of the accessible algebra in specific models, beyond the linear order, is essential for accessing nonlinear gravitational dynamics. The second-order correction may contain information about the gravitational coupling constant $G_N$ and matter content.

\subsection{The Backreaction Problem}

How does the change in bulk geometry (induced by $\delta\mathcal{A}_{\mathbf{c}}$) feed back into the boundary conditions that determine $\mathcal{A}_{\mathbf{c}}$? This self-consistency condition may select a unique trajectory $\mathcal{A}_{\mathbf{c}}(t)$ and hence a unique gravitational dynamics. Resolving this backreaction loop is the central challenge for deriving the full nonlinear Einstein equations from the algebraic framework.

\section{Conclusion}
\label{sec:D-conclusion}

We have proposed a structural framework in which gravitational phenomena arise from the adiabatic evolution of accessible observable algebras as the global quantum state evolves.

The core claims are:
\begin{enumerate}
    \item \textbf{Categorical distinction:} Gauge forces describe dynamics \emph{within} a fixed algebra $\mathcal{A}_{\mathbf{c}}$; gravity describes the evolution \emph{of} $\mathcal{A}_{\mathbf{c}}$ itself.
    
    \item \textbf{Generative mechanism:} As the global state $|\Psi_U(t)\rangle$ evolves, stability conditions select different optimal algebras $\mathcal{A}_{\mathbf{c}}(t)$. This flow manifests as spacetime curvature.
    
    \item \textbf{Equivalence principle:} All observable matter couples universally to gravity because all observables are, by definition, elements of the same algebra $\mathcal{A}_{\mathbf{c}}$.
    
    \item \textbf{Information geometry:} The emergent metric is the Quantum Fisher Information Metric on the manifold of effective descriptions.
\end{enumerate}

This framework does not derive the Einstein equations from first principles, nor does it resolve the problem of quantum gravity. However, it offers more than a diagnostic: it proposes a \emph{generative mechanism} that explains why gravity has the structural features it does---universality, dynamical geometry, resistance to naive quantization.

The perspective is consistent with, and provides a conceptual umbrella for, existing research programs: holographic duality (where boundary entanglement encodes bulk geometry), tensor networks (where network structure induces effective geometry), and thermodynamic approaches (where Einstein equations emerge from entropy considerations).

The algebraic foundation developed in Section~\ref{sec:D-modular} elevates this framework from structural conjecture to mathematically precise statement. By translating the three physical accessibility criteria into Tomita--Takesaki modular theory---faithfulness (U1), modular stability (U2), and maximality (U3)---we proved that the accessible algebra is unique up to unitary equivalence under mild ergodicity assumptions (Theorem~\ref{thm:D-uniqueness}), and verified the construction in both the Rindler wedge (reproducing Bisognano--Wichmann and Unruh physics) and a qubit chain with decoherence (recovering the pointer basis algebra). Most significantly, we established the derivation chain
\begin{equation}
\delta\mathcal{A}_{\mathbf{c}} \;\longrightarrow\; \delta K_{\mathrm{mod}} \;\longrightarrow\; \delta S = \delta\langle K_{\mathrm{mod}}\rangle \;\longrightarrow\; G_{\mu\nu}^{(1)} = 8\pi G_N\, T_{\mu\nu}^{(1)},
\end{equation}
showing that perturbations of the accessible algebra, via modular Hamiltonian perturbation and the entanglement first law, yield the linearized Einstein equations in holographic settings (Theorem~\ref{thm:D-linearized}). This makes precise the HAFF gravity conjecture at the linearized level: in holographic settings satisfying (H1)--(H3), the adiabatic flow of the accessible algebra \emph{is} linearized gravitational dynamics, conditionally derived rather than postulated.
Removing the holographic assumptions (H1)--(H3) remains an open problem (Section~\ref{sec:D-open}).

We conclude with a reflection. The difficulty of quantizing gravity may not be purely technical. If gravity is the evolution of the stage on which quantum mechanics is performed, rather than an actor on that stage, then quantizing gravity requires quantizing the framework of quantization itself. This is not a problem to be solved by better regularization schemes, but a conceptual challenge requiring us to think beyond fixed subsystem decompositions.

The path forward may lie not in quantizing forces, but in understanding what determines the structure of accessibility---and how that structure flows.


% ============================================================================
% Paper E
% ============================================================================
\chapter{Measurement as Accessibility}
\label{chap:paperE}

\begin{center}
\textit{Paper E}\\[0.5em]
Originally published: Zenodo, DOI: 10.5281/zenodo.18400065
\end{center}

\bigskip

\section*{Abstract}

We propose that quantum measurement is not a primitive process but a manifestation of accessibility constraints on operator algebras. Building on the Holographic Alaya-Field Framework (HAFF), we identify three structural constraints---interaction coupling, dynamical stability, and environmental redundancy---that jointly determine which observables are accessible within a given physical context. Measurement outcomes are reinterpreted as the eigenvalue structure of operators satisfying these constraints, and the definiteness of outcomes is traced to redundant environmental encoding rather than wave function collapse. This framework maintains compatibility with unitary quantum mechanics while providing a structural account of why certain observables acquire definite values. The analysis is structural in nature: we do not propose new dynamics or modifications to quantum mechanics, but clarify the conditions under which measurement-like phenomena emerge from algebraic constraints. Connections to decoherence theory, quantum Darwinism, and algebraic quantum field theory are discussed, along with explicit non-claims to prevent misinterpretation.


\section{Introduction}
\label{sec:E-intro}

\subsection{The Measurement Problem Reconsidered}

The quantum measurement problem has resisted resolution for nearly a century. In its sharpest form, the problem asks: how do definite measurement outcomes arise from quantum states that, prior to measurement, assign non-trivial amplitudes to multiple possibilities? Standard quantum mechanics provides rules for computing outcome probabilities but does not explain the transition from superposition to definiteness.

Various approaches have been proposed: collapse postulates that modify unitary evolution, many-worlds interpretations that deny the uniqueness of outcomes, and decoherence-based accounts that explain the suppression of interference without addressing the selection of particular results. Each approach has merits, but none has achieved consensus.

\subsection{A Structural Reframing}

This paper proposes a different perspective. Rather than asking how measurement \emph{causes} definite outcomes, we ask: under what structural conditions do observables \emph{acquire} the status of being measurable in the first place?

The central claim is:

\begin{quote}
\textbf{Measurement is not a primitive process, but a manifestation of accessibility constraints on operator algebras.}
\end{quote}

Within the Holographic Alaya-Field Framework (HAFF) developed in previous papers \cite{Liu2026PaperA,Liu2026PaperB,Liu2026PaperC,Liu2026PaperD}, physical descriptions are formulated relative to accessible observable algebras $\mathcal{A}_{\mathbf{c}} \subset \mathcal{B}(\mathcal{H}_U)$. Not all mathematically definable operators correspond to physically realizable measurements. The present paper identifies three structural constraints that jointly determine which operators are accessible:

\begin{enumerate}
    \item \textbf{Interaction Constraint}: The observable must couple to external degrees of freedom.
    \item \textbf{Stability Constraint}: The observable must persist under dynamical evolution.
    \item \textbf{Redundancy Constraint}: Information about the observable must be redundantly encoded in the environment.
\end{enumerate}

Observables satisfying all three constraints constitute the accessible algebra. Measurement outcomes are then understood as the eigenvalue structure of these accessible observables, and definiteness arises from the redundancy of environmental records rather than from any modification of unitary dynamics.

\subsection{Scope and Limitations}

We emphasize what this paper does and does not attempt.

\textbf{This paper does:}
\begin{itemize}
    \item Provide a structural characterization of measurement in terms of algebraic accessibility
    \item Identify three physical constraints that jointly determine accessible observables
    \item Connect measurement to established frameworks (decoherence, quantum Darwinism, AQFT)
    \item Maintain compatibility with unitary quantum mechanics
\end{itemize}

\textbf{This paper does not:}
\begin{itemize}
    \item Propose new dynamics or modifications to quantum mechanics
    \item Explain why specific measurement outcomes occur (the ``outcome problem'')
    \item Resolve interpretational debates about the ontology of quantum states
    \item Invoke consciousness, observers, or subjective elements
\end{itemize}

The analysis is structural: we clarify conditions under which measurement-like phenomena emerge, without claiming to have solved the measurement problem in its deepest form.

\subsection{Outline}

Section~\ref{sec:E-background} reviews relevant background on algebraic approaches to quantum mechanics and the HAFF framework. Section~\ref{sec:E-constraints} develops the three accessibility constraints in detail. Section~\ref{sec:E-measurement} reframes measurement in terms of these constraints. Section~\ref{sec:E-relations} discusses connections to existing approaches. Section~\ref{sec:E-nonclaims} states explicit non-claims. Section~\ref{sec:E-connection} situates the paper within the broader HAFF program. Section~\ref{sec:E-conclusion} concludes.

\section{Background}
\label{sec:E-background}

\subsection{Algebraic Approaches to Quantum Mechanics}

In the algebraic formulation of quantum mechanics, the fundamental objects are not wave functions or Hilbert spaces, but algebras of observables. A quantum system is characterized by a $*$-algebra $\mathcal{A}$ of bounded operators, and states are positive linear functionals on $\mathcal{A}$ \cite{Haag1996,Araki1999}.

This perspective has several advantages. It does not presuppose a specific Hilbert space representation, accommodates systems with infinitely many degrees of freedom, and naturally incorporates superselection rules. Most importantly for our purposes, it treats the specification of observables as logically prior to the specification of states.

\subsection{Observable Algebras in AQFT}

In algebraic quantum field theory (AQFT), local observable algebras $\mathcal{A}(\mathcal{O})$ are associated with spacetime regions $\mathcal{O}$, without invoking a global tensor product structure \cite{Haag1996}. The key insight is that subsystem structure emerges from the algebra of observables rather than being presupposed.

The HAFF framework extends this perspective by treating the selection of accessible algebras as physically constrained rather than given. Building on foundational work demonstrating that tensor product structures are observable-induced \cite{Zanardi2001,Zanardi2004}, different physical contexts---characterized by different interaction structures, stability conditions, and environmental couplings---yield different accessible algebras, and hence different effective physical descriptions.

\subsection{Accessible Algebras in HAFF}

Following \cite{Liu2026PaperA,Liu2026PaperB}, we define:

\begin{definition}[Accessible Algebra]
\label{def:E-accessible}
An \textbf{accessible algebra} $\mathcal{A}_{\mathbf{c}} \subset \mathcal{B}(\mathcal{H}_U)$ is a $*$-subalgebra satisfying physical constraints that ensure its elements correspond to operationally realizable observables within a given context $\mathbf{c}$.
\end{definition}

The subscript $\mathbf{c}$ denotes the \emph{context}---the totality of physical conditions (interaction Hamiltonian, environmental structure, timescales) that determine which observables are accessible. Different contexts yield different accessible algebras from the same underlying Hilbert space.

The present paper specifies three constraints that jointly determine $\mathcal{A}_{\mathbf{c}}$.

\section{Accessibility as Physical Constraint}
\label{sec:E-constraints}

We now develop the three constraints that determine which observables belong to the accessible algebra.

\subsection{Constraint 1: Interaction Coupling}

\begin{constraint}[Interaction]
\label{const:interaction}
An observable $\hat{O} \in \mathcal{B}(\mathcal{H}_U)$ satisfies the \textbf{interaction constraint} if it couples non-trivially to external degrees of freedom via the interaction Hamiltonian:
\begin{equation}
[\hat{O}, \hat{H}_{\text{int}}] \neq 0.
\end{equation}
\end{constraint}

\paragraph{Physical interpretation.}
An observable that commutes with all interaction terms is dynamically inert: it cannot be probed, recorded, or correlated with any external system. Such observables are mathematically well-defined but physically inaccessible.

\paragraph{Relation to measurement.}
Measurement requires that the system observable become correlated with apparatus degrees of freedom. This correlation is mediated by interaction. Observables that do not couple to any external system cannot, even in principle, be measured.

\begin{remark}
The interaction constraint is necessary but not sufficient for accessibility. An observable may couple to external degrees of freedom yet fail to satisfy stability or redundancy requirements.
\end{remark}

\subsection{Constraint 2: Dynamical Stability}

\begin{constraint}[Stability]
\label{const:stability}
An observable $\hat{O}$ satisfies the \textbf{stability constraint} if it remains approximately invariant under physically relevant dynamical maps $\mathcal{E}_t$:
\begin{equation}
\|\mathcal{E}_t(\hat{O}) - \hat{O}\| < \epsilon
\end{equation}
for timescales $t$ relevant to the physical process under consideration.
\end{constraint}

\paragraph{Physical interpretation.}
Observables that scramble rapidly---spreading their information across many degrees of freedom faster than any recording process can track---cannot be reliably measured. Stability ensures that the observable persists long enough to be correlated with records.

\paragraph{Relation to scrambling.}
In the language of quantum chaos, stable observables are those with slow out-of-time-order correlator (OTOC) growth \cite{Hayden2007}:
\begin{equation}
\langle [\hat{O}(t), \hat{V}(0)]^2 \rangle \ll 1 \quad \text{for } t \ll \tau_{\text{scrambling}}.
\end{equation}
Observables satisfying this condition resist rapid delocalization and maintain their identity under dynamical evolution.

\paragraph{Relation to decoherence.}
The stability constraint is closely related to the selection of pointer observables in decoherence theory \cite{Zurek2003}. Pointer observables are those that remain stable under system-environment interaction, forming the preferred basis in which the density matrix becomes approximately diagonal.

\begin{remark}[Threshold $\epsilon$]
The threshold $\epsilon$ is not a fundamental constant but depends on the physical context: the precision of available recording mechanisms, the timescales of interest, and the noise level of the environment. This context-dependence is a feature, not a bug---it reflects the operational nature of accessibility.
\end{remark}

\subsection{Constraint 3: Environmental Redundancy}

\begin{constraint}[Redundancy]
\label{const:redundancy}
An observable $\hat{O}$ satisfies the \textbf{redundancy constraint} if information about $\hat{O}$ is redundantly encoded across multiple independent environmental fragments $\{E_k\}$:
\begin{equation}
I(\hat{O} : E_k) \approx H(\hat{O}) \quad \text{for many } k,
\end{equation}
where $I(\cdot : \cdot)$ denotes quantum mutual information and $H(\cdot)$ denotes von Neumann entropy.
\end{constraint}

\paragraph{Physical interpretation.}
Redundancy ensures that information about the observable is not localized in a single environmental degree of freedom but is broadcast across many independent fragments. This makes the information robust and intersubjectively accessible: multiple independent observers can extract the same information without disturbing each other's records.

\paragraph{Relation to quantum Darwinism.}
The redundancy constraint formalizes the central insight of quantum Darwinism \cite{Zurek2009}: classical objectivity arises when information about a system is redundantly imprinted on the environment. Observables satisfying this constraint are precisely those for which multiple observers can agree on measurement outcomes.

\paragraph{Operational significance.}
Redundancy distinguishes \emph{objective} from \emph{subjective} information. An observable whose information is encoded in only a single environmental fragment is accessible to at most one observer; different observers would obtain different, incompatible records. Redundancy ensures that the observable's value is a matter of intersubjective fact.

\begin{remark}[Relation to classical objectivity]
The redundancy constraint provides a structural account of why certain observables behave ``classically'': their values are recorded multiply and independently, making them robust against local perturbations and accessible to multiple agents.
\end{remark}

\subsection{The Accessible Algebra}

\begin{definition}[Accessible Algebra via Constraints]
\label{def:E-accessible-full}
The \textbf{accessible algebra} $\mathcal{A}_{\mathbf{c}}$ relative to context $\mathbf{c}$ is the set of all observables satisfying Constraints~\ref{const:interaction}, \ref{const:stability}, and~\ref{const:redundancy}:
\begin{equation}
\mathcal{A}_{\mathbf{c}} = \{\hat{O} \in \mathcal{B}(\mathcal{H}_U) : \hat{O} \text{ satisfies Constraints 1, 2, and 3}\}.
\end{equation}
\end{definition}

The accessible algebra is not fixed \emph{a priori} but is determined by the physical context. Different interaction Hamiltonians, environmental structures, and timescales yield different accessible algebras from the same underlying Hilbert space.

\begin{table}[ht]
\centering
\small
\begin{tabular}{|P{2.5cm}|P{5cm}|P{5cm}|}
\hline
\textbf{Constraint} & \textbf{Condition} & \textbf{Physical Meaning} \\
\hline
Interaction & $[\hat{O}, \hat{H}_{\text{int}}] \neq 0$ & Observable couples to external degrees of freedom \\
\hline
Stability & $\|\mathcal{E}_t(\hat{O}) - \hat{O}\| < \epsilon$ & Observable persists under dynamics \\
\hline
Redundancy & $I(\hat{O} : E_k) \approx H(\hat{O})$ & Information redundantly encoded \\
\hline
\end{tabular}
\caption{Summary of the three accessibility constraints.}
\label{tab:constraints}
\end{table}

\section{Measurement Reframed}
\label{sec:E-measurement}

We now apply the accessibility framework to reinterpret quantum measurement.

\subsection{What Can Be Measured}

Within the present framework, the question ``What can be measured?'' receives a precise answer:

\begin{quote}
\textbf{An observable can be measured if and only if it belongs to the accessible algebra $\mathcal{A}_{\mathbf{c}}$.}
\end{quote}

Observables outside $\mathcal{A}_{\mathbf{c}}$---those failing one or more of the three constraints---are not measurable within context $\mathbf{c}$, regardless of their mathematical definition. This does not mean they ``do not exist'' in any metaphysical sense, but that they do not correspond to operationally realizable measurements within the given physical context.

\subsection{Measurement Outcomes}

Given an accessible observable $\hat{O} \in \mathcal{A}_{\mathbf{c}}$, its measurement outcomes are identified with its eigenvalue structure:

\begin{quote}
\textbf{Measurement outcomes are the eigenvalues of accessible observables.}
\end{quote}

This identification is standard in quantum mechanics. The novelty lies in restricting attention to \emph{accessible} observables: only those satisfying the three constraints yield operationally meaningful outcomes.

\subsection{Definiteness from Redundancy}

The definiteness of measurement outcomes---the fact that measurements yield single, definite results rather than superpositions---is traced to the redundancy constraint rather than to wave function collapse.

When an observable $\hat{O}$ satisfies the redundancy constraint, its eigenvalue is recorded in multiple independent environmental fragments. These records are mutually consistent: any fragment yields the same information about $\hat{O}$. This redundancy constitutes the objective, intersubjective definiteness of the measurement outcome.

\begin{quote}
\textbf{Definiteness is not imposed by collapse but constituted by redundant environmental encoding.}
\end{quote}

The global quantum state remains in superposition; what becomes definite is the content of redundant records, which all agree on the same eigenvalue.

\subsection{The Outcome Problem}

The framework does not explain why a \emph{particular} eigenvalue is recorded rather than another. This ``outcome problem'' remains open:

\begin{quote}
\textbf{We explain why outcomes are definite (redundancy), not why they are what they are.}
\end{quote}

This limitation is shared with decoherence-based approaches. The present framework does not claim to resolve this aspect of the measurement problem, only to clarify the structural conditions under which definite outcomes become possible.

\subsection{Measurement Without Observers}

A crucial feature of the framework is that measurement is characterized without reference to observers, agents, or consciousness:

\begin{quote}
\textbf{The ``observer'' is replaced by the ``interaction context.''}
\end{quote}

Any physical system satisfying the three constraints---be it a photon counter, a mineral surface, or an interstellar dust grain---constitutes a ``measurement site'' for the relevant observables. Human observers are a special case, not a privileged category.

\section{Relation to Existing Approaches}
\label{sec:E-relations}

\subsection{Decoherence Theory}

Decoherence theory explains how interference between quantum states is suppressed through environmental entanglement \cite{Zurek2003}. The present framework is fully compatible with decoherence and may be viewed as extending it in two respects:

\begin{enumerate}
    \item We make explicit the \emph{conditions} under which decoherence selects a preferred basis (the stability and redundancy constraints).
    \item We embed decoherence within the broader HAFF framework, connecting it to emergent geometry and gravitational phenomena.
\end{enumerate}

\subsection{Quantum Darwinism}

Quantum Darwinism \cite{Zurek2009} emphasizes the role of environmental redundancy in establishing classical objectivity. The redundancy constraint (Constraint~\ref{const:redundancy}) formalizes this insight as a criterion for accessibility.

The present framework may be viewed as situating quantum Darwinism within an algebraic setting, treating redundancy as one of three jointly necessary conditions for observability rather than as a standalone principle.

\subsection{QBism}

QBism \cite{Fuchs2014} interprets quantum states as expressions of an agent's beliefs. The present framework differs fundamentally: accessibility is determined by physical interaction structure, not by agent beliefs.

The key difference:
\begin{itemize}
    \item \textbf{QBism}: Dependence on agent's epistemic state (belief-determined)
    \item \textbf{HAFF}: Dependence on interaction structure (interaction-determined)
\end{itemize}

Both reject naive realism about quantum states, but the present framework maintains objectivity by grounding accessibility in physical constraints rather than subjective beliefs.

\subsection{Relational Quantum Mechanics}

Relational quantum mechanics (RQM) \cite{Rovelli1996} holds that quantum states are relational---defined only relative to a reference system. The present framework shares the emphasis on relationality but differs in its treatment of what grounds the relation:

\begin{itemize}
    \item \textbf{RQM}: Relations between systems (system-relative)
    \item \textbf{HAFF}: Stability conditions on algebras (interaction-determined)
\end{itemize}

HAFF may provide the stable ``nodes'' required for RQM's relational network: before relations can exist, there must be relata stable enough to participate in interactions.

\subsection{Algebraic Quantum Field Theory}

The closest structural affinity is with algebraic quantum field theory (AQFT) \cite{Haag1996}. Both frameworks treat observable algebras as primary and states as secondary. The present framework extends AQFT by:

\begin{enumerate}
    \item Providing explicit criteria (the three constraints) for algebra selection
    \item Connecting algebra selection to measurement and emergent geometry
    \item Situating AQFT insights within the broader HAFF program
\end{enumerate}

\begin{table}[ht]
\centering
\small
\begin{tabular}{|P{3cm}|P{4cm}|P{5.5cm}|}
\hline
\textbf{Approach} & \textbf{Key Mechanism} & \textbf{Relation to HAFF} \\
\hline
Decoherence & Environmental entanglement & Compatible; constraints specify conditions \\
\hline
Quantum Darwinism & Redundant encoding & Redundancy constraint formalizes this \\
\hline
QBism & Agent beliefs & Categorically distinct; HAFF is interaction-determined \\
\hline
RQM & System relations & Complementary; HAFF provides stable relata \\
\hline
AQFT & Observable algebras & Closest affinity; HAFF adds selection criteria \\
\hline
\end{tabular}
\caption{Relation of the present framework to existing approaches.}
\label{tab:relations}
\end{table}

\section{What This Paper Does NOT Claim}
\label{sec:E-nonclaims}

To prevent misinterpretation, we state explicitly what the paper does not claim.

\begin{enumerate}
    \item \textbf{No resolution of the outcome problem.} We do not explain why particular measurement outcomes occur, only why outcomes are definite.
    
    \item \textbf{No collapse postulate.} The framework assumes unitary evolution throughout. Definiteness arises from redundancy, not from non-unitary collapse.
    
    \item \textbf{No modification of quantum mechanics.} We do not propose new equations, new dynamics, or modifications to the standard formalism.
    
    \item \textbf{No consciousness or observer-dependence.} Accessibility is determined by physical constraints, not by conscious observers or epistemic states.
    
    \item \textbf{No claim that all measurement problems are solved.} The framework addresses the definiteness problem but leaves other aspects (the preferred basis problem, the tails problem) to be addressed in conjunction with existing approaches.
    
    \item \textbf{No claim of novelty regarding decoherence.} The framework builds on and is compatible with decoherence theory; it does not replace it.
    
    \item \textbf{No claim of universal applicability.} The analysis is confined to the HAFF framework and does not assert that all approaches to measurement must adopt this structure.
    
    \item \textbf{No metaphysical conclusions.} We do not claim that the accessible algebra exhausts reality, only that it exhausts what is operationally measurable within a given context.
    
    \item \textbf{No derivation of Born rule.} The framework does not derive the Born rule for outcome probabilities; it assumes standard quantum probability.
    
    \item \textbf{No claim about quantum-classical divide.} We do not assert a sharp boundary between quantum and classical; accessibility is context-dependent and admits degrees.
    
    \item \textbf{No hidden variables.} The framework does not invoke hidden variables or additional ontology beyond standard quantum mechanics.
    
    \item \textbf{No many-worlds commitment.} The framework is compatible with, but does not require, many-worlds interpretations.
    
    \item \textbf{No claim of interpretational neutrality.} While the framework avoids some interpretational commitments, it does adopt the structural stance of HAFF, which may not be neutral with respect to all interpretations.
\end{enumerate}

\section{Connection to the HAFF Framework}
\label{sec:E-connection}

\subsection{Relation to Previous Papers}

The present paper (Paper E) is part of a series developing the Holographic Alaya-Field Framework:

\begin{itemize}
    \item \textbf{Paper A} \cite{Liu2026PaperA}: Establishes that inequivalent coarse-graining structures induce inequivalent effective geometries from the same global quantum state.
    
    \item \textbf{Paper B} \cite{Liu2026PaperB}: Clarifies the structural (vs.\ epistemic) nature of accessibility and situates HAFF relative to existing interpretations.
    
    \item \textbf{Paper C} \cite{Liu2026PaperC}: Explores philosophical implications for causation, agency, and existence.
    
    \item \textbf{Paper D} \cite{Liu2026PaperD}: Proposes that gravitational dynamics corresponds to the adiabatic evolution of accessible algebras.
    
    \item \textbf{Paper E} (this paper): Reframes measurement as a manifestation of accessibility constraints.
    
    \item \textbf{Paper F} (forthcoming): Addresses temporal asymmetry as accessibility propagation.
\end{itemize}

\subsection{The Diagnostic Triangle: D + E + F}

Papers D, E, and F form a ``diagnostic triangle'' within the HAFF framework:

\begin{table}[ht]
\centering
\small
\begin{tabular}{|P{1.5cm}|P{2.5cm}|P{3.5cm}|P{4.5cm}|}
\hline
\textbf{Paper} & \textbf{Phenomenon} & \textbf{Traditional View} & \textbf{HAFF Reframing} \\
\hline
D & Gravity & Fundamental force & Evolution of accessible algebra \\
\hline
E & Measurement & Primitive process & Selection within accessible algebra \\
\hline
F & Time & Fundamental parameter & Direction of accessibility propagation \\
\hline
\end{tabular}
\caption{The diagnostic triangle: gravity, measurement, and time reframed as aspects of algebraic accessibility.}
\label{tab:triangle}
\end{table}

The unifying insight is that force, measurement, and time are not fundamental but are different projections of the structure of accessible algebras.

\subsection{Structural Link to Gravity}

Paper D establishes that gravity corresponds to the evolution of the accessible algebra $\mathcal{A}_{\mathbf{c}}(t)$. The present paper clarifies what determines $\mathcal{A}_{\mathbf{c}}$ at any given time: the three accessibility constraints.

The connection may be summarized as follows:

\begin{quote}
If gravity (Paper D) describes how information maps to spatial curvature, then measurement (Paper E) reveals the pruning criterion that determines which information participates in that mapping. Without accessibility constraints, the holographic map would include non-physical operators and yield divergent geometry. The finiteness of gravity is grounded in the finiteness of the accessible algebra.
\end{quote}

\section{Conclusion}
\label{sec:E-conclusion}

We have proposed that quantum measurement is not a primitive process but a manifestation of accessibility constraints on operator algebras.

The central results are:

\begin{enumerate}
    \item \textbf{Three accessibility constraints}: Interaction coupling, dynamical stability, and environmental redundancy jointly determine which observables are accessible within a given physical context.
    
    \item \textbf{Measurement reframed}: What can be measured is determined by membership in the accessible algebra; measurement outcomes are eigenvalues of accessible observables; definiteness arises from redundant environmental encoding.
    
    \item \textbf{Observer-independence}: The framework characterizes measurement without reference to observers, consciousness, or subjective elements. The ``observer'' is replaced by the ``interaction context.''
    
    \item \textbf{Compatibility}: The framework is compatible with unitary quantum mechanics, decoherence theory, and quantum Darwinism, while providing a structural account that connects measurement to the broader HAFF program.
\end{enumerate}

The framework does not resolve all aspects of the measurement problem. It does not explain why particular outcomes occur, nor does it derive the Born rule. What it provides is a structural clarification: the conditions under which measurement-like phenomena emerge from physical constraints on operator algebras.

Within the HAFF program, measurement joins gravity and time as phenomena that are not fundamental but emerge from the structure of accessible algebras. This diagnostic unification does not constitute a Theory of Everything, but it suggests that seemingly disparate foundational puzzles may share a common structural origin.


% ============================================================================
% Paper F
% ============================================================================
\chapter{Temporal Asymmetry as Accessibility Propagation}
\label{chap:paperF}

\begin{center}
\textit{Paper F}\\[0.5em]
Originally published: Zenodo, DOI: 10.5281/zenodo.18400425
\end{center}

\bigskip

\section*{Abstract}

We propose that temporal and causal asymmetry arise from the directional structure of accessibility propagation. Building on the Holographic Alaya-Field Framework (HAFF), which characterizes measurement as the selection of stable accessible algebras, we argue that the ``arrow of time'' is not a fundamental parameter but a consequence of how information spreads irreversibly into environmental degrees of freedom. The redundancy constraint central to accessibility---that information must be multiply recorded to be operationally accessible---is inherently asymmetric: information expands from few to many degrees of freedom, but the reverse process is statistically suppressed to the point of physical uninstantiability. This asymmetry defines a preferred direction that we identify with temporal ordering. No fundamental time parameter is assumed; all temporal ordering emerges from a partial order induced by algebraic inclusion and redundancy monotonicity. Causation is reframed as constraint propagation along this direction, with retrocausal trajectories being non-generic (measure zero) rather than forbidden by principle. A minimal mathematical model demonstrating irreversible redundancy expansion is provided in the appendix.


\section{Introduction}
\label{sec:F-intro}

\subsection{The Problem of Time's Arrow}

Among the deepest puzzles in physics is the origin of temporal asymmetry. The fundamental laws of physics---Newtonian mechanics, electromagnetism, quantum mechanics---are time-reversal invariant or nearly so. Yet our experience of the world is profoundly asymmetric: eggs break but do not unbreak; we remember the past but not the future; causes precede effects.

This tension between microscopic reversibility and macroscopic irreversibility has been recognized since Boltzmann's work on statistical mechanics \cite{Boltzmann1896}. The standard resolution appeals to special initial conditions: the universe began in a low-entropy state, and the second law of thermodynamics reflects the statistical tendency to evolve toward higher entropy \cite{Penrose1989,Carroll2010}.

While this explanation is widely accepted, it raises further questions:
\begin{itemize}
    \item Why should initial conditions be ``special''? What selects them?
    \item Is the thermodynamic arrow the only arrow, or are there independent sources of temporal asymmetry?
    \item In quantum gravity, where time itself may be emergent, how does any notion of ``before'' and ``after'' arise?
\end{itemize}

The present work does not claim to resolve these questions definitively. Instead, it offers a structural reframing: temporal asymmetry may be understood as a consequence of how accessible algebras propagate information.

\subsection{Central Thesis}

We propose that temporal asymmetry can be understood as follows:

\begin{quote}
\textbf{Central Thesis:} No fundamental time parameter is assumed. All temporal ordering emerges from a partial order induced by algebraic inclusion and redundancy monotonicity. The ``arrow of time'' is the direction of irreversible accessibility propagation---information spreads from localized degrees of freedom into distributed environmental records, and this expansion is statistically irreversible.
\end{quote}

This thesis builds on the accessibility framework developed in Paper E \cite{Liu2026PaperE}. Recall that an observable is accessible only if information about it is redundantly recorded in multiple environmental fragments (the redundancy constraint). This redundancy is achieved through physical processes that spread information outward---precisely the processes that define thermodynamic irreversibility.

The key insight is that the redundancy constraint is inherently asymmetric:
\begin{itemize}
    \item \textbf{Forward direction}: Information spreads from system to environment, creating multiple records. This satisfies the redundancy constraint.
    \item \textbf{Backward direction}: Contracting distributed information back into a localized system would require precise coordination of many degrees of freedom---a process of measure zero in the space of dynamical trajectories.
\end{itemize}

This asymmetry is not imposed by hand; it follows from the structure of accessibility itself.

\subsection{Scope and Limitations}

We state explicitly what this paper does and does not attempt.

\textbf{This paper does:}
\begin{itemize}
    \item Propose a structural account of temporal asymmetry based on accessibility propagation
    \item Derive temporal ordering from redundancy structure without assuming fundamental time
    \item Connect this account to the thermodynamic and quantum arrows
    \item Reframe causation as constraint propagation along the accessibility direction
    \item Provide a minimal mathematical model (Appendix A) demonstrating irreversible redundancy expansion
    \item Situate the analysis within the broader HAFF framework
\end{itemize}

\textbf{This paper does not:}
\begin{itemize}
    \item Derive the second law of thermodynamics from first principles
    \item Explain why initial conditions are low-entropy
    \item Resolve metaphysical debates about the nature of time (A-theory vs.\ B-theory, presentism vs.\ eternalism)
    \item Address free will, agency, or the phenomenology of temporal experience
    \item Propose new dynamical equations or empirical predictions
\end{itemize}

The analysis is structural. We examine how temporal asymmetry relates to accessibility structure, without claiming that this analysis exhausts the content of the problem.

\begin{remark}[Relation to Papers D and E]
Paper D \cite{Liu2026PaperD} argued that gravity reflects the evolution of accessible algebras. Paper E \cite{Liu2026PaperE} argued that measurement reflects the selection of accessible algebras. The present paper argues that time reflects the \emph{directionality} of accessibility propagation. Together, these three papers characterize the diagnostic layer of the HAFF framework:
\begin{itemize}
    \item D: Geometry (algebra evolution)
    \item E: Measurement (algebra selection)
    \item F: Time (algebra propagation direction)
\end{itemize}
\end{remark}

\subsection{Outline}

Section~\ref{sec:F-background} reviews the status of time in various physical theories. Section~\ref{sec:F-accessibility} develops the core technical content: how the accessibility constraints generate directional structure. Section~\ref{sec:F-causation} reframes causation as constraint propagation along the accessibility direction. Section~\ref{sec:F-relations} compares the present approach to existing accounts of temporal asymmetry. Section~\ref{sec:F-noncommitments} states explicit non-commitments. Section~\ref{sec:F-connection} discusses connections to the HAFF framework. Section~\ref{sec:F-conclusion} concludes. Appendix~\ref{app:model} provides a minimal mathematical model demonstrating the irreversibility of redundancy expansion.

\section{Background: Time in Physics}
\label{sec:F-background}

Before developing the accessibility-based account, we briefly review the status of time in major physical theories. The purpose is to identify a common structural assumption: that time is an external parameter, given rather than derived.

\subsection{Time in Classical and Quantum Mechanics}

In Newtonian mechanics, time is an absolute parameter. The equations of motion are time-reversal invariant: if $\mathbf{x}(t)$ is a solution, so is $\mathbf{x}(-t)$ (with velocities reversed). There is no intrinsic arrow.

In quantum mechanics, time evolution is governed by the Schr\"odinger equation:
\begin{equation}
i\hbar \frac{\partial}{\partial t} |\psi\rangle = \hat{H} |\psi\rangle.
\end{equation}
This equation is unitary and reversible. The apparent irreversibility of measurement is an interpretational issue, not a feature of the formalism itself.

\subsection{Time in Quantum Gravity}

In canonical approaches to quantum gravity, the Wheeler-DeWitt equation takes the form:
\begin{equation}
\hat{H} |\Psi\rangle = 0,
\end{equation}
where $|\Psi\rangle$ is the wave function of the universe. This equation contains no time parameter; the universe is described by a static state satisfying a constraint equation \cite{DeWitt1967}.

The Page-Wootters mechanism \cite{PageWootters1983} recovers effective time evolution from correlations between a ``clock'' subsystem and the rest of the universe within a timeless universal state. However, this mechanism explains how time \emph{ordering} emerges from correlations but does not explain why this ordering is \emph{asymmetric}.

\subsection{The Common Thread}

Across these theories, time appears either as an external parameter or as an emergent concept requiring additional input. The present framework offers a third perspective: time as a \textbf{structural consequence} of accessibility propagation, with directionality arising from the statistical asymmetry of redundancy expansion.

\section{Accessibility and Directionality}
\label{sec:F-accessibility}

We now develop the central technical content: how the accessibility constraints generate a preferred direction that can be identified with temporal ordering.

\subsection{Recap: The Redundancy Constraint}

Paper E \cite{Liu2026PaperE} established that an observable $\hat{O}$ belongs to the accessible algebra $\mathcal{A}_{\mathbf{c}}$ only if it satisfies three constraints, including the \emph{redundancy constraint}:
\begin{equation}
I(\hat{O} : E_k) \approx H(\hat{O}) \quad \text{for many } k,
\end{equation}
where $I(\cdot : \cdot)$ denotes quantum mutual information, $H(\cdot)$ denotes von Neumann entropy, and $\{E_k\}$ are independent environmental fragments.

This constraint ensures that information about accessible observables is distributed across multiple environmental subsystems, enabling intersubjective objectivity.

\subsection{The Redundancy Index}

We introduce a quantitative measure of redundancy:

\begin{definition}[Redundancy Index]
\label{def:F-redundancy}
For an observable $\hat{O}$ and environment $E = \bigotimes_k E_k$ consisting of $N$ fragments, the \textbf{redundancy index} $\mathcal{R}(\hat{O})$ is the number of environmental fragments that have acquired nearly complete information about $\hat{O}$:
\begin{equation}
\mathcal{R}(\hat{O}) = \sum_{k=1}^{N} \Theta\left( I(\hat{O} : E_k) - (1-\delta) H(\hat{O}) \right),
\end{equation}
where $\Theta$ is the Heaviside step function and $\delta \ll 1$ is the information loss tolerance.
\end{definition}

High redundancy ($\mathcal{R} \sim N$) corresponds to classical, objective observables. Low redundancy ($\mathcal{R} \sim 1$) corresponds to quantum, contextual observables.

\subsection{Asymmetry of Redundancy Flow}

The central observation is that redundancy expansion and contraction are radically asymmetric:

\begin{proposition}[Asymmetry of Redundancy Flow]
\label{prop:asymmetry}
Let $\hat{O}$ be an observable of a central system $S$ interacting with an $N$-fragment environment $E$. Then:
\begin{enumerate}
    \item \textbf{Expansion is generic}: Under typical interactions, $\mathcal{R}(\hat{O})$ increases from $\mathcal{R} = 0$ toward $\mathcal{R} \sim N$.
    \item \textbf{Contraction is non-generic}: The phase space volume of trajectories along which $\mathcal{R}$ decreases is exponentially suppressed:
    \begin{equation}
    \frac{\text{Vol}(\mathcal{R} \downarrow)}{\text{Vol}(\mathcal{R} \uparrow)} \sim e^{-\alpha N}
    \end{equation}
    for some $\alpha > 0$ depending on the fragment dimensions.
\end{enumerate}
\end{proposition}

The proof is provided in Appendix~\ref{app:model}. The key insight is that expansion requires only generic spreading of correlations, while contraction requires exponentially precise conspiracy among $N$ independent fragments.

\subsection{Temporal Direction from Redundancy Gradient}

This asymmetry induces a natural ordering on configurations of the accessible algebra:

\begin{definition}[Accessibility Ordering]
\label{def:F-ordering}
Let $\mathcal{A}_\alpha$ and $\mathcal{A}_\beta$ be two configurations of the accessible algebra (corresponding to different redundancy structures). We define the partial order:
\begin{equation}
\mathcal{A}_\alpha \prec \mathcal{A}_\beta \quad \Leftrightarrow \quad \mathcal{A}_\alpha \subset \mathcal{A}_\beta \text{ and } \mathcal{R}(\mathcal{A}_\beta) \geq \mathcal{R}(\mathcal{A}_\alpha).
\end{equation}
\end{definition}

This partial order is not imposed externally but emerges from the statistical structure of redundancy propagation. It constitutes the structural origin of temporal direction.

\begin{tcolorbox}[colback=gray!5!white,colframe=black!75!black,title=\textbf{Clarification: Arrow Without Fundamental Time}]
No fundamental time parameter is assumed. What might conventionally be written as $\mathcal{A}(t_1)$ and $\mathcal{A}(t_2)$ with $t_1 < t_2$ is here understood as $\mathcal{A}_\alpha \prec \mathcal{A}_\beta$---a partial order on algebraic configurations induced by redundancy monotonicity.

``Dynamical trajectories'' are not functions $\hat{O}(t)$ parametrized by external time, but \textbf{directed paths through the space of accessible algebras} $\{\mathcal{A}_\alpha\}$, with direction determined by the redundancy gradient:
\begin{equation}
\mathcal{A}_\alpha \subset \mathcal{A}_\beta \quad \text{with} \quad \mathcal{R}(\mathcal{A}_\beta) \geq \mathcal{R}(\mathcal{A}_\alpha).
\end{equation}

Time is not a parameter but the \textbf{inclusion order of accessible structures}.
\end{tcolorbox}

\subsection{The Statistical Nature of the Arrow}

The arrow of time, in this framework, is neither:
\begin{itemize}
    \item A fundamental law (time-reversal symmetry is not violated)
    \item A thermodynamic accident (entropy is not the primary concept)
    \item A cosmological boundary condition (no special initial state is assumed)
\end{itemize}

Rather, it is a \emph{statistical gradient}: the overwhelming majority of accessible-algebra configurations lie in the direction of increasing redundancy. Trajectories toward decreasing redundancy exist in principle but occupy exponentially vanishing phase space volume.

\begin{quote}
\textbf{Time is the statistical gradient of redundancy.}
\end{quote}

\subsection{Relation to Thermodynamic Arrow}

The accessibility arrow and the thermodynamic arrow are closely related but not identical:

\begin{table}[ht]
\centering
\small
\begin{tabular}{|P{3.5cm}|P{4.5cm}|P{4.5cm}|}
\hline
\textbf{Feature} & \textbf{Thermodynamic Arrow} & \textbf{Accessibility Arrow} \\
\hline
Defined by & Entropy increase & Redundancy expansion \\
\hline
Requires & Coarse-graining choice & Accessibility constraints \\
\hline
Fundamental quantity & $S = -k_B \text{Tr}(\rho \ln \rho)$ & $\mathcal{R}[\mathcal{A}]$ (redundancy index) \\
\hline
Applies to & Macroscopic systems & Any system with environment \\
\hline
\end{tabular}
\caption{Comparison of thermodynamic and accessibility arrows.}
\label{tab:arrows}
\end{table}

The accessibility arrow may be viewed as a \emph{generalization} of the thermodynamic arrow: it applies whenever accessibility constraints are satisfied, even in contexts where thermodynamic entropy is not well-defined (e.g., quantum gravitational regimes where spacetime is emergent).

\section{Causation as Constraint Propagation}
\label{sec:F-causation}

Having established that accessibility propagation defines a preferred direction, we now reframe causation in these terms.

\subsection{Causation Without Fundamental Time}

Traditional accounts of causation presuppose temporal ordering: causes precede effects. But if temporal ordering itself emerges from accessibility structure, causation must be reframed accordingly.

\begin{definition}[Causal Relation]
\label{def:F-causal}
An observable $\hat{A}$ is \textbf{causally prior} to observable $\hat{B}$ (written $\hat{A} \rightsquigarrow \hat{B}$) if:
\begin{enumerate}
    \item $\hat{A}$ and $\hat{B}$ are both accessible: $\hat{A}, \hat{B} \in \mathcal{A}_{\mathbf{c}}$
    \item The redundancy of $\hat{A}$ is established before the redundancy of $\hat{B}$: $\mathcal{R}(\hat{A})$ saturates at algebraic configuration $\mathcal{A}_\alpha$ while $\mathcal{R}(\hat{B})$ saturates at $\mathcal{A}_\beta$ with $\mathcal{A}_\alpha \prec \mathcal{A}_\beta$
    \item Counterfactual dependence holds: perturbations of $\hat{A}$ induce correlated perturbations of $\hat{B}$
\end{enumerate}
\end{definition}

This definition grounds causation in the propagation of accessibility constraints through environmental redundancy.

\subsection{Why Retrocausation is Non-Generic}

A persistent question in philosophy of physics is whether retrocausation---effects preceding causes---is possible. The present framework provides a structural answer:

\begin{proposition}[Suppression of Retrocausation]
\label{prop:retro}
Retrocausal trajectories are not excluded by principle, but are non-generic to the extent of being physically uninstantiable.
\end{proposition}

\begin{proof}[Proof sketch]
For $\hat{B}$ to causally influence $\hat{A}$ when $\mathcal{R}(\hat{A})$ is already saturated (information about $\hat{A}$ distributed across $N$ environmental fragments), the influence would need to:
\begin{enumerate}
    \item Propagate through all $N$ fragments simultaneously
    \item Reconverge the distributed information coherently
    \item Do so without disturbing the existing redundancy structure
\end{enumerate}
The phase space volume for such trajectories scales as $e^{-\alpha N}$ (Appendix~\ref{app:model}), rendering them statistically negligible for macroscopic $N$.
\end{proof}

\begin{remark}
This result does not ``forbid'' retrocausation by fiat. Rather, it explains why retrocausal scenarios---while not logically impossible---do not occur: they require exponentially fine-tuned conspiracies in Hilbert space that generically do not obtain. This is stronger than any ``causal postulate'' because it derives from the geometry of state space, not from an imposed principle.
\end{remark}

\subsection{Causal Structure Without Spacetime}

The causal relation $\rightsquigarrow$ defines a partial order on accessible observables with the following properties:
\begin{itemize}
    \item \textbf{Irreflexive}: $\hat{A} \not\rightsquigarrow \hat{A}$
    \item \textbf{Asymmetric}: $\hat{A} \rightsquigarrow \hat{B}$ implies $\hat{B} \not\rightsquigarrow \hat{A}$ (by Proposition~\ref{prop:retro})
    \item \textbf{Transitive}: $\hat{A} \rightsquigarrow \hat{B}$ and $\hat{B} \rightsquigarrow \hat{C}$ implies $\hat{A} \rightsquigarrow \hat{C}$
\end{itemize}

These properties are characteristic of causal structure and emerge here without presupposing a background temporal manifold.

\section{Relation to Existing Approaches}
\label{sec:F-relations}

We situate the accessibility-based account relative to existing approaches to temporal asymmetry.

\subsection{Comparison Table}

\begin{table}[ht]
\centering
\small
\begin{tabular}{|P{2.8cm}|P{3.5cm}|P{3.5cm}|P{3cm}|}
\hline
\textbf{Approach} & \textbf{Source of Arrow} & \textbf{What It Presupposes} & \textbf{Relation to HAFF} \\
\hline
Thermodynamic & Entropy increase & Coarse-graining choice & Accessibility more fundamental \\
\hline
Cosmological & Low-entropy Big Bang & Boundary conditions & Explains initial conditions \\
\hline
Decoherence & Interference suppression & System-environment split & Special case of accessibility \\
\hline
Causal set & Fundamental partial order & Causal order as primitive & HAFF derives the order \\
\hline
Page-Wootters & Correlations in static $|\Psi\rangle$ & Timeless formulation & HAFF adds directionality \\
\hline
\textbf{Accessibility} & \textbf{Redundancy expansion} & \textbf{Interaction structure} & \textbf{---} \\
\hline
\end{tabular}
\caption{Comparison of approaches to temporal asymmetry.}
\label{tab:comparison}
\end{table}

\subsection{Key Distinctions}

\textbf{Thermodynamic arrow}: The accessibility arrow is closely related but more fundamental. Entropy increase presupposes a coarse-graining; accessibility expansion explains \emph{why} certain coarse-grainings are physically relevant.

\textbf{Page-Wootters}: Both approaches treat time as emergent. Page-Wootters explains how time \emph{appears}; the accessibility framework explains why it has a \emph{direction}.

\textbf{Retrocausality programs}: Some approaches explore retrocausal models \cite{Price2012}. The present framework does not exclude retrocausation in principle but explains its non-occurrence: retrocausal trajectories occupy exponentially vanishing phase space volume.

\section{What This Paper Does NOT Claim}
\label{sec:F-noncommitments}

To prevent misreading, we state explicitly what this paper does \emph{not} claim.

\begin{enumerate}
    \item \textbf{No claim that time is unreal or illusory.} The framework reframes temporal asymmetry as emergent from accessibility structure, but this does not imply that time is ``merely subjective'' or non-existent.
    
    \item \textbf{No adjudication between A-theory and B-theory of time.} The framework is compatible with both presentism and eternalism.
    
    \item \textbf{No explanation of initial conditions.} We do not explain why the universe began with low redundancy, only why redundancy generically increases thereafter.
    
    \item \textbf{No resolution of the problem of time in quantum gravity.} The framework clarifies what temporal asymmetry \emph{means} in accessibility terms but does not derive time from the Wheeler-DeWitt equation.
    
    \item \textbf{No claims about consciousness or subjective time.} The phenomenology of temporal experience is not addressed.
    
    \item \textbf{No novel empirical predictions.} The analysis is structural, not dynamical.
    
    \item \textbf{No claim that retrocausation is impossible.} Retrocausation is statistically suppressed (measure zero), not logically forbidden.
    
    \item \textbf{No modification of quantum mechanics.} The framework assumes standard unitary evolution throughout.
\end{enumerate}

\section{Connection to HAFF Framework}
\label{sec:F-connection}

This paper completes the diagnostic layer of the HAFF framework.

\subsection{The D + E + F Diagnostic Triangle}

Papers D, E, and F form a coherent triad, each addressing a different aspect of how structure emerges from accessible algebras:

\begin{table}[ht]
\centering
\small
\begin{tabular}{|P{1.5cm}|P{2.5cm}|P{4cm}|P{4.5cm}|}
\hline
\textbf{Paper} & \textbf{Phenomenon} & \textbf{Traditional View} & \textbf{HAFF Reframing} \\
\hline
D & Gravity & Force between masses & Evolution of accessible algebra \\
\hline
E & Measurement & Primitive process & Selection within accessible algebra \\
\hline
F & Time & Fundamental parameter & Direction of accessibility propagation \\
\hline
\end{tabular}
\caption{The D + E + F diagnostic triangle.}
\label{tab:triad}
\end{table}

The unifying theme is that features traditionally taken as fundamental---force, measurement, time---may be understood as emergent properties of accessibility structure.

\subsection{Structural Link: D + E + F}

The three papers form a symmetric closed structure:
\begin{itemize}
    \item \textbf{D (Gravity)}: The evolution $\mathcal{A}_{\mathbf{c}}(t)$ of the accessible algebra manifests as curved geometry.
    \item \textbf{E (Measurement)}: The selection of $\mathcal{A}_{\mathbf{c}}$ via physical constraints manifests as objective outcomes.
    \item \textbf{F (Time)}: The direction of redundancy expansion within $\mathcal{A}_{\mathbf{c}}$ manifests as the causal arrow.
\end{itemize}

\subsection{Boundary Note: Toward Layer III}

The completion of Layer II (diagnostic unification) sets the stage for Layer III: the structural limits of the framework itself.

A key insight from D + E + F is that what appears fundamental (force, measurement, time) is actually emergent from accessible algebras. But this raises a question: \emph{What determines the accessible algebra structure itself?}

Layer III (Paper G) will argue that this question admits no complete answer within the framework---not because the framework is incomplete, but because any answer would require a ``meta-framework'' to justify, leading to infinite regress. The boundary is structural, not epistemic.

A theory that claims to explain everything must know where it must stop.

\section{Conclusion}
\label{sec:F-conclusion}

\subsection{Summary of Results}

We have proposed a structural account of temporal asymmetry and causation based on accessibility propagation. The central results are:

\begin{enumerate}
    \item \textbf{Time without fundamental parameter}: All temporal ordering emerges from a partial order induced by algebraic inclusion and redundancy monotonicity. No external time parameter is assumed.
    
    \item \textbf{The accessibility arrow}: Redundancy expansion is generic; redundancy contraction is exponentially suppressed. This asymmetry defines a preferred direction.
    
    \item \textbf{Causation as constraint propagation}: Causal relations emerge from constraint propagation along the accessibility arrow. Causes are sources of redundancy expansion; effects are regions of redundant recording.
    
    \item \textbf{Retrocausation non-generic}: Retrocausal trajectories are not excluded by principle, but are non-generic to the extent of being physically uninstantiable.
    
    \item \textbf{Diagnostic layer complete}: With Papers D, E, and F, the HAFF framework provides unified structural accounts of gravity, measurement, and time.
\end{enumerate}

\subsection{Closing Remark}

The arrow of time has puzzled physicists and philosophers for over a century. We do not claim to have dissolved this puzzle. What we have done is reframe it:

\begin{quote}
The question is not ``Why does entropy increase?'' but ``Why does accessibility expand?''
\end{quote}

The answer---that expansion is generic while contraction requires exponential fine-tuning---follows from the geometry of Hilbert space in interacting systems. This does not explain everything. But by identifying the structural basis of temporal asymmetry, we clarify what remains to be explained---and what, perhaps, lies beyond the reach of structural analysis altogether.

\section{A Minimal Model of Irreversible Redundancy Expansion}
\label{app:model}

To rigorously demonstrate the central thesis of Section~\ref{sec:F-accessibility}---that accessibility expansion is generic while contraction is statistically suppressed---we consider a finite-dimensional model of a central system interacting with a fragmented environment.

\subsection{The Star-Graph Interaction Setup}

Consider a central system $S$ (the ``source'' of accessibility) and an environment $E$ consisting of $N$ independent subsystems (fragments) $E_1, E_2, \ldots, E_N$. The total Hilbert space is:
\begin{equation}
\mathcal{H}_{\text{tot}} = \mathcal{H}_S \otimes \mathcal{H}_{E_1} \otimes \mathcal{H}_{E_2} \otimes \cdots \otimes \mathcal{H}_{E_N}.
\end{equation}

The interaction Hamiltonian generating accessibility is chosen to be of the ``pre-measurement'' type:
\begin{equation}
\hat{H}_{\text{int}} = g \sum_{k=1}^{N} \hat{O}_S \otimes \hat{M}_k,
\end{equation}
where $\hat{O}_S$ is the observable of $S$ becoming accessible, $\hat{M}_k$ are the monitoring operators of the environmental fragments, and $g$ is the coupling strength.

\subsection{Dynamics of Redundancy}

Assume the initial state is uncorrelated:
\begin{equation}
|\Psi(0)\rangle = |s\rangle_S \otimes |e_0\rangle_{E_1} \otimes |e_0\rangle_{E_2} \otimes \cdots \otimes |e_0\rangle_{E_N}.
\end{equation}

Under the unitary evolution $U(t) = e^{-i\hat{H}_{\text{int}}t/\hbar}$, the state evolves into an entangled superposition. For $\hat{O}_S$ with eigenstates $|o_i\rangle$:
\begin{equation}
|\Psi(t)\rangle = \sum_i c_i |o_i\rangle_S \otimes |E_i^{(1)}(t)\rangle \otimes |E_i^{(2)}(t)\rangle \otimes \cdots \otimes |E_i^{(N)}(t)\rangle,
\end{equation}
where $|E_i^{(k)}(t)\rangle$ are the relative states of the environmental fragments.

The mutual information $I(\hat{O}_S : E_k)$ grows as the fragments become correlated with $S$. By standard decoherence results \cite{Zurek2003}, for small $t$:
\begin{equation}
I(\hat{O}_S : E_k) \sim (gt)^2.
\end{equation}

\subsection{Proof of Asymmetry}

\textbf{Forward Evolution (Generic Expansion):}

As $t$ increases, information spreads to more fragments. For $gt \gg 1$, the redundancy index approaches its maximum:
\begin{equation}
\mathcal{R}(\hat{O}_S) \to N.
\end{equation}
This state corresponds to the ``classical plateau'' where the algebra generated by $\hat{O}_S$ is maximally accessible.

\textbf{Backward Evolution (Contraction Suppression):}

Consider the time-reversed evolution from a state of high redundancy. For $\mathcal{R}$ to decrease, the $N$ environmental fragments must conspiratorially un-correlate with $S$ simultaneously.

In the phase space of the total system $\mathcal{H}_{\text{tot}}$, let $V_{\text{low}}$ be the volume of states with low redundancy ($\mathcal{R} < \mathcal{R}_{\text{crit}}$) and $V_{\text{high}}$ be the volume of states with high redundancy ($\mathcal{R} \sim N$).

By counting Hilbert space dimensions, the ratio is exponentially suppressed:
\begin{equation}
\frac{V_{\text{low}}}{V_{\text{high}}} \sim e^{-\alpha N},
\end{equation}
where $\alpha > 0$ depends on the dimension of the fragments.

\subsection{Conclusion of Appendix}

While the dynamical laws ($U(t) = e^{-i\hat{H}t/\hbar}$) are reversible, the \textbf{Accessibility Flow} is structurally irreversible:
\begin{itemize}
    \item A trajectory starting in $V_{\text{low}}$ generically moves to $V_{\text{high}}$ (time arrow $\to$).
    \item A trajectory starting in $V_{\text{high}}$ will almost never spontaneously fluctuate back to $V_{\text{low}}$ within the recurrence time of the universe.
\end{itemize}

The irreversibility comes from \textbf{state space volume}, not from dynamical asymmetry. This is the structural basis of the accessibility arrow:

\begin{center}
\fbox{\textbf{Time is the statistical gradient of Redundancy.}}
\end{center}


% ============================================================================
% Paper G
% ============================================================================
\chapter{Structural Limits of Unification}
\label{chap:paperG}

\begin{center}
\textit{Paper G}\\[0.5em]
Originally published: Zenodo, DOI: 10.5281/zenodo.18402907
\end{center}

\bigskip

\section*{Abstract}

This paper examines the structural conditions under which a unificatory physical framework must terminate its explanatory extension. Building on recent work demonstrating that gravitational phenomena, measurement outcomes, and temporal asymmetries can be jointly reframed as consequences of accessible observable algebra selection, we argue that such frameworks cannot be simultaneously complete and self-grounding. The incompleteness identified here is neither formal (in the G\"{o}del--Turing sense) nor epistemic, but architectural: it arises from the non-self-grounding character of accessibility-based physical description. We establish a structural lemma showing that any attempt to internalize accessibility conditions within the framework they enable leads to either infinite regress or explanatory collapse. The stopping point identified is therefore not discretionary but forced by the framework's own explanatory architecture. This analysis does not claim generality beyond the specific formalism developed; whether alternative approaches would encounter analogous limits remains an open question. The contribution is methodological: to articulate the conditions under which recognizing structural boundaries becomes a requirement of explanatory coherence rather than an admission of incompleteness.


\section{Introduction}
\label{sec:G-intro}

\subsection{The Expectation of Completeness}

The aspiration toward a unified physical description has historically been guided by the expectation that deeper unification corresponds to increased completeness. In this traditional view, apparent multiplicity---of forces, degrees of freedom, or explanatory principles---is taken to signal provisional fragmentation, to be resolved by a more fundamental theory. A Theory of Everything, in its strongest formulation, is therefore often assumed to be both unifying and self-grounding: it should not only subsume all known interactions under a single framework, but also account for the conditions under which its own descriptions are possible.

The present work does not adopt this expectation. Instead, it advances a more restricted claim: that unification may be achievable only up to a structurally imposed boundary, beyond which further explanatory extension would undermine the coherence of the framework itself. This claim does not arise from epistemic modesty, nor from skepticism regarding the scope of physical explanation, but from the internal architecture of the formalism developed in the preceding papers of this series \cite{Liu2026PaperA,Liu2026PaperB,Liu2026PaperC,Liu2026PaperD,Liu2026PaperE,Liu2026PaperF}, which builds on foundational observations regarding the non-uniqueness of tensor factorizations \cite{Zanardi2001,Zanardi2004}.

\subsection{Summary of the Preceding Framework}

Across Papers D--F, gravitational phenomena, measurement outcomes, and temporal asymmetries are jointly reframed as consequences of structural selection: specifically, the selection of accessible observable algebras and associated coarse-grainings. No new fundamental entities are postulated, and no modification of underlying dynamics is proposed. Rather, phenomena traditionally treated as primitive are shown to emerge from constraints on how physical descriptions are stably instantiated.

\begin{itemize}
    \item \textbf{Paper D}: Gravitational dynamics corresponds to the adiabatic flow of the accessible algebra $\mathcal{A}_{\mathbf{c}}(t)$, not to a force operating within a fixed algebra.
    \item \textbf{Paper E}: Measurement is not a primitive process but a manifestation of accessibility constraints on operator algebras---specifically, the selection of observables satisfying interaction, stability, and redundancy criteria.
    \item \textbf{Paper F}: Temporal directionality is identified with the direction of irreversible accessibility propagation, grounded in the asymmetric expansion of redundant environmental records.
\end{itemize}

These results share a common structure: each phenomenon is traced to accessibility conditions rather than to fundamental ontology. This constitutes a genuine unification at the level of explanatory architecture.

\subsection{The Problem of Self-Grounding}

However, this reframing has a nontrivial implication. If the explanatory power of the framework depends essentially on restrictions---on what is accessible, stable, and non-scrambling---then unification cannot consist in the removal of all such restrictions. To do so would be to erase the very conditions that render physical description meaningful. Unification, in this sense, cannot be both total and self-enclosed.

The unifying move, therefore, is not a convergence toward an all-encompassing description, but a clarification of how far structural explanation can be coherently extended before it becomes reflexive. The aim of this final layer is to articulate that stopping point and to demonstrate that it is structurally forced rather than pragmatically chosen.

\subsection{Scope and Limitations}

Several clarifications are necessary at the outset.

First, the incompleteness identified in this paper is not formal in the sense of G\"{o}del's incompleteness theorems. We do not claim that the framework contains undecidable propositions within a formal system, nor do we invoke metamathematical results. The incompleteness is \emph{architectural}: it concerns the explanatory roles within a physical framework, not provability within a formal calculus.

Second, we do not claim that the structural limits identified here apply universally to all conceivable approaches to unification. The present analysis is confined to the algebraic and coarse-graining-based framework developed in Papers A--F. Whether alternative formalisms---category-theoretic, non-algebraic, or radically background-free---would exhibit analogous limits is an open question that we do not address.

Third, the stopping point identified is not temporal, existential, or normative. It does not mark the end of physics, nor a claim about the limits of human knowledge. It marks the point at which the framework's internal explanatory resources are exhausted without circularity.

\subsection{Structure of the Paper}

Section~\ref{sec:G-accessibility} develops the notion of accessibility as a non-global structural constraint and clarifies its distinction from epistemic limitations. Section~\ref{sec:G-lemma} establishes a structural lemma demonstrating that accessibility-based descriptions cannot be self-grounding without collapse or regress. Section~\ref{sec:G-collapse} presents a concrete collapse scenario illustrating what would occur if the framework were extended beyond its structural boundary. Section~\ref{sec:G-cut} characterizes the final cut as a forced stopping point rather than a discretionary choice. Section~\ref{sec:G-nonclaims} states explicit non-claims to prevent misinterpretation. Section~\ref{sec:G-conclusion} concludes with reflections on the methodological significance of the analysis.

\section{Accessibility as a Non-Global Constraint}
\label{sec:G-accessibility}

\subsection{The Role of Accessibility in HAFF}

Central to the HAFF framework is the notion of accessibility. Physical descriptions are not formulated over the full algebra of global observables, but over restricted subalgebras determined by interaction structure, dynamical stability, and environmental redundancy. These restrictions are not introduced as pragmatic simplifications, nor as reflections of limited knowledge. They are constitutive of what counts as a well-defined physical description in the first place.

In Papers D--F, this point is developed across distinct domains:

\begin{itemize}
    \item Effective geometry is shown to depend on coarse-grainings that preserve entanglement structure over relevant timescales.
    \item Measurement outcomes are shown to arise from dynamically stable partitions that resist rapid scrambling.
    \item Temporal directionality is associated with asymmetric information flow under constrained interactions.
\end{itemize}

In each case, the phenomenon under consideration becomes intelligible only relative to a selected accessible algebra.

\subsection{Structural vs.\ Epistemic Constraints}

A crucial distinction must be drawn between epistemic and structural constraints. This distinction is contested in philosophy of physics, and we acknowledge that what follows adopts it as a working criterion rather than a demonstrated result.

\begin{definition}[Epistemic Constraint]
A constraint is \textbf{epistemic} if it concerns what can be known, inferred, or verified by agents, given their informational position.
\end{definition}

\begin{definition}[Structural Constraint]
A constraint is \textbf{structural} if it concerns what descriptions are well-defined, given a pattern of physical interactions, independent of any agent's knowledge or epistemic state.
\end{definition}

The HAFF framework relies exclusively on the latter notion. Accessibility is determined by stability criteria---dynamical invariance, environmental redundancy (quantum Darwinism), and non-scrambling behavior---that are properties of the Hamiltonian and the global quantum state, not of observers.

Crucially, the selection of an accessible algebra is not arbitrary. Given a fixed interaction structure, different agents---or no agents at all---will identify the same accessible observables. This is the sense in which accessibility is structural rather than epistemic: it is interaction-determined, not belief-determined.

\begin{remark}[Contested Distinction]
We acknowledge that this distinction between epistemic and structural constraints is philosophically contested. The framework does not claim to have resolved this broader debate. However, the burden of argument lies with the critic to demonstrate that the stability criteria invoked in Papers A--F reduce to epistemic conditions, rather than with the framework to prove a negative. The working distinction is adopted on the grounds that interaction-determined constraints are conceptually prior to agent-relative knowledge.
\end{remark}

\subsection{Non-Globality of Accessibility}

Accessibility is not a global property of the underlying theory. There is no privileged, all-encompassing accessible algebra from which all others can be derived. Each effective description presupposes its own restrictions, and those restrictions cannot be fully specified from within the description they enable.

This asymmetry is decisive. Any attempt to internalize the conditions of accessibility would require a further level of description, governed by its own accessibility conditions. The implications of this observation are developed in the following section.

\section{A Structural Lemma on Self-Grounding}
\label{sec:G-lemma}

\subsection{Statement of the Lemma}

We now state the central structural result of this paper.

\begin{lemma}[Structural Non-Self-Grounding]
\label{lem:nonsg}
Within the HAFF framework, no description can simultaneously:
\begin{enumerate}
    \item[(i)] specify the structure of accessibility, and
    \item[(ii)] be formulated entirely within that same accessibility structure,
\end{enumerate}
without collapse into circularity or triviality.
\end{lemma}

\subsection{Argument}

The argument proceeds in five steps.

\paragraph{Step 1: All physical descriptions in HAFF are formulated relative to an accessible algebra.}
This is not an optional modeling choice but the basic condition under which any observable, geometry, or temporal ordering becomes definable. Papers D--F establish that gravitational dynamics, measurement outcomes, and causal direction all presuppose restriction to a stable accessible subalgebra $\mathcal{A}_{\mathbf{c}} \subset \mathcal{B}(\mathcal{H}_U)$.

\paragraph{Step 2: Accessibility itself is defined by selection criteria.}
Stability, redundancy, and interaction locality determine which subalgebras are accessible. These criteria are conditions of possibility for description, not objects described within the description.

\paragraph{Step 3: Attempting to internalize accessibility requires re-applying accessibility criteria to themselves.}
That is, one would need an accessible algebra that describes the selection of the accessible algebra itself. The framework would have to render the conditions of its own applicability as objects within its descriptive scope.

\paragraph{Step 4: This generates a fixed-point requirement.}
The framework would have to identify an algebra $\mathcal{A}^*$ that:
\begin{itemize}
    \item is accessible because it satisfies the stability criteria, and
    \item simultaneously encodes the criteria by which it is judged accessible.
\end{itemize}
Symbolically, one would require:
\begin{equation}
\mathcal{A}^* \in \text{Acc}(\mathcal{A}^*),
\end{equation}
where $\text{Acc}(\cdot)$ denotes the set of algebras satisfying the accessibility criteria defined within the argument algebra.

\paragraph{Step 5: Such a fixed point is generically unavailable.}
Except in degenerate cases---trivial algebras (containing only the identity) or total algebras (the full $\mathcal{B}(\mathcal{H}_U)$ that erases all structure)---the selection criteria cannot be satisfied by their own output. A non-trivial accessible algebra defines distinctions (between accessible and inaccessible, stable and scrambled, redundant and local); encoding the criteria for those distinctions within the algebra would require the algebra to contain its own meta-description, which exceeds the information available at the object level.

\subsection{Conclusion of the Lemma}

Therefore, the framework cannot close on itself without either:
\begin{itemize}
    \item collapsing into triviality (everything accessible, nothing distinguished), or
    \item introducing an external meta-structure (violating internal coherence).
\end{itemize}

The stopping point is not pragmatic. It is forced by the non-self-grounding character of accessibility-based description.

\begin{remark}[Framework-Relative Claim]
This is a necessity claim internal to the framework's architecture, not a universal limitation on explanation. We do not claim that all physical theories must exhibit this structure, only that the HAFF framework, as developed, does.
\end{remark}

\section{A Collapse Scenario}
\label{sec:G-collapse}

To make the structural lemma concrete, we now present a hypothetical extension of HAFF that attempts to fully internalize accessibility as an object-level dynamical variable, and show that this attempt fails.

\subsection{Hypothetical Extension}

\paragraph{Step 1: Treat accessibility as a physical observable.}
Suppose one introduces an operator or state variable $\hat{A}$ encoding ``degree of accessibility'' for subalgebras. This variable would quantify, for each subalgebra $\mathcal{A} \subset \mathcal{B}(\mathcal{H}_U)$, the extent to which it satisfies the stability criteria.

\paragraph{Step 2: Demand dynamical laws for accessibility.}
To be explanatory, $\hat{A}$ must:
\begin{itemize}
    \item evolve under some dynamics, and
    \item be measurable within the theory.
\end{itemize}

\paragraph{Step 3: Apply accessibility criteria to $\hat{A}$.}
But measurability requires that $\hat{A}$ itself satisfy:
\begin{itemize}
    \item stability under interaction,
    \item redundancy across environments, and
    \item non-scrambling behavior.
\end{itemize}
That is, $\hat{A}$ must belong to some accessible algebra $\mathcal{A}_{\hat{A}}$.

\subsection{Two Fatal Outcomes}

\paragraph{Outcome (a): Infinite regress.}
The algebra $\mathcal{A}_{\hat{A}}$ that makes $\hat{A}$ accessible is itself defined by accessibility criteria. To explain why $\mathcal{A}_{\hat{A}}$ is accessible, one would need a further algebra $\mathcal{A}_{\mathcal{A}_{\hat{A}}}$, and so on. Each level of accessibility-description requires a higher-level accessibility structure to define its observables. The regress does not terminate.

\paragraph{Outcome (b): Totalization collapse.}
To avoid regress, one might declare everything accessible---that is, take $\mathcal{A}_{\mathbf{c}} = \mathcal{B}(\mathcal{H}_U)$. But then:
\begin{itemize}
    \item no algebra selection remains,
    \item no measurement distinction exists (all observables are equally accessible),
    \item effective geometry loses definition (no coarse-graining induces structure), and
    \item temporal direction vanishes (no asymmetric accessibility propagation).
\end{itemize}
The framework either never terminates or destroys the very distinctions it set out to explain.

\subsection{Conclusion of the Scenario}

Any attempt to go ``beyond'' Papers D--F by internalizing accessibility eliminates the explanatory power already achieved. This is not philosophical caution. It is structural self-destruction.

The collapse scenario demonstrates concretely what the structural lemma establishes abstractly: the framework cannot extend itself to explain its own conditions of applicability without losing the capacity to explain anything at all.

\section{The Necessity of a Final Cut}
\label{sec:G-cut}

\subsection{Stopping as Structural Necessity}

The preceding analyses motivate a specific sense in which the HAFF framework is incomplete. This incompleteness is neither formal nor metaphysical. It does not arise from undecidable propositions, nor from claims about the limits of human cognition. Rather, it is structural: a consequence of the fact that explanatory resources cannot simultaneously function as both explanans and explanandum.

Within the framework developed here, gravity, measurement, and time are unified at the level of structural selection. They are shown to depend on how observable algebras are restricted and stabilized. This constitutes a genuine unification, insofar as disparate phenomena are traced to a common architectural feature. Yet the framework does not, and cannot, provide a further account of why those accessibility conditions obtain, without appealing to structures that would themselves require explanation under the same terms.

\subsection{The Final Cut}

The notion of a ``final cut'' is introduced to mark this boundary. It does not denote a temporal endpoint, nor a claim about the completion of physics. It denotes the point at which the internal explanatory strategy of the framework reaches saturation. Beyond this point, further elaboration would no longer clarify structure, but obscure it by erasing the asymmetries that make explanation possible.

The introduction of a final cut is not a discretionary methodological choice, nor a gesture of philosophical modesty. It is the point at which the framework exhausts its own internal resources without contradiction.

Beyond this point, any further extension would require the framework to explain the conditions of its own applicability using those very conditions---a requirement that admits no non-degenerate solution (Lemma~\ref{lem:nonsg}).

The stopping point is therefore not selected but encountered. It is the boundary at which explanation ceases to be generative and becomes self-consuming.

\subsection{Incompleteness as Internal Limit}

In this sense, the incompleteness identified here is not provisional, nor external, nor epistemic. It is structural and internal: a limit imposed by the framework's success in making accessibility do explanatory work.

The framework terminates not in absence, but at the level of unselected structure---a domain that admits no geometry, no temporal ordering, and no observational standpoint, yet functions as the necessary substrate from which all three are selectively realized.

A theory may approach totality only by knowing where it must stop---relative to its own structure.

\begin{remark}[Non-Arbitrary Stopping]
The stopping point is non-arbitrary in the following precise sense: any proposed extension beyond this point can be shown to lead either to regress (Section~\ref{sec:G-collapse}, Outcome a) or to collapse (Section~\ref{sec:G-collapse}, Outcome b). The burden of argument therefore shifts to those who claim that further extension is possible: they must specify how regress or collapse is avoided.
\end{remark}

\section{Scope and Non-Claims}
\label{sec:G-nonclaims}

To prevent misinterpretation, several non-claims must be stated explicitly.

\begin{enumerate}
    \item \textbf{No invocation of G\"{o}del's theorems.} This work does not invoke G\"{o}del's incompleteness theorems, nor does it rely on any formal analogy to them. The incompleteness identified here is architectural, not metamathematical. It concerns explanatory roles within a physical framework, not provability within a formal system.
    
    \item \textbf{No claim of universal applicability.} No claim is made that accessibility constraints apply universally across all conceivable approaches to unification. The present analysis is confined to the algebraic and coarse-graining-based framework developed in Papers A--F.
    
    \item \textbf{Contested distinction acknowledged.} The distinction between structural and epistemic constraints is acknowledged to be contested in the philosophy of physics. The framework adopts this distinction as a working criterion, grounded in interaction-determined stability conditions. It does not claim to have resolved the broader philosophical debate.
    
    \item \textbf{No foreclosure of alternative frameworks.} The identification of a final cut does not preclude further physical progress, alternative models, or deeper insights within other frameworks. It merely states that, within the present formalism, further internal extension would be incoherent.
    
    \item \textbf{No uniqueness claim.} We do not claim that this stopping point is unique. Other frameworks may stop elsewhere, or may not require stopping at all. The claim is only that, given the explanatory architecture adopted here, the stopping point identified is forced.
    
    \item \textbf{No philosophical finality.} This is a structural observation within a specific formalism, not a philosophical conclusion about the ultimate nature of reality or the limits of knowledge as such. The stopping point identified is methodological and structural, not existential.
    
    \item \textbf{No consciousness or observer-creation claims.} The framework does not claim that consciousness plays a fundamental role, that observers create reality, or that accessibility is observer-relative in any subjective sense.
    
    \item \textbf{No claim of novelty regarding theoretical presupposition.} We acknowledge that the observation that explanatory frameworks have presuppositions they cannot fully explain is familiar in philosophy of science. The contribution here is to show that the specific HAFF framework forces this conclusion through its reliance on accessibility, not merely that it is compatible with it.
\end{enumerate}

\section{Conclusion}
\label{sec:G-conclusion}

\subsection{Summary of Results}

This paper has examined the structural conditions under which the HAFF framework must terminate its explanatory extension.

The central results are:

\begin{enumerate}
    \item \textbf{Accessibility as non-global constraint}: Physical descriptions in HAFF are formulated relative to accessible algebras, which are determined by stability criteria that cannot be fully specified from within the descriptions they enable.
    
    \item \textbf{Structural non-self-grounding} (Lemma~\ref{lem:nonsg}): No description within HAFF can simultaneously specify the structure of accessibility and be formulated entirely within that accessibility structure, without collapse or regress.
    
    \item \textbf{Collapse scenario}: Any attempt to internalize accessibility as an object-level variable leads to infinite regress or totalization collapse, eliminating the framework's explanatory power.
    
    \item \textbf{Forced stopping point}: The final cut is not discretionary but encountered as a structural necessity---the point at which further extension would be self-consuming rather than generative.
\end{enumerate}

\subsection{Methodological Significance}

Unification is often equated with the elimination of boundaries. The analysis presented here suggests a different criterion: that a unifying framework should distinguish between boundaries that are provisional and those that are structural.

Papers D--F identify such structural boundaries in the treatment of gravity, measurement, and time. This final layer marks the point at which acknowledging those boundaries becomes a condition of explanatory clarity rather than an admission of incompleteness.

The value of the framework lies not in its ability to say everything, but in its ability to determine what cannot be said without loss of coherence. At that point, stopping is not a retreat, but a completion.

\subsection{Open Questions}

Several questions remain beyond the scope of this analysis:

\begin{itemize}
    \item Whether alternative frameworks (category-theoretic, non-algebraic, or background-free) would exhibit analogous structural limits.
    \item Whether the structural/epistemic distinction adopted here can be given a more robust philosophical foundation.
    \item Whether the collapse scenario admits any non-trivial avoidance strategies not considered here.
\end{itemize}

These questions are left for future investigation.


% ============================================================================
% Postscript
% ============================================================================
\chapter*{Postscript: On the Closure of Structure}
\addcontentsline{toc}{chapter}{Postscript}

\begin{center}
\textit{Originally published: Zenodo, DOI: 10.5281/zenodo.18407367}
\end{center}

\bigskip

% ============================================================================
% POSTSCRIPT
% ============================================================================

In the Holographic Alaya--Field Framework, the universe has been treated not as a collection of fundamental objects, but as a bounded domain of accessibility within a single global operator structure \cite{Liu2026PaperA,Liu2026PaperB}. Throughout this work, gravity, time, and measurement have appeared only insofar as stable distinctions are sustained by a persistent separation---what we have called the Cut---between accessible subalgebras and the total algebra \cite{Liu2026PaperE,Liu2026PaperG}. The philosophical implications of this structural stance---for causation, agency, and existence---have been explored in the accompanying essay \cite{Liu2026PaperC}.

Pushing this framework to its logical limit raises a natural question: what becomes of the theory when such distinctions can no longer be maintained?

\section*{Heat Death as Accessibility Saturation}

From a structural perspective, the conventional notion of heat death admits a reinterpretation. Rather than signifying the disappearance of physical existence, it corresponds to the saturation of accessibility. As informational redundancy becomes maximal, differences between subsystems cease to be stably recordable \cite{Zurek2009}. The gradients that underwrite locality, temporal ordering, and effective classicality flatten into a homogeneous configuration.

In this limit, the Cut loses operational meaning. No stable partition remains that could support observers, records, or localized descriptions. The system approaches the undifferentiated operator structure introduced at the beginning of this work---a state of maximal symmetry and minimal distinguishability.

\section*{Structural Equivalence of Origin and Terminus}

Crucially, this endpoint is not structurally distinct from the origin. The state of maximal entropy reached at late times is, in algebraic terms, indistinguishable from the maximally symmetric pre-differentiated configuration. The difference between ``beginning'' and ``end'' is therefore not ontological, but structural: it reflects whether accessibility constraints are present or dissolved.

Mathematically, let $\mathcal{A}_{\text{total}}$ denote the full operator algebra and let $S[\rho]$ denote the von Neumann entropy of a state $\rho$. At both temporal extremes:
\begin{equation}
\lim_{t \to 0^+} S[\rho(t)] \approx \lim_{t \to \infty} S[\rho(t)] \approx S_{\max},
\end{equation}
where the limits are understood in terms of accessible structure rather than absolute time. The initial state (pre-Cut) and the final state (post-dissolution) occupy the same region of algebraic configuration space---both correspond to conditions under which no stable coarse-graining can be sustained.

\section*{Bounded Evolution Without Cyclicity}

Seen this way, cosmic evolution traces neither a linear narrative nor a teleological arc. It is instead bounded by two structurally equivalent limits: one preceding the emergence of stable distinctions, and one following their dissolution. The domain in which physics, observation, and meaning are possible occupies only the intermediate regime, where the Cut is sustained.

This observation carries no additional dynamical claims, nor does it posit a cosmological cycle in the sense of a Big Bounce or oscillating universe model \cite{Penrose2010}. It merely completes the logical closure of the framework developed here. The theory describes the conditions under which structure can appear, persist, and ultimately fail. Beyond those conditions, no further physical description is available---not because reality ends, but because the criteria for description are no longer satisfied.

\section*{The Contingency of Intelligibility}

If the work has a final implication, it is a modest one: intelligibility itself is contingent. The universe is describable only while distinctions endure. Understanding this boundary does not diminish the value of structure; it clarifies the narrow window in which structure---and thus physics---is possible.

The framework terminates not in absence, but at the level of unselected structure: a domain that admits no geometry, no temporal ordering, and no observational standpoint, yet functions as the necessary substrate from which all three are selectively realized.

\begin{remark}[On Structural Closure]
The identification of origin and terminus as algebraically equivalent does not constitute a prediction about cosmological dynamics. It is a statement about the explanatory boundaries of accessibility-based description. Within those boundaries, the framework provides a unified account of gravity, measurement, and time \cite{Liu2026PaperD,Liu2026PaperE,Liu2026PaperF}. Beyond them, no description formulated in terms of accessible algebras can be coherently maintained \cite{Liu2026PaperG}.
\end{remark}

\section*{Concluding Reflection}

The value of a theoretical framework lies not only in what it explains, but in what it determines cannot be explained without loss of coherence. The stopping point identified in this work is not a failure of explanation but its completion. A theory that claims to explain everything must know where it must stop.

\bigskip

\noindent\textit{Clarity does not require totality. And knowing where to stop is sometimes the most precise act of understanding.}

\newpage

% ============================================================================
% REFERENCES
% ============================================================================

% ============================================================================
% Acknowledgments
% ============================================================================
\chapter*{Acknowledgments}
\addcontentsline{toc}{chapter}{Acknowledgments}

The author thanks the anonymous reviewers for their insightful comments and suggestions, which greatly improved the clarity and rigor of this work.

% ============================================================================
% Consolidated Bibliography
% ============================================================================
\backmatter
\begin{thebibliography}{99}

\bibitem{Araki1999}
H. Araki, \emph{Mathematical Theory of Quantum Fields}, Oxford University Press (1999).

\bibitem{Asanga_Mahayana}
Asa\.{n}ga, \emph{Mah\={a}y\={a}nasa\d{m}graha (Summary of the Great Vehicle)}, trans.\ \'E. Lamotte, Peeters (1973).

\bibitem{Bengtsson2006}
I. Bengtsson and K. \.Zyczkowski, \emph{Geometry of Quantum States: An Introduction to Quantum Entanglement}, Cambridge University Press (2006).

\bibitem{BisognanoWichmann1975}
J.~J.~Bisognano and E.~H.~Wichmann,
\emph{On the duality condition for a Hermitian scalar field},
J.\ Math.\ Phys.\ \textbf{16}, 985 (1975).

\bibitem{BisognanoWichmann1976}
J.~J.~Bisognano and E.~H.~Wichmann,
\emph{On the duality condition for quantum fields},
J.\ Math.\ Phys.\ \textbf{17}, 303 (1976).

\bibitem{Boltzmann1896}
L. Boltzmann, \emph{Vorlesungen \"uber Gastheorie}, J. A. Barth, Leipzig (1896).

\bibitem{BratteliRobinson1997}
O.~Bratteli and D.~W.~Robinson,
\emph{Operator Algebras and Quantum Statistical Mechanics},
Vols.\ 1--2, 2nd ed., Springer (1997).

\bibitem{Carroll2010}
S. Carroll, \emph{From Eternity to Here: The Quest for the Ultimate Theory of Time}, Dutton (2010).

\bibitem{Chalmers1996}
D. J. Chalmers, \emph{The Conscious Mind: In Search of a Fundamental Theory}, Oxford University Press (1996).

\bibitem{CasiniHuertaMyers2011}
H.~Casini, M.~Huerta, and R.~C.~Myers,
\emph{Towards a derivation of holographic entanglement entropy},
JHEP \textbf{1105}, 036 (2011),
arXiv:1102.0440.

\bibitem{Connes1994}
A.~Connes, \emph{Noncommutative Geometry}, Academic Press (1994).

% Neuroscience

\bibitem{DeWitt1967}
B. S. DeWitt, \emph{Quantum Theory of Gravity. I. The Canonical Theory}, Phys. Rev. \textbf{160}, 1113 (1967).

\bibitem{Eisert2010}
J.~Eisert, M.~Cramer, and M.~B.~Plenio,
\emph{Colloquium: Area laws for the entanglement entropy},
Rev.\ Mod.\ Phys.\ \textbf{82}, 277 (2010).

\bibitem{Faulkner2014}
T. Faulkner, M. Guica, T. Hartman, R. C. Myers, and M. Van Raamsdonk, \emph{Gravitation from Entanglement in Holographic CFTs}, JHEP \textbf{03}, 051 (2014).

\bibitem{French2014}
S. French, \emph{The Structure of the World: Metaphysics and Representation}, Oxford University Press (2014).

% Buddhist Primary Texts (Yogacara)

\bibitem{Fuchs2014}
C. A. Fuchs, \emph{QBism, the Perimeter of Quantum Bayesianism}, arXiv:1003.5209 (2014).

\bibitem{FuchsMerminSchack2014}
C. A. Fuchs, N. D. Mermin, and R. Schack, \emph{An Introduction to QBism with an Application to the Locality of Quantum Mechanics}, Am. J. Phys. \textbf{82}, 749 (2014).

\bibitem{Garfield2002}
J. L. Garfield, \emph{Empty Words: Buddhist Philosophy and Cross-Cultural Interpretation}, Oxford University Press (2002).

\bibitem{Gombrich1996}
R. F. Gombrich, \emph{How Buddhism Began: The Conditioned Genesis of the Early Teachings}, Athlone Press (1996).

\bibitem{Graybiel2008}
A. M. Graybiel, \emph{Habits, rituals, and the evaluative brain}, Annual Review of Neuroscience \textbf{31}, 359--387 (2008).

\bibitem{Haag1996}
R. Haag, \emph{Local Quantum Physics: Fields, Particles, Algebras}, Springer-Verlag (1996).

\bibitem{Haagerup1975}
U.~Haagerup,
\emph{The standard form of von Neumann algebras},
Math.\ Scand.\ \textbf{37}, 271--283 (1975).

\bibitem{Haggard2005}
P. Haggard, \emph{Conscious intention and motor cognition}, Trends in Cognitive Sciences \textbf{9}(6), 290--295 (2005).

\bibitem{Hastings2007}
M.~B.~Hastings,
\emph{An area law for one-dimensional quantum systems},
J.\ Stat.\ Mech.\ \textbf{2007}, P08024 (2007),
arXiv:0705.2024.

% Structural Realism

\bibitem{Harvey2000}
P. Harvey, \emph{An Introduction to Buddhist Ethics: Foundations, Values and Issues}, Cambridge University Press (2000).

% Early Buddhist Texts

\bibitem{Hayden2007}
P. Hayden and J. Preskill, \emph{Black holes as mirrors: quantum information in random subsystems}, JHEP \textbf{09}, 120 (2007).

\bibitem{HaydenLeungWinter2006}
P.~Hayden, D.~W.~Leung, and A.~Winter,
\emph{Aspects of generic entanglement},
Commun.\ Math.\ Phys.\ \textbf{265}, 95--117 (2006).

\bibitem{Jacobson1995}
T. Jacobson, \emph{Thermodynamics of Spacetime: The Einstein Equation of State}, Phys. Rev. Lett. \textbf{75}, 1260 (1995).

\bibitem{Jacobson2016}
T.~Jacobson,
\emph{Entanglement equilibrium and the Einstein equation},
Phys.\ Rev.\ Lett.\ \textbf{116}, 201101 (2016),
arXiv:1505.04753.

\bibitem{JLMS2016}
D.~L.~Jafferis, A.~Lewkowycz, J.~Maldacena, and S.~J.~Suh,
\emph{Relative entropy equals bulk relative entropy},
JHEP \textbf{1606}, 004 (2016),
arXiv:1512.06431.

\bibitem{Kiefer2012}
C. Kiefer, \emph{Quantum Gravity}, 3rd ed., Oxford University Press (2012).

\bibitem{Ladyman2007}
J. Ladyman and D. Ross, \emph{Every Thing Must Go: Metaphysics Naturalized}, Oxford University Press (2007).

\bibitem{Lashkari2014}
N. Lashkari, M. B. McDermott, and M. Van Raamsdonk, \emph{Gravitational dynamics from entanglement ``thermodynamics''}, JHEP \textbf{04}, 195 (2014).

\bibitem{LaudisaRovelli2021}
F. Laudisa and C. Rovelli, \emph{Relational Quantum Mechanics}, Stanford Encyclopedia of Philosophy (2021).

\bibitem{Levine1983}
J. Levine, \emph{Materialism and qualia: The explanatory gap}, Pacific Philosophical Quarterly \textbf{64}, 354 (1983).

\bibitem{Liu2026}
S. Liu, \emph{Emergent Geometry from Coarse-Grained Observable Algebras: The Holographic Alaya-Field Framework}, Zenodo (2026), DOI: 10.5281/zenodo.18361706.

\bibitem{Liu2026PaperA}
S. Liu, \emph{Emergent Geometry from Coarse-Grained Observable Algebras: The Holographic Alaya-Field Framework}, Zenodo (2026), DOI: 10.5281/zenodo.18361706.

\bibitem{Liu2026PaperB}
S. Liu, \emph{Accessibility, Stability, and Emergent Geometry: Conceptual Clarifications on the Holographic Alaya-Field Framework}, Zenodo (2026), DOI: 10.5281/zenodo.18367060.

% Causation and Philosophy of Science

\bibitem{Liu2026PaperC}
S. Liu, \emph{Causation, Agency, and Existence: Structural Constraints and Interpretive Bridges}, Zenodo (2026), DOI: 10.5281/zenodo.18374805.

\bibitem{Liu2026PaperD}
S. Liu, \emph{Gravitational Phenomena as Emergent Properties of Observable Algebra Selection: A Structural Analysis}, Zenodo (2026), DOI: 10.5281/zenodo.18388881.

\bibitem{Liu2026PaperE}
S. Liu, \emph{Measurement as Accessibility: A Structural Analysis of Observable Algebra Selection}, Zenodo (2026), DOI: 10.5281/zenodo.18400065.

\bibitem{Liu2026PaperF}
S. Liu, \emph{Temporal Asymmetry as Accessibility Propagation: A Structural Analysis of Causal Direction}, Zenodo (2026), DOI: 10.5281/zenodo.18400425.

\bibitem{Lusthaus2002}
D. Lusthaus, \emph{Buddhist Phenomenology: A Philosophical Investigation of Yog\=ac\=ara Buddhism and the Ch'eng Wei-shih lun}, Routledge (2002).

\bibitem{Maldacena1999}
J. M. Maldacena, \emph{The Large N limit of superconformal field theories and supergravity}, Int. J. Theor. Phys. \textbf{38}, 1113 (1999).

\bibitem{Nagarjuna_MMK_Garfield1995}
N\={a}g\={a}rjuna, \emph{The Fundamental Wisdom of the Middle Way: N\={a}g\={a}rjuna's M\={u}lamadhyamakak\={a}rik\={a}}, trans.\ J. L. Garfield, Oxford University Press (1995).

% Buddhist Secondary Sources

\bibitem{Page1993}
D. N. Page, \emph{Average entropy of a subsystem}, Phys. Rev. Lett. \textbf{71}, 1291 (1993).

\bibitem{PageWootters1983}
D. N. Page and W. K. Wootters, \emph{Evolution without evolution: Dynamics described by stationary observables}, Phys. Rev. D \textbf{27}, 2885 (1983).

\bibitem{Pearl2009}
J. Pearl, \emph{Causality: Models, Reasoning, and Inference}, 2nd ed., Cambridge University Press (2009).

\bibitem{Penrose1989}
R. Penrose, \emph{The Emperor's New Mind}, Oxford University Press (1989).

\bibitem{Petz1996}
D. Petz, \emph{Monotone metrics on matrix spaces}, Linear Algebra Appl. \textbf{244}, 81 (1996).

\bibitem{Price2012}
H. Price, \emph{Does Time-Symmetry Imply Retrocausality?}, Found. Phys. \textbf{42}, 724 (2012).

\bibitem{Rovelli1996}
C. Rovelli, \emph{Relational Quantum Mechanics}, Int. J. Theor. Phys. \textbf{35}, 1637 (1996).

\bibitem{Rovelli2004}
C. Rovelli, \emph{Quantum Gravity}, Cambridge University Press (2004).

\bibitem{RyuTakayanagi2006}
S. Ryu and T. Takayanagi, \emph{Holographic Derivation of Entanglement Entropy from AdS/CFT}, Phys. Rev. Lett. \textbf{96}, 181602 (2006).

\bibitem{Siderits2007}
M. Siderits, \emph{Buddhism as Philosophy: An Introduction}, Hackett Publishing (2007).

% Karma Studies

\bibitem{Smolin2006}
L. Smolin, \emph{The case for background independence}, in \emph{The Structural Foundations of Quantum Gravity}, eds. D. Rickles, S. French, J. Saatsi, Oxford University Press (2006).

\bibitem{SuttaNipata}
\emph{Sutta Nip\={a}ta}, in \emph{The Group of Discourses (Sutta Nip\={a}ta)}, trans.\ K. R. Norman, Pali Text Society (2001).

\bibitem{Swingle2012}
B. Swingle, \emph{Entanglement Renormalization and Holography}, Phys. Rev. D \textbf{86}, 065007 (2012).

\bibitem{Takesaki2003}
M.~Takesaki,
\emph{Theory of Operator Algebras II},
Encyclopaedia of Mathematical Sciences \textbf{125}, Springer (2003).

\bibitem{Thiemann2007}
T. Thiemann, \emph{Modern Canonical Quantum General Relativity}, Cambridge University Press (2007).

\bibitem{VanRaamsdonk2010}
M. Van Raamsdonk, \emph{Building up spacetime with quantum entanglement}, Gen. Relativ. Gravit. \textbf{42}, 2323 (2010).

\bibitem{Vasubandhu_Trimsika}
Vasubandhu, \emph{Tri\d{m}\'sik\={a}-vij\~{n}aptim\={a}trat\={a} (Thirty Verses on Consciousness Only)}, trans.\ S. Anacker, in \emph{Seven Works of Vasubandhu}, Motilal Banarsidass (1984).

% Buddhist Primary Texts (Madhyamaka)

\bibitem{Vidal2008}
G. Vidal, \emph{Class of quantum many-body states that can be efficiently simulated}, Phys. Rev. Lett. \textbf{101}, 110501 (2008).

\bibitem{Viola2004}
H. Barnum, E. Knill, G. Ortiz, R. Somma, and L. Viola, \emph{A Subsystem-Independent Generalization of Entanglement}, Phys. Rev. Lett. \textbf{92}, 107902 (2004).

\bibitem{Waldron2003}
W. S. Waldron, \emph{The Buddhist Unconscious: The \={A}laya-vij\~{n}\={a}na in the Context of Indian Buddhist Thought}, Routledge (2003).

\bibitem{Wallace2012}
D. Wallace, \emph{The Emergent Multiverse}, Oxford University Press (2012).

\bibitem{Wallace2012Time}
D. Wallace, \emph{The Emergent Multiverse: Quantum Theory according to the Everett Interpretation}, Oxford University Press (2012), Chapter 8.

\bibitem{Wiesbrock1993}
H.-W.~Wiesbrock,
\emph{Half-sided modular inclusions of von Neumann algebras},
Commun.\ Math.\ Phys.\ \textbf{157}, 83 (1993).

\bibitem{Wolf2006}
M.~M.~Wolf,
\emph{Violation of the entropic area law for fermions},
Phys.\ Rev.\ Lett.\ \textbf{96}, 010404 (2006).

\bibitem{Woodward2003}
J. Woodward, \emph{Making Things Happen: A Theory of Causal Explanation}, Oxford University Press (2003).

\bibitem{Yin2006}
H. H. Yin and B. J. Knowlton, \emph{The role of the basal ganglia in habit formation}, Nature Reviews Neuroscience \textbf{7}, 464--476 (2006).

\bibitem{Yngvason2005}
J.~Yngvason,
\emph{The role of type III factors in quantum field theory},
Rep.\ Math.\ Phys.\ \textbf{55}, 135 (2005),
arXiv:math-ph/0411058.

\bibitem{Zanardi2001}
P. Zanardi, \emph{Virtual Quantum Subsystems}, Phys. Rev. Lett. \textbf{87}, 077901 (2001).

\bibitem{Zanardi2004}
P. Zanardi, D. A. Lidar, and S. Lloyd, \emph{Quantum Tensor Product Structures are Observable Induced}, Phys. Rev. Lett. \textbf{92}, 060402 (2004).

\bibitem{Zurek2003}
W. H. Zurek, \emph{Decoherence, einselection, and the quantum origins of the classical}, Rev. Mod. Phys. \textbf{75}, 715 (2003).

\bibitem{Zurek2009}
W. H. Zurek, \emph{Quantum Darwinism}, Nature Physics \textbf{5}, 181 (2009).


\end{thebibliography}

\end{document}
