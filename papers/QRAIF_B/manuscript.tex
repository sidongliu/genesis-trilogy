\documentclass[aps,pre,onecolumn,superscriptaddress,floatfix,longbibliography]{revtex4-2}
\usepackage{amsmath,amssymb,amsthm,bm,color}
\usepackage{hyperref}
\usepackage{booktabs}
\usepackage{array}
\newcolumntype{P}[1]{>{\raggedright\arraybackslash}p{#1}}
\usepackage{tikz}
\usetikzlibrary{shapes,arrows,positioning,calc,decorations.pathmorphing}

\newtheorem{theorem}{Theorem}
\newtheorem{lemma}[theorem]{Lemma}
\newtheorem{proposition}[theorem]{Proposition}
\newtheorem{definition}[theorem]{Definition}
\newtheorem{remark}[theorem]{Remark}
\newtheorem{corollary}[theorem]{Corollary}

\begin{document}
\emergencystretch=2em
\raggedbottom
\hbadness=5000
\vbadness=5000

\title{Thermodynamic Stability Constraints on the Operator Algebra\\
of Persistent Open Quantum Subsystems}

\author{Sidong Liu, PhD \\
\small iBioStratix Ltd \\
\small \texttt{sidongliu@hotmail.com}}

\date{February 2026}

\begin{abstract}
We investigate the necessary conditions for an open quantum subsystem to maintain a non-equilibrium steady state (NESS) with bounded internal entropy under continuous environmental interaction.
We formulate the persistence requirement as a variational constraint and analyze its Lyapunov stability.

We identify two structural constraints on the control algebra: (i)~associativity (a structural boundary condition automatically satisfied in standard quantum mechanics but delineating the regime of consistent dynamics), and (ii)~a non-degenerate bilinear form of indefinite signature (required to distinguish qualitatively different environmental coupling channels).
We systematically examine alternative algebraic frameworks---von Neumann factors, $C^*$-algebras, Lie algebras---and argue that Clifford algebra $Cl(V,q)$ is the minimal structure satisfying both constraints simultaneously with a built-in quadratic form.

A worked example of a controlled qubit under Lindblad dynamics illustrates how Clifford structure enables stable feedback while alternative algebraic structures do not.
The analysis complements the Holographic Alaya-Field Framework (HAFF)~\cite{Liu2026PaperA,Liu2026PaperB} and the companion paper on algebraic constraints in entropic gravity~\cite{Liu2026QRAIF_A}.
\end{abstract}

\maketitle

% ============================================================
\section{Introduction}
\label{sec:intro}

The interaction of a quantum system with a large environment typically leads to decoherence and thermalization~\cite{Breuer2002,Weiss2012}.
Maintaining a NESS requires continuous energetic cost~\cite{Seifert2012,Jarzynski2011}, which can be modeled as a feedback control process~\cite{Sagawa2012,Parrondo2015}.
We address: \textit{What algebraic structures allow the internal control dynamics to remain Lyapunov stable?}

\subsection{Three-Paper Structure}

\begin{center}
\begin{tabular}{@{}lll@{}}
\toprule
\textbf{Paper} & \textbf{Question} & \textbf{Analogy} \\
\midrule
HAFF~\cite{Liu2026PaperA} & How does geometry emerge? & Ocean \\
Q-RAIF A~\cite{Liu2026QRAIF_A} & What algebra does geometry need? & Water \\
This work & What algebra does survival need? & Fish \\
\bottomrule
\end{tabular}
\end{center}

\subsection{Anti-Solipsism Disclaimer}

A potential misreading is that the observer ``creates'' geometry through survival.
We explicitly reject this.
The claim is structural: any subsystem maintaining persistence must encode its environment using a Clifford-compatible algebra.
Within HAFF, geometry exists as a stable organizational phase~\cite{Liu2026PaperB}---contingent on physical conditions but objective within them.
The present paper argues that subsystems embedded in such a phase must reflect that geometry in their internal algebra---not generate it.

\subsection{Scope}

This work does not claim to derive Clifford algebra from first principles.
It argues that, within the variational framework of persistence under Lindblad dynamics, Clifford algebra is the minimal algebraic structure compatible with stable feedback.
The argument proceeds by exclusion of alternatives, not by uniqueness proof.

% ============================================================
\section{Variational Bounds on Persistence}
\label{sec:variational}

Consider $\mathcal{H}_{\mathrm{tot}} = \mathcal{H}_R \otimes \mathcal{H}_E$, with reduced dynamics:
\begin{equation}
\dot{\rho}_R = -i[H_{\mathrm{eff}}, \rho_R] + \mathcal{D}[\rho_R].
\end{equation}

\begin{definition}[Persistence Action]
$\mathcal{A}[Q] = \int_0^\tau dt\, D_{KL}(\rho(t) \| \rho_{\mathrm{NESS}})$.
\end{definition}

Minimizing $\delta\mathcal{A} = 0$ implies a control Hamiltonian $H_{\mathrm{ctrl}}(t)$ generated by an operator algebra $\mathcal{O}$.

\subsection{Why Lie Algebras Are Insufficient}

Standard quantum control theory uses Lie algebra generators~\cite{WisemanMilburn2009}: the control Hamiltonian $H_{\mathrm{ctrl}} = \sum_k u_k(t) G_k$ where $\{G_k\}$ generate a Lie algebra $\mathfrak{g}$ via commutators $[G_i, G_j] = i f_{ijk} G_k$.

Lie algebras encode \emph{infinitesimal symmetries}---they specify \emph{which directions} in state space are accessible via control.
However, they do not encode \emph{distances} between states.
The commutator $[G_i, G_j]$ determines the algebra's structure, but there is no built-in notion of ``how far'' a correction moves the state.

For error correction, the subsystem must quantify both the \emph{direction} and the \emph{magnitude} of environmental perturbations.
This requires a quadratic form $q(v) = \eta_{\mu\nu} v^\mu v^\nu$ on the space of perturbations---which is precisely the additional structure that Clifford algebras provide over Lie algebras.

% ============================================================
\section{Algebraic Constraints on Control Stability}
\label{sec:stability}

\subsection{Constraint I: Associativity as Structural Boundary}
\label{sec:assoc}

\begin{lemma}[Associativity Boundary]
\label{lem:assoc}
Consistent composition of sequential control operations requires an associative algebra.
\end{lemma}

We acknowledge that this constraint is automatically satisfied by operator algebras on Hilbert spaces, where composition of linear maps is inherently associative~\cite{Breuer2002}.
Non-associative algebras (Jordan, octonion) are not realistic candidates for quantum dynamics.

Lemma~\ref{lem:assoc} therefore functions as a \emph{structural boundary marker}: it delineates the minimal algebraic condition separating consistent from inconsistent dynamics, analogous to how the second law delineates irreversibility without claiming that reversible processes are a realistic threat.
For the mathematical structure of non-associative algebras and their dynamical instabilities, see~\cite{Schafer1966,Gunaydin1973}.
The physical pathologies of non-associative alternatives are discussed in Paper~A~\cite{Liu2026QRAIF_A} (Remark~3, ``Physical pathologies of non-associative dynamics'').

The substantive constraint is Constraint~II, which discriminates among \emph{associative} algebras.

\subsection{Constraint II: Indefinite Metric for Channel Discrimination}
\label{sec:metric}

\begin{lemma}[Metric Constraint]
\label{lem:metric}
For a persistent subsystem to distinguish qualitatively different environmental coupling channels and implement directed error correction, the control algebra must carry a non-degenerate bilinear form of indefinite signature.
\end{lemma}

\begin{proof}
Lyapunov stability requires $\dot{V} < 0$ for $V(\delta\rho) \geq 0$, implying gradient flow:
\begin{equation}
\dot{\lambda} = -\Gamma\, G^{-1} \nabla_\lambda V,
\label{eq:gradient}
\end{equation}
where $G$ is a metric on the control parameter manifold.

\textbf{Important distinction}: $G$ here is an information-geometric metric on the space of control parameters, not the spacetime metric.
Standard quantum state metrics (Bures, Fisher--Rao~\cite{Petz1996}) are positive-definite and satisfy non-degeneracy.
However, they are \emph{isotropic}: they treat all perturbation directions equivalently.

In realistic open quantum systems, the environment couples to the subsystem through qualitatively different channels---dissipative (population decay), dephasing (coherence loss), and unitary (Hamiltonian shift).
Effective error correction requires distinguishing \emph{qualitatively} between these channel types.

A positive-definite anisotropic metric (e.g., $\mathrm{diag}(a, b, c)$ with $a \neq b \neq c > 0$) can encode quantitative differences between directions, but cannot encode a \emph{qualitative} dichotomy: all directions remain ``of the same type.''
The distinction between thermodynamically reversible perturbations (unitary, entropy-preserving) and irreversible perturbations (dissipative, entropy-producing) is not a matter of degree but of kind.
An indefinite quadratic form $q(v) = \eta_{\mu\nu} v^\mu v^\nu$ with $\mathrm{sig}(\eta) = (p,r)$, $p,r \geq 1$, encodes this: positive-norm directions correspond to one class, negative-norm directions to another.
This is precisely the structure built into Clifford algebras via $v^2 = q(v)\mathbf{1}$.

Standard quantum state metrics (Bures, Fisher--Rao) are positive-definite and therefore cannot algebraically encode the reversible/irreversible dichotomy without external structure.
\end{proof}

\subsection{Exclusion of Alternative Algebras}
\label{sec:exclusion}

\begin{table}[h]
\centering
\begin{tabular}{@{}lccl@{}}
\toprule
\textbf{Algebra} & \textbf{I: Assoc.} & \textbf{II: Indef.~$q$} & \textbf{Status} \\
\midrule
von Neumann (III$_1$) & \checkmark & $\times$ & No built-in $q$ \\
$C^*$-algebra & \checkmark & $\times$ & Positive-definite only \\
Lie algebra & \checkmark$^*$ & $\times$ & Killing form, no $q$ \\
Jordan algebra & $\times$ & --- & Non-associative \\
\textbf{Clifford} $Cl(V,q)$ & \checkmark & \checkmark & \textbf{Minimal} \\
\bottomrule
\end{tabular}
\caption{Systematic evaluation of candidate control algebras.
$^*$Via universal enveloping algebra.}
\label{tab:exclusion}
\end{table}

The exclusion argument shifts the burden from ``why Clifford?'' to ``why not the alternatives?''---and the answer is that no other standard algebraic framework carries a built-in indefinite quadratic form encoding channel discrimination.

\subsection{Worked Example: Controlled Qubit Under Lindblad Dynamics}
\label{sec:toymodel}

\paragraph{Setup.}
Consider a single qubit coupled to a thermal bath at inverse temperature $\beta$, with Lindblad dissipator:
\begin{equation}
\mathcal{D}[\rho] = \gamma_\downarrow \mathcal{L}[\sigma_-]\rho + \gamma_\uparrow \mathcal{L}[\sigma_+]\rho + \gamma_\phi \mathcal{L}[\sigma_z]\rho,
\end{equation}
where $\mathcal{L}[L]\rho = L\rho L^\dagger - \frac{1}{2}\{L^\dagger L, \rho\}$, and $\gamma_\downarrow$, $\gamma_\uparrow$, $\gamma_\phi$ are decay, excitation, and dephasing rates.

\paragraph{Control algebra.}
The control Hamiltonian is $H_{\mathrm{ctrl}} = \sum_i u_i(t)\, \sigma_i$ where $\{\sigma_x, \sigma_y, \sigma_z\}$ are Pauli operators.
These satisfy $\{\sigma_i, \sigma_j\} = 2\delta_{ij}\mathbf{1}$---the defining relation of $Cl(3,0)$.

\paragraph{Lyapunov function.}
Take $V = D_{KL}(\rho \| \rho_{\mathrm{NESS}})$ where $\rho_{\mathrm{NESS}}$ is the thermal state.
The gradient $\nabla_u V$ is well-defined because the Pauli algebra carries a natural inner product (the quadratic form $q(\sigma_i) = +1$).
The control protocol $u_i(t) = -\alpha\, \partial V / \partial u_i$ yields:
\begin{equation}
\dot{V} = -\alpha \sum_i \left(\frac{\partial V}{\partial u_i}\right)^2 \leq 0,
\end{equation}
which is strictly negative away from the NESS.

\paragraph{Failure mode without built-in metric.}
If the control algebra were an abstract Lie algebra $\mathfrak{su}(2)$ (same generators, but with only the commutator structure $[\sigma_i, \sigma_j] = 2i\epsilon_{ijk}\sigma_k$ and no anticommutator/metric), the control protocol could specify \emph{rotation directions} in Bloch sphere but could not canonically quantify \emph{how large} a correction to apply.
The gradient flow~\eqref{eq:gradient} would require importing an external metric (e.g., the Killing form of $\mathfrak{su}(2)$, which happens to be proportional to $\delta_{ij}$).

In the Pauli/Clifford case, the metric is \emph{internal}: the same algebraic structure that generates rotations also defines distances.
This unification is what makes Clifford algebras uniquely suited for feedback control where both direction and magnitude matter.

\paragraph{Extension to indefinite signature.}
\begin{sloppypar}
When the subsystem must distinguish dissipative from unitary perturbations---e.g., $\gamma_\downarrow \neq 0$ (dissipative) versus Hamiltonian noise (unitary)---the control space naturally splits into positive-norm (unitary) and negative-norm (dissipative) sectors.
Encoding this distinction algebraically requires an indefinite quadratic form, upgrading $Cl(3,0)$ to $Cl(p,q)$ with appropriate signature.
\end{sloppypar}

% ============================================================
\subsection{The Algebraic Compatibility Theorem}

\begin{theorem}[Persistence Compatibility]
\label{thm:main}
Among associative algebras encoding $n$ orthogonal control channels with a built-in non-degenerate quadratic form, the Clifford algebra $Cl(V,q)$ is the universal minimal structure, by its universal property.
\end{theorem}

\begin{proof}
Constraint~I (associativity) is given.
Constraint~II requires a non-degenerate quadratic form $q$ on the generating space $V$, with indefinite signature when channel discrimination is required.
The universal property of Clifford algebras~\cite{Hestenes1966} identifies $Cl(V,q)$ as the unique associative algebra generated by $V$ subject to $v^2 = q(v)\mathbf{1}$.
\end{proof}

\begin{corollary}
Any subsystem maintaining NESS for $\tau \gg \tau_{\mathrm{relax}}$ while discriminating among environmental channels must encode its boundary using a Clifford-compatible algebra.
\end{corollary}

\begin{remark}[Natural Selection, Not Design]
The theorem establishes a selection principle.
Subsystems do not ``choose'' Clifford algebra; only Clifford-compatible structures persist when channel discrimination is required.
This is algebraic natural selection.
\end{remark}

\begin{remark}[Content of the Argument]
\label{rem:scope}
As with the Clifford Compatibility Theorem in Paper~A~\cite{Liu2026QRAIF_A}, the final step of the proof---invoking the universal property of $Cl(V,q)$---is a standard algebraic identity.
The substantive content resides in the constraints:
Lemma~\ref{lem:assoc} (associativity as a boundary condition, with non-associative alternatives producing quantifiable pathologies; see Paper~A~\cite{Liu2026QRAIF_A}, Remark~3),
and Lemma~\ref{lem:metric} (indefinite metric from the requirement of channel discrimination in directed error correction).
The worked example in Section~\ref{sec:toymodel} demonstrates that the constraints are simultaneously satisfiable in a concrete Lindblad model.
\end{remark}

% ============================================================
\section{Contextual Relations}
\label{sec:context}

\subsection{Convergence with Paper A}

The companion paper~\cite{Liu2026QRAIF_A} argues that $Cl(1,3)$ is the minimal algebra compatible with emergent Lorentzian geometry in entropic gravity.

\begin{center}
\begin{tabular}{@{}lcc@{}}
\toprule
& \textbf{Paper A} & \textbf{This work} \\
\midrule
Starting point & Holographic boundary & Open subsystem \\
Method & Signature selection & Lyapunov stability \\
Key constraint & Causal ordering & Channel discrimination \\
Result & $Cl(1,3)$ & $Cl(V,q)$ \\
\bottomrule
\end{tabular}
\end{center}

We note explicitly that this convergence is \emph{heuristic rather than deductive}.
Two arguments pointing to the same algebraic structure from different directions is suggestive but does not constitute proof.
The convergence motivates further investigation through explicit models and connections to established frameworks, not a claim of mathematical necessity.

\subsection{Relation to Quantum Control Theory}

Standard quantum control operates within a Lie algebraic framework~\cite{WisemanMilburn2009}: controllability is characterized by the Lie algebra generated by the drift and control Hamiltonians.
This framework is complete for determining \emph{reachability} of target states.

However, Lie algebras encode symmetries (via commutators) without encoding distances (via quadratic forms).
When the control objective is not merely reachability but \emph{stabilization against stochastic perturbations}---as in NESS maintenance---both direction and magnitude of corrections must be specified.
Clifford algebras provide this additional structure through their built-in quadratic form, complementing rather than replacing the Lie algebraic framework.

\subsection{Relation to Decoherence-Free Subspaces}

Decoherence-free subspaces (DFS)~\cite{Zanardi2001,Lidar2003} represent subsystems that are passively protected from environmental noise by symmetry.
The present analysis addresses the \emph{active} counterpart: subsystems that maintain coherence through continuous feedback.
In both cases, the algebraic structure of the system-environment interaction determines which subsystems can persist.
The Clifford constraint identified here applies to the active case; DFS theory applies to the passive case.
A unified treatment remains an open problem.

% ============================================================
\section{Discussion}

\textbf{What this result does show:}
Clifford algebra is the minimal algebraic structure satisfying both associativity and built-in indefinite metric among standard algebraic candidates.
The exclusion of alternatives (Table~\ref{tab:exclusion}) and the controlled qubit example (Section~\ref{sec:toymodel}) provide concrete support.

\textbf{What this result does not show:}
That Clifford algebra is the \emph{unique} solution---larger structures are also compatible.
That non-Clifford feedback is impossible in all settings---it is possible when channel discrimination is not required.
A derivation from first principles independent of the variational framework assumed here.

\section{Conclusion}

We have argued that geometric (Clifford) algebra structure is a natural minimal requirement for persistent subsystems that must discriminate among environmental coupling channels:
(i)~associativity is a structural boundary condition;
(ii)~a built-in indefinite quadratic form is required for channel discrimination and directed error correction;
(iii)~no standard alternative algebra satisfies both with built-in structure.

Combined with the companion paper's holographic constraints, this suggests $Cl(V,q)$ occupies a distinguished position as the algebraic structure simultaneously compatible with geometric consistency and thermodynamic persistence.

% ============================================================
\begin{thebibliography}{99}

\bibitem{Breuer2002}
H.-P.~Breuer and F.~Petruccione, \emph{The Theory of Open Quantum Systems}, Oxford University Press (2002).

\bibitem{Weiss2012}
U.~Weiss, \emph{Quantum Dissipative Systems}, 4th ed., World Scientific (2012).

\bibitem{Seifert2012}
U.~Seifert, \emph{Stochastic thermodynamics, fluctuation theorems and molecular machines}, Rep.\ Prog.\ Phys.\ \textbf{75}, 126001 (2012).

\bibitem{Jarzynski2011}
C.~Jarzynski, \emph{Equalities and inequalities: irreversibility and the second law at the nanoscale}, Annu.\ Rev.\ Condens.\ Matter Phys.\ \textbf{2}, 329 (2011).

\bibitem{Sagawa2012}
T.~Sagawa and M.~Ueda, \emph{Fluctuation theorem with information exchange}, Phys.\ Rev.\ E \textbf{85}, 021104 (2012).

\bibitem{Parrondo2015}
J.~M.~R.~Parrondo, J.~M.~Horowitz, and T.~Sagawa, \emph{Thermodynamics of information}, Nat.\ Phys.\ \textbf{11}, 131 (2015).

\bibitem{Hestenes1966}
D.~Hestenes, \emph{Space-Time Algebra}, Gordon and Breach (1966).

\bibitem{Doran2003}
C.~Doran and A.~Lasenby, \emph{Geometric Algebra for Physicists}, Cambridge University Press (2003).

\bibitem{Petz1996}
D.~Petz, \emph{Monotone metrics on matrix spaces}, Linear Algebra Appl.\ \textbf{244}, 81 (1996).

\bibitem{Schafer1966}
R.~D.~Schafer, \emph{An Introduction to Nonassociative Algebras}, Academic Press (1966).

\bibitem{Gunaydin1973}
M.~G\"unaydin and F.~G\"ursey, \emph{Quark structure and octonions}, J.\ Math.\ Phys.\ \textbf{14}, 1651 (1973).

\bibitem{WisemanMilburn2009}
H.~M.~Wiseman and G.~J.~Milburn, \emph{Quantum Measurement and Control}, Cambridge University Press (2009).

\bibitem{Zanardi2001}
P.~Zanardi, \emph{Virtual Quantum Subsystems}, Phys.\ Rev.\ Lett.\ \textbf{87}, 077901 (2001).

\bibitem{Lidar2003}
D.~A.~Lidar and K.~B.~Whaley, \emph{Decoherence-free subspaces and subsystems}, in \emph{Irreversible Quantum Dynamics}, Springer, pp.\ 83--120 (2003).

\bibitem{Liu2026PaperA}
S.~Liu, \emph{Emergent Geometry from Coarse-Grained Observable Algebras: The Holographic Alaya-Field Framework}, Zenodo (2026), DOI: 10.5281/zenodo.18361706.

\bibitem{Liu2026PaperB}
S.~Liu, \emph{Accessibility, Stability, and Emergent Geometry: Conceptual Clarifications on the Holographic Alaya-Field Framework}, Zenodo (2026), DOI: 10.5281/zenodo.18367060.

\bibitem{Liu2026QRAIF_A}
S.~Liu, \emph{Algebraic Constraints on the Emergence of Lorentzian Metrics in Entropic Gravity Frameworks}, Zenodo (2026), DOI: 10.5281/zenodo.18525876.

\end{thebibliography}

\end{document}
