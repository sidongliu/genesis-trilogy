% ============================================================
% T-DOME Paper II: The Ego
% T-DOME three-paper series, Paper II
% ============================================================

\documentclass[12pt]{article}
\usepackage[margin=1in]{geometry}
\usepackage{amsmath,amssymb,amsthm}
\usepackage[hidelinks]{hyperref}
\usepackage{booktabs}
\usepackage{physics}
\usepackage{graphicx}
\usepackage{caption}
\usepackage{float}

% Theorem environments — matching HAFF/Q-RAIF/T-DOME I style
\newtheorem{theorem}{Theorem}
\newtheorem{lemma}[theorem]{Lemma}
\newtheorem{proposition}[theorem]{Proposition}
\newtheorem{definition}[theorem]{Definition}
\newtheorem{remark}[theorem]{Remark}
\newtheorem{corollary}[theorem]{Corollary}

\title{Spontaneous Symmetry Breaking of Reference Frames\\
as a Computational Cost Minimization Strategy}

\author{Sidong Liu, PhD \\
\small iBioStratix Ltd \\
\small \texttt{sidongliu@hotmail.com}}

\date{February 2026}

\begin{document}
\maketitle

% ============================================================
\begin{abstract}
We investigate the computational constraints on persistent
open quantum systems that carry non-Markovian memory
(Paper~I~\cite{Liu2026TDOME_I}).
Paper~I established that memory is a thermodynamic necessity
for survival beyond the Markovian ceiling, but revealed a
secondary crisis: the \emph{Memory Catastrophe}, in which
the Landauer cost of maintaining unbounded history exceeds
any finite free-energy budget.

We prove a \textbf{Computational Ceiling}: any agent that
processes its memory \emph{symmetrically}---treating all
components of its internal Clifford algebra $Cl(V,q)$ as
equally relevant---reaches computational paralysis at a
finite critical time $t_{\mathrm{par}}$.

We then show that the resolution requires
\textbf{spontaneous symmetry breaking} of the agent's
internal reference frame: the selection of a privileged basis
(a gauge fixing of the automorphism group
$G = \mathrm{Aut}(Cl(V,q))$) that compresses the memory
kernel into a tractable, low-dimensional representation.
The optimal compression is governed by a survival-weighted
rate-distortion bound; under generic conditions, the agent
retains $k^{*} = \mathcal{C}_{\mathrm{budget}}/h_\mu$
components and discards the rest.

This establishes \textbf{reference-frame selection as the
survival-optimal strategy under bounded rationality}:
the ``self'' (a privileged computational basis) is not an
additional hypothesis but the minimal structure that makes
memory computationally tractable.

The broken phase introduces four systematic bias terms---basis
selection, frame drag, objective centering, and model
incompleteness---that are generic consequences
of gauge fixing under assumptions~(B1)--(B5).
We show that under environmental drift, a fixed reference
frame leads to the \textbf{Delusion Trap}: an exponential
divergence of prediction error that the agent cannot detect
from within its own frame, establishing the crisis that
Paper~III must resolve.
\end{abstract}

% ============================================================
\section{Introduction}
\label{sec:intro}

% ------------------------------------------------------------
\subsection{Context: The Problem of Overload}
\label{subsec:overload}

Paper~I of this series~\cite{Liu2026TDOME_I} established that
non-Markovian memory is a thermodynamic necessity for
persistent far-from-equilibrium systems: under open-loop
Markovian (GKSL) dynamics, the survival functional satisfies
$\mathcal{S} \leq 0$ (the Markovian Ceiling), while agents
carrying memory kernels can achieve $\mathcal{S} > 0$ by
consuming stored system--environment correlations.

This result, however, carries a price.
The \emph{Memory Catastrophe} (Paper~I, Proposition~10)
shows that the Landauer cost of maintaining a memory archive
of depth $\tau_{\mathrm{mem}}$ grows at a rate
\begin{equation}
\label{eq:mem_cost_recall}
\dot{W}_{\mathrm{mem}}
\geq k_B T \ln 2 \cdot h_\mu,
\end{equation}
where $h_\mu$ is the entropy rate of the environmental
process~\cite{CrutchfieldYoung1989,ShaliziCrutchfield2001}.
For any finite free-energy budget $\dot{W}_{\mathrm{budget}}$,
there exists a critical time $t_{\mathrm{crit}}$ beyond which
$\dot{W}_{\mathrm{mem}} > \dot{W}_{\mathrm{budget}}$:
the agent's memory consumes more resources than are available.

But thermodynamic cost is only half the crisis.
Even if unlimited free energy were available for memory
maintenance, the agent must still \emph{process} the stored
correlations---evaluate the survival functional as a function
of its ever-growing archive---using finite computational
resources. This is the problem that the present paper
addresses.

% ------------------------------------------------------------
\subsection{Position within the Series}
\label{subsec:series_position}

This paper is the second of three constituting the
\textbf{T-DOME} (Thermodynamic Dynamics of Observer-Memory
Entanglement) framework, the third pillar of a three-paper
program.

\begin{center}
\small
\setlength{\tabcolsep}{4pt}%
\begin{tabular}{@{}lp{3.2cm}cp{3.4cm}c@{}}
\toprule
\textbf{Framework} & \textbf{Question} & & \textbf{Result}
  & \textbf{Status} \\
\midrule
HAFF~\cite{Liu2026HAFF_A,Liu2026HAFF_B}
  & How does geometry emerge?
  & Ocean
  & Algebra $\to$ Geometry
  & Complete \\[3pt]
Q-RAIF~\cite{Liu2026QRAIF_A,Liu2026QRAIF_B}
  & What algebra must an observer have?
  & Fish
  & $Cl(V,q) \hookrightarrow Cl(1,3)$
  & Complete \\[3pt]
T-DOME~I~\cite{Liu2026TDOME_I}
  & Why must agents carry memory?
  & Seed
  & Markovian ceiling; memory as necessity
  & Complete \\[3pt]
\textbf{T-DOME~II} (this work)
  & Why must agents break symmetry?
  & Ego
  & Reference-frame selection under bounded computation
  & \textbf{This paper} \\[3pt]
T-DOME~III
  & How does self-calibration arise?
  & Loop
  & Fisher self-referential bound
  & Planned \\
\bottomrule
\end{tabular}
\end{center}

The three T-DOME papers form an irreversible logical chain.
Each resolves a survival crisis created by its predecessor:
\begin{enumerate}
\item \textbf{Paper~I (The Seed):} Without memory, a system
  is trapped in the \emph{Markovian present}---no accumulation,
  no temporal arrow, inevitable thermal death.
  Memory breaks this trap but floods the system with unbounded
  historical data.
\item \textbf{Paper~II (The Ego, this work):} Unbounded memory
  under finite computational resources causes processing
  collapse. Spontaneous symmetry breaking of the reference
  frame (establishing a ``self'') resolves the overload but
  introduces systematic bias.
\item \textbf{Paper~III (The Loop):} Uncorrected bias diverges
  from a changing environment. A self-referential calibration
  loop (monitoring one's own prediction error) resolves the
  bias but requires the system to ``observe its own
  observation''---closing the self-calibration loop.
\end{enumerate}

% ------------------------------------------------------------
\subsection{Relation to Q-RAIF}
\label{subsec:qraif}

Q-RAIF Paper~B~\cite{Liu2026QRAIF_B} established that any
persistent open quantum subsystem maintaining a
non-equilibrium steady state (NESS) requires an internal
control algebra isomorphic to a Clifford algebra $Cl(V,q)$.
Paper~C~\cite{Liu2026QRAIF_C} showed that this algebra must
embed in the environmental observable algebra via a
realizability homomorphism
$\phi: Cl(V,q) \hookrightarrow Cl(1,3)$.

The Clifford algebra $Cl(V,q)$, however, admits a non-trivial
\emph{automorphism group} $G = \mathrm{Aut}(Cl(V,q))$.
In the absence of external constraints, all elements of $G$
yield physically equivalent representations---the choice of
basis within the algebra is a \emph{gauge freedom}.
This gauge freedom is the mathematical substrate of the
symmetry that the present paper breaks.

The ``ego'' is not a new algebraic structure imposed from
outside the Q-RAIF framework; it is a \emph{gauge fixing}
of the already-present internal symmetry, driven by
computational optimality under bounded resources.

% ------------------------------------------------------------
\subsection{Relation to HAFF Paper~G}
\label{subsec:haff_g}

HAFF Paper~G established \emph{architectural
incompleteness}: the observable-algebra framework cannot
self-ground~\cite{Liu2026HAFF_G}.
The present paper provides a partial operational resolution:
under bounded computation, an agent satisfying~(B1)--(B5)
is driven to choose a computational basis (break symmetry)
precisely \emph{because} the framework is incomplete.
The ego is an operational response to incompleteness, not a
metaphysical addition.

% ------------------------------------------------------------
\subsection{Scope and Disclaimers}
\label{subsec:scope}

To prevent interpretational overreach, we state at the
outset what this paper does \emph{not} claim:
\begin{enumerate}
\item We do not claim that symmetry breaking is
  \emph{sufficient} for persistence.
  Paper~III addresses the additional requirements.
\item We do not claim that the specific form of the
  privileged basis is unique---only that \emph{some} basis
  selection is necessary under bounded computation.
\item The term ``ego'' or ``self'' is used in the
  control-theoretic sense: a fixed reference frame within
  the agent's internal algebra. It carries no implication
  of consciousness or subjective experience.
\item A broader structural analogy with classical philosophical
  concepts of selfhood exists but is outside the scope of
  this paper.
\end{enumerate}

\paragraph{Related work.}
The idea that bounded agents must compress their
representations has roots in Simon's bounded
rationality~\cite{Simon1955}, Shannon's
rate-distortion theory~\cite{Shannon1959,CoverThomas2006},
and Sims's rational inattention~\cite{Sims2003}, which
models finite-capacity decision-makers as solving a
rate-distortion problem---precisely the economic
counterpart of our $\mathcal{C}_{\mathrm{budget}}$
formalism.
The information bottleneck~\cite{Tishby2000}
formalises relevance-weighted compression and has been
applied to neural coding and deep
learning.
The role of decoherence in selecting preferred bases
(pointer states) is well established via quantum
Darwinism~\cite{Zurek2009}; our contribution is to
show that the same selection arises as a \emph{computational}
necessity, independent of the decoherence mechanism.
Measures of non-Markovianity and their thermodynamic
consequences are reviewed
in~\cite{RivasHuelgaPlenio2014,BreuerPetruccione2002};
the connection to survival was established in Paper~I.

\paragraph{Summary of contributions.}
This paper establishes three main results:
\begin{enumerate}
\item \textbf{Computational Ceiling scaling law}
  (Theorem~\ref{thm:comp_ceiling}): symmetric processing
  of a $Cl(V,q)$ memory kernel requires rate
  $\mathcal{R} \geq h_\mu \cdot D$, leading to paralysis
  at a finite $\tau_{\mathrm{par}}$.
\item \textbf{Survival-weighted rate-distortion bound}
  (Theorem~\ref{thm:compression}): the optimal gauge-fixed
  representation retains
  $k^* = \lfloor\mathcal{C}_{\mathrm{budget}}/h_\mu\rfloor$
  components.
\item \textbf{Delusion dynamics}
  (Theorem~\ref{thm:delusion}): a fixed reference frame
  decouples from a drifting environment on the logarithmic
  timescale $t_{\mathrm{del}}
  = \Lambda^{-1}\ln(\pi/4\theta_0)$.
\end{enumerate}

% ============================================================
\section{Mathematical Preliminaries}
\label{sec:prelim}

% ------------------------------------------------------------
\subsection{Inherited Framework from Paper~I}
\label{subsec:inherited}

We briefly recall the key objects from
Paper~I~\cite{Liu2026TDOME_I} that the present work builds
upon.  The reader is referred to Paper~I for full definitions
and proofs.

\paragraph{Survival functional.}
For an open quantum system $S$ coupled to an environment $E$
at inverse temperature $\beta$, with dynamics $\Lambda$ and
external control protocol $H_{\mathrm{ctrl}}(t)$, the
survival functional is
\begin{equation}
\label{eq:survival_recall}
\mathcal{S}[\Lambda, \tau]
:= \Delta F - W[0,\tau],
\end{equation}
where $\Delta F = F(\rho(\tau)) - F(\rho(0))$ is the change
in non-equilibrium free energy and
$W = \int_0^\tau \tr(\rho(t)\,\dot{H}_{\mathrm{ctrl}}(t))\,dt$
is the work performed by the external protocol.

\paragraph{Markovian Ceiling.}
Under open-loop GKSL dynamics with no feedback
(control class $\mathcal{C}_{\mathrm{M}}$, Paper~I,
Definition~6):
\begin{equation}
\label{eq:ceiling_recall}
\mathcal{S}[\Lambda^{\mathrm{M}}, \tau] \leq 0
\qquad\text{for all } \tau \geq 0.
\end{equation}

\paragraph{Non-Markovian advantage identity.}
For arbitrary initial states:
\begin{equation}
\label{eq:advantage_recall}
\beta\,\mathcal{S}
= -\Delta I(S{:}E)
  - \Delta D_{\mathrm{KL}}(\rho_E \| \rho_E^{\mathrm{th}})
  - \beta\,\Delta\langle H_{\mathrm{ctrl}} \rangle.
\end{equation}

\paragraph{Memory Catastrophe.}
The Landauer cost of maintaining a memory archive of depth
$\tau_{\mathrm{mem}}$ satisfies
$\dot{W}_{\mathrm{mem}} \geq k_BT\ln 2 \cdot h_\mu$
(Paper~I, Proposition~10),
where $h_\mu$ is the \emph{per-component} entropy rate of the
environmental process~\cite{CrutchfieldYoung1989}, defined by
\begin{equation}
\label{eq:entropy_rate}
h_\mu
:= \lim_{T\to\infty}\frac{1}{T}\,H(X_{0:T}),
\end{equation}
measuring the asymptotic information (in bits per unit time)
generated by a single algebraic component of the memory
kernel (we work in units where the sampling interval equals
the environmental correlation time
$\tau_E$)\footnote{For a continuous-valued process sampled at
resolution $b$ bits, $h_\mu$ includes the quantisation cost:
$h_\mu = h_\mu^{(\mathrm{diff})} + b\,f_s$, where
$h_\mu^{(\mathrm{diff})}$ is the differential entropy rate
and $f_s$ the sampling frequency.  All budget inequalities in
this paper hold with $h_\mu$ so defined.}---and the stored
mutual
information grows as
$\mathfrak{i}_{\mathrm{stored}}(\tau_{\mathrm{mem}})
\geq \min(I_{\mathrm{pred}},\; h_\mu \,\tau_{\mathrm{mem}})$,
with $I_{\mathrm{pred}}$ the \emph{predictive information}
(excess entropy)~\cite{BialekNemenmanTishby2001,ShaliziCrutchfield2001},
defined as the mutual information between past and future
of the environmental process:
\begin{equation}
\label{eq:I_pred}
I_{\mathrm{pred}}
:= I\!\left(\overleftarrow{X};\, \overrightarrow{X}\right)
= H(\overrightarrow{X})
  - H(\overrightarrow{X} \mid \overleftarrow{X}),
\end{equation}
where $\overleftarrow{X}$ and $\overrightarrow{X}$ denote
the semi-infinite past and future, respectively.
For a stationary process,
$I_{\mathrm{pred}}$ relates to $h_\mu$ via the entropy-rate
decomposition
$H(X_{1:T}) = I_{\mathrm{pred}} + h_\mu\,T + o(1)$
as $T \to \infty$~\cite{CrutchfieldYoung1989}.

% ------------------------------------------------------------
\subsection{The Agent's Internal Algebra}
\label{subsec:algebra}

Following Q-RAIF~\cite{Liu2026QRAIF_B,Liu2026QRAIF_C},
the agent's internal control algebra is a Clifford algebra
$\mathcal{O}_{\mathrm{int}} = Cl(V,q)$
for a real vector space $V$ equipped with a non-degenerate
quadratic form $q$.
The algebra satisfies the fundamental relation
$v^2 = q(v)\,\mathbf{1}$ for all $v \in V$.

The \emph{automorphism group}
\begin{equation}
\label{eq:gauge_group}
G := \mathrm{Aut}(Cl(V,q))
\end{equation}
is the group of algebra automorphisms that preserve the
grading and quadratic form.\footnote{\label{fn:gauge_group}%
We use $G$ as an effective symmetry group acting
transitively on admissible frames.
The detailed Lie-algebraic structure of $G$ is not required
for our results; only the existence of a non-trivial
symmetry that must be broken (assumption~(B5)).
In concrete models, one may replace $G$ by its image
under the adjoint representation---typically
$O(V,q)$ or a pin/spin subgroup.}
For $Cl(1,3)$, $G$ contains the spin group
$\mathrm{Spin}(1,3) \cong SL(2,\mathbb{C})$
as a subgroup---a six-real-dimensional Lie group.

In the absence of computational constraints, all
$g \in G$ yield physically equivalent descriptions of
the agent's internal state.
The choice of basis within $Cl(V,q)$ is a \emph{gauge
freedom}---the symmetry that will be broken.

The realizability embedding
$\phi: Cl(V,q) \hookrightarrow Cl(1,3)$
(Q-RAIF Paper~C) constrains the physically accessible
reference frames: only gauge choices compatible with
$\mathrm{Im}(\phi) \subset Cl(1,3)$ are realizable.

\paragraph{Dimensional convention.}
Two distinct notions of dimension appear throughout:
\begin{center}
\small
\begin{tabular}{@{}lll@{}}
\toprule
Symbol & Meaning & Scaling \\
\midrule
$n := \dim V$ & number of generators
  (degrees of freedom) & --- \\
$D := \dim Cl(V,q) = 2^n$ & full multivector space
  (algebra basis size) & exponential in $n$ \\
\bottomrule
\end{tabular}
\end{center}
The Computational Ceiling (Section~\ref{sec:ceiling})
scales with~$D$, not~$n$; the distinction matters
whenever one compares generator-level and
algebra-level quantities.

% ------------------------------------------------------------
\subsection{Rate-Distortion Theory}
\label{subsec:RD}

We require the classical rate-distortion framework
of Shannon~\cite{Shannon1959}.

\begin{definition}[Rate-distortion function]
\label{def:RD}
Let $X$ be a random source with distribution $p(x)$,
$\hat{X}$ a reconstruction, and
$d: \mathcal{X} \times \hat{\mathcal{X}} \to [0,\infty)$
a distortion measure.
The \emph{rate-distortion function} is
\begin{equation}
\label{eq:RD}
R(D) := \min_{\substack{p(\hat{x}|x):\\
  \mathbb{E}[d(X,\hat{X})] \leq D}} I(X; \hat{X}),
\end{equation}
the minimum mutual information between source and
reconstruction that achieves average distortion at most $D$.
\end{definition}

$R(D)$ is a convex, non-increasing function of $D$ with
$R(0) = H(X)$ (lossless) and
$R(D_{\max}) = 0$ (maximum distortion).
It provides the fundamental limit on lossy
compression~\cite{CoverThomas2006}.
The \emph{information bottleneck} method of
Tishby et al.~\cite{Tishby2000} generalises this framework
to the case where the relevant variable is not the source
itself but a downstream prediction target---precisely the
situation in our survival-weighted compression problem
(Section~\ref{subsec:RD_bound}).

% ------------------------------------------------------------
\subsection{Bounded Rationality}
\label{subsec:bounded}

Following Simon~\cite{Simon1955}, we model computational
limitations as a hard constraint on the agent's information
processing rate.

\begin{definition}[Computational budget]
\label{def:budget}
The agent's \emph{computational budget}
$\mathcal{C}_{\mathrm{budget}}$ (measured in bits per unit
time) is the maximum rate at which the agent can evaluate
functions of its stored correlations.
We assume $\mathcal{C}_{\mathrm{budget}} < \infty$.
\end{definition}

Physically, finiteness of $\mathcal{C}_{\mathrm{budget}}$
reflects the finite number of degrees of freedom in the
agent's physical substrate: finite Hilbert space dimension,
finite memory register size, and finite energy available for
computation (Landauer's
principle~\cite{Landauer1961,Bennett1982}).

% ------------------------------------------------------------
\subsection{Fiber Bundle Formalism}
\label{subsec:fiber}

The geometric setting for reference-frame selection is a
principal fiber bundle.

\begin{definition}[Gauge bundle]
\label{def:bundle}
The \emph{gauge bundle} is the principal $G$-bundle
\begin{equation}
\label{eq:bundle}
\pi: P \to M, \qquad G = \mathrm{Aut}(Cl(V,q)),
\end{equation}
where:
\begin{itemize}
\item $M$ is the base space of \emph{effective memory
  kernels}---equivalently, the space of induced
  sufficient-statistic processes accessible to the agent
  (a finite-dimensional manifold that admits local
  parametrisation by the environmental spectral-density
  couplings);
\item $G$ is the structure group acting transitively on
  admissible frames (see
  footnote~\ref{fn:gauge_group} for the effective
  subgroup);
\item the fiber $\pi^{-1}(\kappa)$ over a kernel
  $\kappa \in M$ is the $G$-orbit of equivalent algebraic
  representations (frames) for describing~$\kappa$
  in $Cl(V,q)$;
\item a \emph{section} $\sigma: M \to P$ constitutes a
  global gauge-fixing policy---a systematic choice
  of reference frame for every kernel configuration.
\end{itemize}
\end{definition}

A \emph{connection} on $P$ specifies how the reference frame
is parallel-transported as the agent's state evolves.
The curvature of this connection measures the extent to
which the reference frame ``twists'' along different paths
through state space.

% ------------------------------------------------------------
\subsection{Standing Assumptions}
\label{subsec:standing}

\begin{definition}[Standing Assumptions]
\label{def:assumptions_B}
Throughout this paper, the following conditions are assumed:
\begin{enumerate}
\item[\textup{(B1)}] \textbf{Inherited framework.}
  All assumptions (A1)--(A5) of
  Paper~I~\cite{Liu2026TDOME_I} remain in force
  (open quantum system coupled to a thermal bath,
  well-defined free energy, non-equilibrium initial state,
  finite-dimensional system Hilbert space,
  and weak-coupling or controlled-coupling regime).
  Additionally, the agent possesses an internal control
  algebra $\mathcal{O}_{\mathrm{int}} = Cl(V,q)$
  with realizability embedding
  $\phi: Cl(V,q) \hookrightarrow Cl(1,3)$
  (Q-RAIF~\cite{Liu2026QRAIF_C}).
\item[\textup{(B2)}] \textbf{Finite computational budget.}
  The agent's information processing rate satisfies
  $\mathcal{C}_{\mathrm{budget}} < \infty$
  (Definition~\ref{def:budget}).
\item[\textup{(B3)}] \textbf{Non-trivial environment.}
  The entropy rate satisfies $h_\mu > 0$ and the
  memory depth satisfies $\tau_{\mathrm{mem}} > 0$.
  In Sections~\ref{sec:breaking}--\ref{sec:cost} we
  additionally require that the Computational Ceiling
  is binding:
  $\tau_{\mathrm{mem}} > \tau_{\mathrm{par}}$
  (Theorem~\ref{thm:comp_ceiling}), i.e., the symmetric
  phase is computationally intractable.
\item[\textup{(B4)}] \textbf{Survival imperative.}
  The agent's dynamics must maintain
  $\mathcal{S} \geq \mathcal{S}_{\min}$
  over survival horizons
  $T \gg \tau_{\mathrm{mem}}$.
  This is a persistence constraint, not an optimization
  objective.
\item[\textup{(B5)}] \textbf{Gauge symmetry of bare
  algebra.}
  The automorphism group $G = \mathrm{Aut}(Cl(V,q))$
  is non-trivial ($G \neq \{e\}$).
  In the absence of computational constraints, all
  $g \in G$ yield physically equivalent descriptions.
\end{enumerate}
\end{definition}

% ============================================================
\section{The Computational Ceiling}
\label{sec:ceiling}

We now establish the fundamental computational limitation of
symmetric agents---those that treat all components of their
internal algebra as equally relevant.
The result is the computational analogue of Paper~I's
Markovian Ceiling: where that theorem showed that
\emph{memoryless} dynamics cannot achieve
$\mathcal{S} > 0$, the present theorem shows that
\emph{unbiased processing} of memory leads to computational
paralysis.

% ------------------------------------------------------------
\subsection{The Information Processing Inequality
for Bounded Agents}
\label{subsec:processing}

\paragraph{Accounting convention.}
To ensure dimensional consistency throughout, we distinguish
two quantities:
\begin{itemize}
\item $\mathcal{C}_{\mathrm{budget}}$: the agent's processing
  \emph{rate} (bits per unit time).
\item $\mathcal{I}_{\mathrm{proc}}(\tau)$: the total
  information (bits) that must be processed per evaluation
  cycle when the memory archive has depth $\tau$.
\end{itemize}
The agent must complete one evaluation cycle per
environmental correlation time $\tau_E$.
The \emph{processing rate} required for a memory depth
$\tau$ is
\begin{equation}
\label{eq:rate_def}
\mathcal{R}_{\mathrm{proc}}(\tau)
:= \frac{\mathcal{I}_{\mathrm{proc}}(\tau)}
        {\tau_E}.
\end{equation}
Paralysis occurs when
$\mathcal{R}_{\mathrm{proc}}(\tau_{\mathrm{mem}})
> \mathcal{C}_{\mathrm{budget}}$.
Hereafter we measure time in units of $\tau_E$
(i.e., set $\tau_E = 1$), so that rates and per-cycle
information quantities are numerically equal.

\begin{definition}[Symmetric processing]
\label{def:symmetric}
An agent processes its memory \emph{symmetrically} if both
its cost functional $\mathcal{C}[\cdot]$ and its distortion
measure $D(\cdot)$ are $G$-invariant:
$\mathcal{C}[g \cdot \mathcal{K}]
= \mathcal{C}[\mathcal{K}]$ and
$D(g \cdot \mathcal{F}) = D(\mathcal{F})$
for every $g \in G = \mathrm{Aut}(Cl(V,q))$.
In operational terms: for every stored correlation $c_i$ in the
memory kernel $\mathcal{K}(t,s)$ and every $g \in G$, the
cost of evaluating $c_i$ equals the cost of evaluating
$g \cdot c_i$, and no basis direction is
\emph{a priori} preferred for survival evaluation.
\end{definition}

\begin{remark}[Operational meaning of processing rate]
\label{rem:proc_rate}
We define the processing rate
$\mathcal{R}_{\mathrm{proc}}$ as an
\emph{information-throughput} measure: the number of
algebraic components that must be updated per unit time,
multiplied by the innovation rate $h_\mu$ per component.
It captures the \emph{bandwidth} cost of maintaining an
internal representation, not the algorithmic gate
complexity of individual operations.
\end{remark}

\begin{theorem}[Computational Ceiling]
\label{thm:comp_ceiling}
Let an agent satisfy assumptions~\textup{(B1)--(B5)} with
memory depth $\tau_{\mathrm{mem}}$ and per-component entropy
rate $h_\mu > 0$.
Assume the environment is \textbf{unstructured} in the
following two senses:
\textup{(i)}~the effective activated dimension satisfies
$D_{\mathrm{eff}} \approx D$ (all grades of $Cl(V,q)$
carry non-negligible correlations), and
\textup{(ii)}~the predictive information is not concentrated
on a known sub-algebra (the agent possesses no
\emph{a priori} knowledge of the environmental symmetry
group and cannot exploit group-theoretic shortcuts such as
irreducible representations or Schur
decompositions).\footnote{%
If the agent knows the environmental symmetry group $H$,
symmetric processing can be restricted to the isotypic
components of $H$, reducing the effective dimension to
$D_{\mathrm{eff}} \leq D$.
The ceiling applies to the \emph{generic} (worst-case)
scenario.
All subsequent results hold \emph{a fortiori}
when $D$ is replaced by $D_{\mathrm{eff}}$.}
Within the class of \emph{symmetric representations}
that retain all $D$ components with equal fidelity
(permitting no privileged subspace)---thereby precluding
structured compression techniques such as sparse coding
or Johnson--Lindenstrauss
embeddings~\cite{JohnsonLindenstrauss1984}, as these
inherently implement a form of symmetry
breaking---the minimum processing rate satisfies
\begin{equation}
\label{eq:comp_ceiling}
\mathcal{R}_{\mathrm{proc}}^{\mathrm{sym}}
\;\geq\; h_\mu \cdot D,
\qquad D := \dim Cl(V,q) = 2^n,
\end{equation}
where $n = \dim V$ is the number of generators.
This rate scales \emph{linearly} in the algebra dimension
$D$ and \emph{exponentially} in~$n$.

For any finite $\mathcal{C}_{\mathrm{budget}}$, the maximum
memory depth that can be processed before correlations
expire is
\begin{equation}
\label{eq:t_par}
\tau_{\mathrm{par}}
:= \frac{\mathcal{C}_{\mathrm{budget}}}
        {h_\mu \cdot D}.
\end{equation}
Here $\tau_{\mathrm{par}}$ is measured in units of
$\tau_E$ (environmental correlation times), not seconds;
cf.\ the accounting convention at the start of this section.

For $\tau_{\mathrm{mem}} > \tau_{\mathrm{par}}$, the agent's
evaluation cycle cannot complete within one correlation time:
\begin{equation}
\label{eq:paralysis}
\mathcal{I}_{\mathrm{proc}}(\tau_{\mathrm{mem}})
= h_\mu \cdot \tau_{\mathrm{mem}} \cdot D
> \mathcal{C}_{\mathrm{budget}}.
\end{equation}
Stored correlations go stale before they can be used.
\end{theorem}

\begin{proof}
Under symmetric processing, the agent maintains $D$
parallel correlation channels---one for each independent
algebraic component of $Cl(V,q)$.
The environment generates innovations at rate $h_\mu$ bits
per unit time in each channel
(Remark~\ref{rem:proc_rate}).
Over a memory depth $\tau_{\mathrm{mem}}$, the total
information load is therefore
$\mathcal{I}_{\mathrm{proc}}(\tau_{\mathrm{mem}})
= D \cdot h_\mu \cdot \tau_{\mathrm{mem}}$
bits~\cite{CoverThomas2006}, and the required
\emph{rate} is
$\mathcal{R}_{\mathrm{proc}}^{\mathrm{sym}}
= D \cdot h_\mu$ bits per unit time.

The agent must complete one evaluation cycle within
$\tau_E$ (one environmental correlation time); otherwise
the oldest correlations expire before use.
Setting
$\mathcal{R}_{\mathrm{proc}}^{\mathrm{sym}}
= \mathcal{C}_{\mathrm{budget}}$
and solving for $\tau_{\mathrm{mem}}$ gives
$\tau_{\mathrm{par}}$~\eqref{eq:t_par}.
\end{proof}

\begin{corollary}[The Symmetry Tax]
\label{cor:symmetry_tax}
Maintaining full gauge invariance imposes a multiplicative
overhead of $D = 2^n$ on all computational operations
relative to a fixed-basis agent that processes only $k$
components.
The overhead ratio is $D/k$, which for
$Cl(1,3)$ ($D = 16$, $k = 2$) is $8\times$, and grows
exponentially with the number of generators $n$.
\end{corollary}

\begin{remark}[Effective vs.\ full dimension]
\label{rem:D_eff}
The ceiling uses $D = \dim Cl(V,q) = 2^n$, the full
multivector dimension.
In practice, the environment may couple to only a subset of
grades (e.g., grade-1 generators), yielding an effective
dimension $D_{\mathrm{eff}} \leq D$.
For a \emph{structured} environment where the agent knows
which grades are active, the ceiling can be tightened to
$\mathcal{R}_{\mathrm{proc}} \gtrsim h_\mu
\cdot D_{\mathrm{eff}}$.
The unstructured assumption~(B3) represents the worst case;
all subsequent results hold \emph{a fortiori} when
$D$ is replaced by $D_{\mathrm{eff}}$.
\end{remark}

% ------------------------------------------------------------
\subsection{Processing Collapse}
\label{subsec:collapse}

\begin{proposition}[Processing Collapse]
\label{prop:collapse}
Under~\textup{(B1)--(B5)}, an agent that maintains full
gauge symmetry reaches computational paralysis at time
$\tau_{\mathrm{par}}$~\eqref{eq:t_par}.
Beyond $\tau_{\mathrm{par}}$, the agent's processing latency
$\delta t_{\mathrm{proc}}$ exceeds the environmental
correlation time $\tau_E$:
\begin{equation}
\label{eq:latency}
\delta t_{\mathrm{proc}}(\tau_{\mathrm{mem}})
= \frac{D \cdot \tau_{\mathrm{mem}}}
       {\mathcal{C}_{\mathrm{budget}} / h_\mu}
> 1
\qquad\text{(in units of $\tau_E$)}.
\end{equation}
Every stored correlation becomes stale before it can be
evaluated, rendering the entire memory archive operationally
useless.
\end{proposition}

\begin{remark}[Comparison with Paper~I's Memory Catastrophe]
\label{rem:comparison}
Paper~I's Memory Catastrophe is \emph{thermodynamic}:
the \emph{cost of storing} memory exceeds the energy budget.
The Computational Ceiling is \emph{informational}: the
\emph{cost of processing} memory exceeds the computational
budget.
The two crises are complementary---an agent with unlimited
energy but finite computation is still paralyzed, and vice
versa.
The resolution of both crises is the same: compression
through symmetry breaking.
\end{remark}

% ============================================================
\section{The Symmetry Breaking Resolution}
\label{sec:breaking}

% ------------------------------------------------------------
\subsection{Reference Frame as Gauge Fixing}
\label{subsec:frame}

\begin{definition}[Reference frame]
\label{def:frame}
A \emph{reference frame} $\mathcal{F}$ is a section
$\sigma: M \to P$ of the gauge bundle
(Definition~\ref{def:bundle}).
Choosing $\sigma$ is equivalent to selecting a preferred
orthonormal basis $\{e_1, \ldots, e_n\}$ of the generating
vector space $V$ at each point in state space~$M$, thereby
fixing the gauge freedom of $Cl(V,q)$.
\end{definition}

\begin{definition}[Projected memory kernel]
\label{def:projected}
Given a reference frame $\mathcal{F}$, let
$V_{\mathrm{fg}}(\mathcal{F}) \subset Cl(V,q)$ be the
$k^*$-dimensional \emph{foreground subspace} selected by
the rate-distortion optimization
(Theorem~\ref{thm:compression}).
Let $\Pi_{\mathcal{F}}$ denote the orthogonal projection
onto $V_{\mathrm{fg}}(\mathcal{F})$ with respect to the
trace inner product
$\langle A, B \rangle := \tr(A^\dagger B)$.
The \emph{projected memory kernel} is
\begin{equation}
\label{eq:projected_kernel}
\mathcal{K}_{\mathcal{F}}(t,s)
:= \Pi_{\mathcal{F}}\,\mathcal{K}(t,s)\,
   \Pi_{\mathcal{F}}.
\end{equation}
The complementary projection
$\Pi_{\mathcal{F}}^{\perp}
= \mathbf{1} - \Pi_{\mathcal{F}}$ defines the
\emph{background subspace} $V_{\mathrm{bg}}(\mathcal{F})$.
The decomposition
$Cl(V,q) = V_{\mathrm{fg}} \oplus V_{\mathrm{bg}}$
is determined by $\mathcal{F}$, not by any \emph{a priori}
ordering of basis vectors.
\end{definition}

% ------------------------------------------------------------
\subsection{The Rate-Distortion Bound}
\label{subsec:RD_bound}

We now apply rate-distortion theory to the problem of
optimal memory compression under the survival constraint.

\paragraph{Processing rate of a frame.}
If the agent retains $k$ algebraic components (the foreground
subspace $V_{\mathrm{fg}}$), each generating $h_\mu$ bits
per unit time, the processing rate of frame $\mathcal{F}$ is
\begin{equation}
\label{eq:R_frame}
R_{\mathcal{F}}(k) \;=\; k \cdot h_\mu
\qquad\text{(bits per unit time)}.
\end{equation}
The budget constraint $R_{\mathcal{F}} \leq
\mathcal{C}_{\mathrm{budget}}$ thus bounds the number of
maintainable components.

\begin{definition}[Survival distortion]
\label{def:survival_distortion}
The \emph{survival distortion} of a reference frame
$\mathcal{F}$ is
\begin{equation}
\label{eq:distortion}
D(\mathcal{F})
:= \mathbb{E}_{\xi}\!\left[
  \ell\!\bigl(\mathcal{S}_{\mathrm{full}}(\xi)
  - \mathcal{S}_{\mathcal{F}}(\xi)\bigr)
\right],
\end{equation}
where $\xi$ denotes environmental realizations,
$\ell: \mathbb{R} \to [0,\infty)$ is a convex,
non-decreasing loss function (we use squared error
$\ell(x) = x^2$ throughout),
$\mathcal{S}_{\mathrm{full}}(\xi)$ is the survival
functional evaluated using the full memory kernel
$\mathcal{K}(t,s)$, and $\mathcal{S}_{\mathcal{F}}(\xi)$
is evaluated using the projected kernel
$\mathcal{K}_{\mathcal{F}}(t,s)$.
\end{definition}

\begin{remark}[Information-theoretic objects]
\label{rem:info_objects}
Strictly speaking, rate-distortion theory and mutual
information apply to stochastic processes, not to
superoperator kernels directly.
Throughout
Sections~\ref{sec:breaking}--\ref{sec:cost},
$I(\mathcal{K}_{\mathcal{F}};\,\mathcal{K})$ is shorthand
for $I(\hat{X};\,X)$, where
$X = \{c_i(t)\}_{i=1}^{D}$ is the sufficient-statistic
record process induced by the full kernel $\mathcal{K}$
acting on the agent's internal coordinates, and
$\hat{X} = \{c_i(t)\}_{i \in V_{\mathrm{fg}}}$ is the
projected record induced by
$\mathcal{K}_{\mathcal{F}}$.
The distortion measure~\eqref{eq:distortion} acts on the
survival functional $\mathcal{S}$ evaluated on these
records.
\end{remark}

\begin{theorem}[Optimal Compression under Survival Constraint]
\label{thm:compression}
Let an agent with computational budget
$\mathcal{C}_{\mathrm{budget}}$ and per-component entropy
rate $h_\mu$ choose a reference frame $\mathcal{F}$ that
minimizes the survival distortion~\eqref{eq:distortion}
subject to $R_{\mathcal{F}} \leq
\mathcal{C}_{\mathrm{budget}}$~\eqref{eq:R_frame}.
Then:
\begin{enumerate}
\item[\textup{(a)}] Assuming that $D(\mathcal{F})$ is
  non-increasing in the available rate $R_{\mathcal{F}}$
  (retaining more components cannot worsen survival
  distortion), the set of optimal reference frames
  $\mathfrak{F}^*
  := \operatorname*{arg\,min}_{\mathcal{F}}
  D(\mathcal{F})$
  subject to the budget constraint is non-empty, and
  any $\mathcal{F}^* \in \mathfrak{F}^*$ saturates the
  budget:
  $R_{\mathcal{F}^*} = \mathcal{C}_{\mathrm{budget}}$
  (the set $\mathfrak{F}^*$ may contain multiple
  elements; see Theorem~\ref{thm:SSB}(c)).
\item[\textup{(b)}] The compressed representation retains
  \begin{equation}
  \label{eq:k_star}
  k^* = \left\lfloor
    \frac{\mathcal{C}_{\mathrm{budget}}}{h_\mu}
  \right\rfloor
  \end{equation}
  effective algebraic components
  (the maximum integer number of components whose processing
  rate $k^* \cdot h_\mu$ fits within the budget;
  in practice the floor function ensures
  $k^* \in \mathbb{Z}_{\geq 1}$).
\item[\textup{(c)}] The fraction of algebraic structure
  discarded (in component count) is
  \begin{equation}
  \label{eq:discard_fraction}
  1 - \frac{k^*}{D},
  \end{equation}
  For $Cl(1,3)$ ($D = 16$) with a budget allowing
  $k^* = 2$, the discarded fraction is $1 - 2/16 = 87.5\%$.
  For $k^* = 1$, it exceeds $93\%$.
  In the regime $k^* \ll D$, the fraction approaches
  $1 - 1/D$ and grows with algebra dimension.
\end{enumerate}
\end{theorem}

\begin{proof}
The survival functional $\mathcal{S}$ is a function of the
full density operator $\rho(t)$, which in turn depends on
the full memory kernel $\mathcal{K}(t,s)$.
The agent's task is to evaluate $\mathcal{S}$ using only
$k$ components of $\mathcal{K}$, chosen to minimize the
mean-squared error in $\mathcal{S}$.

Strictly, rate-distortion theory applies to
\emph{random processes}, not to superoperator kernels
directly.
The bridge is the \emph{induced record process}: the
memory kernel $\mathcal{K}(t,s)$, acting on the agent's
internal coordinates, generates a $D$-component time series
of sufficient statistics $\{c_i(t)\}_{i=1}^{D}$ whose
entropy rate per component is~$h_\mu$.
Rate-distortion is applied to this record
stream~(Section~\ref{subsec:RD};
cf.\ Tishby et al.~\cite{Tishby2000}),
with source $X = \{c_i(t)\}$ (the full record),
reconstruction
$\hat{X} = \{c_i(t)\}_{i \in V_{\mathrm{fg}}}$
(the projected record), and distortion measure
$d = |\mathcal{S}_{\mathrm{full}}
- \mathcal{S}_{\mathcal{F}}|^2$.

By Shannon's rate-distortion theorem~\cite{Shannon1959},
the minimum rate required to achieve distortion $\delta$ is
$R(\delta)$, a convex non-increasing function.
The budget constraint~\eqref{eq:R_frame} limits the
processing rate to $R_{\mathcal{F}} = k \cdot h_\mu
\leq \mathcal{C}_{\mathrm{budget}}$.
The optimal frame $\mathcal{F}^*$ saturates this bound.

For part~(b): by~\eqref{eq:R_frame}, tracking $k$ components
costs $k \cdot h_\mu$ bits per unit time.
The maximum integer $k$ satisfying
$k \cdot h_\mu \leq \mathcal{C}_{\mathrm{budget}}$ is
$k^* = \lfloor \mathcal{C}_{\mathrm{budget}} / h_\mu
\rfloor$.

The discard fraction~(c) follows by counting:
$k^*$ of $D$ components are retained.
For $Cl(1,3)$ ($D = 16$, $k^* = 2$), the discarded fraction
is $87.5\%$; for higher-dimensional algebras it exceeds
$99\%$.
\end{proof}

% ------------------------------------------------------------
\subsection{Spontaneous Symmetry Breaking}
\label{subsec:SSB}

\begin{theorem}[Necessity of Symmetry Breaking]
\label{thm:SSB}
Under assumptions~\textup{(B1)--(B5)}, with the
Computational Ceiling binding
($\tau_{\mathrm{mem}} > \tau_{\mathrm{par}}$, both
measured in units of~$\tau_E$), and
assuming \emph{non-degeneracy}: the survival
distortion~\eqref{eq:distortion} satisfies
$D(\mathcal{F}) \neq D(\mathcal{F}')$ for almost all
pairs $\mathcal{F} \neq \mathcal{F}'$ in the space of
frames\footnote{%
Non-degeneracy is generically satisfied when the
environment's pointer basis~\cite{Zurek2009} assigns
different survival values to different algebraic
components, breaking the continuous symmetry of the
distortion landscape.
In degenerate cases, a finite set of local minima may
coexist---multiple ``ego attractors''---analogous to
the discrete magnetization directions in a
crystal-field anisotropic ferromagnet.},
the agent's survival-optimal strategy requires:
\begin{enumerate}
\item[\textup{(a)}] \textbf{Gauge fixing}: selection of a
  section $\sigma$ of the gauge bundle
  (Definition~\ref{def:bundle}), breaking the
  $G$-symmetry of the bare algebra.
\item[\textup{(b)}] \textbf{Privileged decomposition}:
  partition of the algebra into foreground and background
  subspaces,
  $Cl(V,q) = V_{\mathrm{fg}} \oplus V_{\mathrm{bg}}$,
  with
  $\dim V_{\mathrm{fg}} = k^* \ll \dim V_{\mathrm{bg}}$.
\item[\textup{(c)}] \textbf{Non-uniqueness}: the gauge
  fixing is generically \emph{not} unique.
  Different initial conditions, environmental histories,
  or stochastic fluctuations lead to different choices of
  $\sigma$, just as different initial conditions in a
  ferromagnet lead to different magnetization directions.
\end{enumerate}
The symmetry breaking is \emph{spontaneous} in the precise
physical sense: the underlying algebra $Cl(V,q)$ retains
its full $G$-symmetry, but the agent's operational
representation necessarily breaks it.
\end{theorem}

\begin{proof}
By Theorem~\ref{thm:comp_ceiling}, symmetric processing
leads to paralysis at $\tau_{\mathrm{par}}$.
By assumption~(B4) (survival imperative), the agent must
maintain $\mathcal{S} \geq \mathcal{S}_{\min}$ beyond
$\tau_{\mathrm{par}}$.
This requires evaluating $\mathcal{S}$ within the
computational budget $\mathcal{C}_{\mathrm{budget}}$, which
by Theorem~\ref{thm:compression} requires projecting onto
$k^* < \dim Cl(V,q)$ components.

Such a projection \emph{is} a gauge fixing: it selects
$k^*$ basis vectors $\{e_1, \ldots, e_{k^*}\}$ from the
generating space $V$, thereby breaking the
$G$-invariance that treats all bases equivalently.

Part~(b) follows from the definition of the projected kernel
(Definition~\ref{def:projected}).
Part~(c) follows from the non-degeneracy assumption:
the rate-distortion optimization
(Theorem~\ref{thm:compression}) generically admits
finitely many local minima.
Different initial conditions or environmental histories
select different minima, analogous to the
spontaneous magnetization of a ferromagnet below $T_c$.
The breaking is \emph{spontaneous}: the algebra retains
$G$-symmetry, but any operational solution breaks it.
\end{proof}

% ------------------------------------------------------------
\subsection{The Four Bias Terms}
\label{subsec:bias}

\begin{proposition}[Structure of the Broken Phase]
\label{prop:bias}
When gauge symmetry is broken by a reference frame
$\mathcal{F}$, the agent's operational representation
acquires four systematic deviations from the symmetric
phase:
\begin{enumerate}
\item[\textup{(i)}] \textbf{Basis selection bias}
  ($\mathcal{B}_{\mathrm{select}}$):
  The choice of $\{e_1, \ldots, e_{k^*}\}$ privileges
  certain algebraic components over others.
  Information aligned with the chosen basis is processed
  efficiently; misaligned information is discarded or
  distorted.
  \emph{Observable consequence:} systematic blindness to
  off-basis environmental perturbations (orthogonal
  masking).
\item[\textup{(ii)}] \textbf{Frame drag}
  ($\mathcal{B}_{\mathrm{frame}}$):
  The connection on the gauge bundle
  (Section~\ref{subsec:fiber}) induces a systematic
  preference for states near the current gauge choice.
  The agent's predictions are biased toward confirming
  its existing frame.
  \emph{Observable consequence:} hysteresis in belief
  updating; the agent's model lags behind rapid
  environmental shifts.
\item[\textup{(iii)}] \textbf{Objective centering}
  ($\mathcal{B}_{\mathrm{center}}$):
  The survival functional $\mathcal{S}$, when evaluated
  in the projected basis, becomes centered on the agent's
  own state rather than a global optimum.
  The agent optimizes \emph{locally} within its frame.
  \emph{Observable consequence:} inability to detect
  global survival optima located in the background
  subspace.
\item[\textup{(iv)}] \textbf{Model incompleteness}
  ($\mathcal{B}_{\mathrm{inc}}$):
  The compression from $Cl(V,q)$ to $V_{\mathrm{fg}}$
  is lossy.
  The discarded components $V_{\mathrm{bg}}$ contain
  correlations that are invisible to the agent but
  physically real.
  \emph{Observable consequence:} systematic
  underestimation of total thermodynamic uncertainty
  (overconfidence).
\end{enumerate}
\end{proposition}

\begin{proof}
(i)~follows directly from the definition of the projection
$\Pi_{\mathcal{F}}$: components orthogonal to the selected
basis are annihilated.

(ii)~The parallel transport of the gauge connection
preserves the agent's basis choice along its trajectory.
Under perturbation, the connection's holonomy creates a
restoring ``force'' toward the established frame---a
systematic confirmation bias.

(iii)~In the projected representation,
$\mathcal{S}_{\mathcal{F}}$ is a function of the
$k^*$-dimensional foreground state only.
The gradient $\nabla \mathcal{S}_{\mathcal{F}}$ lies
entirely in $V_{\mathrm{fg}}$, so the agent's
optimization is blind to directions in
$V_{\mathrm{bg}}$.
This is equivalent to centering the objective function
on the agent's own representational subspace.

(iv)~By Theorem~\ref{thm:compression}(c), a fraction
$\geq 1 - k^*/\dim Cl(V,q)$ of information is discarded.
The discarded components exist physically (they contribute
to $\mathcal{S}_{\mathrm{full}}$) but are invisible to the
agent's evaluation of $\mathcal{S}_{\mathcal{F}}$.
\end{proof}

\begin{table}[H]
\centering
\small
\setlength{\tabcolsep}{3pt}
\begin{tabular}{@{}lp{3.5cm}p{4.2cm}l@{}}
\toprule
\textbf{Bias} & \textbf{Origin}
  & \textbf{Observable consequence}
  & \textbf{Determines} \\
\midrule
$\mathcal{B}_{\mathrm{select}}$ (selection)
  & projection $\Pi_{\mathcal{F}}$
  & Systematic blindness to off-basis
    perturbations (orthogonal masking)
  & \emph{what} is seen \\[4pt]
$\mathcal{B}_{\mathrm{frame}}$ (frame drag)
  & bundle connection / holonomy
  & Hysteresis in belief updating;
    model lags behind rapid drift
  & \emph{duration} \\[4pt]
$\mathcal{B}_{\mathrm{center}}$ (centering)
  & $\nabla\mathcal{S} \in V_{\mathrm{fg}}$
  & Local frame-relative optima;
    global background optima invisible
  & \emph{target} \\[4pt]
$\mathcal{B}_{\mathrm{inc}}$ (incompleteness)
  & lossy compression $k^* \ll D$
  & Underestimation of thermodynamic
    uncertainty (structural overconfidence)
  & \emph{blind spot} \\
\bottomrule
\end{tabular}
\caption{The four bias terms of the broken phase.
  All four are generic consequences of gauge fixing
  under assumptions~(B1)--(B5).}
\label{tab:bias}
\end{table}

\begin{remark}[Nature of the bias terms]
\label{rem:bias_nature}
The four bias terms (Table~\ref{tab:bias}) are not
pathologies---they are \emph{generic} consequences of gauge
fixing under bounded computation.
Any agent satisfying~(B1)--(B5) acquires all four.
\end{remark}

% ============================================================
\section{Emergent Structure: The Architecture of Ego}
\label{sec:architecture}

We consolidate the gauge-fixed compressed representation
into a single mathematical object.
Throughout this section, ``ego'' is used purely as shorthand
for a gauge-fixed compressed representation; no claims about
phenomenal consciousness, subjective experience, or qualia
are intended or implied.

\begin{definition}[Ego]
\label{def:ego}
The \emph{ego} of an agent satisfying~\textup{(B1)--(B5)}
is the pair
\begin{equation}
\label{eq:ego}
\mathfrak{E}
:= \bigl(\mathcal{F}^*,\;
   V_{\mathrm{fg}}^*\bigr),
\end{equation}
where $\mathcal{F}^* \in \mathfrak{F}^*$
(Theorem~\ref{thm:compression}) is the chosen gauge
(providing the coordinate system) and
$V_{\mathrm{fg}}^*
:= V_{\mathrm{fg}}(\mathcal{F}^*)$ is the
$k^*$-dimensional foreground subspace selected by
the rate-distortion bound (providing the compression).
The projected memory kernel
$\mathcal{K}_{\mathfrak{E}}
:= \Pi_{V_{\mathrm{fg}}^*}\,\mathcal{K}\,
\Pi_{V_{\mathrm{fg}}^*}$
is induced by this pair.
All bias terms, distortion bounds, and delusion dynamics
are functions of~$\mathfrak{E}$.
\end{definition}

% ------------------------------------------------------------
\subsection{The Ego as a Fiber Bundle Section}
\label{subsec:section}

The reference frame $\mathcal{F}$, understood as a section
$\sigma: M \to P$, is the mathematical object we call the
\emph{ego}.
It has three key properties:

\paragraph{Smoothness.}
The section $\sigma$ varies continuously with the agent's
state $\rho \in M$.
Small changes in $\rho$ produce small changes in the
preferred basis---the ego is not a discrete switch but a
smooth deformation of perspective.

\paragraph{Holonomy.}
If the agent's state traces a closed loop
$\gamma: [0,1] \to M$ with $\gamma(0) = \gamma(1) = \rho_0$,
the parallel-transported frame need not return to its
initial value:
\begin{equation}
\label{eq:holonomy}
\sigma(\gamma(1))
= \mathrm{Hol}(\gamma) \cdot \sigma(\gamma(0)),
\end{equation}
where $\mathrm{Hol}(\gamma) \in G$ is the holonomy of the
connection around $\gamma$.
Non-trivial holonomy means the agent can ``learn''---its
reference frame shifts after a complete cycle of experience.

\paragraph{Topological obstruction.}
In general, a \emph{global} section $\sigma: M \to P$ may
not exist.
The obstruction is measured by the characteristic classes of
the bundle $P$.
When a global section does not exist, the ego must have
``singularities''---states where the preferred basis is
undefined or discontinuous.
This connects to the crisis of Paper~III: the delusion trap
can be understood as the agent approaching a topological
obstruction of its own reference frame.

% ------------------------------------------------------------
\subsection{The Effective Survival Functional}
\label{subsec:effective}

\begin{proposition}[Survival decomposition]
\label{prop:decomp}
In the broken phase, the survival functional decomposes as
\begin{equation}
\label{eq:decomp}
\mathcal{S}
= \mathcal{S}_{\mathrm{vis}}(\mathcal{F})
  + \mathcal{S}_{\mathrm{hid}}(\mathcal{F}),
\end{equation}
where:
\begin{itemize}
\item $\mathcal{S}_{\mathrm{vis}}(\mathcal{F})$
  is the contribution from the foreground subspace
  $V_{\mathrm{fg}}$, computable within the agent's
  reference frame;
\item $\mathcal{S}_{\mathrm{hid}}(\mathcal{F})$
  is the contribution from the background subspace
  $V_{\mathrm{bg}}$, invisible to the agent.
\end{itemize}
The agent maximizes $\mathcal{S}_{\mathrm{vis}}$ while
being structurally blind to
$\mathcal{S}_{\mathrm{hid}}$.
\end{proposition}

\begin{proof}
The survival functional $\mathcal{S} = \Delta F - W$ depends
on $\rho(t)$, which is a function of the full memory kernel
$\mathcal{K}(t,s)$.
Decomposing
$\mathcal{K} = \Pi_{\mathcal{F}}\,\mathcal{K}\,
\Pi_{\mathcal{F}}
+ \Pi_{\mathcal{F}}^{\perp}\,\mathcal{K}\,
\Pi_{\mathcal{F}}^{\perp}
+ \text{cross terms}$,
the leading contributions are
$\mathcal{S}_{\mathrm{vis}} := \mathcal{S}[
\Pi_{\mathcal{F}}\,\mathcal{K}\,\Pi_{\mathcal{F}}]$
and
$\mathcal{S}_{\mathrm{hid}} := \mathcal{S} -
\mathcal{S}_{\mathrm{vis}}$
(collecting background and cross terms).
The agent computes only
$\mathcal{S}_{\mathrm{vis}}$, as the projected kernel
$\mathcal{K}_{\mathcal{F}}$ discards all background
components.
\end{proof}

% ------------------------------------------------------------
\subsection{The Computational Speedup}
\label{subsec:speedup}

\begin{proposition}[Ego dividend]
\label{prop:speedup}
After symmetry breaking, the computational cost of
processing memory drops from
$\mathcal{C}_{\mathrm{proc}} \sim h_\mu \cdot
\tau_{\mathrm{mem}} \cdot D$
(symmetric case, $D = \dim Cl(V,q)$) to
\begin{equation}
\label{eq:speedup}
\mathcal{C}_{\mathrm{proc}}^{(\mathcal{F})}
\sim h_\mu \cdot \tau_{\mathrm{mem}}
     \cdot k^*.
\end{equation}
The speedup factor is
\begin{equation}
\label{eq:speedup_factor}
\frac{D}{k^*} = \frac{2^n}{k^*}.
\end{equation}
\end{proposition}

This is the computational advantage of reference-frame
selection.
For $Cl(1,3)$ ($D = 16$) with $k^* = 2$, the speedup is
$8\times$.
For higher-dimensional algebras, the speedup grows
exponentially in $n$.

% ------------------------------------------------------------
\subsection{The Ego-Entropy Trade-off}
\label{subsec:tradeoff}

\begin{theorem}[Ego-Entropy Trade-off]
\label{thm:tradeoff}
Let $X = \{c_i(t)\}_{i=1}^{D}$ denote the full stochastic
record process induced by the memory kernel $\mathcal{K}$
on the agent's internal coordinates, and let
$\hat{X} = \{c_i(t)\}_{i \in V_{\mathrm{fg}}}$ denote the
projected record retained by the ego.
The mutual information between compressed and full records,
denoted $I(\mathcal{K}_{\mathcal{F}};\,\mathcal{K})
\equiv I(\hat{X};\,X)$, satisfies
\begin{equation}
\label{eq:I_retained}
I(\hat{X};\,X) \;\leq\; H(\hat{X})
\;\leq\; k^* \cdot h_\mu \cdot \tau_{\mathrm{mem}}.
\end{equation}
Under the additional assumption that
$I_{\mathrm{pred}}$~\eqref{eq:I_pred} is approximately
uniformly distributed across the $D$ algebraic components
in the symmetric phase\footnote{%
This ``uniformity assumption'' is the information-theoretic
counterpart of the unstructured-environment condition in
Theorem~\ref{thm:comp_ceiling}.
When some components carry disproportionately more predictive
information, the bound tightens or loosens depending on the
alignment between $V_{\mathrm{fg}}$ and the high-information
subspace.},
the information discarded by the ego is bounded below
(up to $O(1)$ constants under uniformity):
\begin{equation}
\label{eq:I_discarded}
I_{\mathrm{discarded}}
:= H(X) - I(\hat{X};\,X)
\gtrsim \left(1 - \frac{k^*}{D}\right)
\cdot I_{\mathrm{pred}}.
\end{equation}
\end{theorem}

\begin{proof}
By the data processing inequality,
$I(\hat{X};\,X) \leq H(\hat{X})$.
The projected record $\hat{X}$ has $k^*$ components,
each carrying at most $h_\mu$ bits per unit time over a
window of $\tau_{\mathrm{mem}}$, giving
$H(\hat{X}) \leq k^* \cdot h_\mu \cdot
\tau_{\mathrm{mem}}$~\cite{CoverThomas2006}.
This yields~\eqref{eq:I_retained}.
The total predictive information in the full record is
$I_{\mathrm{pred}}$~\eqref{eq:I_pred}.
Under the uniformity assumption, each of the $D$ components
carries $\sim I_{\mathrm{pred}}/D$, so the $k^*$ retained
components account for $\sim (k^*/D)\,I_{\mathrm{pred}}$.
The discarded fraction follows by subtraction.
\end{proof}

\begin{remark}[The price of selfhood]
\label{rem:price}
Equation~\eqref{eq:I_discarded} quantifies the
\emph{information cost of having an ego}: the agent
sacrifices at least a fraction
$1 - k^*/\dim Cl(V,q)$ of all predictive information
about its environment in exchange for computational
tractability.
This is not a deficiency---it is a \emph{design
constraint} forced by bounded resources.
The ego is the optimal lossy compression under survival
weighting.
\end{remark}

% ============================================================
\section{Worked Example: Qubit in a Two-Channel Bath}
\label{sec:example}

% ------------------------------------------------------------
\subsection{Model Setup}
\label{subsec:model}

We extend Paper~I's spin-boson model to demonstrate
symmetry breaking explicitly.
Consider a qubit ($\dim \mathcal{H}_S = 2$) with internal
algebra $Cl(0,2) \cong \mathbb{H}$ (the quaternions,
$\dim = 4$).

\paragraph{Symbol mapping.}
The general framework of Sections~\ref{sec:ceiling}--\ref{sec:architecture}
specialises as follows:
\begin{center}
\small
\begin{tabular}{@{}lll@{}}
\toprule
General & This example & Value \\
\midrule
$Cl(V,q)$ & $Cl(0,2) \cong \mathbb{H}$
  & $D = 4$ \\
$G = \mathrm{Aut}(Cl(V,q))$ & $SO(3)$
  & acting on $\{\mathbf{i},\mathbf{j},\mathbf{k}\}$\\
$\mathcal{C}_{\mathrm{budget}}$ & $2\,h_\mu$
  & bits/time \\
$k^*$ (Thm.~\ref{thm:compression})
  & $\lfloor 2h_\mu / h_\mu \rfloor = 2$
  & components \\
$V_{\mathrm{fg}}$ & $\mathrm{span}\{1,\mathbf{k}\}$
  & dephasing subspace \\
$V_{\mathrm{bg}}$ & $\mathrm{span}\{\mathbf{i},\mathbf{j}\}$
  & dissipative subspace \\
$\tau_{\mathrm{par}}$ (Thm.~\ref{thm:comp_ceiling})
  & $2h_\mu / (4h_\mu) = 0.5$
  & $\omega_0^{-1}$ \\
\bottomrule
\end{tabular}
\end{center}
The qubit is coupled to a bosonic environment through
\emph{two} independent channels:
\begin{itemize}
\item A \emph{dephasing channel} via $\sigma_z$, with
  spectral density
  \begin{equation}
  \label{eq:J_z}
  J_z(\omega)
  = \frac{2\lambda_z\,\gamma_z\,\omega}
         {\omega^2 + \gamma_z^2}
  \qquad\text{(Lorentz--Drude)},
  \end{equation}
  producing a memory kernel $\mathcal{K}_z(t,s)$ with
  non-Markovian backflow.
\item A \emph{dissipative channel} via $\sigma_x$, with
  spectral density
  \begin{equation}
  \label{eq:J_x}
  J_x(\omega)
  = \frac{2\lambda_x\,\gamma_x\,\omega}
         {\omega^2 + \gamma_x^2}
  \qquad\text{(Lorentz--Drude)},
  \end{equation}
  producing a memory kernel $\mathcal{K}_x(t,s)$.
\end{itemize}

The full memory kernel is
$\mathcal{K}(t,s)
= \mathcal{K}_z(t,s) \oplus \mathcal{K}_x(t,s)$,
and the quaternionic algebra
$\mathbb{H} = \mathrm{span}\{1, \mathbf{i}, \mathbf{j},
\mathbf{k}\}$
has automorphism group
$G = \mathrm{Aut}(\mathbb{H}) \cong SO(3)$
(rotations of the pure quaternion subspace).

\paragraph{Parameters.}
We set $\omega_0 = 1$ (energy unit),
$\lambda_z = 1$, $\gamma_z = 0.5$ (underdamped, strong
non-Markovian effects in the dephasing channel),
$\lambda_x = 0.3$, $\gamma_x = 5.0$ (overdamped,
approximately Markovian in the dissipative channel),
and the low-temperature regime $\beta\omega_0 \gg 1$.

\paragraph{Computational budget.}
The agent has
$\mathcal{C}_{\mathrm{budget}} = 2\,h_\mu$
bits per unit time---sufficient to track two components
of $\mathbb{H}$ but not all four.

\paragraph{Parameter-to-theorem mapping.}
Table~\ref{tab:params} collects the example parameters and
confirms that the Computational Ceiling binds.

\begin{table}[t]
\centering
\small
\begin{tabular}{@{}lcll@{}}
\toprule
\textbf{Quantity} & \textbf{Symbol} & \textbf{Value}
  & \textbf{Theorem check} \\
\midrule
Full dimension & $D$ & $4$ &
  Thm.~\ref{thm:comp_ceiling} \\
Entropy rate & $h_\mu$ & $1.0$ (normalised) &
  per-component rate \\
Budget & $\mathcal{C}_{\mathrm{budget}}$ & $2\,h_\mu$
  & Def.~\ref{def:budget} \\
\midrule
Ceiling check & $h_\mu D$ vs $\mathcal{C}_{\rm budget}$
  & $4 > 2$ & \textbf{ceiling binds} \\
Optimal $k$ & $k^*$ & $\lfloor 2/1 \rfloor = 2$
  & Thm.~\ref{thm:compression}(b) \\
Discard fraction & $1 - k^*/D$ & $1/2 = 50\%$
  & Thm.~\ref{thm:tradeoff} \\
Paralysis time & $\tau_{\mathrm{par}}$ & $2/(4) = 0.5$
  & Eq.~\eqref{eq:t_par} \\
\bottomrule
\end{tabular}
\caption{Parameter mapping for the two-channel qubit
example.  The ceiling check confirms that symmetry
breaking is necessary; the budget is exactly saturated
after breaking ($R_{\mathcal{F}} = k^* h_\mu
= \mathcal{C}_{\mathrm{budget}}$).}
\label{tab:params}
\end{table}

% ------------------------------------------------------------
\subsection{The Unbroken Phase: Paralysis}
\label{subsec:unbroken}

In the symmetric phase, the agent tracks all four
quaternionic components
$\{1, \mathbf{i}, \mathbf{j}, \mathbf{k}\}$
simultaneously.
The computational cost is
\begin{equation}
\label{eq:cost_full}
\mathcal{C}_{\mathrm{proc}}
= h_\mu \cdot \tau_{\mathrm{mem}} \cdot D
= 4\,h_\mu \cdot \tau_{\mathrm{mem}},
\qquad D := \dim Cl(0,2) = 4.
\end{equation}
The paralysis time is
\begin{equation}
\label{eq:t_par_example}
\tau_{\mathrm{par}}
= \frac{\mathcal{C}_{\mathrm{budget}}}
       {h_\mu \cdot D}
= \frac{2\,h_\mu}{4\,h_\mu} = 0.5
\qquad\text{(in units of $\omega_0^{-1}$)}.
\end{equation}
Beyond $\tau_{\mathrm{mem}} = 0.5\,\omega_0^{-1}$,
the agent cannot process both channels
simultaneously---it is paralyzed.

% ------------------------------------------------------------
\subsection{Symmetry Breaking: Choosing $\sigma_z$}
\label{subsec:choosing}

The agent breaks the $SO(3)$ symmetry of $\mathbb{H}$
by selecting $\sigma_z$ as the privileged basis direction,
retaining the $\{1, \mathbf{k}\}$ subspace (the dephasing
channel) as foreground and discarding
$\{\mathbf{i}, \mathbf{j}\}$ (the dissipative channel)
as background:
\begin{equation}
\label{eq:decomp_example}
\mathbb{H}
= \underbrace{\mathrm{span}\{1, \mathbf{k}\}}_{%
  V_{\mathrm{fg}}\; (k^* = 2)}
\oplus
\underbrace{\mathrm{span}\{\mathbf{i}, \mathbf{j}\}}_{%
  V_{\mathrm{bg}}}.
\end{equation}

\paragraph{Why $\sigma_z$?}
The dephasing channel ($\lambda_z = 1$, $\gamma_z = 0.5$)
is strongly non-Markovian and carries the dominant
survival-relevant information (the backflow revivals that
enable $\mathcal{S} > 0$, as demonstrated in Paper~I).
The dissipative channel ($\lambda_x = 0.3$, $\gamma_x = 5.0$)
is approximately Markovian and contributes primarily to
decoherence---its survival value is negative.

This choice coincides with the \emph{pointer basis}
selected by environmental decoherence (quantum
Darwinism~\cite{Zurek2009}): the $\sigma_z$ eigenstates
are the states that survive decoherence and become
redundantly encoded in the environment.
The ego ``accepts the suggestion'' of decoherence,
aligning its computational resources with the
environmentally stable basis.

% ------------------------------------------------------------
\subsection{The Broken Phase: Effective Processing}
\label{subsec:broken}

In the broken phase, the projected memory kernel
$\mathcal{K}_{\mathcal{F}} = \mathcal{K}_z$ retains only
the dephasing-channel dynamics.
The computational cost drops to
\begin{equation}
\label{eq:cost_broken}
\mathcal{C}_{\mathrm{proc}}^{(\mathcal{F})}
= h_\mu \cdot \tau_{\mathrm{mem}} \cdot k^*
= 2\,h_\mu \cdot \tau_{\mathrm{mem}},
\end{equation}
exactly half the symmetric cost~\eqref{eq:cost_full}.
The agent can now process memory up to depth
$\tau_{\mathrm{mem}} = 1\,\omega_0^{-1}$ before reaching
its budget---twice the paralysis time.

The survival functional in the broken phase is
\begin{equation}
\label{eq:S_broken}
\mathcal{S}_{\mathrm{vis}}(\mathcal{F})
= \mathcal{S}[\mathcal{K}_z],
\end{equation}
which, as shown in Paper~I, achieves
$\beta\,\mathcal{S}_{\mathrm{vis}} \approx +0.093$ at
the first backflow revival.

The hidden component
$\mathcal{S}_{\mathrm{hid}} = \mathcal{S}[\mathcal{K}_x]$
is the survival contribution from the dissipative channel,
which the agent can no longer evaluate.
For the chosen parameters,
$|\mathcal{S}_{\mathrm{hid}}|
\ll |\mathcal{S}_{\mathrm{vis}}|$
(the dissipative channel contributes primarily negative
survival value), so the distortion is small.

% ------------------------------------------------------------
\subsection{Quantitative Evaluation}
\label{subsec:quantitative}

We now evaluate the ego dividend explicitly.
Each channel's decoherence function follows from the exact
$T \to 0$ solution of the Lorentz--Drude pure-dephasing
model~\cite{BreuerPetruccione2002,Liu2026TDOME_I}:
\begin{equation}
\label{eq:decoherence_II}
p_\alpha(t) = e^{-\gamma_\alpha t/2}\!\left[
  \cos(\Omega_\alpha t)
  + \frac{\gamma_\alpha}{2\Omega_\alpha}\,
    \sin(\Omega_\alpha t)
\right],
\quad
\Omega_\alpha
:= \tfrac{1}{2}\sqrt{4\lambda_\alpha\gamma_\alpha
   - \gamma_\alpha^2},
\end{equation}
for $\alpha \in \{z, x\}$.
When $4\lambda_\alpha\gamma_\alpha < \gamma_\alpha^2$
(the overdamped regime), $\Omega_\alpha$ becomes imaginary
and the trigonometric functions are replaced by hyperbolic
functions (monotonic decay, no backflow).

For our parameters:
\begin{itemize}
\item \textbf{$z$-channel} ($\lambda_z = 1$,
  $\gamma_z = 0.5$):
  $\Omega_z = \frac{1}{2}\sqrt{1.75} \approx 0.661$.
  Underdamped; $|p_z(t)|$ exhibits oscillatory backflow.
\item \textbf{$x$-channel} ($\lambda_x = 0.3$,
  $\gamma_x = 5.0$):
  Discriminant $4\lambda_x\gamma_x - \gamma_x^2
  = 6 - 25 = -19 < 0$.
  Overdamped; $|p_x(t)|$ decays monotonically with no
  backflow.
\end{itemize}

The survival proxy from Paper~I,
$\beta\,\mathcal{S} \propto |p(t)|^2 - 1$
(valid for the pure-dephasing model with maximally coherent
initial state and pointer-basis measurement), applies to
each channel independently.
Backflow intervals---where $d|p_\alpha|/dt > 0$---produce
$\mathcal{S} > 0$ over those subintervals (Paper~I,
Theorem~2).

\textbf{Key result.}
For the $z$-channel with $\gamma_z = 0.5$, the first
backflow interval begins at
$t^* \approx 2.9\,\omega_0^{-1}$---well after the
paralysis time $\tau_{\mathrm{par}} = 0.5\,\omega_0^{-1}$.
The symmetric agent, paralyzed at $\tau_{\mathrm{par}}$,
can harvest \emph{zero} backflow.
The ego agent, tracking only the $z$-channel, can process
memory to depth $1\,\omega_0^{-1}$ and exploits \emph{all
three} backflow revivals visible in
Figure~\ref{fig:ego}(a).

The cumulative backflow harvested by the ego agent
(Figure~\ref{fig:ego}(b)) totals approximately $0.10$
(in dimensionless $\beta\,\mathcal{S}$ units) over
$t \in [0, 15\,\omega_0^{-1}]$.
The symmetric agent harvests exactly zero.
This infinite ratio is the \emph{ego dividend}: the entire
non-Markovian survival advantage is accessible only to the
agent that has broken symmetry.

Crucially, visual inspection of Figure~\ref{fig:ego}(a)
reveals a timeline of tragedy for the symmetric agent.
The paralysis time $\tau_{\mathrm{par}} = 0.5$ occurs
\emph{before} the onset of the first backflow interval
($t^* \approx 2.9$).
The symmetric agent is computationally dead before the
environment offers its first gift.
The ratio of survival profit is not merely large; it is
singular.
In this framework, to remain symmetric is to starve in the
midst of plenty.

\begin{figure}[t]
\centering
\includegraphics[width=\textwidth]{fig_ego.pdf}
\caption{%
Two-channel qubit model (Section~\ref{sec:example}) with
Lorentz--Drude spectral density.
\textbf{Parameters:}
$\omega_0 = 1$ (energy unit);
$\lambda_z = 1$, $\gamma_z = 0.5$ (dephasing, non-Markovian);
$\lambda_x = 0.3$, $\gamma_x = 5.0$ (dissipative,
$\sim$Markovian);
$\mathcal{C}_{\mathrm{budget}} = 2\,h_\mu$.
\textbf{Units:} time in $\omega_0^{-1}$.
\textbf{Regime:} low temperature ($\beta\omega_0 \gg 1$);
using the standard $T \to 0$ analytic
expression~\eqref{eq:decoherence_II}~%
\cite{BreuerPetruccione2002}.
\textbf{(a)}~Decoherence amplitudes $|p_z(t)|$ (blue,
non-Markovian, with backflow in green bands) and
$|p_x(t)|$ (orange, monotonic decay).
Red dashed: paralysis time $\tau_{\mathrm{par}} = 0.5$.
\textbf{(b)}~Cumulative backflow harvested.
Blue: ego agent (broken $\to \sigma_z$) exploits all three
revival intervals.
Red dashed: symmetric agent, paralyzed at
$\tau_{\mathrm{par}}$, harvests zero---all backflow occurs
after paralysis onset.
The growing gap is the \emph{ego dividend}.}
\label{fig:ego}
\end{figure}

\paragraph{Consistency check.}
We verify the Computational Ceiling
(Theorem~\ref{thm:comp_ceiling}) directly:
$\mathcal{R}_{\mathrm{proc}}^{\mathrm{sym}}
= h_\mu \cdot D = 4\,h_\mu
> \mathcal{C}_{\mathrm{budget}} = 2\,h_\mu$,
confirming that the ceiling binds and symmetry breaking is
required.
After breaking ($k^* = 2$),
$R_{\mathcal{F}} = 2\,h_\mu
= \mathcal{C}_{\mathrm{budget}}$:
the budget is exactly saturated, as predicted by
Theorem~\ref{thm:compression}(a).

% ------------------------------------------------------------
\subsection{The Pointer-State Connection}
\label{subsec:pointer}

The optimal basis choice coincides with the einselection
(environment-induced superselection) basis of decoherence
theory~\cite{Zurek2009}.
This is not a coincidence: the pointer states are precisely
those that generate the most redundant records in the
environment---i.e., the most predictive correlations.
The rate-distortion optimization
(Theorem~\ref{thm:compression}) selects the components
with the highest survival value per bit, which are
generically the pointer-state components.

\begin{remark}[Decoherence as symmetry-breaking catalyst]
\label{rem:decoherence}
The environment does not \emph{force} a specific gauge
fixing; it merely breaks the degeneracy among possible
fixings by making some bases more informationally
efficient than others.
The agent's bounded computation does the rest: once
the degeneracy is broken, the survival imperative~(B4)
selects the pointer-aligned frame as optimal.
This is the precise sense in which decoherence ``catalyzes''
the spontaneous symmetry breaking of the ego.
\end{remark}

% ============================================================
\section{The Cost of Ego}
\label{sec:cost}

The ego resolves the computational crisis of
Section~\ref{sec:ceiling}, but it introduces a new
vulnerability. A fixed reference frame is a \emph{static}
gauge choice in a \emph{dynamic} environment.
If the environment changes, the ego becomes progressively
maladaptive.

\paragraph{Drift layer.}
Environmental change can occur at multiple levels:
parameter drift ($\lambda_\alpha(t)$, $\gamma_\alpha(t)$),
spectral-density deformation ($J(\omega, t)$), or
full process-distribution shift ($P_t(X)$).
For analytical tractability, we model drift at the
\emph{spectral-density parameter level} throughout this
section; the results generalise monotonically to deeper
levels (faster drift $\Rightarrow$ shorter
$t_{\mathrm{del}}$).

% ------------------------------------------------------------
\subsection{The Rigidity Trap}
\label{subsec:rigidity}

\begin{proposition}[Frame Rigidity under Drift]
\label{prop:rigidity}
Let the environment undergo slow drift: the spectral density
parameters change as
$\lambda_\alpha(t) = \lambda_\alpha^{(0)}
+ \varepsilon\,f_\alpha(t)$
for $\alpha \in \{z, x\}$, with drift rate
$\varepsilon > 0$.
The optimal reference frame $\mathcal{F}^*(t)$ (the
instantaneous minimizer of survival distortion)
rotates continuously in the gauge group~$G$.

If the agent's reference frame $\mathcal{F}$ is held fixed
(no recalibration), the mismatch between $\mathcal{F}$
and $\mathcal{F}^*(t)$ grows as
\begin{equation}
\label{eq:mismatch}
\delta(t)
:= d_G\bigl(\mathcal{F},\, \mathcal{F}^*(t)\bigr)
\sim \varepsilon \int_0^t \bigl|\dot{f}(s)\bigr|\,ds,
\end{equation}
where $d_G$ is the geodesic distance in the gauge group.
\end{proposition}

\begin{proof}
The instantaneous optimal frame $\mathcal{F}^*(t)$ is a
continuous function of the spectral density parameters
$\{\lambda_\alpha(t), \gamma_\alpha(t)\}$.
Under the drift $\lambda_\alpha(t)
= \lambda_\alpha^{(0)} + \varepsilon\,f_\alpha(t)$,
the chain rule gives
$\dot{\mathcal{F}}^*(t)
= \varepsilon\,\sum_\alpha
(\partial \mathcal{F}^*/\partial\lambda_\alpha)\,
\dot{f}_\alpha(t)$.
Integrating and taking the norm in $G$ gives
the bound~\eqref{eq:mismatch}.
\end{proof}

% ------------------------------------------------------------
\subsection{Stylized Drift Model}
\label{subsec:drift_model}

To quantify the collapse of a fixed frame, we introduce
a minimal drift model that makes the exponential divergence
and the logarithmic delusion time algebraically explicit.

\begin{definition}[Rotating optimal frame]
\label{def:drift}
Let the mismatch angle $\theta(t)$ between the agent's
fixed frame $\mathcal{F}$ and the instantaneous optimal
frame $\mathcal{F}^*(t)$ evolve as
\begin{equation}
\label{eq:theta}
\theta(t) = \theta_0\,e^{\Lambda t}
\qquad\text{(chaotic drift)},
\end{equation}
where $\theta_0 \in (0, \pi/4)$ is the initial misalignment
(so that $t_{\mathrm{del}} > 0$) and
$\Lambda > 0$ is the environmental Lyapunov exponent
(the rate at which nearby environmental trajectories
diverge in spectral-density space).
Operationally, $\Lambda$ is determined by the drift rate
$\varepsilon$ and the adaptation timescale
$\tau_{\mathrm{adapt}}$ of the spectral-density
parameters via the scaling
\begin{equation}
\label{eq:Lambda_scaling}
\Lambda \;\sim\; \frac{\varepsilon}
  {\tau_{\mathrm{adapt}}}\,;
\end{equation}
cf.~\eqref{eq:mismatch}.
For slow linear drift ($\theta(t) = \varepsilon\,t$,
$\Lambda \to 0$), the crossover time is
$t_{\mathrm{del}} = \pi/(4\varepsilon)$
(Remark~\ref{rem:linear_drift}).

The visible and hidden survival components decompose
geometrically:
\begin{equation}
\label{eq:vis_hid_theta}
\mathcal{S}_{\mathrm{vis}}(t)
= \mathcal{S}_{\mathrm{tot}}\,\cos^2\theta(t),
\qquad
\mathcal{S}_{\mathrm{hid}}(t)
= \mathcal{S}_{\mathrm{tot}}\,\sin^2\theta(t),
\end{equation}
where $\mathcal{S}_{\mathrm{tot}}$ is the full survival
functional (invariant under frame rotation).
\end{definition}

% ------------------------------------------------------------
\subsection{The Prediction Error Divergence}
\label{subsec:divergence}

\begin{proposition}[Divergence of Hidden Survival]
\label{prop:divergence}
Under the drift model~\eqref{eq:theta}--\eqref{eq:vis_hid_theta},
the hidden survival component grows as
\begin{equation}
\label{eq:divergence}
\bigl|\mathcal{S}_{\mathrm{hid}}(t)\bigr|
= |\mathcal{S}_{\mathrm{tot}}|\,\sin^2\!\bigl(\theta_0\,
  e^{\Lambda t}\bigr).
\end{equation}
For small angles ($\theta_0 e^{\Lambda t} \ll 1$):
$|\mathcal{S}_{\mathrm{hid}}| \approx
|\mathcal{S}_{\mathrm{tot}}|\,\theta_0^2\,e^{2\Lambda t}$
(exponential growth).
\end{proposition}

\begin{proof}
Direct substitution of~\eqref{eq:theta}
into~\eqref{eq:vis_hid_theta}.
The small-angle expansion
$\sin^2\theta \approx \theta^2$ gives the exponential
form.
\end{proof}

% ------------------------------------------------------------
\subsection{The Delusion Trap}
\label{subsec:delusion}

\begin{theorem}[The Delusion Trap]
\label{thm:delusion}
Under~\textup{(B1)--(B5)} with the drift
model~\eqref{eq:theta} and initial misalignment
$\theta_0 \in (0,\,\pi/4)$, an agent with a fixed
reference frame $\mathcal{F}$ reaches a critical
\textbf{delusion time}
\begin{equation}
\label{eq:t_delusion}
t_{\mathrm{del}}
= \frac{1}{\Lambda}\,
\ln\!\left(\frac{\pi/4}{\theta_0}\right),
\end{equation}
beyond which:
\begin{enumerate}
\item[\textup{(a)}]
  $|\mathcal{S}_{\mathrm{hid}}(t)|
  > |\mathcal{S}_{\mathrm{vis}}(t)|$:
  the invisible component dominates the survival functional.
\item[\textup{(b)}]
  The agent's update direction becomes
  \emph{anti-correlated} with the true optimal direction:
  the inner product of survival gradients
  (with respect to the agent's control variables
  $u \in V_{\mathrm{fg}}$) satisfies
  \begin{equation}
  \label{eq:anti_correlation}
  \bigl\langle \nabla_u\mathcal{S}_{\mathrm{vis}},\;
    \nabla_u\mathcal{S}_{\mathrm{full}}
  \bigr\rangle < 0.
  \end{equation}
  Updating $u$ to maximise $\mathcal{S}_{\mathrm{vis}}$
  actually \emph{decreases}
  $\mathcal{S}_{\mathrm{full}}$.
\item[\textup{(c)}]
  The agent cannot detect this failure from within its
  own reference frame, because all four bias terms
  ($\mathcal{B}_{\mathrm{select}}$,
  $\mathcal{B}_{\mathrm{frame}}$,
  $\mathcal{B}_{\mathrm{center}}$,
  $\mathcal{B}_{\mathrm{inc}}$)
  operate within $V_{\mathrm{fg}}$ and cannot register
  changes in $V_{\mathrm{bg}}$.
\end{enumerate}
\end{theorem}

\begin{proof}
Part~(a):
The crossover $|\mathcal{S}_{\mathrm{hid}}|
= |\mathcal{S}_{\mathrm{vis}}|$ occurs when
$\sin^2\theta = \cos^2\theta$, i.e.,
$\theta(t_{\mathrm{del}}) = \pi/4$.
Substituting~\eqref{eq:theta}:
$\theta_0\,e^{\Lambda\,t_{\mathrm{del}}} = \pi/4$,
which gives~\eqref{eq:t_delusion}.
The logarithmic dependence on $1/\theta_0$ means that
even a very small initial misalignment ($\theta_0 \sim 10^{-3}$)
delays the trap only by $\sim 7/\Lambda$---a modest
multiple of the environmental Lyapunov time.

Part~(b):
Beyond $t_{\mathrm{del}}$, the gradient
$\nabla\mathcal{S}_{\mathrm{full}}$ points primarily
into $V_{\mathrm{bg}}$ (the hidden sector now carrying
$> 50\%$ of survival weight), while
$\nabla\mathcal{S}_{\mathrm{vis}}$ remains confined to
$V_{\mathrm{fg}}$.
Since the foreground and background subspaces are
orthogonal by construction, the angle between the two
gradients exceeds $\pi/2$, yielding anti-correlation.

Part~(c):
The bias terms
$\mathcal{B}_{\mathrm{select}}$ through
$\mathcal{B}_{\mathrm{inc}}$
(Proposition~\ref{prop:bias}) are defined
\emph{within} $V_{\mathrm{fg}}$.
The agent's performance metric
$\mathcal{S}_{\mathrm{vis}} =
\mathcal{S}_{\mathrm{tot}}\cos^2\theta$
decreases only at second order in $\theta$,
so it remains positive and shows no anomaly until
$\theta$ is already $O(1)$.
The growing signal in $V_{\mathrm{bg}}$ maps to the null
space of $\Pi_{\mathcal{F}}$ and is strictly invisible.
\end{proof}

\begin{remark}[Linear drift limit]
\label{rem:linear_drift}
For slow linear drift ($\theta(t) = \varepsilon\,t$,
$\Lambda \to 0$), the crossover occurs at
$t_{\mathrm{del}} = \pi/(4\varepsilon)$.
With $\varepsilon = 0.01\,\omega_0$,
$t_{\mathrm{del}} \approx 79\,\omega_0^{-1}$---long
enough for the agent to accumulate a false sense of
security, yet short on environmental timescales.
\end{remark}

\begin{remark}[Why dithering does not help]
\label{rem:dithering}
One might ask whether the agent could escape the delusion
trap by randomly ``probing'' the background subspace
$V_{\mathrm{bg}}$---temporarily rotating its frame to
sample hidden components.
This fails for two reasons.
First, each probe costs
$\sim h_\mu \cdot D$ bits of
computation (the Symmetry Tax, Corollary~\ref{cor:symmetry_tax}),
directly competing with the budget allocated to foreground
processing.
Second---and more fundamentally---the agent has no
\emph{gradient signal} to indicate \emph{when} or
\emph{where} to probe.
As long as $|\mathcal{S}_{\mathrm{hid}}|
< |\mathcal{S}_{\mathrm{vis}}|$ (pre-delusion), the
in-frame performance metric $\mathcal{S}_{\mathrm{vis}}$
shows no anomaly.
The exponential divergence~\eqref{eq:divergence} is
invisible until it dominates---at which point it is too late.
Systematic correction requires monitoring the \emph{rate of
change} of prediction error, which is a second-order
operation: the subject of Paper~III.
\end{remark}

\begin{remark}[The ego as medicine and poison]
\label{rem:medicine_poison}
The ego cures computational paralysis
(Theorem~\ref{thm:comp_ceiling}) but creates the delusion
trap (Theorem~\ref{thm:delusion}).
It is simultaneously the \emph{medicine} for Paper~I's
crisis and the \emph{poison} that generates Paper~III's
crisis.
This duality is a structural consequence of the irreversible
logic chain: each resolution creates the conditions for the
next crisis.
\end{remark}

% ------------------------------------------------------------
\subsection{The Origin of Paper~III}
\label{subsec:paper3}

To escape the delusion trap, the agent needs a mechanism
to monitor the quality of its own reference frame---to
``observe its own observation.''
This requires a \emph{second-order control loop}: a
meta-controller that adjusts the gauge fixing $\sigma$ in
response to accumulated prediction errors.

The key difficulty is that the prediction errors the agent
can measure ($\mathcal{S}_{\mathrm{vis}} -
\mathcal{S}_{\mathrm{vis}}^{\mathrm{predicted}}$) all
lie within $V_{\mathrm{fg}}$.
To detect frame drift, the agent must compare these
in-frame errors to an estimate of out-of-frame
contributions---a self-referential operation that requires
\emph{Fisher information about the agent's own parameters}.

This is the subject of Paper~III: the Fisher information
geometry of self-referential calibration, and the
thermodynamic cost of the loop that closes the chain
\emph{Chaos $\to$ Time $\to$ Self $\to$ Calibration}.

% ============================================================
\section{Numerical Demonstration}
\label{sec:numerical}

The preceding sections establish analytic bounds and a
worked example with a qubit in a two-channel bath.
We now provide a numerical illustration showing that the
core symmetry-breaking signature---attention entropy collapse
under budget constraints---and the resulting selection
advantage are reproduced in a minimal multi-dimensional
system.
Full code and parameters are provided for reproducibility.

% ------------------------------------------------------------
\subsection{Model}
\label{subsec:demo_model}

\paragraph{Environment.}
A $D$-dimensional linear prediction task with sparse
rotating support:
$y(t) = \mathbf{w}^*(t)^\top \mathbf{x}(t) + \xi(t)$,
$\mathbf{x}(t) \sim \mathcal{N}(\mathbf{0}, \mathbf{I}_D)$,
$\xi \sim \mathcal{N}(0, \sigma^2)$.
Only $m \ll D$ dimensions carry nonzero weight at any time;
the active support rotates every $\tau_{\mathrm{switch}}$
steps, modelling environmental drift.

\paragraph{Hard budget constraint.}
Per step, the agent may update only $k$ coordinates of its
weight vector (a hard processing budget), mirroring the
bounded computation assumption~(B2).

\paragraph{Agents.}
\begin{itemize}
\item \textbf{Budgeted selector (SSB)}: selects the top-$k$
  dimensions by importance score---an exponential moving
  average of the signed per-coordinate gradient.  Signed
  accumulation ensures that noise dimensions (zero expected
  signal) cancel over time while signal dimensions persist,
  enabling reliable discrimination without access to the true
  support.
\item \textbf{Random-$k$ baseline}: selects $k$ dimensions
  uniformly at random each step.  This provides a
  budget-fair comparison: identical mechanism, no symmetry
  breaking.
\end{itemize}

\noindent
The choice of \emph{signed} gradient EMA (rather than
squared-gradient magnitude) is structurally motivated:
for noise dimensions $\mathbb{E}[r\,x_i] = 0$, so the
signed accumulation cancels over time; for signal
dimensions $\mathbb{E}[r\,x_i] \neq 0$, so a consistent
directional bias persists.  The signed EMA thus acts as a
\emph{directional coherence filter} that discriminates
signal from noise without access to the true support---a
minimal realisation of the ``reference-frame bias'' that
emerges from symmetry breaking.

\paragraph{Parameters.}
\begin{center}
\small
\begin{tabular}{@{}lcl@{}}
\toprule
\textbf{Quantity} & \textbf{Value} & \textbf{Role} \\
\midrule
$D$ & 64 & ambient dimension \\
$m$ & 8  & signal dimensions (sparse support) \\
$T$ & $10{,}000$ & horizon per trial \\
Seeds & 10 & independent replications \\
$\sigma$ & 0.3 & observation noise std \\
$\eta$ & 0.02 & SGD learning rate \\
$\lambda$ & 0.995 & weight decay per step \\
$k$ & 2, 4, 6, 8, 10, 12, 16, 20, 24, 32, 48, 64
  & budget grid \\
$\tau_{\mathrm{switch}}$ & $\{500, 1000, 2000\}$
  & support rotation period \\
\bottomrule
\end{tabular}
\end{center}

\paragraph{Attention entropy.}
Let $n_i$ be the number of updates coordinate $i$ receives
in a measurement window of the last $1{,}000$ steps.
The normalised update frequency
$p_i = n_i / \sum_j n_j$ defines the attention entropy:
\begin{equation}
\label{eq:H_attn}
H_{\mathrm{attn}} = -\sum_{i=1}^{D} p_i \ln p_i.
\end{equation}
Under symmetric processing (no SSB), $p_i = 1/D$ and
$H_{\mathrm{attn}} = \ln D$.  Under budget-constrained
selection, $H_{\mathrm{attn}}$ collapses away from $\ln D$,
serving as an order parameter for symmetry breaking.

\paragraph{Oracle metric.}
Neither agent has access to $\mathbf{w}^*(t)$.
Performance is evaluated externally using the
weight-space mean-squared error
$\mathrm{MSE} = \|\hat{\mathbf{w}} - \mathbf{w}^*\|^2$,
averaged over post-burn-in steps.

% ------------------------------------------------------------
\subsection{Results}
\label{subsec:demo_results}

Figure~\ref{fig:kstar_scaling} shows the two key signatures.

\paragraph{Result 1: Attention entropy collapse
(Figure~\ref{fig:kstar_scaling}a).}
Under fixed support (no rotation), the attention entropy
$H_{\mathrm{attn}}$ exhibits a sharp collapse away from
$\ln D = \ln 64 \approx 4.16$ and increases monotonically
with $k$, consistent with a budget-induced concentration
of update mass onto signal-carrying dimensions.
For budgets near and below the signal scale ($k \leq m$),
$H_{\mathrm{attn}}$ remains $O(\ln m)$, consistent with
confinement to the signal subspace.
We use the collapse of $H_{\mathrm{attn}}$ away from
$\ln D$ as the order parameter of symmetry breaking;
a strict plateau at $\ln m$ is not expected under the
present re-selection dynamics and finite-window estimator.

\paragraph{Result 2: Selection advantage
(Figure~\ref{fig:kstar_scaling}b).}
Under rotating support, the mean-squared error gap
$\Delta\mathrm{MSE} = \mathrm{MSE}_{\mathrm{rnd}}
- \mathrm{MSE}_{\mathrm{sel}}$ is positive for
$k \lesssim 3m$ and peaks at tight budgets ($k = 2$)
where the selection advantage is strongest.
For $k \gg m$ the gap turns slightly negative
(the selector's commitment to stale dimensions costs
more than the random baseline's diversification),
before returning to zero at $k = D$.
The three $\tau_{\mathrm{switch}}$ curves are ordered:
slower drift (larger $\tau$) yields a larger peak gap,
with the ordering most visible at small $k$.

\begin{figure}[t]
\centering
\includegraphics[width=\textwidth]%
  {fig_paper2_kstar_scaling.pdf}
\caption{%
\textbf{Budget-induced symmetry breaking.}
$D = 64$, $m = 8$, $T = 10{,}000$, 10 seeds, 95\% CI bands.
\textbf{(a)}~Attention entropy $H_{\mathrm{attn}}$ vs
budget $k$ (fixed support).  The empirical curve (blue)
collapses from $\ln D \approx 4.16$ toward an
$O(\ln m)$ floor as budget tightens.
For $k \leq m$, $H_{\mathrm{attn}}$ remains below
$\ln m \approx 2.08$ (green dashed), consistent with
confinement to the signal subspace.
\textbf{(b)}~Selection advantage
$\Delta\mathrm{MSE} = \mathrm{MSE}_{\mathrm{rnd}}
- \mathrm{MSE}_{\mathrm{sel}}$ vs budget $k$ under
rotating support.  The gap is positive for $k \lesssim 3m$
(selection helps), turns slightly negative at large $k$
(commitment cost exceeds diversification), and returns
to zero at $k = D$.  Slower drift ($\tau = 2000$)
yields a larger peak advantage.}
\label{fig:kstar_scaling}
\end{figure}

% ------------------------------------------------------------
\subsection{Scope of This Demonstration}
\label{subsec:demo_scope}

These simulations illustrate the symmetry-breaking
phenomenon predicted by Theorem~\ref{thm:SSB} under
the stated model class; they do not constitute a proof
beyond this class.

\medskip
This demonstration \textbf{does} show:
\begin{enumerate}
\item Under hard budget constraints, attention entropy
  collapses sharply away from $\ln D$ and remains
  $O(\ln m)$ for $k \leq m$---the agent confines its
  updates to the signal subspace.
  This is the computational analogue of spontaneous
  symmetry breaking (Theorem~\ref{thm:SSB}).
\item A budgeted selector that exploits importance-weighted
  selection systematically outperforms a budget-fair
  random baseline, consistent with a survival advantage
  in the broken phase
  (cf.~Proposition~\ref{prop:speedup}).
\item The advantage scales with both budget tightness
  (smaller $k$) and environmental stability (larger
  $\tau_{\mathrm{switch}}$).
\end{enumerate}

\noindent
In summary, this demonstration validates the
\emph{existence} and \emph{measurability} of
budget-induced symmetry breaking in a minimal linear
setting; it does not claim universality across
architectures or environment classes.

\medskip
This demonstration does \textbf{not} show:
\begin{enumerate}
\item That $H_{\mathrm{attn}}$ reaches a strict plateau
  at $\ln m$ for all $k \leq m$.  Under the adaptive
  re-selection dynamics used here, the selector cycles
  within the signal subspace, producing
  $H_{\mathrm{attn}}$ values near but not locked to
  $\ln m$.  The relevant signature is the collapse
  \emph{away from} $\ln D$, not convergence to a
  specific lower bound.
\item That the specific form of the importance score
  (signed gradient EMA) is optimal.  It is one
  realisation of the selection mechanism.
\item That the results generalise to all environment
  classes.  The model uses Gaussian features, linear
  regression, and sparse rotating support.
\item That the delusion--correction cycle is addressed.
  This is the subject of Paper~III.
\end{enumerate}

\paragraph{Reproducibility.}
The complete simulation is a self-contained Python script
(\texttt{paper2\_kstar\_scaling\_demo.py}, ${\sim}\,540$
lines, requiring only NumPy and Matplotlib) with fixed
random seeds.  All figures in this section can be
reproduced by executing the script.  The following files
are included in the supplementary archive:
\begin{itemize}
\item \texttt{paper2\_kstar\_scaling\_demo.py} --- simulation script
\item \texttt{fig\_paper2\_kstar\_scaling.pdf} --- Figure~\ref{fig:kstar_scaling}
\item \texttt{kstar\_scaling\_data.csv} --- raw performance gap data
\end{itemize}

% ============================================================
\section{Discussion}
\label{sec:discussion}

% ------------------------------------------------------------
\subsection{Summary of Results}
\label{subsec:summary}

\begin{center}
\small
\setlength{\tabcolsep}{4pt}%
\begin{tabular}{@{}lp{6cm}c@{}}
\toprule
\textbf{Result} & \textbf{Statement} & \textbf{Sec.} \\
\midrule
Computational Ceiling
  & Symmetric processing cost exceeds
    $\mathcal{C}_{\mathrm{budget}}$ at
    $\tau_{\mathrm{par}}$
  & \ref{sec:ceiling} \\[3pt]
Rate-Distortion Bound
  & Optimal compression retains
    $k^* = \mathcal{C}_{\mathrm{budget}}/h_\mu$
    components
  & \ref{subsec:RD_bound} \\[3pt]
Necessity of SSB
  & Under bounded computation, survival requires gauge fixing
  & \ref{subsec:SSB} \\[3pt]
Four Bias Terms
  & Broken phase acquires $\mathcal{B}_{\mathrm{select}}$,
    $\mathcal{B}_{\mathrm{frame}}$,
    $\mathcal{B}_{\mathrm{center}}$,
    $\mathcal{B}_{\mathrm{inc}}$
  & \ref{subsec:bias} \\[3pt]
Survival Decomposition
  & $\mathcal{S} = \mathcal{S}_{\mathrm{vis}} +
    \mathcal{S}_{\mathrm{hid}}$
  & \ref{subsec:effective} \\[3pt]
Ego-Entropy Trade-off
  & $\gtrsim 1 - k^*/\dim Cl(V,q)$ of
    $I_{\mathrm{pred}}$ discarded (uniformity assumption)
  & \ref{subsec:tradeoff} \\[3pt]
Delusion Trap
  & Fixed frame diverges from optimal under environmental
    drift; agent cannot self-detect
  & \ref{subsec:delusion} \\[3pt]
Numerical demo
  & Budget-induced SSB and selection advantage
    (Fig.~\ref{fig:kstar_scaling})
  & \ref{sec:numerical} \\
\bottomrule
\end{tabular}
\end{center}

% ------------------------------------------------------------
\subsection{What This Paper Does and Does Not Show}
\label{subsec:claims}

This paper \textbf{does} show:
\begin{enumerate}
\item Under bounded computation~(B2) and non-trivial
  environment~(B3), symmetric processing of memory leads
  to computational paralysis (Theorem~\ref{thm:comp_ceiling}).
\item The survival-optimal response is spontaneous symmetry
  breaking of the internal reference frame
  (Theorem~\ref{thm:SSB}), governed by a rate-distortion
  bound (Theorem~\ref{thm:compression}).
\item The broken phase acquires four generic bias
  terms under~(B1)--(B5) (Proposition~\ref{prop:bias}).
\item Under environmental drift, a fixed frame leads to
  exponential divergence of prediction error---the
  Delusion Trap (Theorem~\ref{thm:delusion}).
\item A minimal computational demonstration reproduces
  the budget-induced symmetry-breaking signature (attention
  entropy collapse away from $\ln D$) and selection
  advantage over a budget-fair random baseline
  (Section~\ref{sec:numerical},
  Figure~\ref{fig:kstar_scaling}).
\end{enumerate}

\medskip
This paper does \textbf{not} show:
\begin{enumerate}
\item That the privileged basis is uniquely determined by
  computational constraints. The basis is constrained but
  not unique---different histories lead to different gauge
  fixings, as in a ferromagnet.
\item That symmetry breaking is sufficient for persistence.
  It is the survival-optimal strategy under bounded computation;
  sufficiency
  requires the self-referential calibration of Paper~III.
\item That the ``ego'' implies or requires consciousness,
  subjective experience, or phenomenal awareness.
  The term is used strictly in the control-theoretic sense.
\item That this framework constitutes a theory of
  consciousness. It is a theory of computational optimality
  under thermodynamic constraints.
\item That the four bias terms exhaust the phenomenology of
  self-reference. They are the minimal structural
  consequences of gauge fixing in a Clifford algebra.
\item That the rate-distortion bound is achievable by any
  specific physical implementation. It is an
  information-theoretic lower bound.
\item That the Delusion Trap is inescapable. Paper~III will
  show it can be mitigated by self-referential calibration.
\item That the framework constitutes or implies a
  philosophical or metaphysical claim about the nature
  of selfhood.
\item That this framework applies to all possible physical
  systems. It applies to systems
  satisfying~(B1)--(B5)---persistent agents with finite
  computation in non-trivial environments.
\item That the Clifford algebra is the only possible
  algebraic setting. It is the minimal setting inherited
  from Q-RAIF. Other algebras may yield analogous results.
\end{enumerate}

% ============================================================
% REFERENCES
% ============================================================
\begin{thebibliography}{99}

\bibitem{Liu2026TDOME_I}
S.~Liu,
\emph{Non-Markovian Memory and the Thermodynamic Necessity
of Temporal Accumulation},
Zenodo (2026), DOI: 10.5281/zenodo.18574342.

\bibitem{BreuerPetruccione2002}
H.-P.~Breuer and F.~Petruccione,
\emph{The Theory of Open Quantum Systems},
Oxford University Press (2002).

\bibitem{CrutchfieldYoung1989}
J.~P.~Crutchfield and K.~Young,
\emph{Inferring statistical complexity},
Phys.\ Rev.\ Lett.\ \textbf{63}, 105 (1989).

\bibitem{ShaliziCrutchfield2001}
C.~R.~Shalizi and J.~P.~Crutchfield,
\emph{Computational mechanics: Pattern and prediction,
structure and simplicity},
J.\ Stat.\ Phys.\ \textbf{104}, 817 (2001).

\bibitem{BialekNemenmanTishby2001}
W.~Bialek, I.~Nemenman, and N.~Tishby,
\emph{Predictability, complexity, and learning},
Neural Computation \textbf{13}, 2409 (2001).

\bibitem{Shannon1959}
C.~E.~Shannon,
\emph{Coding theorems for a discrete source with a
fidelity criterion},
IRE Nat.\ Conv.\ Rec., Part~4, pp.~142--163 (1959).

\bibitem{CoverThomas2006}
T.~M.~Cover and J.~A.~Thomas,
\emph{Elements of Information Theory},
2nd ed., Wiley (2006).

\bibitem{Simon1955}
H.~A.~Simon,
\emph{A behavioral model of rational choice},
Quarterly J.\ of Economics \textbf{69}, 99 (1955).

\bibitem{Tishby2000}
N.~Tishby, F.~C.~Pereira, and W.~Bialek,
\emph{The information bottleneck method},
in Proc.\ 37th Allerton Conf.\ on Communication, Control,
and Computing (1999); arXiv:physics/0004057 (2000).

\bibitem{Sims2003}
C.~A.~Sims,
\emph{Implications of rational inattention},
J.\ Monetary Economics \textbf{50}, 665 (2003).

\bibitem{JohnsonLindenstrauss1984}
W.~B.~Johnson and J.~Lindenstrauss,
\emph{Extensions of Lipschitz mappings into a Hilbert space},
Contemp.\ Math.\ \textbf{26}, 189 (1984).

\bibitem{Landauer1961}
R.~Landauer,
\emph{Irreversibility and heat generation in the
computing process},
IBM J.\ Res.\ Dev.\ \textbf{5}, 183 (1961).

\bibitem{Bennett1982}
C.~H.~Bennett,
\emph{The thermodynamics of computation---a review},
Int.\ J.\ Theor.\ Phys.\ \textbf{21}, 905 (1982).

\bibitem{Zurek2009}
W.~H.~Zurek,
\emph{Quantum Darwinism},
Nature Physics \textbf{5}, 181 (2009).

\bibitem{RivasHuelgaPlenio2014}
\'{A}.~Rivas, S.~F.~Huelga, and M.~B.~Plenio,
\emph{Quantum non-Markovianity: characterization,
quantification and detection},
Rep.\ Prog.\ Phys.\ \textbf{77}, 094001 (2014).

\bibitem{Liu2026HAFF_A}
S.~Liu,
\emph{Emergent Geometry from Coarse-Grained Observable
Algebras},
Zenodo (2026), DOI: 10.5281/zenodo.18361707.

\bibitem{Liu2026HAFF_B}
S.~Liu,
\emph{Accessibility, Stability, and Emergent Geometry},
Zenodo (2026), DOI: 10.5281/zenodo.18367061.

\bibitem{Liu2026HAFF_G}
S.~Liu,
\emph{Structural Limits of Unification: Accessibility,
Incompleteness, and the Necessity of a Final Cut},
Zenodo (2026), DOI: 10.5281/zenodo.18402908.

\bibitem{Liu2026QRAIF_A}
S.~Liu,
\emph{Algebraic Constraints on the Emergence of Lorentzian
Metrics in Entropic Gravity Frameworks},
Zenodo (2026), DOI: 10.5281/zenodo.18525877.

\bibitem{Liu2026QRAIF_B}
S.~Liu,
\emph{Thermodynamic Stability Constraints on the Operator
Algebra of Persistent Open Quantum Subsystems},
Zenodo (2026), DOI: 10.5281/zenodo.18525891.

\bibitem{Liu2026QRAIF_C}
S.~Liu,
\emph{The Realizability Bridge: Algebraic Closure in the
Q-RAIF Framework},
Zenodo (2026), DOI: 10.5281/zenodo.18528935.

\end{thebibliography}

\end{document}
