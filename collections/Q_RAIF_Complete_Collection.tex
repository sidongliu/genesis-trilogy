% ============================================================================
% Q-RAIF: Quantum Reference Algebra for Information Flow
% Complete Collected Volume
% ============================================================================

\documentclass[12pt,a4paper,openany]{book}

\usepackage[utf8]{inputenc}
\usepackage[T1]{fontenc}
\usepackage{amsmath,amssymb,amsfonts,amsthm}
\usepackage{physics}
\usepackage{graphicx}
\usepackage[hidelinks]{hyperref}
\usepackage{geometry}
\usepackage{fancyhdr}
\usepackage{array}
\usepackage{booktabs}
\usepackage{xcolor}

\geometry{a4paper, margin=1in, headheight=14pt}

\newtheorem{theorem}{Theorem}[chapter]
\newtheorem{proposition}[theorem]{Proposition}
\newtheorem{lemma}[theorem]{Lemma}
\newtheorem{definition}[theorem]{Definition}
\newtheorem{remark}[theorem]{Remark}
\newtheorem{corollary}[theorem]{Corollary}

\pagestyle{fancy}
\fancyhf{}
\fancyhead[LE,RO]{\thepage}
\fancyhead[RE]{\textit{Quantum Reference Algebra for Information Flow}}
\fancyhead[LO]{\textit{\leftmark}}
\renewcommand{\headrulewidth}{0.4pt}

\title{
  \vspace{-2cm}
  {\Huge\textbf{Quantum Reference Algebra\\for Information Flow}}\\[1cm]
  {\Large\textit{Algebraic Constraints on Emergent Geometry,\\
  Thermodynamic Persistence, and Realizability}}\\[2cm]
  {\large Complete Collected Volume}\\[0.5cm]
  {\normalsize Papers A, B, and C}
}

\author{
  \textbf{Sidong Liu, PhD}\\[0.5em]
  iBioStratix Ltd\\[0.3em]
  \texttt{sidongliu@hotmail.com}
}

\date{February 2026}

\begin{document}

\frontmatter
\maketitle

\newpage
\thispagestyle{empty}
\vspace*{\fill}
\begin{center}
\textbf{Publication Record}\\[2em]
\begin{tabular}{ll}
Paper A & DOI: 10.5281/zenodo.18525877 \\
Paper B & DOI: 10.5281/zenodo.18525891 \\
Paper C & DOI: 10.5281/zenodo.18528935 \\
\end{tabular}
\\[3em]
\textit{This collected volume compiles previously published works\\
for archival and reference purposes.}
\\[2em]
\copyright{} 2026 Sidong Liu. All rights reserved.
\end{center}
\vspace*{\fill}

\chapter*{Abstract}
\addcontentsline{toc}{chapter}{Abstract}

This volume develops a set of algebraic consistency constraints governing the relationship between emergent spacetime geometry and the persistence of subsystems embedded within it.

The central result is a three-part logical chain:

\begin{enumerate}
\item \textbf{Paper A (Water):} If an emergent bulk geometry is to be Lorentzian, the boundary observable algebra must contain a Clifford algebra $Cl(1,3)$ as its minimal compatible structure.

\item \textbf{Paper B (Fish):} If a subsystem is to maintain a non-equilibrium steady state (persist) under continuous environmental interaction, its internal control algebra must be Cliffordian $Cl(V,q)$.

\item \textbf{Paper C (Bridge):} Any physically realizable control algebra must embed into the environmental observable algebra. The fish must be built from the water's algebraic atoms.
\end{enumerate}

Together, Papers A, B, and C close the logical loop: the world constrains the algebra of geometry (A), survival constrains the algebra of control (B), and realizability forces the two to be compatible (C).

The framework builds on the Holographic Alaya-Field Framework (HAFF), which establishes that geometry emerges from coarse-graining of observable algebras. Q-RAIF investigates what algebraic structures are \emph{necessary}---as opposed to merely convenient---for this emergence and for persistence within it.

\bigskip
\noindent\textbf{Keywords}: emergent geometry, Clifford algebra, entropic gravity, holographic principle, operator algebra, open quantum systems, Lyapunov stability, algebraic closure, realizability

\tableofcontents

\mainmatter


% ============================================================================
% PAPER A
% ============================================================================
\chapter{Algebraic Constraints on the Emergence of Lorentzian Metrics in Entropic Gravity Frameworks}
\label{chap:paperA}

\begin{center}
\textit{Paper A --- ``The Water''}\\[0.5em]
Originally published: Zenodo, DOI: 10.5281/zenodo.18525877
\end{center}

\bigskip

% ============================================================
\section*{Abstract}
We investigate the algebraic conditions under which an emergent bulk geometry acquires a Lorentzian signature within the framework of entropic gravity.
While thermodynamic approaches to gravity~\cite{Jacobson1995,Verlinde2011} and the Ryu--Takayanagi formula~\cite{RyuTakayanagi2006} relate entanglement entropy to geometric data, the specific algebraic mechanism constraining the spacetime signature remains an open question.

We identify three independent constraints on the boundary algebra---associativity, metric compatibility, and causal channel encoding---and argue that their simultaneous satisfaction naturally selects a \textbf{Clifford algebra} $Cl(V,q)$ as the minimal compatible structure.
We support this claim by systematically examining alternative algebraic frameworks (von Neumann factors, $C^*$-algebras, Jordan algebras, Lie algebras) and demonstrating that each fails to satisfy at least one constraint.
A worked example using a qubit tensor network illustrates how the three constraints operate in a concrete setting.

The analysis complements the Holographic Alaya-Field Framework (HAFF)~\cite{Liu2026HAFF_A,Liu2026HAFF_B}, which establishes that geometry emerges from coarse-graining of observable algebras: the present work characterizes the algebraic constraints that any such emergent geometry must satisfy to be Lorentzian.

\medskip
\noindent\textbf{Keywords}: emergent geometry, Clifford algebra, entropic gravity, holographic principle, signature selection, algebraic quantum gravity


% ============================================================
\section{Introduction}
\label{A-sec:intro}

\subsection{Context and Motivation}

The AdS/CFT correspondence~\cite{Maldacena1999} and the Ryu--Takayanagi formula~\cite{RyuTakayanagi2006} have established that spacetime geometry can be viewed as an emergent property of quantum entanglement.
Thermodynamic approaches~\cite{Jacobson1995,Verlinde2011} further suggest that the Einstein equations arise as an equation of state.
Yet a critical question remains: \textit{What algebraic constraints ensure that the emergent geometry is Lorentzian?}

Most models assume the $(1,3)$ signature \textit{a priori}.
We argue that this assumption can be partially justified by examining the algebraic consistency conditions on the boundary degrees of freedom.

\subsection{Relation to HAFF}

The Holographic Alaya-Field Framework~\cite{Liu2026HAFF_A,Liu2026HAFF_B} demonstrates that inequivalent coarse-graining structures on a global quantum state induce inequivalent emergent geometries.
HAFF establishes \emph{that} geometry emerges from observable algebras; the present work addresses \emph{what algebraic constraints} such emergent geometry must satisfy to be Lorentzian.

\begin{center}
\begin{tabular}{ll}
\textbf{HAFF} & Geometry is coarse-graining-dependent (the ``ocean'') \\
\textbf{This paper} & Lorentzian signature is algebraically constrained (the ``water'')
\end{tabular}
\end{center}

\subsection{Scope and Disclaimers}

This work does not propose a new fundamental theory of gravity, nor does it claim to derive $Cl(1,3)$ from first principles alone.
Rather, it identifies a set of physically motivated algebraic constraints and argues that Clifford algebra is the minimal structure satisfying all of them simultaneously.
The argument is presented as a \emph{consistency analysis}, not a uniqueness proof.

The specific value $(1,3)$ for the signature requires additional input beyond the algebraic constraints developed here (e.g., observational dimensionality or anomaly cancellation arguments).
We do not address why spacetime has $3+1$ dimensions.

% ============================================================
\section{Candidate Algebraic Structures}
\label{A-sec:candidates}

Before deriving constraints, we survey the landscape of algebraic structures that could, in principle, describe boundary degrees of freedom in a holographic setting.
This survey serves as the basis for the exclusion argument in Section~\ref{A-sec:exclusion}.

\begin{definition}[Boundary Algebra]
Let $\mathcal{A}_\partial$ be the algebra of observables on a holographic boundary.
We require: (a)~$\mathcal{A}_\partial$ acts faithfully on $\mathcal{H}_\partial$; (b)~$\mathcal{A}_\partial$ admits a trace compatible with the holographic entropy bound; (c)~coarse-graining of $\mathcal{A}_\partial$ induces an effective bulk description.
\end{definition}

The following algebraic families are candidates:

\begin{enumerate}
\item \textbf{von Neumann algebras} (Type I, II, III): Associative, closed under adjoint, weakly closed. Standard in algebraic QFT~\cite{Haag1996}. Type~III$_1$ factors are generic in relativistic QFT.
\item \textbf{$C^*$-algebras}: Associative Banach algebras with involution. More general than von Neumann algebras. Standard framework for quantum observables.
\item \textbf{Lie algebras}: Antisymmetric bracket $[A,B] = -[B,A]$, satisfying the Jacobi identity. Encode infinitesimal symmetries. The universal enveloping algebra is associative.
\item \textbf{Jordan algebras}: Commutative but generally non-associative: $A \circ B = B \circ A$, satisfying the Jordan identity. Proposed for quantum mechanics by Jordan, von Neumann, and Wigner (1934).
\item \textbf{Octonion algebras}: Non-associative division algebra. Explored in the context of exceptional structures in string theory~\cite{Gunaydin1973}.
\item \textbf{Clifford algebras} $Cl(V,q)$: Associative, generated by a vector space $V$ with quadratic form $q$, subject to $v^2 = q(v)\mathbf{1}$. Encode both metric and algebraic structure~\cite{Hestenes1966,Doran2003}.
\end{enumerate}

% ============================================================
\section{Three Algebraic Constraints}
\label{A-sec:constraints}

We now derive three constraints from physically motivated requirements and examine which candidate algebras survive.

\subsection{Constraint I: Associativity}

\begin{lemma}[Associativity from Compositional Consistency]
\label{A-lem:assoc}
If the boundary algebra supports well-defined time evolution (evolution operators forming a semigroup), it must be associative.
\end{lemma}

\begin{proof}
The semigroup property requires $(U(t_1)U(t_2))U(t_3) = U(t_1)(U(t_2)U(t_3))$ for all $t_i \geq 0$.
In a non-associative algebra, different bracketings of $n$ sequential operations produce $C_n \sim 4^n / n^{3/2}$ distinct results (Catalan numbers), generating uncontrolled ambiguity that grows exponentially with the number of time steps.

We note that this constraint is automatically satisfied by operator algebras on Hilbert spaces, where composition of linear maps is inherently associative.
The constraint therefore functions as a \emph{structural boundary condition}: it delineates the algebraic regime in which consistent dynamics is possible, rather than excluding a plausible physical alternative.
For analysis of non-associative dynamics and their instabilities, see~\cite{Schafer1966,Gunaydin1973}.
\end{proof}

\textbf{Exclusions}: Jordan algebras and octonion algebras are non-associative and are excluded by Constraint~I.

\subsection{Constraint II: Metric Compatibility}
\label{A-sec:constraint2}

\begin{lemma}[Non-Degenerate Bilinear Form from Holographic Error Correction]
\label{A-lem:metric}
For the boundary algebra to support error correction compatible with holographic bulk reconstruction, a non-degenerate bilinear form must be available on the space of boundary operators.
\end{lemma}

\begin{proof}
Error correction in the holographic context requires quantifying the ``distance'' between the actual boundary state and the target code subspace.
This requires a Lyapunov-type function $V(\delta\rho) \geq 0$ with $\dot{V} < 0$ under the correction protocol, which in turn requires a gradient flow:
\begin{equation}
\dot{\lambda} = -\Gamma\, G^{-1} \nabla_\lambda V,
\end{equation}
where $G$ is a metric on the parameter manifold of boundary states.

\textbf{Important distinction}: The metric $G$ appearing here is an \emph{information-geometric} metric on the space of boundary states (analogous to the Fisher--Rao metric~\cite{Petz1996}), not the emergent spacetime metric $g_{\mu\nu}$.
However, recent results in holographic entanglement~\cite{Faulkner2014,Lashkari2014} establish that linearized perturbations of the bulk metric $\delta g_{\mu\nu}$ are encoded in the boundary modular Hamiltonian and its associated Fisher information.
Specifically, the quantum-corrected Ryu--Takayanagi formula~\cite{Faulkner2014} implies:
\begin{equation}
\delta S_A = \delta \langle K_A \rangle + \delta S_{\text{bulk}},
\end{equation}
where $K_A$ is the boundary modular Hamiltonian and $S_A$ is the boundary entanglement entropy.
The structure of $G_{\text{info}}$ on the boundary therefore constrains the structure of $g_{\mu\nu}$ in the bulk.

The requirement is thus that the boundary algebra carries a non-degenerate bilinear form compatible with this holographic encoding.
Standard quantum-state metrics (Bures, Fisher--Rao) are positive-definite and satisfy non-degeneracy, but they do not encode signature information (see Constraint~III).
\end{proof}

\textbf{Exclusions}: Lie algebras carry a Killing form, but it may be degenerate (for non-semisimple algebras) and does not naturally encode a quadratic form on the generating vector space.
General $C^*$-algebras and von Neumann algebras support multiple choices of metric (Bures, Hilbert--Schmidt, etc.) but none is canonically ``built in'' to the algebraic structure itself.

\subsection{Constraint III: Causal Channel Encoding}

\begin{lemma}[Indefinite Signature from Causality]
\label{A-lem:signature}
For the emergent geometry to distinguish time-like from space-like separation, the bilinear form must have indefinite signature $(p,q)$ with $p \geq 1$, $q \geq 1$.
\end{lemma}

\begin{proof}
A positive-definite metric treats all directions identically---no causal cone structure exists.
Distinguishing causal from acausal domains requires at least one time-like and one space-like direction, hence indefinite signature.

We emphasize that this constraint does not determine the specific values of $p$ and $q$.
The identification $(p,q) = (1,3)$ requires additional input: the observational dimensionality of macroscopic spacetime.
The present argument establishes only that indefiniteness is necessary for causality, not that $(1,3)$ is the unique solution.
\end{proof}

\textbf{Exclusions}: All positive-definite metrics (including standard Bures and Fisher--Rao on quantum state spaces) fail Constraint~III.
This is the constraint that separates Clifford algebras (which carry a built-in quadratic form of arbitrary signature) from generic associative algebras with positive-definite metrics.

% ============================================================
\subsection{Exclusion of Alternative Algebras}
\label{A-sec:exclusion}

We now systematically evaluate each candidate from Section~\ref{A-sec:candidates}:

\begin{table}[h]
\centering
\begin{tabular}{@{}lcccl@{}}
\toprule
\textbf{Algebra} & \textbf{I: Assoc.} & \textbf{II: Metric} & \textbf{III: Indef.} & \textbf{Status} \\
\midrule
von Neumann (Type III$_1$) & \checkmark & $\sim$ & $\times$ & No built-in signature \\
$C^*$-algebra (general) & \checkmark & $\sim$ & $\times$ & Metric not canonical \\
Lie algebra & \checkmark$^*$ & $\times$ & --- & No quadratic form \\
Jordan algebra & $\times$ & \checkmark & --- & Non-associative \\
Octonion algebra & $\times$ & \checkmark & --- & Non-associative \\
\textbf{Clifford} $Cl(V,q)$ & \checkmark & \checkmark & \checkmark & \textbf{All satisfied} \\
\bottomrule
\end{tabular}
\caption{Evaluation of candidate algebras against three constraints.
$\checkmark$: satisfied; $\times$: violated; $\sim$: partially satisfied (metric exists but is not built-in or canonical).
$^*$Lie algebras are not associative, but their universal enveloping algebras are.}
\label{A-tab:exclusion}
\end{table}

The key observation is that Clifford algebras are distinguished by having the quadratic form $q$ \emph{built into} the algebraic structure via the defining relation $v^2 = q(v)\mathbf{1}$.
Other associative algebras (von Neumann, $C^*$) can be \emph{equipped with} metrics, but do not carry a canonical one; the metric is an additional choice external to the algebra.
In the holographic context, where the boundary algebra must encode bulk metric information, this built-in feature becomes a substantive advantage rather than a mere convenience.

% ============================================================
\subsection{Worked Example: Qubit Tensor Network}
\label{A-sec:toymodel}

To illustrate how the three constraints operate concretely, consider a tensor network model of holographic bulk reconstruction.

\paragraph{Setup.}
Take $N$ qubits arranged on a MERA (multiscale entanglement renormalization ansatz) tensor network~\cite{Swingle2012,Vidal2008}.
The boundary algebra is generated by tensor products of Pauli operators $\{\sigma_x, \sigma_y, \sigma_z\}$ acting on individual qubits.

\paragraph{Constraint I.}
The Pauli algebra is associative (it consists of $2 \times 2$ matrices).
If we were to replace the Pauli operators with elements of an octonion algebra (which is non-associative), the isometry conditions defining the MERA network---specifically, $V^\dagger V = \mathbf{1}$, which requires associative composition---would fail.

\paragraph{Constraint II.}
The Pauli operators satisfy $\{\sigma_i, \sigma_j\} = 2\delta_{ij}\mathbf{1}$, which defines a \emph{positive-definite} quadratic form.
This is the Clifford algebra $Cl(3,0)$ (or equivalently $Cl(0,3)$, since $Cl(3,0) \cong Cl(0,3)$ as algebras over $\mathbb{R}$).
The quadratic form is built into the anticommutation relation.

\paragraph{Constraint III.}
The Pauli algebra alone encodes a Euclidean signature $(3,0)$.
To obtain a Lorentzian signature, we must introduce a distinguished direction corresponding to the modular Hamiltonian $K$, which generates modular flow (the boundary analog of time evolution in the bulk).
The extended algebra $\{i K, \sigma_x, \sigma_y, \sigma_z\}$ then satisfies anticommutation relations encoding signature $(1,3)$:
\begin{equation}
(iK)^2 = -\mathbf{1}, \qquad \sigma_i^2 = +\mathbf{1}, \qquad \{iK, \sigma_i\} = 0.
\end{equation}
This is precisely $Cl(1,3)$.

\paragraph{Lesson.}
The tensor network example illustrates how associativity (isometry conditions), built-in metric (Pauli anticommutation), and indefinite signature (modular flow direction) naturally combine to produce Clifford structure in a concrete holographic model.

% ============================================================
\subsection{The Algebraic Compatibility Theorem}

\begin{theorem}[Clifford Compatibility]
\label{A-thm:selection}
Among finitely-generated associative algebras over a vector space $V$ equipped with a non-degenerate quadratic form $q$, the Clifford algebra $Cl(V,q)$ is the universal (and hence minimal) such structure, by its universal property.
\end{theorem}

\begin{proof}
Constraint~I requires associativity; Constraint~II requires a non-degenerate quadratic form on the generating space; Constraint~III requires indefinite signature.
The universal property of Clifford algebras~\cite{Hestenes1966,Doran2003} states that $Cl(V,q)$ is the unique (up to isomorphism) associative algebra generated by $V$ subject to $v^2 = q(v)\mathbf{1}$.
Any other associative algebra satisfying these constraints contains $Cl(V,q)$ as a subalgebra (or quotient), making Clifford the minimal compatible structure.
\end{proof}

\begin{remark}[Scope of the Claim]
Theorem~\ref{A-thm:selection} is a statement about \emph{algebraic compatibility}, not physical uniqueness.
It asserts that Clifford algebra is the natural minimal framework for encoding the three constraints simultaneously.
It does not exclude larger structures, nor does it claim that physics \emph{must} use the minimal option.
The theorem should be understood as identifying an algebraic bottleneck rather than deriving a unique physical theory.
\end{remark}

% ============================================================
\section{Entropic Gravity from Algebraic Structure}
\label{A-sec:entropic}

\subsection{Holographic Screen and Einstein Equations}

Following Jacobson~\cite{Jacobson1995}, the entropic force $F = T \nabla S$ and the holographic entropy bound $S \leq A / 4G$ reproduce the Einstein field equations in the thermodynamic limit.
This derivation assumes local Lorentz invariance---a condition naturally satisfied when the boundary algebra is Clifford-compatible, since $Cl(1,3)$ contains $\mathrm{Spin}(1,3)$ (the double cover of the Lorentz group) as its even subalgebra.

\subsection{Relation to HAFF Emergence Chain}

Within HAFF, geometry emerges via:
\[
\text{Observable Algebra} \to \text{Representation} \to \text{Entanglement} \to \text{Connectivity} \to \text{Geometry}
\]
The present work adds a constraint on the final arrow: among geometrically admissible coarse-grainings~\cite{Liu2026HAFF_A}, those producing Lorentzian geometry must induce effective algebras compatible with $Cl(1,3)$.

\subsection{Relation to Algebraic QFT}
\label{A-sec:aqft}

In algebraic quantum field theory (AQFT)~\cite{Haag1996}, local observable algebras associated with spacetime regions are generically Type~III$_1$ von Neumann factors.
These algebras are associative and support rich mathematical structure, but they do not carry a canonical metric of indefinite signature.

The connection to Lorentzian structure emerges through the Tomita--Takesaki theorem: for any cyclic and separating state, the modular operator $\Delta$ generates a one-parameter group (modular flow) that, in the Bisognano--Wichmann theorem, coincides with the boost generator in Rindler spacetime.
This modular flow singles out a \emph{time-like direction} within the algebraic structure.

In the HAFF framework, the accessible algebra $\mathcal{A}_{\mathbf{c}}$ can be understood as a stable subalgebra of a Type~III$_1$ factor.
The present analysis suggests that when such a subalgebra supports a Lorentzian bulk description, it must admit a $Cl(1,3)$ representation---where the modular flow direction provides the time-like generator and spatial locality provides the space-like generators.

This perspective connects the present work to Witten's observation~\cite{Witten2018} that Type~III$_1$ algebras are essential in gravitational settings, and to Connes' noncommutative geometry program~\cite{Connes1994}, where Clifford algebras play a central role in the spectral characterization of Riemannian (and pseudo-Riemannian) manifolds.

% ============================================================
\section{Discussion}
\label{A-sec:discussion}

\subsection{What This Result Does and Does Not Show}

\textbf{Does show:}
Clifford algebra is the minimal algebraic structure simultaneously satisfying associativity, metric compatibility, and causal channel encoding.
The exclusion argument (Table~\ref{A-tab:exclusion}) demonstrates that alternative algebras fail at least one constraint.
The tensor network example (Section~\ref{A-sec:toymodel}) illustrates the constraints in a concrete model.

\textbf{Does not show:}
Why $3+1$ dimensions rather than some other $(p,q)$---the argument constrains to $Cl(p,q)$ for any $p \geq 1$; the value $(1,3)$ requires additional input.
That gravity \emph{is} entropic---we derive consistency conditions within the entropic gravity framework.
That $Cl(1,3)$ is the \emph{unique} boundary algebra---larger algebras containing $Cl(1,3)$ as a subalgebra are also compatible.
A complete theory of quantum gravity.

\subsection{Convergence with Paper B}

The companion paper (Chapter~\ref{chap:paperB}) arrives at $Cl(V,q)$ from a completely different direction: thermodynamic stability of persistent open quantum subsystems.

\begin{center}
\begin{tabular}{@{}lll@{}}
\toprule
& \textbf{Paper A (this work)} & \textbf{Paper B} \\
\midrule
Question & What algebra does geometry need? & What algebra does persistence need? \\
Method & Holographic consistency & Lyapunov stability \\
Perspective & The world (``ocean'') & The subsystem (``fish'') \\
Result & $Cl(1,3)$ from signature & $Cl(V,q)$ from error stability \\
\bottomrule
\end{tabular}
\end{center}

We note that this convergence is \emph{heuristic rather than deductive}: it suggests that Clifford algebra occupies a distinguished position in the landscape of emergent algebraic structures, but does not constitute a proof.
The convergence motivates further investigation, particularly through more elaborate models and deeper connections to established algebraic frameworks.

\subsection{Open Problems}

\begin{enumerate}
\item \textbf{Dimensionality}: What additional constraints (anomaly cancellation, stability of persistent subsystems, observational input) fix $(p,q) = (1,3)$?
\item \textbf{Constructive derivation}: Can Clifford generators be explicitly constructed from modular Hamiltonians or Tomita--Takesaki data in holographic models?
\item \textbf{Relation to noncommutative geometry}: How does the present analysis connect to Connes' spectral triples, where Clifford algebras characterize the Dirac operator?
\item \textbf{Tensor network realization}: Can the qubit toy model of Section~\ref{A-sec:toymodel} be made rigorous in the context of holographic error-correcting codes?
\end{enumerate}

% ============================================================
\section{Conclusion}

We have argued that the Lorentzian metric structure in entropic gravity frameworks is algebraically constrained by three independent requirements: associativity, metric compatibility (with holographic encoding), and indefinite signature.
A systematic exclusion of alternative algebras (von Neumann, $C^*$, Jordan, Lie, octonion) shows that Clifford algebra $Cl(V,q)$ is the minimal structure satisfying all three simultaneously.

This result connects the top-down perspective of HAFF (geometry emerges from observable algebras) with bottom-up algebraic constraints (any causal geometry must be Clifford-compatible).
It does not constitute a derivation of Lorentzian gravity from first principles, but identifies an algebraic bottleneck through which any emergent causal geometry must pass.

% ============================================================


% ============================================================================
% PAPER B
% ============================================================================
\chapter{Thermodynamic Stability Constraints on the Operator Algebra of Persistent Open Quantum Subsystems}
\label{chap:paperB}

\begin{center}
\textit{Paper B --- ``The Fish''}\\[0.5em]
Originally published: Zenodo, DOI: 10.5281/zenodo.18525891
\end{center}

\bigskip

% ============================================================
\section{Introduction}
\label{B-sec:intro}

The interaction of a quantum system with a large environment typically leads to decoherence and thermalization~\cite{Breuer2002,Weiss2012}.
Maintaining a NESS requires continuous energetic cost~\cite{Seifert2012,Jarzynski2011}, which can be modeled as a feedback control process~\cite{Sagawa2012,Parrondo2015}.
We address: \textit{What algebraic structures allow the internal control dynamics to remain Lyapunov stable?}

\subsection{Three-Paper Structure}

\begin{center}
\begin{tabular}{@{}lll@{}}
\toprule
\textbf{Paper} & \textbf{Question} & \textbf{Analogy} \\
\midrule
HAFF~\cite{Liu2026HAFF_A} & How does geometry emerge? & Ocean \\
Q-RAIF A~\cite{Liu2026QRAIF_A} & What algebra does geometry need? & Water \\
This work & What algebra does survival need? & Fish \\
\bottomrule
\end{tabular}
\end{center}

\subsection{Anti-Solipsism Disclaimer}

A potential misreading is that the observer ``creates'' geometry through survival.
We explicitly reject this.
The claim is structural: any subsystem maintaining persistence must encode its environment using a Clifford-compatible algebra.
Within HAFF, geometry exists as a stable organizational phase~\cite{Liu2026HAFF_B}---contingent on physical conditions but objective within them.
The present paper argues that subsystems embedded in such a phase must reflect that geometry in their internal algebra---not generate it.

\subsection{Scope}

This work does not claim to derive Clifford algebra from first principles.
It argues that, within the variational framework of persistence under Lindblad dynamics, Clifford algebra is the minimal algebraic structure compatible with stable feedback.
The argument proceeds by exclusion of alternatives, not by uniqueness proof.

% ============================================================
\section{Variational Bounds on Persistence}
\label{B-sec:variational}

Consider $\mathcal{H}_{\mathrm{tot}} = \mathcal{H}_R \otimes \mathcal{H}_E$, with reduced dynamics:
\begin{equation}
\dot{\rho}_R = -i[H_{\mathrm{eff}}, \rho_R] + \mathcal{D}[\rho_R].
\end{equation}

\begin{definition}[Persistence Action]
$\mathcal{A}[Q] = \int_0^\tau dt\, D_{KL}(\rho(t) \| \rho_{\mathrm{NESS}})$.
\end{definition}

Minimizing $\delta\mathcal{A} = 0$ implies a control Hamiltonian $H_{\mathrm{ctrl}}(t)$ generated by an operator algebra $\mathcal{O}$.

\subsection{Why Lie Algebras Are Insufficient}

Standard quantum control theory uses Lie algebra generators~\cite{WisemanMilburn2009}: the control Hamiltonian $H_{\mathrm{ctrl}} = \sum_k u_k(t) G_k$ where $\{G_k\}$ generate a Lie algebra $\mathfrak{g}$ via commutators $[G_i, G_j] = i f_{ijk} G_k$.

Lie algebras encode \emph{infinitesimal symmetries}---they specify \emph{which directions} in state space are accessible via control.
However, they do not encode \emph{distances} between states.
The commutator $[G_i, G_j]$ determines the algebra's structure, but there is no built-in notion of ``how far'' a correction moves the state.

For error correction, the subsystem must quantify both the \emph{direction} and the \emph{magnitude} of environmental perturbations.
This requires a quadratic form $q(v) = \eta_{\mu\nu} v^\mu v^\nu$ on the space of perturbations---which is precisely the additional structure that Clifford algebras provide over Lie algebras.

% ============================================================
\section{Algebraic Constraints on Control Stability}
\label{B-sec:stability}

\subsection{Constraint I: Associativity as Structural Boundary}
\label{B-sec:assoc}

\begin{lemma}[Associativity Boundary]
\label{B-lem:assoc}
Consistent composition of sequential control operations requires an associative algebra.
\end{lemma}

We acknowledge that this constraint is automatically satisfied by operator algebras on Hilbert spaces, where composition of linear maps is inherently associative~\cite{Breuer2002}.
Non-associative algebras (Jordan, octonion) are not realistic candidates for quantum dynamics.

Lemma~\ref{B-lem:assoc} therefore functions as a \emph{structural boundary marker}: it delineates the minimal algebraic condition separating consistent from inconsistent dynamics, analogous to how the second law delineates irreversibility without claiming that reversible processes are a realistic threat.
For the mathematical structure of non-associative algebras and their dynamical instabilities, see~\cite{Schafer1966,Gunaydin1973}.

The substantive constraint is Constraint~II, which discriminates among \emph{associative} algebras.

\subsection{Constraint II: Indefinite Metric for Channel Discrimination}
\label{B-sec:metric}

\begin{lemma}[Metric Constraint]
\label{B-lem:metric}
For a persistent subsystem to distinguish qualitatively different environmental coupling channels and implement directed error correction, the control algebra must carry a non-degenerate bilinear form of indefinite signature.
\end{lemma}

\begin{proof}
Lyapunov stability requires $\dot{V} < 0$ for $V(\delta\rho) \geq 0$, implying gradient flow:
\begin{equation}
\dot{\lambda} = -\Gamma\, G^{-1} \nabla_\lambda V,
\label{B-eq:gradient}
\end{equation}
where $G$ is a metric on the control parameter manifold.

\textbf{Important distinction}: $G$ here is an information-geometric metric on the space of control parameters, not the spacetime metric.
Standard quantum state metrics (Bures, Fisher--Rao~\cite{Petz1996}) are positive-definite and satisfy non-degeneracy.
However, they are \emph{isotropic}: they treat all perturbation directions equivalently.

In realistic open quantum systems, the environment couples to the subsystem through qualitatively different channels---dissipative (population decay), dephasing (coherence loss), and unitary (Hamiltonian shift).
Effective error correction requires distinguishing these channel types, which demands an \emph{anisotropic} metric that assigns different signs to different directions.

An indefinite quadratic form $q(v) = \eta_{\mu\nu} v^\mu v^\nu$ with $\mathrm{sig}(\eta) = (p,q)$, $p,q \geq 1$, encodes this distinction: positive-norm directions correspond to one class of perturbations, negative-norm directions to another.
This is precisely the structure built into Clifford algebras via $v^2 = q(v)\mathbf{1}$.

Algebras with only positive-definite metrics (generic von Neumann factors, $C^*$-algebras with Bures metric) cannot distinguish channel types at the algebraic level, requiring external structure to do so.
\end{proof}

\subsection{Exclusion of Alternative Algebras}
\label{B-sec:exclusion}

\begin{table}[h]
\centering
\begin{tabular}{@{}lccl@{}}
\toprule
\textbf{Algebra} & \textbf{I: Assoc.} & \textbf{II: Indef.~$q$} & \textbf{Status} \\
\midrule
von Neumann (III$_1$) & \checkmark & $\times$ & No built-in $q$ \\
$C^*$-algebra & \checkmark & $\times$ & Positive-definite only \\
Lie algebra & \checkmark$^*$ & $\times$ & Killing form, no $q$ \\
Jordan algebra & $\times$ & --- & Non-associative \\
\textbf{Clifford} $Cl(V,q)$ & \checkmark & \checkmark & \textbf{Minimal} \\
\bottomrule
\end{tabular}
\caption{Systematic evaluation of candidate control algebras.
$^*$Via universal enveloping algebra.}
\label{B-tab:exclusion}
\end{table}

The exclusion argument shifts the burden from ``why Clifford?'' to ``why not the alternatives?''---and the answer is that no other standard algebraic framework carries a built-in indefinite quadratic form encoding channel discrimination.

\subsection{Worked Example: Controlled Qubit Under Lindblad Dynamics}
\label{B-sec:toymodel}

\paragraph{Setup.}
Consider a single qubit coupled to a thermal bath at inverse temperature $\beta$, with Lindblad dissipator:
\begin{equation}
\mathcal{D}[\rho] = \gamma_\downarrow \mathcal{L}[\sigma_-]\rho + \gamma_\uparrow \mathcal{L}[\sigma_+]\rho + \gamma_\phi \mathcal{L}[\sigma_z]\rho,
\end{equation}
where $\mathcal{L}[L]\rho = L\rho L^\dagger - \frac{1}{2}\{L^\dagger L, \rho\}$, and $\gamma_\downarrow$, $\gamma_\uparrow$, $\gamma_\phi$ are decay, excitation, and dephasing rates.

\paragraph{Control algebra.}
The control Hamiltonian is $H_{\mathrm{ctrl}} = \sum_i u_i(t)\, \sigma_i$ where $\{\sigma_x, \sigma_y, \sigma_z\}$ are Pauli operators.
These satisfy $\{\sigma_i, \sigma_j\} = 2\delta_{ij}\mathbf{1}$---the defining relation of $Cl(3,0)$.

\paragraph{Lyapunov function.}
Take $V = D_{KL}(\rho \| \rho_{\mathrm{NESS}})$ where $\rho_{\mathrm{NESS}}$ is the thermal state.
The gradient $\nabla_u V$ is well-defined because the Pauli algebra carries a natural inner product (the quadratic form $q(\sigma_i) = +1$).
The control protocol $u_i(t) = -\alpha\, \partial V / \partial u_i$ yields:
\begin{equation}
\dot{V} = -\alpha \sum_i \left(\frac{\partial V}{\partial u_i}\right)^2 \leq 0,
\end{equation}
which is strictly negative away from the NESS.

\paragraph{Failure mode without built-in metric.}
If the control algebra were an abstract Lie algebra $\mathfrak{su}(2)$ (same generators, but with only the commutator structure $[\sigma_i, \sigma_j] = 2i\epsilon_{ijk}\sigma_k$ and no anticommutator/metric), the control protocol could specify \emph{rotation directions} in Bloch sphere but could not canonically quantify \emph{how large} a correction to apply.
The gradient flow~\eqref{B-eq:gradient} would require importing an external metric (e.g., the Killing form of $\mathfrak{su}(2)$, which happens to be proportional to $\delta_{ij}$).

In the Pauli/Clifford case, the metric is \emph{internal}: the same algebraic structure that generates rotations also defines distances.
This unification is what makes Clifford algebras uniquely suited for feedback control where both direction and magnitude matter.

\paragraph{Extension to indefinite signature.}
When the subsystem must distinguish dissipative from unitary perturbations---e.g., $\gamma_\downarrow \neq 0$ (dissipative) versus Hamiltonian noise (unitary)---the control space naturally splits into positive-norm (unitary) and negative-norm (dissipative) sectors.
Encoding this distinction algebraically requires an indefinite quadratic form, upgrading $Cl(3,0)$ to $Cl(p,q)$ with appropriate signature.

% ============================================================
\subsection{The Algebraic Compatibility Theorem}

\begin{theorem}[Persistence Compatibility]
\label{B-thm:main}
Among associative algebras encoding $n$ orthogonal control channels with a built-in non-degenerate quadratic form, the Clifford algebra $Cl(V,q)$ is the universal minimal structure, by its universal property.
\end{theorem}

\begin{proof}
Constraint~I (associativity) is given.
Constraint~II requires a non-degenerate quadratic form $q$ on the generating space $V$, with indefinite signature when channel discrimination is required.
The universal property of Clifford algebras~\cite{Hestenes1966} identifies $Cl(V,q)$ as the unique associative algebra generated by $V$ subject to $v^2 = q(v)\mathbf{1}$.
\end{proof}

\begin{corollary}
Any subsystem maintaining NESS for $\tau \gg \tau_{\mathrm{relax}}$ while discriminating among environmental channels must encode its boundary using a Clifford-compatible algebra.
\end{corollary}

\begin{remark}[Natural Selection, Not Design]
The theorem establishes a selection principle.
Subsystems do not ``choose'' Clifford algebra; only Clifford-compatible structures persist when channel discrimination is required.
This is algebraic natural selection.
\end{remark}

% ============================================================
\section{Contextual Relations}
\label{B-sec:context}

\subsection{Convergence with Paper A}

The companion paper~\cite{Liu2026QRAIF_A} argues that $Cl(1,3)$ is the minimal algebra compatible with emergent Lorentzian geometry in entropic gravity.

\begin{center}
\begin{tabular}{@{}lcc@{}}
\toprule
& \textbf{Paper A} & \textbf{This work} \\
\midrule
Starting point & Holographic boundary & Open subsystem \\
Method & Signature selection & Lyapunov stability \\
Key constraint & Causal ordering & Channel discrimination \\
Result & $Cl(1,3)$ & $Cl(V,q)$ \\
\bottomrule
\end{tabular}
\end{center}

We note explicitly that this convergence is \emph{heuristic rather than deductive}.
Two arguments pointing to the same algebraic structure from different directions is suggestive but does not constitute proof.
The convergence motivates further investigation through explicit models and connections to established frameworks, not a claim of mathematical necessity.

\subsection{Relation to Quantum Control Theory}

Standard quantum control operates within a Lie algebraic framework~\cite{WisemanMilburn2009}: controllability is characterized by the Lie algebra generated by the drift and control Hamiltonians.
This framework is complete for determining \emph{reachability} of target states.

However, Lie algebras encode symmetries (via commutators) without encoding distances (via quadratic forms).
When the control objective is not merely reachability but \emph{stabilization against stochastic perturbations}---as in NESS maintenance---both direction and magnitude of corrections must be specified.
Clifford algebras provide this additional structure through their built-in quadratic form, complementing rather than replacing the Lie algebraic framework.

\subsection{Relation to Decoherence-Free Subspaces}

Decoherence-free subspaces (DFS)~\cite{Zanardi2001,Lidar2003} represent subsystems that are passively protected from environmental noise by symmetry.
The present analysis addresses the \emph{active} counterpart: subsystems that maintain coherence through continuous feedback.
In both cases, the algebraic structure of the system-environment interaction determines which subsystems can persist.
The Clifford constraint identified here applies to the active case; DFS theory applies to the passive case.
A unified treatment remains an open problem.

% ============================================================
\section{Discussion}

\textbf{What this result does show:}
Clifford algebra is the minimal algebraic structure satisfying both associativity and built-in indefinite metric among standard algebraic candidates.
The exclusion of alternatives (Table~\ref{B-tab:exclusion}) and the controlled qubit example (Section~\ref{B-sec:toymodel}) provide concrete support.

\textbf{What this result does not show:}
That Clifford algebra is the \emph{unique} solution---larger structures are also compatible.
That non-Clifford feedback is impossible in all settings---it is possible when channel discrimination is not required.
A derivation from first principles independent of the variational framework assumed here.

\section{Conclusion}

We have argued that geometric (Clifford) algebra structure is a natural minimal requirement for persistent subsystems that must discriminate among environmental coupling channels:
(i)~associativity is a structural boundary condition;
(ii)~a built-in indefinite quadratic form is required for channel discrimination and directed error correction;
(iii)~no standard alternative algebra satisfies both with built-in structure.

Combined with the companion paper's holographic constraints (Chapter~\ref{chap:paperA}), this suggests $Cl(V,q)$ occupies a distinguished position as the algebraic structure simultaneously compatible with geometric consistency and thermodynamic persistence.

% ============================================================


% ============================================================================
% PAPER C
% ============================================================================
\chapter{The Realizability Bridge: Algebraic Closure in the Q-RAIF Framework}
\label{chap:paperC}

\begin{center}
\textit{Paper C --- ``The Bridge''}\\[0.5em]
Originally published: Zenodo, DOI: 10.5281/zenodo.18528935
\end{center}

\bigskip

% ============================================================
\section*{Abstract}
This addendum provides a minimal mathematical bridge between the two foundational papers of the \textbf{Q-RAIF (Quantum Reference Algebra for Information Flow)} framework.
Paper A (Chapter~\ref{chap:paperA}) establishes that the observable algebra of a holographically consistent universe must contain $Cl(1,3)$ as its minimal Clifford-compatible structure.
Paper B (Chapter~\ref{chap:paperB}) establishes that the control algebra of a persistent subsystem must be Cliffordian $Cl(V,q)$ to ensure Lyapunov stability under entropic constraints.

Here we prove the \textbf{Closure Theorem}: any \emph{physically realizable} control algebra must embed into the environmental algebra as a subalgebra.
We formalize the required feedback synchrony via a \emph{Same-Clock} co-indexing lemma, ensuring the feedback loop is thermodynamically potent.

This note does not modify Papers A or B; it supplies only the realizability bridge needed for algebraic closure.

\medskip
\noindent\textbf{Keywords}: Q-RAIF, realizability, representation, operator algebra, Clifford algebra, open quantum systems, Lyapunov stability, algebraic closure


% ============================================================
\section{Introduction}
\label{C-sec:intro}

\subsection{Context: The Q-RAIF Program}

The Quantum Reference Algebra for Information Flow (Q-RAIF) framework investigates what algebraic structures are \emph{necessary}---as opposed to merely convenient---for the self-consistent description of physical reality and persistence within it.
The program builds on the Holographic Alaya-Field Framework (HAFF)~\cite{Liu2026HAFF_A,Liu2026HAFF_B}, which establishes that geometry emerges from coarse-graining of observable algebras.

\begin{center}
\begin{tabular}{@{}llll@{}}
\toprule
\textbf{Paper} & \textbf{Question} & \textbf{Analogy} & \textbf{Result} \\
\midrule
HAFF~\cite{Liu2026HAFF_A} & How does geometry emerge? & Ocean & Algebra $\to$ Geometry \\
Q-RAIF A~\cite{Liu2026QRAIF_A} & What algebra does geometry need? & Water & $Cl(1,3)$ \\
Q-RAIF B~\cite{Liu2026QRAIF_B} & What algebra does survival need? & Fish & $Cl(V,q)$ \\
This work & Must the fish fit the water? & Bridge & $Cl(V,q) \hookrightarrow Cl(1,3)$ \\
\bottomrule
\end{tabular}
\end{center}

\subsection{The Logical Gap}

Papers A and B independently arrive at Clifford algebra from opposite directions.
Both papers explicitly note that this convergence is \emph{heuristic rather than deductive}~\cite{Liu2026QRAIF_A,Liu2026QRAIF_B}.
The present note closes the gap by proving a realizability constraint: the internal control algebra of any persistent subsystem must be representable within the external observable algebra.

\subsection{Scope}

This addendum introduces no new physical assumptions.
It uses only the objects and results already established in Papers A and B, and derives their mutual constraint.
Papers A and B remain unmodified.

% ============================================================
\section{Setup and Prerequisites}
\label{C-sec:setup}

Let $\mathcal{U}$ be a universe described by the Q-RAIF framework.
\begin{itemize}
    \item \textbf{Environment (``water'').} Let $\mathcal{A}_{\mathrm{ext}}$ denote the algebra of observables accessible at the holographic boundary.
    Paper A (Chapter~\ref{chap:paperA}) argues that $\mathcal{A}_{\mathrm{ext}}$ must contain $Cl(1,3)$ as its minimal Clifford-compatible subalgebra (Theorem~1 of Paper A, ``Clifford Compatibility'').
    \item \textbf{Subsystem (``fish'').} Let $\mathcal{O}_{\mathrm{int}}$ denote the internal control algebra of a persistent subsystem $R\subset\mathcal{U}$.
    Paper B (Chapter~\ref{chap:paperB}) argues that thermodynamic persistence requires $\mathcal{O}_{\mathrm{int}} \cong Cl(V,q)$ for some $(V,q)$ (Theorem~1 of Paper B, ``Persistence Compatibility'').
\end{itemize}

The remaining logical gap is the relationship between $\mathcal{O}_{\mathrm{int}}$ and $\mathcal{A}_{\mathrm{ext}}$: can a stable Clifford control algebra exist while being structurally disjoint from the available environmental observables?

% ============================================================
\section{Realizability and Same-Clock Co-Indexing}
\label{C-sec:realizability}

\begin{definition}[Algebraic Realizability]
\label{C-def:realizability}
A control algebra $\mathcal{O}_{\mathrm{int}}$ is \textbf{physically realizable} within an environment $\mathcal{A}_{\mathrm{ext}}$ if there exists a homomorphism
\begin{equation}
    \phi: \mathcal{O}_{\mathrm{int}} \to \mathcal{A}_{\mathrm{ext}}
\end{equation}
such that $\mathrm{Im}(\phi)$ has non-zero action on the interaction Hamiltonian $H_{\mathrm{int}}$, i.e., $[\mathrm{Im}(\phi), H_{\mathrm{int}}] \neq 0$.
This ensures that the controller can physically influence the system-environment boundary.
\end{definition}

Let $I$ be an operational/causal index set (e.g., proper-time frames or discretized event slices).
For a subset $J\subseteq I$, write $\mathcal{A}|_J$ for the restriction of an algebra $\mathcal{A}$ to the index set $J$.

\begin{lemma}[Same-Clock / Co-Indexing]
\label{C-lem:clock}
For a feedback loop to be causally closed and thermodynamically potent (capable of entropy export~\cite{Seifert2012}), there must exist non-null index overlap between control and feedback windows: there exist $J_{\mathrm{ctrl}},J_{\mathrm{env}}\subseteq I$ such that
\begin{enumerate}
    \item \textbf{Non-null intersection:} $J_{\mathrm{ctrl}}\cap J_{\mathrm{env}}\neq\emptyset$.
    \item \textbf{Window integrity:} on any critical lookback window $W\subseteq J_{\mathrm{ctrl}}\cap J_{\mathrm{env}}$ used to define the controller, $\mathcal{A}_{\mathrm{ext}}|_W$ is well-defined (no holes on $W$).
\end{enumerate}
\end{lemma}

\begin{proof}
If $J_{\mathrm{ctrl}}\cap J_{\mathrm{env}}=\emptyset$, the control action is operationally decoupled from environmental feedback, so no entropy export channel exists; persistence (NESS~\cite{Seifert2012}) fails.
If window integrity fails on a critical lookback window $W$, the feedback map---and thus the Lyapunov descent condition (Eq.~(4) of Paper B (Chapter~\ref{chap:paperB}))---is not definable on the operational window.
Therefore both conditions are necessary.
\end{proof}

% ============================================================
\section{The Closure Theorem}
\label{C-sec:closure}

\begin{theorem}[Q-RAIF Algebraic Closure]
\label{C-thm:closure}
Assume $\mathcal{A}_{\mathrm{ext}} \supseteq Cl(1,3)$ (Paper A (Chapter~\ref{chap:paperA})).
Let $R$ be a persistent subsystem whose control algebra satisfies $\mathcal{O}_{\mathrm{int}}\cong Cl(V,q)$ (Paper B (Chapter~\ref{chap:paperB})).
If $\mathcal{O}_{\mathrm{int}}$ is realizable in $\mathcal{A}_{\mathrm{ext}}$ (Definition~\ref{C-def:realizability}) and the Same-Clock conditions of Lemma~\ref{C-lem:clock} hold, then the effective control algebra
\begin{equation}
    \mathcal{O}_{\mathrm{eff}} := \mathrm{Im}(\phi) \subseteq \mathcal{A}_{\mathrm{ext}}
\end{equation}
is a Clifford subalgebra of the external geometry.
\end{theorem}

\begin{proof}
By realizability, there exists a homomorphism $\phi:\mathcal{O}_{\mathrm{int}} \to \mathcal{A}_{\mathrm{ext}}$ with non-trivial image.
The operational content of the controller is its image $\mathcal{O}_{\mathrm{eff}} = \mathrm{Im}(\phi)$.
Since $\mathcal{O}_{\mathrm{int}} \cong Cl(V,q)$ by the persistence requirement (Theorem~1 of Paper B), and $\phi$ is structure-preserving, $\mathcal{O}_{\mathrm{eff}}$ inherits the Clifford relations $v^2 = q(v)\mathbf{1}$~\cite{Hestenes1966}.
Since $\mathcal{O}_{\mathrm{eff}} \subseteq \mathcal{A}_{\mathrm{ext}}$, the internal geometry $(V,q)$ is induced by a restriction of the ambient algebraic structure.
\end{proof}

\begin{corollary}[No Ghost Algebra]
A control algebra that is mathematically stable (Cliffordian) but not representable in $\mathcal{A}_{\mathrm{ext}}$ is not physically realizable.
In particular, a control structure with signature incompatible with $(1,3)$ cannot underwrite persistent feedback in a universe whose observable algebra contains $Cl(1,3)$.
\end{corollary}

% ============================================================
\section{Discussion}
\label{C-sec:discussion}

\subsection{What This Result Does and Does Not Show}

\textbf{Does show:}
Realizability forces the internal control algebra of a persistent subsystem to embed into the external observable algebra.
Combined with Papers A and B, this converts the previously heuristic convergence ($Cl(V,q)$ from stability, $Cl(1,3)$ from geometry) into a constrained embedding: $Cl(V,q) \hookrightarrow Cl(1,3)$.

\textbf{Does not show:}
That $\phi$ must be injective (faithful)---the theorem holds for any non-trivial homomorphism.
That the specific signature $(V,q)$ is uniquely determined---only that it must be compatible with $(1,3)$.
That this constitutes a derivation of physics from first principles---it is a consistency constraint within the Q-RAIF framework.

\subsection{The Bridge Statement}

\begin{remark}[Closing the Loop]
Paper A fixes the realizable operator content of the world ($\mathcal{A}_{\mathrm{ext}}$).
Paper B fixes the algebraic form required for persistence ($\mathcal{O}_{\mathrm{int}}$).
Theorem~\ref{C-thm:closure} locks them together: realizable persistence forces the agent's control algebra to be built from the same algebraic atoms as its environment.
The fish's gills must be made of water's molecules.
\end{remark}

\subsection{Connection to HAFF}

Within the HAFF program~\cite{Liu2026HAFF_A,Liu2026HAFF_B}, geometry emerges from coarse-graining of observable algebras.
The Closure Theorem adds a further structural consequence: not only does the world's geometry emerge from its algebra, but any persistent subsystem's internal geometry is \emph{constrained to be a restriction} of that emergent geometry.
This is algebraic natural selection operating at the level of geometric structure.

% ============================================================


% ============================================================================
% UNIFIED BIBLIOGRAPHY
% ============================================================================
\begin{thebibliography}{99}

\bibitem{Breuer2002}
H.-P.~Breuer and F.~Petruccione, \emph{The Theory of Open Quantum Systems}, Oxford University Press (2002).

\bibitem{Connes1994}
A.~Connes, \emph{Noncommutative Geometry}, Academic Press (1994).

\bibitem{Doran2003}
C.~Doran and A.~Lasenby, \emph{Geometric Algebra for Physicists}, Cambridge University Press (2003).

\bibitem{Faulkner2014}
T.~Faulkner, M.~Guica, T.~Hartman, R.~C.~Myers, and M.~Van~Raamsdonk, \emph{Gravitation from Entanglement in Holographic CFTs}, JHEP \textbf{03}, 051 (2014).

\bibitem{Gunaydin1973}
M.~G\"unaydin and F.~G\"ursey, \emph{Quark structure and octonions}, J.\ Math.\ Phys.\ \textbf{14}, 1651 (1973).

\bibitem{Haag1996}
R.~Haag, \emph{Local Quantum Physics: Fields, Particles, Algebras}, Springer-Verlag (1996).

\bibitem{Hestenes1966}
D.~Hestenes, \emph{Space-Time Algebra}, Gordon and Breach (1966).

\bibitem{Jacobson1995}
T.~Jacobson, \emph{Thermodynamics of Spacetime: The Einstein Equation of State}, Phys.\ Rev.\ Lett.\ \textbf{75}, 1260 (1995).

\bibitem{Jarzynski2011}
C.~Jarzynski, \emph{Equalities and inequalities: irreversibility and the second law at the nanoscale}, Annu.\ Rev.\ Condens.\ Matter Phys.\ \textbf{2}, 329 (2011).

\bibitem{Lashkari2014}
N.~Lashkari, M.~B.~McDermott, and M.~Van~Raamsdonk, \emph{Gravitational dynamics from entanglement ``thermodynamics''}, JHEP \textbf{04}, 195 (2014).

\bibitem{Lidar2003}
D.~A.~Lidar and K.~B.~Whaley, \emph{Decoherence-free subspaces and subsystems}, in \emph{Irreversible Quantum Dynamics}, Springer, pp.~83--120 (2003).

\bibitem{Maldacena1999}
J.~M.~Maldacena, \emph{The Large N limit of superconformal field theories and supergravity}, Int.\ J.\ Theor.\ Phys.\ \textbf{38}, 1113 (1999).

\bibitem{Parrondo2015}
J.~M.~R.~Parrondo, J.~M.~Horowitz, and T.~Sagawa, \emph{Thermodynamics of information}, Nat.\ Phys.\ \textbf{11}, 131 (2015).

\bibitem{Petz1996}
D.~Petz, \emph{Monotone metrics on matrix spaces}, Linear Algebra Appl.\ \textbf{244}, 81 (1996).

\bibitem{RyuTakayanagi2006}
S.~Ryu and T.~Takayanagi, \emph{Holographic Derivation of Entanglement Entropy from AdS/CFT}, Phys.\ Rev.\ Lett.\ \textbf{96}, 181602 (2006).

\bibitem{Sagawa2012}
T.~Sagawa and M.~Ueda, \emph{Fluctuation theorem with information exchange}, Phys.\ Rev.\ E \textbf{85}, 021104 (2012).

\bibitem{Schafer1966}
R.~D.~Schafer, \emph{An Introduction to Nonassociative Algebras}, Academic Press (1966).

\bibitem{Seifert2012}
U.~Seifert, \emph{Stochastic thermodynamics, fluctuation theorems and molecular machines}, Rep.\ Prog.\ Phys.\ \textbf{75}, 126001 (2012).

\bibitem{Swingle2012}
B.~Swingle, \emph{Entanglement Renormalization and Holography}, Phys.\ Rev.\ D \textbf{86}, 065007 (2012).

\bibitem{Verlinde2011}
E.~Verlinde, \emph{On the Origin of Gravity and the Laws of Newton}, JHEP \textbf{04}, 029 (2011).

\bibitem{Vidal2008}
G.~Vidal, \emph{Class of quantum many-body states that can be efficiently simulated}, Phys.\ Rev.\ Lett.\ \textbf{101}, 110501 (2008).

\bibitem{Weiss2012}
U.~Weiss, \emph{Quantum Dissipative Systems}, 4th ed., World Scientific (2012).

\bibitem{WisemanMilburn2009}
H.~M.~Wiseman and G.~J.~Milburn, \emph{Quantum Measurement and Control}, Cambridge University Press (2009).

\bibitem{Witten2018}
E.~Witten, \emph{APS Medal for Exceptional Achievement in Research: Invited article on entanglement properties of quantum field theory}, Rev.\ Mod.\ Phys.\ \textbf{90}, 045003 (2018).

\bibitem{Zanardi2001}
P.~Zanardi, \emph{Virtual Quantum Subsystems}, Phys.\ Rev.\ Lett.\ \textbf{87}, 077901 (2001).

\bibitem{Liu2026HAFF_A}
S.~Liu, \emph{Emergent Geometry from Coarse-Grained Observable Algebras: The Holographic Alaya-Field Framework}, Zenodo (2026), DOI: 10.5281/zenodo.18361707.

\bibitem{Liu2026HAFF_B}
S.~Liu, \emph{Accessibility, Stability, and Emergent Geometry: Conceptual Clarifications on the Holographic Alaya-Field Framework}, Zenodo (2026), DOI: 10.5281/zenodo.18367061.

\bibitem{Liu2026QRAIF_A}
S.~Liu, \emph{Algebraic Constraints on the Emergence of Lorentzian Metrics in Entropic Gravity Frameworks}, Zenodo (2026), DOI: 10.5281/zenodo.18525877.

\bibitem{Liu2026QRAIF_B}
S.~Liu, \emph{Thermodynamic Stability Constraints on the Operator Algebra of Persistent Open Quantum Subsystems}, Zenodo (2026), DOI: 10.5281/zenodo.18525891.

\bibitem{Liu2026QRAIF_C}
S.~Liu, \emph{The Realizability Bridge: Algebraic Closure in the Q-RAIF Framework}, Zenodo (2026), DOI: 10.5281/zenodo.18528935.

\end{thebibliography}

\end{document}
